\section*{狭义相对论}

\subsubsection*{一. 选择题}

1. 宇宙飞船相对于地面以速度$v$作匀速直线飞行,某一时刻飞船头部的宇航员向飞船尾部发出一个光讯号,经过$\Delta t$(飞船上的钟)时间后,被尾部的接收器收到,则由此可知飞船的固有长度为\fbox{$c\Delta t$}.

2. 下列说法正确的是\fbox{(1)(2)(3)}.\pp
    (1) 所有惯性系对物理基本规律都是等价的.\pp
    (2) 在真空中,光的速度与光的频率、光源的运动状态无关.\pp
    (3) 任何惯性系中,光在真空中沿任何方向的传播速率都相同.

3. 一火箭的固有长度为$L$,相对与地面做匀速直线运动的速度为$v_1$,火箭上有一个人从火箭的后端向火箭前端上的一个靶子发射一颗相对于火箭速度为$v_2$的子弹.在火箭上测得子弹从射出到击中靶子的时间间隔是\fbox{$\frac{L}{v_2}$}.

4. 在狭义相对论中,下列说法正确的是\fbox{(1)(2)(4)}.\pp
    (1) 一切运动物体相对于观察者的速度都不能大于真空中的光速.\pp
    (2) 质量、长度、时间的测量结果都是随物体与观察者的相对运动状态而改变的.\pp
    (3) 在一惯性系中发生于同一时刻、不同地点的两个事件在其他一切惯性系中也是同时发生的.\pp
    (4) 惯性系中的观察者观察一个与他做匀速相对运动的时钟时,会看到这时钟比与他相对静止的相同的时钟走得慢些.

5. 关于同时性的以下结论中正确的是\fbox{(C)}.\pp
    (C) 在一惯性系同一地点同时发生的两个事件,在另一惯性系一定同时发生.

6. 在某地发生两件事,静止位于该地的甲测得时间间隔为4s.若相对于甲做匀速直线运动的乙测得时间间隔为5s,则乙相对于甲的运动速度是\fbox{$\frac{3}{5}c$}.

7. 一宇航员要到离地球为5光年的星球去旅行.如果宇航员希望把这路程缩短为3光年,则他所称的火箭相对于地球的速度应是\fbox{$v=\frac{4}{5}c$}.

10.
    (1) 对某观察者来说,发生在某惯性系中的同一地点、同一时刻的两个事件,对于相对该惯性系下匀速直线运动的其他惯性系中的观察者来说,它们\fbox{同时}发生.
    (2) 在某惯性系中的同一时刻、不同地点的两个事件,它们在其他惯性系中\fbox{不同时}发生.

12. 一匀质矩形薄板,在它静止时测得其长为$a$,宽为$b$,质量为$m_0$,由此可算出其面积密度为$m_0/ab$.假定该薄板沿长度方向以接近光速的速度$v$做匀速直线运动,此时再测算该矩形薄板的面积密度则为\fbox{$\frac{m_0}{ab[1-{(\frac{v}{c})}^2]}$}.

13. 设某微观粒子的总能量是它静止能量的$K$倍,则其运动速度的大小为\fbox{$\frac{c}{K}\sqrt{K^2-1}$}.

\subsubsection*{二. 填空题}

16. 狭义相对论的两条基本原理中,相对性原理说的是\ul{一切彼此相对做匀速直线运动的惯性系对于物理学定律都是等价的};光速不变原理说的是\ul{一切惯性系中真空中的光速都是相等的}.

20. $\pi^+$介子是不稳定的粒子,在它自己的参照系中测得平均寿命是$2.6\E{-8}$s,如果它相对与实验室已$0.8c$的速率运动,那么实验室坐标系中测得的$\pi^+$介子寿命是\fbox{$4.33\E{-8}$}s.

21. 一观察者测得一沿米尺长度方向匀速运动着的米尺长度为0.5m,则此米尺以速度$v=$\fbox{$2.60\E8$}$\ms$接近观察者.

26. 狭义相对论确认,时间和空间的测量值都是\ul{相对的},它们与观察者的\ul{运动}密切相关.

28. 狭义相对论中,一质点的质量$m$与速度$v$的关系式为\fbox{$m=\frac{m_0}{\sqrt{1-{(\frac{v}{c})}^2}}$};其动能的表达式为\fbox{$E_k=mc^2-m_0c^2$}.

29. 质子在加速器中被加速,当其动能为静止能量的3倍时,其质量为静止质量的\fbox{4}倍.

30. 观察者甲以$0.8c$的速度相对于静止的观察者乙运动,若甲携带一长度为$l$、截面积为$S$,质量为$m$的棒,这根棒安放在运动方向上,则
    (1) 甲测得此棒的密度为\fbox{$\frac{m}{lS}$};
    (2) 乙测得此棒的密度为\fbox{$\frac{25m}{9lS}$}.

31. 已知一静止质量为$m_0$的粒子,其固有寿命为实验室测量到的寿命的$1/n$,则此粒子的动能是\fbox{$m_0c^2(n-1)$}.

\subsubsection*{三. 计算题}

32. 一艘宇宙飞船的船身固有长度为$L_0=90$m,相对于地面以$v=0.8c$的匀速度在地面观测站的上空飞过.\pp
    (1) 观测站测得飞船的船身通过观测站的时间间隔是多少?\pp
    (2) 宇航员测得船身通过观测站的时间间隔是多少?
\SL{(1) 观测站测得飞船船身的长度为$L=L_0\sqrt{1-{(\frac{v}{c})}^2}=54$m,\\
        则$\Delta t_1=L/v=2.25\E{-7}$s.\\
    (2) 宇航员测得飞船船身的长度为$L_0$,
        则$\Delta t_2=L_0/v=3.75\E{-7}$s.
}
33. 设有宇宙飞船$A$和$B$,固有长度均为$l_0=100$m,沿同一方向匀速飞行,在飞船$B$上观测到飞船$A$的船头、船尾经过飞船$B$船头的时间间隔$\Delta t=\frac{5}{3}\E{-7}$s,求飞船$B$相对于飞船$A$的速度的大小.
\SL{设飞船$A$相对于飞船$B$的速度大小为$v$,这也就是飞船$B$相对于飞船$A$的速度大小.在飞船上测得飞船$A$的长度为$l=l_0\sqrt{1-{(\frac{v}{c})}^2}$,\\
    故在飞船$B$上测得飞船$A$相对于飞船$B$的速度为$v=\frac{l}{\Delta t}=\frac{l_0}{\Delta t}\sqrt{1-{(\frac{v}{c})}^2}$,\\
    解得$v=\frac{\frac{l_0}{\Delta t}}{\sqrt{1+(\frac{l_0}{c\Delta t})^2}}=2.68\E8\ms$.
}
34. 在惯性系$S$中,有两事件发生于同一地点,且第二事件比第一事件晚发生$\Delta t=2$s;而在另一惯性系$S'$中,观测第二事件比第一事件晚发生$\Delta t'=3$s.那么在$S'$系中发生两事件的地点之间的距离是多少?
\SL{令$S'$系与$S$系的相对速度为$v$,有$\Delta t'=\frac{\Delta t}{\sqrt{1-{(\frac{v}{c})}^2}}$, $(\frac{\Delta t}{\Delta t'})^2=1-{(\frac{v}{c})}^2$,\\
    则$v=c\sqrt{1-(\frac{\Delta t}{\Delta t'})^2}=2.24\E8\ms$.\\
    那么,在$S'$系中测得两事件之间距离为$\Delta x'=v\Delta t'=c\sqrt{\Delta t'^2-\Delta t^2}=6.72\E{8}$m.
}
