\section{前言}

社会矛盾,指社会事物、现象内部既同一又斗争的关系。社会矛盾既具有普遍性又具有特殊性。在不同社会形态中,在社会生活的各个领域都自始至终存在着社会矛盾;不同社会形态、同一社会形态的不同发展阶段都包含社会矛盾,不同社会生活方面所包含的社会矛盾,以及这些矛盾的各个方面,都是不一样的、有差别的,解决这些矛盾的方式各有不同。研究社会矛盾的特殊性,具体地分析具体情况,才能科学地认识社会,找到社会矛盾的恰当解决方式。在社会矛盾中,社会基本矛盾是生产力和生产关系、经济基础和上层建筑的矛盾,它规定和制约着其它社会矛盾,决定社会的性质。

人类社会,则是以人为中心在劳动生产中按一定方式结成的各种物质和思想关系的有机系统。人类社会具有复杂的结构,包括人们改造自然、创造新的“人化自然”的生产力系统,与生产力性质和发展水平相适应的各种物质经济关系,建立在一定经济基础之上的上层建筑以及语言思维、教育科学和技术与政治、法律、道德、哲学、艺术、宗教等形成的社会意识形态(含有风俗、习惯、传统等社会心理),由特定社会关系或纽带联结起来的各种人群共同体(如国家、阶级、政党、社团等等)。人类社会的发展是各种社会要素、各种社会力量综合作用的结果,其中生产力是社会发展的物质基础,是最终动力。社会宣传涉及社会的各种因素,宣传的目的是为了促进社会内部矛盾和生产力发展相适应。人类社会按经济生产关系制度划分,先后经历了原始社会、奴隶社会、封建社会、资本主义社会和初级的社会主义社会,按生产交换方式大体经历了原始采集社会、自然经济的农耕社会、商品经济社会。每一种社会形态都伴随着社会宣传,以维护或批评某种社会形态。

\section{正文}

社会矛盾是如何推动人类社会向前发展的?范围不能太大,我能力有限,所以这里我们以维基百科为准,但史论还是以我的想法,来讲讲东亚文化圈(主要是中日关系)的一些旧事。

战国末期,秦始皇统一六国,大批移民迁往日本列岛,并带去了农耕文明。汉时倭奴国开始奉贡朝贺。隋时小野妹子(好名字)来随表明独立国地位。唐时,日本进行大化改新,并随之学习去打量唐文化。后来,唐日因朝鲜半岛开战,虽然胜利,但新罗随后独立。日本既已知大唐强盛,故又开始学习,佛教开始在日本生根。安史之乱爆发,随著藩镇割据日益严重,内乱严重,日本遣唐逐渐减少。五代十国时,汉族与其他民族矛盾严重激化,战争频频,对外势力减弱,交趾(越南)因此独立为藩属国,西夏得以残延,契丹辽国得以建立。宋立,北伐辽国屡次不成,西夏也未能消灭,但由于经济政策开放,宋日之间仍有民间来往。女真灭辽后建立金国,并灭北宋。南宋时期,由于大部份民众不愿开战,导致金国、西夏依然割据。大蒙古国开始征伐并灭金夏,分裂后忽必烈改国号大元。厓山海战,宋军全军覆没,丞相负幼帝投海殉国,元统一全国。元自恃孤傲,希望四夷来朝,日本以“厓山以后无中国”理由拒绝,元发动元日战争以失败告终。元朝短命,后期日本开始在中国沿海采取报复行动即倭寇。明太祖灭元并统一全国,明成祖派遣郑和下西洋并向日本提出朝贡要求,日本再次成为中国附属国。丰臣秀吉欲求独立,发动万历朝鲜战争,以大败告终,但也削弱了明朝万历以来积攒的国力。倭寇增多,明朝实施海禁。明末,欧洲文艺复兴逐渐显现进步力量。崇祯年间,一方面皇帝鼓励向欧洲吸取先进文化,一方面由于粮食收获不敌人口增长而民变四起,一方面女真开始觊觎中原。李自成入京,崇祯帝自缢,满洲军入关。清军入关以后,实行剃发易服令,并进行了惨无人道的扬州十日和嘉定三屠,总死亡人数有五十万更甚。同时南明成立与清廷对抗,日本也有向南明朝贡。南明郑成功收复被荷兰占领的台湾,自此台湾有了汉人政府。后明郑降清,台湾进入清治时期,日本不再朝贡于清。清只有朝鲜和琉球为藩属国,越南也因中法战争而归为法国统治。为争夺朝鲜半岛,日清战争(甲午中日战争)爆发,以清政府惨败告终,朝鲜独立归日本管辖,台湾也进入日治时期。琉球随后也因清廷保护不力,被日本占据。其间太平天国乃至义和团都给中国历史留下了不光彩的一记。

由于满清政府对汉人的压迫,自清初就有民间汉人组织的白莲教、洪门等反清组织。清末汉人复国运动兴致高涨,孙文留学日本,后任中国同盟会总理,十次革命,终于武昌首义成功,创立中华民国。但由于国内仍然大乱,民国政府不能够掌握全国权力,故请袁世凯逼迫清帝退位,并改组中国国民党,随之袁世凯就任民国大总统。宋教仁遭暗杀后,孙文发动二次革命失败,随后于日本成立中国革命党讨袁。此时一战爆发,袁世凯政府则在与日本谈判,在国力衰弱的情况下力争权益,终于签订了二十一条,但只接受了前四条。为巩固国民基础,袁世凯决心建立君主立宪制度,成立中华帝国,改元洪宪。孙文护国运动成功,袁世凯无奈宣布退位并很快驾崩,国内又复军阀割据局面。孙文采取联俄容共与国共合作的策略,并双手奉送外蒙古,借此壮大了自己军事力量,开始北伐,并在途中逝世。北伐途中国民党内部分而复合,共产党趁机发动数次运动,蒋介石不得不在北伐军阀的同时著手解决共产党的问题,国共合作宣布破裂进入对峙状态。北伐军则随后宣布了北伐的胜利,完成了明讨元以来第二次成功的北伐,民国政府正式成立,并改换青天白日满地红旗。随后一方面进入了国民经济建设发展的黄金十年,一方面政府开始了对共产党军队的围剿,一方面日本又开始觊觎中国并发动挑衅。蒋介石的隐忍建设政策被共产党发动的七七西安事变打破,中国进入了八年抗日战争时期。最终美英苏以中国正式让出外蒙古的代价向日本施压,投下历史上唯二的原子弹迫使日本投降,日本右翼政府随即倒台。民国政府随之收复了台湾,却发生了二二八事件。用空间和时间换取来了抗日战争的胜利,却因此消耗了国民党军队大量元气,也使黄金十年的建设成果所剩无几,共产党军队也在养育了一支强大的武装力量。民国政府开始行宪,但和平谈判破裂,国共战争全面爆发。于是八年抗日积攒下来的解放军力量在短短三年内击败了疲惫不堪的国民军队,国民党败退台湾,中国共产党正式在中国大陆建立中华人民共和国政权。大陆政府随后代替台湾政府称为联合国常任理事国,得到国际承认。

\section{总结}

可以看到,社会矛盾正是推动人类社会向前发展的原动力。从上面的描述来看,民族矛盾是中国历史中关键的部分,也是某些时期的主要社会矛盾。作为农耕民族的汉族,和周围的游牧民族产生冲突。大部分情况下都是汉族取得了胜利,少数情况下被少数民族获得了政权,是矛盾交替的正常结果。在和外民族的斗争中,日本作为特殊的民族,扮演了朋友和敌人的双重角色。在汉族强盛的时候,大和民族对我们俯首称臣;在汉族有难的时候,有时互帮互助,有时落井下石;在汉族衰弱的时候,自立小中华;在汉族混乱的时候,趁机发动战争获取利益。这种矛盾正体现了利益的不可调和性,也正是本文的论题所在。