\section{调查问卷}
\label{wenquan}

\subsubsection{网络语言测试}

请完成以下选择题,在每一题中选出您认为恰当的答案:
\begin{enumerate}
\item “愤怒哥”面对记者的镜头问:“我能说脏话吗?”请问,他对何事如此愤怒?
  \\\begin{enumerate*}
  \item 房价上涨 %
  \item 油价上涨
  \item 飞机燃油附加费上涨
  \item 他的自行车被偷了
  \end{enumerate*}
\item 日本人河源启一郎骑自行车环游世界,到中国哪个城市时,自行车被偷了?
  \\\begin{enumerate*}
  \item 西安
  \item 南京
  \item 武汉 %
  \item 长沙
  \end{enumerate*}
\item 被恶搞的“杜甫很忙”,是中学语文课本中哪首诗的插图?
  \\\begin{enumerate*}
  \item 《石壕吏》
  \item 《登高》 %
  \item 《客至》
  \item 《春望》
  \end{enumerate*}
\item “哲♂学”的主要人物是?
  \\\begin{enumerate*}
  \item 花泽香菜
  \item 姚明
  \item 郭敬明
  \item 比利$\cdot$海灵顿 %
  \end{enumerate*}
\item 下列哪个不是“Aji人”的特点?
  \begin{enumerate}
  \item 喜欢玩扩散性百万亚瑟王、偶像大师 灰姑娘女孩、LoveLive!学园偶像祭等课金游戏
  \item 单纯地喜欢 LoveLive! 这一企划 %
  \item 经常盲目跟风大型建造却毫无收获以至于垃圾人群内经常出现满屏哀嚎,夜以继日肝船
  \item 看到朋友出UR或者自己想要却抽不到的SR卡会开玩笑地骂人和打人
  \end{enumerate}
\end{enumerate}

\subsubsection{熟悉度}

对于下列网络语言,请根据自己对它们的了解程度填写表格:
\newcounter{thei}[subsubsection]
\newcommand\ii{\stepcounter{thei}\arabic{thei}}
\begin{longtable}{|c|c|c|c|c|c|c|}
\hline& 网络语言单字 & 熟悉 & 了解 & 一般 & 不清楚 & 没见过\\\hline
\ii & 囧 &  &  &  &  & \\\hline
\ii & 雷 &  &  &  &  & \\\hline
\ii & 槑 &  &  &  &  & \\\hline
\ii & 宅 &  &  &  &  & \\\hline
\ii & 腐 &  &  &  &  & \\\hline
\ii & 艹 &  &  &  &  & \\\hline
\ii & 萌 &  &  &  &  & \\\hline
\ii & 赞 &  &  &  &  & \\\hline
\ii & 顶 &  &  &  &  & \\\hline
\ii & 踩 &  &  &  &  & \\\hline
\end{longtable}
\begin{longtable}{|c|c|c|c|c|c|c|}
\hline & 网络语言词语 & 熟悉 & 了解 & 一般 & 不清楚 & 没见过\\\hline
\ii & 233 &  &  &  &  & \\\hline
\ii & NB &  &  &  &  & \\\hline
\ii & prpr &  &  &  &  & \\\hline
\ii & wwww &  &  &  &  & \\\hline
\ii & 兹磁 &  &  &  &  & \\\hline
\ii & 喜大普奔 &  &  &  &  & \\\hline
\ii & 拙计 &  &  &  &  & \\\hline
\ii & 楼主 &  &  &  &  & \\\hline
\ii & 吐槽 &  &  &  &  & \\\hline
\ii & 卧槽 &  &  &  &  & \\\hline
\end{longtable}
\begin{longtable}{|c|l|c|c|c|c|c|}
\hline & 网络语言成语 & 熟悉 & 了解 & 一般 & 不清楚 & 没见过\\\hline
\ii & 你妈喊你回家吃饭 &  &  &  &  & \\\hline
\ii & $\times\times$大法好 &  &  &  &  & \\\hline
\ii & 图样图森破 &  &  &  &  & \\\hline
\ii & 开门,顺丰快递 &  &  &  &  & \\\hline
\ii & 德国骨科世界第一 &  &  &  &  & \\\hline
\ii & 我兔一盘大棋 &  &  &  &  & \\\hline
\ii & 你国药丸 &  &  &  &  & \\\hline
\ii & 我和我的小伙伴们都惊呆了 &  &  &  &  & \\\hline
\ii & 我觉得自己是零 &  &  &  &  & \\\hline
\ii & 神秘代码呢 &  &  &  &  & \\\hline
\end{longtable}

\subsubsection{喜欢网络语言的原因}

\begin{enumerate}
\item 你喜欢使用网络语言吗?
  \\\begin{enumerate*}
  \item 非常喜欢
  \item 喜欢
  \item 一般
  \item 不喜欢
  \item 讨厌
  \end{enumerate*}
\item 你喜欢使用网络语言的原因是什么(多选):
  \\\begin{enumerate*}
  \item 简单易懂
  \item 幽默生动
  \item 个性时尚
  \item 方便快捷
  \item 气氛轻松
  \item 更好表达
  \item 跟风从众
  \end{enumerate*}
\item 你不喜欢网络语言的原因是什么(多选):
  \\\begin{enumerate*}
  \item 完全看不懂
  \item 觉得没必要
  \item 不够规范
  \item 破坏语言规则
  \item 难以掌握
  \item 其他:
  \end{enumerate*}
\end{enumerate}

\subsubsection{使用范围及态度}

\begin{enumerate}
\item 您在以下媒介中能听到或看到网络语言符号吗(多选):
  \\\begin{enumerate*}
  \item 广播
  \item 电视
  \item 报纸
  \item 杂志
  \item 手机
  \end{enumerate*}
\item 您会在网络聊天时使用网络语言吗?
  \\\begin{enumerate*}
  \item 频繁
  \item 经常
  \item 一般
  \item 偶尔
  \item 从不
  \end{enumerate*}
\item 您一般会对哪些人使用网络语言(多选):
  \\\begin{enumerate*}
  \item 网友
  \item 同学
  \item 朋友
  \item 长辈
  \item 上级
  \item 下级
  \item 学生
  \item 其他:
  \end{enumerate*}
\item 网络人际交往中,您一般在什么情况下使用网络语言(多选):
  \\\begin{enumerate*}
  \item 微博
  \item QQ
  \item 贴吧
  \item 论坛
  \item 博客
  \item 电子邮件
  \item 游戏
  \item 其他:
  \end{enumerate*}
\item 网络语言中经常使用缩略语和自造语,您会“自造”网络词汇吗?
  \\\begin{enumerate*}
  \item 经常会
  \item 较多会
  \item 偶尔会
  \item 从来不会
  \end{enumerate*}
\item 你对网络语言的态度是:
  \\\begin{enumerate*}
  \item 举双手赞成
  \item 拥护
  \item 中立
  \item 不赞成
  \item 坚决反对
  \end{enumerate*}
\end{enumerate}

\subsubsection{被调查者上网情况}

\begin{enumerate}
\item 您有可以上网的电脑吗?
  \begin{enumerate}
  \item 有
  \item 有,但不能上网
  \item 没有
  \end{enumerate}
\item 您有可以上网的手机吗?
  \begin{enumerate}
  \item 有可以上网的智能手机
  \item 有可以上网的手机
  \item 有手机,但不能上网
  \item 没有
  \end{enumerate}
\item 您第一次上网是在\underline{\makebox[5em][c]{}}年。
\item 您每周上网时间大概有\underline{\makebox[5em][c]{}}小时。
\item 您一般都在哪里上网?
  \\\begin{enumerate*}
  \item 家里
  \item 机房
  \item 图书馆
  \item 网吧
  \item 宿舍
  \item 别人家
  \item 其他:
  \end{enumerate*}
\item 您主要使用哪些网络服务?
  \begin{enumerate}
  \item 搜索引擎:百度、Google
  \item 电子邮件
  \item 即时通讯:QQ
  \item 网络购物:淘宝、京东、当当、亚马逊
  \item 网络游戏
  \item 论坛和 BBS
  \item 贴吧和揭示板
  \item 博客
  \item 微博
  \item 浏览网页
  \item 其他:
  \end{enumerate}
\end{enumerate}

\subsubsection{信息采集}

\begin{enumerate}
\item 您的性别是?
  \\\begin{enumerate*}
  \item 男
  \item 女
  \item 其他
  \end{enumerate*}
\item 您的身份是?
  \begin{enumerate}
  \item 初中生:
    \begin{enumerate*}
    \item 初一
    \item 中二
    \item 初三
    \end{enumerate*}
  \item 高中生:
    \begin{enumerate*}
    \item 高一
    \item 高二
    \item 高三
    \end{enumerate*}
  \item 大学本科生:
    \begin{enumerate*}
    \item 大一
    \item 大二
    \item 大三
    \item 大四
    \end{enumerate*}
  \item 硕士生
  \item 博士生
  \end{enumerate}
\end{enumerate}
