\begin{thebibliography}{99}
\bibitem{cnnic} 中国互联网络信息中心. 第33次中国互联网络发展状况统计报告. %\url{http://www.cnnic.net.cn/hlwfzyj/hlwxzbg/hlwtjbg/201403/P020140305346585959798.pdf}.
 2014年1月.
\bibitem{cao} 曹进. 网络语言传播导论. 北京:清华大学出版社. 2012.
\bibitem{webm} 吕恒怡. 中国大陆网络语言:一种新形态的社会方言. Sprak-och litteraturcentrum, kinesisk. 2010.
\bibitem{webt} 于根元. 中国网络语言词典. 北京:中国经济出版社. 2001.
\bibitem{bai} 柏定国. 网络传播与文学. 北京:中国文史出版社. 2008.
\bibitem{qin} 秦秀白. 网语和网话. 外语电化教学. 教育技术通讯. 2003.
\bibitem{deng} 邓雪. 浅议网络语言的形成与发展趋势. 《新闻研究导刊》2014年 第3期 29-31页.
\bibitem{qi} 祁卉璇, 张红雷. 大学校园网络语言和流行语教育价值研究. 《中国外资》2014年 第6期 250-251页.
\bibitem{zuo} 刘建明, 张明根. 应用写作大百科. 北京:中央民族大学出版社. 1994.
\bibitem{2ch} {\mincho 2ちゃんねる用語}. \url{http://2channel-vocabulary.com/}.
\bibitem{2ten} {\mincho 2典}. \url{http://media-k.co.jp/jiten/}.
\bibitem{evchk} 香港网络大典. \url{http://evchk.wikia.com/}.
\bibitem{moegirl} 萌娘百科. \url{http://wiki.moegirl.org/}.
\bibitem{wiki} 维基百科. \url{http://zh.wikipedia.com}.
\bibitem{baike} 百度百科. \url{http://baike.baidu.com}.
\end{thebibliography}
