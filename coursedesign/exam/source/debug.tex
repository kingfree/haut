\section{调试分析}

基于版本控制的编程开发是非常有效的,Git 和 Github 为我的程序代码管理提供了很大的帮助。

程序的调试时一个需要耐心而艰苦反复的工作。我使用多种方法进行调试:
\begin{description}[labelindent=3em,labelwidth=3em]
\item[静态] 使用 \verb|fprintf(stderr, ...)| 函数来输出中间变量的值;
\item[GDB] 在命令行下使用 \verb|gdb| 来跟踪程序并设置端点等;
\item[IDE] 后期换用了 Visual Studio 2013,其调试的功能也是非常强大。
\end{description}

我遵循了类似敏捷的开发方式,保证每个版本可用的情况下进行迭代开发。每增加一个新功能我都会在版本号加上 0.0.1,所以通过版本号可以看出我已经进行了 40 多次的迭代开发了,大概用了一个星期的时间。

有几次迭代实际上不能正常运行,所以没有提交到版本库。以下是 Git 的提交历史记录:
\begin{asparaenum}[1.]
\item 2014-07-02 16:49:30 Init.
\item 2014-07-02 17:42:48 Use
\item 2014-07-03 14:15:42 界面可用
\item 2014-07-03 18:46:13 实现浏览和插入
\item 2014-07-04 16:18:31 兼容VS和nmake
\item 2014-07-04 16:50:27 更好的解耦
\item 2014-07-04 18:29:48 实现修改
\item 2014-07-05 14:16:40 修补内存泄露
\item 2014-07-05 15:12:02 实现删除
\item 2014-07-05 18:10:24 实现条件查询
\item 2014-07-05 21:13:37 实现所有单项查询
\item 2014-07-05 21:53:33 实现模糊查询
\item 2014-07-05 22:54:17 增强输入稳定性
\item 2014-07-05 23:21:13 命令行下跨平台兼容
\item 2014-07-06 16:36:48 开始编写试卷类
\item 2014-07-06 18:20:21 实现随机组题
\item 2014-07-06 19:27:57 布置好主要界面
\item 2014-07-07 11:29:25 实现几个组卷算法
\item 2014-07-07 11:36:28 实现组卷
\item 2014-07-07 16:27:41 不稳定的用户登录功能
\item 2014-07-07 17:40:38 用户登录注册可用
\item 2014-07-07 21:05:32 使用封装的链表操作题目
\item 2014-07-07 22:48:45 对用户界面隐藏单链表
\item 2014-07-08 10:14:52 试卷封装
\item 2014-07-08 11:24:45 做题界面和成绩初始化
\item 2014-07-08 14:22:35 考试成绩读写
\item 2014-07-08 14:45:52 完成所有既定功能
\item 2014-07-09 11:01:06 整理程序架构
\end{asparaenum}

程序编写过程中,必然出现过许多 Bug。当然,这些 Bug 都已经被及时修复了,如今复现难度比较大。

课程设计的调试目的是学会使用调试工具和学会修补 Bug,我想目的已经达到,不必要再走形式,详细地把调试过程列出来了。
