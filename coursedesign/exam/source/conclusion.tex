\section{心得总结}

(废话挺多,不看也罢。)

我的编程经验,大概有六年了吧。开始初中学习 Pascal 参加信息学竞赛,高中开始改用 C/C++,虽然没有在竞赛里拿到保送资格,不过还是积累了一定的编程经验。工具语言也从初中的易语言到高中的 Python 和 PHP,至今又非常喜欢 Ruby,也学习了各种有趣的编程语言比如 Go、Lisp、JavaScript 之流。这次课程完成后,预定要重新深入学习一下 C\# 和 Java 了。

当然,语言是很简单的东西。我认为,重要的是这些语言相关的工具链。语言没有好的生态环境和社区支持,没有优秀的强大的类库,是很难发展起来的,因为会很难用。C 语言之所以能走到今天,其实完全是 UNIX 的功劳 \cite{krc_c}。

从前受王垠\cite{wylinux}的影响,非常推崇 Linux 的生活方式,顺便也接触到了 \TeX —— 也就是这篇文档所用的排版软件。我用 \LaTeXe 已经有四年了吧,期间这玩意儿一直没怎么成长,我也只是用而已。

说一下这篇文档吧,目录啊、交叉引用啊、参考文献啊都是自动生成的。页面格式也是经过精细控制得到的结果。最头疼的部分是画图,尝试了数十个画图的宏包,最终选定了 \verb+tikz-uml+ 来画 UML 图。实际上我不会什么 UML,OO 也是一知半解,临阵磨枪,反正硬着头皮上也就是这结果了。实在有的地方画不出来图,就直接上代码了。当然我觉得使用伪代码更好,不过本程序不怎么偏重于算法,用伪代码也会显得很单调。倒是 \verb+smartdiagram+ 画出来需求分析的几个彩色圆圈挺好看的,赏心悦目。这些宏包都是基于 Ti{\it k}Z \& PGF 的,也是足够强大。

为什么不用 Visio 之类的软件呢?是因为我实在不喜欢这种图形界面的绘图软件画出来的图难以控制,导出来生成一堆乱七八糟的代码,实在恶心的受不了。当然话数学几何图形用 GeoGebra 是能生成非常好看的图形的,可惜 UML 似乎没有这样好用的软件,也可能是我没发现吧。不过什么东西都是熟能生巧,多用用就好了。

说说程序本身吧。完全遵循 ANSI C,可以在 Linux GCC 编译器上编译通过正常运行,也可以在 Windows 平台上 使用 MinGW GCC 编译运行。后来是尝试了 Visual Studio 感觉很好用,就移植到 Microsoft Visual C++ 2013 平台上,经过修改终于可以完美运行。于是这个程序就是跨平台的了,在 Windows 上设置好 \verb+vcvars32+ 环境变量,就可以直接 \verb+nmake /f Makefile.vs+ 来自动编译并连接生成 EXE 可执行文件,当然用 VS 建立工程后把文件导入也可以正常编译的;在 Linux 上也是直接 \verb+make+ 就可以编译连接生成 ELF 可执行文件。当然,有些平台相关特性,比如清屏、暂停、输入掩码,我都用预编译命令进行了处理,以兼容不同操作系统。

本程序是开源的,以 BSD 许可证发布,是个挺宽容的许可证。以前挺疯狂追崇 GPL 的,不过现在也不那么狂热了,宽容点好。SList 类不是我写的,是 LGPL 协议。代码托管在 Github 上,网址是 \url{https://github.com/kingfree/haut/tree/master/clang/exam/exam}。

\vspace{1em}

好像说了很多技术上的问题,该说说这次课程应该得到的收获吧。总的来说就是 C 语言的综合应用,目的是数量掌握 C 语言,掌握程序设计的基本方法,对软件工程产生初步的认识,为后来的学习打下基础。

这次程序开发,难度有点超乎想象了。说实话,用 C 语言写这么大的项目确实还是第一次,为了遵循相关开发标准也适应了一段时间。实际上整个开发时间是六个半天,代码有 2400 多行,源文件达到 17 个。连续几日以比较高负荷的状态来编写代码,确实是非常疲累。但实际上编码的过程也是很充实的,因为我本身就喜欢编程,所以倒也没什么厌恶情绪,其实也是乐在其中。

这次课程呢,作为一次锻炼的机会,可以投入很大的精力来专心做一件事,是平常的学习中所没有的。尤其是作为一个准程序员,这样的训练其实应该有更多。既苦逼又充实的生活,其实也是挺想要的。

\vspace{1em}

再有,通过这次实践,我发现了更多不会的东西。我对面向对象和设计模式还是一知半解,计划通过学习 C\# 和 Java 来深入学习,至于 C++ 我有点不太喜欢就暂时搁置。另外也要继续深入学习 Linux,包括 TCP/IP、UNIX 编程、套接字、进程间通信等。对于 Web 开发,要继续深入掌握 JavaScript,熟练运用 Python 和 Ruby配置服务器和编写脚本。在思维方面,要读一读侯世德的《集异璧之大成》。基本上计划如此,不过还是会在途中增加很多内容的,毕竟知识实在太多,我会的实在太少。

学习是一个漫长而复杂的过程,我希望我能够坚持走下去。当然我这个人平常是比较懒的,每天刷刷微博,看看动画片,一天就过去了。所以还是要静下心来,好好看看书,写写程序,多思考、多运动。只有充实自己,才能够有能力月入一狗,成为菊苣。总之朝着目标努力吧。
