\section{需求分析}

\subsection{任务探究}

大一最后的课程——程序设计实践,目的是加深对 C 语言的理解和使用,为后来的学习打好基础。课程设计要求做一个标准化考试系统。

% 任务书要求如下:
% \begin{quote}\small
% \tasktext
% \end{quote}

考虑基本功能,即是对题目数据库的增、删、改、查(用文件存取数据),以及按一定策略生成试卷。

考虑扩展功能,需要实现用户模块和权限处理,还有简单的成绩管理。

\subsection{界面模块}

对于本程序而言,我们不需要复杂的 GUI(图形用户界面)。因为标准 C 中并没有直接对图形操作的库,直接调用 Windows API 会无法移植到类 UNIX 操作系统,并且程序会相当复杂。如果可能的话,我更喜欢用 Web 的方式呈现,当然这也需要使用相应的编程工具链,不是标准 C 可以简单做到的。

所以我决定采用纯文本命令行界面,如非必要不对终端界面进行特殊设置。本来是要实现命令行参数接口的,限于时间关系,未能完成。目前的界面只是简单地文本状态下的用户交互,提供简单的容错处理。

考虑用户界面,提供学生和教师两个入口。如图~\ref{al_main}~所示,主界面提供了两个用户入口,并提供了帮助和关于。

\begin{figure}[htp]
\pictext
% \begin{tikzpicture}[align=center](10,4)
% % \draw[help lines] (0,0) grid (10,3);
% \node (main) at (5,4) [draw] {入口主界面};
%   \node (student) at (2,2) [draw] {学\\生\\登\\录};
%   \draw[-] (student) |- (5,3.3) -| (main);
%   \node (teacher) at (4,2) [draw] {教\\师\\登\\录};
%   \draw[-] (teacher) |- (5,3.3) -| (main);
%   \node (help) at (6,2) [draw] {帮\\助};
%   \draw[-] (help) |- (5,3.3) -| (main);
%   \node (about) at (8,2) [draw] {关\\于};
%   \draw[-] (about) |- (5,3.3) -| (main);
% \end{tikzpicture}
\smartdiagram[bubble diagram]{主界面,
帮助,
学生\\登录,
关于,
教师\\登录
}
\caption{\label{al_main}主要模块}  
\end{figure}

\subsubsection{学生功能模块}

学生模块属于扩展功能,如图~\ref{al_student}~,提供做卷子和看成绩两个模块入口。

\begin{figure}[htp]
\pictext
\smartdiagram[constellation diagram]{学生\\主界面,
做卷子,
看成绩
}
\caption{\label{al_student}学生功能模块}  
\end{figure}

\subsubsection{教师功能模块}

教师模块是本程序的主要部分。如图~\ref{al_teacher}~所示,应该实现试题的增、删、改、查功能,实现一定策略的“智能”组卷算法。作为扩展功能,实现对成绩的查看。

\begin{figure}[htp]
\pictext
\smartdiagram[constellation diagram]{教师\\主界面,
浏览\\试题,
添加\\试题\\(增),
删除\\试题\\(删),
修改\\试题\\(改),
查询\\试题\\(查),
智能\\组卷,
成绩\\分析
}
\caption{\label{al_teacher}教师功能模块}  
\end{figure}

目前不需要做到批量增、删、改,其对象是应当是唯一的,这里我们用题目编号来指定。

对于查询操作,应该提供多种方式。如图~\ref{al_select}~,可以根据试题某个属性单独查询,也应提供模糊查询以查询所有选项。

\begin{figure}[htp]
\pictext
\smartdiagram[constellation diagram]{查询\\试题,
编号,
描述,
选项,
难度,
标签,
章节,
模糊\\查询
}
\caption{\label{al_select}试题查询模块}  
\end{figure}

对于组卷功能,我们应该提供多种算法。如图~\ref{al_generate}~,可以完全随机生成,也可以指定标签、章节、难度对试卷进行生成。另外,这里也可以查看已生成试卷列表和内容。

\begin{figure}[htp]
\pictext
\smartdiagram[constellation diagram]{智能\\组卷,
随机\\生成,
按标签\\生成,
按章节\\生成,
按难度\\生成,
试卷\\列表,
查看\\试卷
}
\caption{\label{al_generate}试卷模块}  
\end{figure}
