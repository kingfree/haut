\section{总结}

这个项目我是从2015年4月20日开始,基本代码完成于5月19日。最后到小学期,完成的这个文档。
整个过程是很有意思的,也成为了我在GitHub上提交最频繁的一段记录。
当然,还是一直在参考书\cite{osask}中的内容再做。
书的内容是中文的,但是给出的随书源代码是日文注释,不过正好会日语,所以对源代码的理解和重写也就轻松了许多。
许多函数名和变量名,实际上也都是日语的一些东西,看懂了是挺有意思的,看得出作者的用心。
比如Haribote这个词就是{\ja 「張りぼて」},意思是“纸糊的戏剧用小道具”。
也就是说,这个操作系统,就像纸糊的一样,只是个玩具,看起来还不错,其实是空的。
所以,我们也可以看到,这个操作系统,真的就像玩具一样,可以玩玩,但并不能实用。
比如它不能真正地读写文件,没有虚拟内存,不能连接网络等。

在现代操作系统中,进程、存储、I/O、文件、网络,显然是必要的五个元素。
这里呢,实际上只是对进程的实现,是相对比较丰富的。
从代码中我们可以看到,这个操作系统,是分时多任务的。
也就是说,我们为每个进程分配一个时间片,然后切换上下文来运行在队列中的进程。
这里我们也加了一个优先级,来调度不同等级的任务。
查找算法,也都是非常简单的线性算法。
因为任务很少,也没有明显的性能问题。

对于存储器的管理,实际上更简单了。
我们维护一个空闲内存块的链表,然后分配和释放都在这个链表上进行。
算法也是线性的。

I/O存取,其实这里就是对系统中断的处理了。
主要是鼠标和键盘中断,以及屏幕的绘制。

屏幕绘制,专门写了一个窗口和图层管理器。
这个算是代码量最大的一部分了,所以看起来这个操作系统代码很多,大部分都不是核心。
写屏幕很有意思,因为能够看到花花绿绿的结果,很开心,有成就感。
但实际上这些东西并不是操作系统的核心。

当然完成了这些东西,提供了一些API供应用程序使用,精力也就比较有限了。
所以文件系统没有实现,只是简单地在汇编阶段把软盘(而且只支持FAT12格式的软盘)内容全部读到内存中,再去操作内存。
同样,网络也没有实现,下学期学《计算机网络》的时候,倒不妨写一个TCP/IP玩玩。

在此期间我又读了读MINIX\cite{minix3}的源代码,这是一个分模块的操作系统实现,也从中学到了很多东西。
不过还没有去读Linux的内核源代码,这是计划了。

其实这个学期对我影响最深的是CS:APP——《深入理解计算机系统》\cite{csappv2}。
花了两个月的时间读完了。
这本书从头到尾把计算机硬件到软件讲了一遍,是很好的计算机入门书。
而且也正式从这本书中学到了汇编,对C的理解更深了一些。
不过学习的教学环境还没有这种课程,我对于这本书的理解也只是浅尝辄止,后面的作业其实没有做多少。
参考文献列在最后,\LaTeX 用来排版以及{\sc Bib}\TeX 用来列参考文献,倒是挺方便的。

操作系统原理性的东西,其实在上个世纪80年代已经全部搞明白了。
作为一个科班出身的程序员,能写OS也不是什么稀奇事了。
但是,玩具好做,真正能够把操作系统上的API设计得好,能够构建出一系列应用程序,
是一项非常浩大的工程。从内核构建到虚拟机,再到用户界面和应用程序,
只能说只有微软完成了这个完整地流程。
就算是Mac OS X和iOS,也是基于UNIX实现改过来的,说白了不过加个壳。
而Linux至今在桌面领域一团糟,衍生而来的Android虽然在移动领域二分天下,
不过也不是Google独立开发出来的,还面临着Java的版权问题。
所以这么一个庞大的生态系统,可以说难于上青天。当然现在上天倒是很随意的事情了。

话说回来,这个操作系统,还只能是玩具。
作为课程设计,这个可能有点太大了,但我觉得这一轮下来也是值得的。
这回课程设计,能够借机完成一个操作系统,也是我一直以来的愿望。
不过其实对这个玩具并不满意,除了可能GUI设计的扁平化了一些,看起来更像现代操作系统以外,
真是没有一点可爱的地方。本来还想实现汉字显示,结果折腾了半个月没弄好,
还好一直用git做版本控制,就回滚到能用的版本提交了这个版本。
如果有时间,我倒是想写一个UNIX兼容(POSIX)的内核,而不是现在这个自有API的东西。
当然这是后话了。

更多的探索还在后面,这个课程设计的完成也只是更大目标的开始。
要学的东西还有很多,所以就去更多地学习和实践了!

\vfill

\centering
\includegraphics[width=12cm]{image/laala.png}

\vspace*{1cm}
