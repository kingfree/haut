\newpage
\section*{\underline{~软件工程~}专业课程设计任务书}
% \addcontentsline{toc}{section}{课程设计任务书}
% \ThisCenterWallPaper{1}{source/task.pdf}
\begin{center}

\begin{tabular}{|c|c|}\hline
{\bf 学生姓名} & \makebox[5em][c]{\tjf} \vil \makebox[4em][c]{\bf 专业班级} \vil \quad 软件 1305 班 \quad \vil \makebox[4em][c]{\bf 学~号} \vil 201316920311 \\\hline
{\bf 题~目} & \quad \titlec \\\hline
{\bf 课题性质} & \makebox[12em][c]{其他} \vil {\bf 课题来源} \vil \makebox[11em][c]{自拟课题} \\\hline
{\bf 指导教师} & \makebox[12em][c]{刘扬} \vil {\bf 同组姓名} \vil \makebox[11em][c]{无} \\\hline
{\bf 主要内容} & {\begin{minipage}[c][5cm][c]{12cm}
操作系统是控制应用程序执行的程序,并充当应用程序和计算机硬件之间的接口。一个操作系统的主要功能有:
\begin{enumerate}
\item 处理器管理
\item 存储器管理
\item 设备管理
\item 文件管理
\end{enumerate}
现在的桌面操作系统多是多任务分时操作系统。
\end{minipage}} \\\hline
{\bf 任务要求} & {\begin{minipage}[c][5cm][c]{12cm}
目标是完成一个基本可用的图形界面操作系统,包括如下基本模块:
\begin{enumerate}
\item 进程:中断处理、多任务调度、系统保护
\item 存储管理:内存分配、进程空间管理
\item I/O 系统:鼠标、键盘和屏幕的控制
\item 文件系统:文件与可执行程序的读取和加载
\end{enumerate}

系统提供命令行用户接口和图形化用户接口,允许使用 C 语言编写系统应用程序,可以从 FAT12 格式软盘启动。
\end{minipage}} \\\hline
{\bf 参考文献} & {\begin{minipage}[c][5cm][c]{12cm}
川合秀实. 30天自制操作系统. 人民邮电出版社, 2012\\
W. Stallings. 操作系统: 精髓与设计原理(第6版). 机械工业出版社, 2010\\
R. E. Bryant, 等. 深入理解计算机系统系统(第2版). 机械工业出版社, 2010\\
A.S.Tanenbaum, 等. 操作系统设计与实现. 电子工业出版社, 2007\\
W. R. Stevens, 等. UNIX环境高级编程(第3版). 人民邮电出版社, 2014
\end{minipage}} \\\hline
{\bf 审查意见} & {\begin{minipage}[c][5cm][c]{12cm}
    指导教师签字:\\
    \vspace*{2cm}\\
    教研室主任签字:\hspace{6cm} 2015~年~6~月~25~日
\end{minipage}} \\\hline
\end{tabular}

\small 说明:本表由指导教师填写,由教研室主任审核后下达给选题学生,装订在设计(论文)首页

\end{center}

\newpage
