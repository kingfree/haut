\section{设计}

该课程设计内容,主要是以{\ja 川合秀実}老师所著《30天自制操作系统》\cite{osask}~一书中介绍的OSASK操作系统为基础的。

代码以C语言和汇编写成,其中汇编是nasm的一个方言NASK,而C语言则是ANSI C,使用gcc编译器可以编译。
编译成为启动镜像文件的\verb|Makefile|适用于Windows平台(可以移植到其他平台),在\verb|z_tools|目录中提供了
所需要的编译程序和链接库。

% \lstset{basicstyle=\small\tt}

\subsection{引导程序}
\label{sub:引导程序}

\lstset{language={[x64]Assembler}}

系统存放在一个1.44MB软盘中,其第一扇区为引导程序\verb|ipl10.bin|,作用是将软盘中的前10个柱面读入内存中。

该系统只支持读取FAT12格式,所以首先是格式化的代码。
{\linespread{1}\lstinputlisting[caption={FAT12格式软盘格式化},linerange={10-28}]{osask/src/sys/ipl10.nas}}

下面分别列出了读取一个扇区、18个扇区、10个柱面的汇编代码:
{\linespread{1}\lstinputlisting[caption={读取一个扇区},linerange={40-44,49-62,85-86,102-108}]{osask/src/sys/ipl10.nas}}
{\linespread{1}\lstinputlisting[caption={读取18个扇区},linerange={46-47,64-70}]{osask/src/sys/ipl10.nas}}
{\linespread{1}\lstinputlisting[caption={读取10个柱面},linerange={71-78,82-83,88-96,98-100}]{osask/src/sys/ipl10.nas}}

将磁盘上的内容读入到内存之后,开始载入操作系统内核。我们让操作系统进入图形模式:
{\linespread{1}\lstinputlisting[caption={启动信息},linerange={5-23}]{osask/src/sys/asmhead.nas}}

对于支持VESA BIOS扩展的BIOS,我们进入高分辨率模式(640 x 400 x 8位色):
{\linespread{1}\lstinputlisting[caption={判断VBE并进入高分辨率模式},linerange={27-73}]{osask/src/sys/asmhead.nas}}
对于不支持VBE的主板,进入低分辨率模式:
{\linespread{1}\lstinputlisting[caption={低分辨率模式},linerange={75-82}]{osask/src/sys/asmhead.nas}}

获取键盘指示灯和屏蔽终端后,开始切换进入32位模式:
{\linespread{1}\lstinputlisting[caption={进入32位模式转存数据},linerange={111-153}]{osask/src/sys/asmhead.nas}}
然后调用主函数,正式启动操作系统:
{\linespread{1}\lstinputlisting[caption={启动bootpack},linerange={157-170,200-201}]{osask/src/sys/asmhead.nas}}

\lstset{language={C}}

操作系统首先初始化中断描述符表、系统FIFO队列、鼠标键盘等:
{\linespread{1}\lstinputlisting[caption={初始化设备},linerange={53-63}]{osask/src/sys/bootpack.c}}

然后初始化内存管理器:
{\linespread{1}\lstinputlisting[caption={初始化内存管理器},linerange={65-68}]{osask/src/sys/bootpack.c}}

初始化调色板和桌面,启动一个默认的终端窗口:
{\linespread{1}\lstinputlisting[caption={初始化桌面},linerange={70-71,77-79,83-84}]{osask/src/sys/bootpack.c}}

初始化鼠标指针:
{\linespread{1}\lstinputlisting[caption={初始化鼠标},linerange={89-92}]{osask/src/sys/bootpack.c}}

这时候系统就已经算是启动完成了。接下来进入一个无限循环,该循环查询CPU中断事件,并给与响应:
{\linespread{1}\lstinputlisting[caption={主循环},linerange={104-106,110-113,126-129,136-137,239-239,330-332}]{osask/src/sys/bootpack.c}}

\subsubsection{中断处理}
\label{subs:中断处理}

首先是初始化GDT和IDT:
{\linespread{1}\lstinputlisting[caption={初始化GDT和IDT},linerange={5-36}]{osask/src/sys/dsctbl.c}}

初始化PIC:
{\linespread{1}\lstinputlisting[caption={初始化PIC},linerange={6-25}]{osask/src/sys/int.c}}

\subsection{设备管理}
\label{sub:设备管理}

\subsubsection{键盘}
\label{subs:键盘}

{\linespread{1}\lstinputlisting[caption={PS/2键盘中断},linerange={9-16}]{osask/src/sys/keyboard.c}}
{\linespread{1}\lstinputlisting[caption={键盘初始化},linerange={36-47}]{osask/src/sys/keyboard.c}}

\subsubsection{鼠标}
\label{subs:鼠标}

{\linespread{1}\lstinputlisting[caption={PS/2鼠标中断},linerange={9-17}]{osask/src/sys/mouse.c}}

因为鼠标中断是多个字节,所以需要特殊处理:
{\linespread{1}\lstinputlisting[caption={鼠标中断处理},linerange={36-74}]{osask/src/sys/mouse.c}}

\subsubsection{屏幕}
\label{subs:屏幕}

初始化一个$6\times6\times6$的调色板:
{\linespread{1}\lstinputlisting[caption={初始化调色板},linerange={7-40,55-72}]{osask/src/sys/graphic.c}}

初始化鼠标光标:
{\linespread{1}\lstinputlisting[caption={初始化鼠标光标},linerange={140-179}]{osask/src/sys/graphic.c}}

在屏幕上显示一个半角字符:
{\linespread{1}\lstinputlisting[caption={初始化鼠标光标},linerange={94-127}]{osask/src/sys/graphic.c}}

显示字符串:
{\linespread{1}\lstinputlisting[caption={初始化鼠标光标},linerange={129-138}]{osask/src/sys/graphic.c}}

\subsubsection{窗口管理器}
\label{subs:窗口管理器}

{\linespread{1}\lstinputlisting[caption={初始化图层管理器},linerange={7-30}]{osask/src/sys/sheet.c}}
{\linespread{1}\lstinputlisting[caption={为新图层分配内存},linerange={32-46}]{osask/src/sys/sheet.c}}

重绘一个图层相对比较麻烦,需要处理透明色,以及一个小的优化:
{\linespread{1}\lstinputlisting[caption={重绘图层},linerange={57-131}]{osask/src/sys/sheet.c}}

改变图层层次:
{\linespread{1}\lstinputlisting[caption={改变图层层次},linerange={233-291}]{osask/src/sys/sheet.c}}

% {\linespread{1}\lstinputlisting[caption={},linerange={129-138}]{osask/src/sys/window.c}}


\subsection{进程管理}
\label{sub:进程管理}

\subsection{内存管理}
\label{sub:内存管理}

\subsection{文件管理}
\label{sub:文件管理}

\subsection{系统接口}
\label{sub:系统接口}

\subsection{应用程序}
\label{sub:应用程序}
