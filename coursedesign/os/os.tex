\documentclass[c5size,a4paper,nofonts,twoside]{ctexart}
\usepackage[utf8]{inputenc}
\def\tjf{{\tt{田劲锋}}}
\def\titlec{一个小型图形界面操作系统}
\usepackage[top=2.5cm,bottom=2.5cm,left=3.1cm,right=3.1cm]{geometry} % 页面设置
\usepackage[unicode,breaklinks=true,
colorlinks=true,linkcolor=black,anchorcolor=black,citecolor=black,urlcolor=black,
pdftitle={\titlec},pdfauthor={\tjf}]{hyperref}
\usepackage{tikz} % 画图
\usetikzlibrary{shapes,arrows}
\usepackage{multicol} % 分栏
\usepackage{multirow} % 跨行
\usepackage{longtable} % 长表格
\usepackage{tabularx} % 变宽表格
\usepackage{booktabs} % 表格画线
\usepackage{graphicx} % 图形
\usepackage{color} % 颜色
\usepackage{xcolor} % 颜色
\usepackage{wallpaper} % 背景图片
\usepackage{listings} % 排版代码
\usepackage{fancyvrb}
\usepackage{url} % 排版链接
\usepackage{shortvrb}
\usepackage{uml} % 画 UML
\usepackage{smartdiagram} % 智能画图
\usepackage{nameref}
\usepackage{rotating} % 横排大图
\usepackage{caption}
\captionsetup{font={small}} % 标题字体大小

\usepackage{tikz}
\usetikzlibrary{arrows,shadows} % for pgf-umlsd
\usepackage[underline=true,rounded corners=false]{pgf-umlsd}
% \usepackage{tikz-uml}

\usepackage{clrscode3e}

\setmainfont{Times New Roman}
\setmonofont{Ubuntu Mono}
\setCJKmainfont[BoldFont={SimHei},ItalicFont={STFangsong}]{SimSun}  % 主要字体:宋体、黑体
\setCJKsansfont[BoldFont={STZhongsong}]{STFangsong} % 次要字体:仿宋、中宋
\setCJKmonofont{KFKai} % 等宽字体:楷体
\setCJKfamilyfont{liti}{隶书} \newcommand{\liti}{\CJKfamily{liti}}
\setCJKfamilyfont{ja}{MS PMincho} \newcommand{\ja}{\CJKfamily{ja}}

\CJKsetecglue{\hspace{0.1em}}
\renewcommand\CJKglue{\hskip -0.3pt plus 0.08\baselineskip}
\frenchspacing
\widowpenalty=10000
\linespread{1.5} % 1.5 倍行距
\setlength{\parskip}{2pt plus 2pt}
\renewcommand{\baselinestretch}{1.5}

\usepackage[inline]{enumitem} % 调整列表样式
\setlist{noitemsep,align=right}
\setlist[itemize]{topsep=0pt,partopsep=0pt,itemsep=0pt,parsep=0pt}
\setlist[enumerate]{topsep=0pt,partopsep=0pt,itemsep=0pt,parsep=0pt}

\makeindex
\pagestyle{plain}

\begin{document}
\newcommand{\tabincell}[2]{\begin{tabular}{@{}#1@{}}#2\end{tabular}}
\newcommand{\tabincelll}[3]{\begin{tabular*}{#1}{@{}#2@{}}#3\end{tabular*}}
\renewcommand{\tabularxcolumn}[1]{m{#1}}
\newcommand{\vil}{\,\vline \hspace{.5em} }


\newcommand{\des}[2]{\makebox[2em][s]{\bf #1}\quad #2}
\newcommand{\function}[5]{\CTEXnoindent\vspace{1em}\par
\begin{minipage}{\textwidth}
    \underline{\makebox[\textwidth][l]{\it{#1}}}\\
    {\tt {{#3}} {\bfseries #1}(#2);}\\
    \des{返回}{#4}\\
    \des{描述}{#5}
\end{minipage}\par
\CTEXindent}

\newcommand\pictext{\linespread{1}\centering}


%%%% 开始 %%%%

\pagestyle{empty}
\begin{titlepage}

\begin{center}

\includegraphics[height=1cm]{image/haut.png}

\vspace*{1cm}
{\liti\fontsize{48pt}{50pt}{课\quad 程\quad 设\quad 计}}

\vspace*{4cm}
{\fontsize{36}{50}\sf\bfseries \titlec}

\vspace*{1cm}
{\huge\sf\bfseries \titlee}

\vfill
{\large
\newcommand{\ctline}[2]{\makebox[6em][s]{\bf #1}:\underline{\makebox[14em][c]{\qquad #2\qquad}}\\}
\ctline{课程设计名称}{数据结构课程设计}
\ctline{专业班级}{计算机 1303 班}
\ctline{学生姓名}{\tjf}
\ctline{学号}{201316920311}
\ctline{指导教师}{白\quad 浩}
\ctline{课程设计时间}{\today}
}

\end{center}

\end{titlepage}

\cleardoublepage
\newpage
\section*{\underline{~软件工程~}专业课程设计任务书}
% \addcontentsline{toc}{section}{课程设计任务书}
% \ThisCenterWallPaper{1}{source/task.pdf}
\CTEXnoindent
\begin{tabularx}{\textwidth}{|c|X|}\hline
{\bf 学生姓名} & \quad \tjf \quad \vil {\bf 专业班级} \vil 软件 1305 班 \vil {\bf 学\quad 号} \vil 201316920311 \\\hline
{\bf 题\qquad 目} & \quad \titlec \\\hline
{\bf 课题性质} & \makebox[11em][c]{其他} \vil {\bf 课题来源} \vil \makebox[11em][c]{自拟课题} \\\hline
{\bf 指导教师} & \makebox[11em][c]{刘扬} \vil {\bf 同组姓名} \vil \makebox[11em][c]{无} \\\hline
{\bf 主要内容} & \tabincell{l}{\begin{minipage}[c][5cm][c]{12cm}

\end{minipage}} \\\hline
{\bf 任务要求} & \tabincell{l}{\begin{minipage}[c][5cm][c]{12cm}

\end{minipage}} \\\hline
{\bf 参考文献} & \tabincell{l}{\begin{minipage}[c][5cm][c]{12cm}

\end{minipage}} \\\hline
{\bf 审查意见} & \tabincell{l}{
    指导教师签字:\\
    \vspace*{2cm}\\
    教研室主任签字:\hspace{5cm} \today\\[0.5em]
} \\\hline
\end{tabularx}

{\small 说明:本表由指导教师填写,由教研室主任审核后下达给选题学生,装订在设计(论文)首页}

\CTEXindent
\newpage

\cleardoublepage
\setcounter{page}{1}
\pagestyle{plain}
\tableofcontents
\cleardoublepage

\section{详细设计}

\newcommand\class[3]{\node[rectangle split, rectangle split parts=3, draw, text width=14cm]{
{\bf #1}
\nodepart{two}#2
\nodepart{three}#3
};}
\newcommand\method[2]{{#1}:~{\it #2}}
\newcommand\vart[2]{{#1}:~{\it #2}}
\newcommand\argu[2]{{\sf #2}:~{\it #1}}

\begin{figure}[htp]
  \centering
% \begin{tikzpicture}[outline/.style={draw=#1,thick,fill=#1!20},
% outline/.default=black]
% \node [outline] at (0,1) {default};
% \node [outline=blue] at (0,0) {blue};
% \end{tikzpicture}
\fontsize{10pt}{11pt}
\linespread{1}
\begin{tikzpicture}[every text node part/.style={align=center}]
\class{SList}{
+ \argu{SList *}{next} \\
+ \argu{const void *}{userdata}
}{
+ \method{slist\_concat(\argu{SList *}{head}, \argu{SList *}{tail})}{SList *} \\
+ \method{slist\_cons(\argu{SList *}{item}, \argu{SList *}{slist})}{SList *} \\
+ \method{slist\_delete(\argu{SList *}{slist}, \argu{回调函数}{delete\_fct})}{SList *} \\
+ \method{slist\_remove(\argu{SList **}{phead}, \argu{回调函数}{find}, \argu{void *}{matchdata})}{SList *} \\
+ \method{slist\_reverse(\argu{SList *}{slist})}{SList *} \\
- \method{slist\_sort\_merge(\argu{SList *}{left}, \argu{SList *}{right}, \argu{比较函数}{compare}, \argu{void *}{userdata})}{SList *} \\
+ \method{slist\_sort(\argu{SList *}{slist}, \argu{比较函数}{compare}, \argu{void *}{userdata})}{SList *} \\
+ \method{slist\_tail(\argu{SList *}{slist})}{SList *} \\
+ \method{slist\_nth(\argu{SList *}{slist}, \argu{size\_t}{n})}{SList *} \\
+ \method{slist\_find(\argu{SList *}{slist}, \argu{回调函数}{find}, \argu{void *}{matchdata})}{void *} \\
+ \method{slist\_length(\argu{SList *}{slist})}{size\_t} \\
+ \method{slist\_foreach(\argu{SList *}{slist}, \argu{回调函数}{find}, \argu{void *}{userdata})}{void *} \\
+ \method{slist\_box(\argu{const void *}{userdata})}{SList *} \\
+ \method{slist\_unbox(\argu{SList *}{item})}{void *}
}
\end{tikzpicture}
  \caption{SList 类}
\end{figure}

\begin{figure}
  \centering
  \begin{sequencediagram}
    \newthread{ss}{:SimulationServer}
    \newinst{ctr}{:SimControlNode}
    \newinst{ps}{:PhysicsServer}
    \newinst[1]{sense}{:SenseServer}
    
    \begin{call}{ss}{Initialize()}{sense}{}
    \end{call}
    \begin{sdblock}{Run Loop}{The main loop}
      \begin{call}{ss}{StartCycle()}{ctr}{}
        \begin{call}{ctr}{ActAgent()}{sense}{}
        \end{call}
      \end{call}
      \begin{call}{ss}{Update()}{ps}{}
        \begin{messcall}{ps}{PrePhysicsUpdate()}{sense}{state}
        \end{messcall}
        \begin{sdblock}{Physics Loop}{}
          \begin{callself}{ps}{PhysicsUpdate()}{}
          \end{callself}
        \end{sdblock}
        \begin{call}{ps}{PostPhysicsUpdate()}{sense}{}
        \end{call}
      \end{call}
      \begin{call}{ss}{EndCycle()}{ctr}{}
        \begin{call}{ctr}{SenseAgent()}{sense}{}
        \end{call}
      \end{call}
    \end{sdblock}
  \end{sequencediagram}
  \caption{UML sequence diagram demo}
\end{figure}


% \umlDiagram[box=,sizeX=15cm, sizeY=16cm,ref=ADTdiagram,
%             grayness=0.92]{}% End of diagram
%   \umlSchema[pos=\umlTopRight{ADTdiagram}, posDelta={-.5,-.5},
%     refpoint=tr]{ADT}{% Attributes
%     \umlAttribute[visibility,type=String]{name}}{}{}{}{}
%   \umlSchema[pos=\umlTopLeft{ADTdiagram}, posDelta={.5,-1},
%              refpoint=lt, abstract,
%              ref=ADTexample]{ADT-example}{%
%     \umlAttribute[visibility=-,
%       type=\emph{\umlColorsArgument\umlColorsAdjust type},default=null]{%
%       firstNode}
%     }{%Methods
%     }{%Arguments
%     \umlArgument[type=Metaclass]{type}
%     }{%Constraints
%     }{%Structure
%       \umlDiagram[box=,innerBorder=2mm,outerBorder]{%
%         \umlClass[pos={.5,.5}, ref=adtNode,box=]{Node}{%
%           \umlAttribute[visibility,
%             type=\emph{\umlColorsArgument\umlColorsAdjust type}]{%
%             data}}{}%
%         \umlAssociation[angleA=20, angleB=-20,
%                      arm=1em, arm=1em]{adtNode}{adtNode}%
%      }\cr% End of Diagram
%    }% End of ADT-example
%   \umlInstance{ADTexample}{ADT}%
%   \umlSchema[pos=\umlRight{ADTexample}, posDelta={3,-1},
%              refpoint=tl, ]{Graph}{% Attributes
%     }{% Methods
%     \umlMethod[visibility,]{%
%       insert}{\emph{\umlColorsArgument\umlColorsAdjust type} x}
%     \umlMethod[visibility,
%         type=\emph{\umlColorsArgument\umlColorsAdjust type}]{%
%       dijkstra}{\emph{\umlColorsArgument\umlColorsAdjust type} x}
%     \umlMethod[visibility, type=boolean]{%
%       insertEdge}{\emph{\umlColorsArgument\umlColorsAdjust type} x}
%     \umlMethod[visibility, ]{%
%       delete}{\emph{\umlColorsArgument\umlColorsAdjust type} x}
%     }{% Arguments
%     }{% Constraints
%     }{% Structure
%       \umlDiagram[box=,innerBorder=2mm, outerBorder,
%                   sizeX=11em,sizeY=3.5em,ref=GraphDiagram]{%
%            \begin{umlColors}{\umlColorsSub}
%         \umlClass[pos=\umlBottomLeft{GraphDiagram},
%                   posDelta={1,1}, ref=graphNode]{Node}{}{}%
%         \umlAssociation[angleA=20, angleB=-20, armA=1em, armB=1em
%                      ]{graphNode}{graphNode}%
%            \end{umlColors}
%           \umlLabelA[height=0mm,offset=1ex]{graphNodegraphNode}{*}%
%           \umlLabelB[height=0mm,offset=1ex,refpoint=t
%                      ]{graphNodegraphNode}{*}%
%         \umlSymbol[fraction=.5]{graphNodegraphNode}{\pnode{gngn}}
%         \umlClass[pos=gngn, posDelta={2,0},
%                            ref=graphEdge, refpoint=l]{Edge}{%
%         \umlAttribute[type=real]{cost}}{}%
%       \umlAssociationClass[]{graphEdge}{gngn}%
%       }\cr% End of diagram
%     }% End of Graph
%   \umlSubclass{Graph}{ADTexample}
%   \umlSchema[posX=\umlLeft{ADTexample}, posDelta={3em,-1em},
%              posY=\umlBottom{Graph},
%              refpoint=tl, ref=searchTree]{Search Tree}{% Attributes
%     }{% Methods
%     \umlMethod[visibility,]{%
%       insert}{\emph{\umlColorsArgument\umlColorsAdjust type} x}
%     \umlMethod[visibility,
%          type=\emph{\umlColorsArgument\umlColorsAdjust type}]{%
%       search}{\emph{\umlColorsArgument\umlColorsAdjust type} x}
%     \umlMethod[visibility, type=boolean]{%
%       search}{\emph{\umlColorsArgument\umlColorsAdjust type} x}
%     \umlMethod[visibility, ]{%
%       delete}{\emph{\umlColorsArgument\umlColorsAdjust type} x}
%     }{% Arguments
%     \umlArgument[type=Integer, initialValue=2]{arity}
%     \umlArgument[type={${\umlColorsArgument\umlColorsAdjust type}
%       \times {\umlColorsArgument\umlColorsAdjust type}\rightarrow$
%         boolean}, default=>]{sort}
%     }{% Constraints
%     }{% Structure
%       \umlDiagram[box=, sizeX=14em, sizeY=4em,
%                   innerBorder=2mm, outerBorder]{%
%         \begin{umlColors}{\umlColorsSub}
%         \umlClass[pos={.5, .5}, ref=treeNode]{Node}{}{}%
%         \umlAssociation[angleA=30, angleB=-30, armA=1em, armB=1em
%                      ]{treeNode}{treeNode}%
%          \end{umlColors}
%           \umlLabelA[height=1mm, offset=4mm,refpoint=l
%             ]{treeNodetreeNode}{%
%             \emph{\umlColorsArgument\umlColorsAdjust arity}}%
%           \umlLabelB[refpoint=tl,height=-1mm, offset=4mm,
%                      ]{treeNodetreeNode}{1}%
%         }\cr% End of diagram
%     }%
%   \umlSubclass{searchTree}{ADTexample}%
%   \umlSchema[pos=\umlBottomLeft{searchTree},posDelta={0,-1},
%              refpoint=lt]{List}{%Attributes
%     }{% Methods
%     }{% Arguments
%     }{% Constraints
%     }{% Structure
%       \umlDiagram[box=, sizeX=6em, sizeY=4em,
%                   innerBorder=2mm, outerBorder]{%
%         \begin{umlColors}{\umlColorsSub}
%           \umlClass[pos={.5, .5}, ref=listNode]{Node}{}{}%
%           \umlAssociation[angleA=30, angleB=-30, armA=1em, armB=1em
%                        ]{listNode}{listNode}%
%         \end{umlColors}
%           \umlLabelA[height=1mm, offset=5mm, refpoint=l
%                      ]{listNodelistNode}{1}%
%           \umlLabelB[refpoint=tl, height=-1mm,offset=5mm
%                      ]{listNodelistNode}{1}%
%         }\cr% End of diagram
%     }% End of Schema List
%   \umlPlaceNode[leftside, top, down=1em]{List}{Listtl}
%   \umlPlaceNode[leftside,right=2em,bottom]{ADTexample}{ADTexamplebl}
%   \umlSubclass[armA=1.4142em, armAngleA=135]{Listtl}{ADTexamplebl}%
%   \umlSchema[pos=\umlTopRight{List},posDelta={1em,-2em},
%              refpoint=tl]{Queue}{% Attributes
%     }{\umlMethod[visibility]{%
%       enqueue}{\emph{\umlColorsArgument\umlColorsAdjust type} x}
%     \umlMethod[visibility, type=\emph{\umlColorsArgument\umlColorsAdjust type}]{%
%       dequeue}{}}{%Arguments
%     }{%Constraints
%     \umlCompartmentline{First come, first served.}
%     }{% Structure
%     }% End of Queue
%   \umlPlaceNode[rightside, top, down=1em]{List}{Listtr}
%   \umlSubclass[angleA=90, armAngleA=135, armA=1.4142em]{Queue}{Listtr}%
%   \umlSchema[pos=\umlTopRight{Queue},posDelta={\umlNodeSep,0em},
%              refpoint=tl]{Stack}{%Attributes
%     }{% Methods
%     \umlMethod[visibility]{
%       push}{\emph{\umlColorsArgument\umlColorsAdjust type} x}
%     \umlMethod[visibility, type=\emph{\umlColorsArgument\umlColorsAdjust type}]{%
%       pop}{}
%     }{% Arguments
%     }{% Constraints
%     \umlCompartmentline{S:Stack = S.push(x).pop()}
%     }{% Structure
%     }% End of Stack
%   \umlSubclass[angleA=90, armAngleA=135, armA=1.4142em]{Stack}{Listtr}%


\section{详细设计}

整个系统有七个大类,上百个方法。由于时间关系,不能够一一说明,我这里拣几个比较主要的方法来说明其程序流程,其他代码实现还请查看附录~\ref{codes}。

\subsection{已实现功能}

\subsubsection{插入题目}

图~\ref{insert_problem}~展示了插入题目的流程。

\begin{figure}[htp]
  \pictext\small
\begin{tikzpicture}
\begin{umlseqdiag}
\umlactor[class=UI]{ui}
\umlmulti[class=List, fill=blue!20]{list}
\umlobject[class=Problem]{p}
\begin{umlcall}[op={list\_new()},return={list\_free(list)}]{ui}{list}
  \begin{umlcall}[op={problem\_read\_file(list)}]{list}{list}
  \end{umlcall}
  \begin{umlcall}[op={ui\_output\_count(list)}]{ui}{list}
  \end{umlcall}
  \begin{umlcall}[op={ui\_input\_problem()}, return={ui\_output\_problem(p)}]{ui}{p}
    \begin{umlcall}[op={p.id}, return={list.max\_id + 1}]{p}{list}
    \end{umlcall}
  \end{umlcall}
  \begin{umlcall}[op={list\_insert(list, p)}]{p}{list}
  \end{umlcall}
  \begin{umlcall}[op={problem\_write\_file(list)}]{list}{list}
  \end{umlcall}
\end{umlcall}
\end{umlseqdiag}
\end{tikzpicture}
  \caption{\label{insert_problem}插入题目的流程}
\end{figure}

\subsubsection{修改题目}
\subsubsection{模糊查询}
\subsubsection{生成试卷}
\subsubsection{用户登录}
\subsubsection{学生考试}


\subsection{未实现功能}

系统完成的时候,老师提出了加入多选题功能的要求。由于时间关系,不再具体实现,大概叙述一下实现的思路。























\section{总结}

这个项目我是从2015年4月20日开始,基本代码完成于5月19日。最后到小学期,完成的这个文档。
整个过程是很有意思的,也成为了我在GitHub上提交最频繁的一段记录。
当然,还是一直在参考书\cite{osask}中的内容再做。
书的内容是中文的,但是给出的随书源代码是日文注释,不过正好会日语,所以对源代码的理解和重写也就轻松了许多。
许多函数名和变量名,实际上也都是日语的一些东西,看懂了是挺有意思的,看得出作者的用心。
比如Haribote这个词就是{\ja 「張りぼて」},意思是“纸糊的戏剧用小道具”。
也就是说,这个操作系统,就像纸糊的一样,只是个玩具,看起来还不错,其实是空的。
所以,我们也可以看到,这个操作系统,真的就像玩具一样,可以玩玩,但并不能实用。
比如它不能真正地读写文件,没有虚拟内存,不能连接网络等。

在现代操作系统中,进程、存储、I/O、文件、网络,显然是必要的五个元素。
这里呢,实际上只是对进程的实现,是相对比较丰富的。
从代码中我们可以看到,这个操作系统,是分时多任务的。
也就是说,我们为每个进程分配一个时间片,然后切换上下文来运行在队列中的进程。
这里我们也加了一个优先级,来调度不同等级的任务。
查找算法,也都是非常简单的线性算法。
因为任务很少,也没有明显的性能问题。

对于存储器的管理,实际上更简单了。
我们维护一个空闲内存块的链表,然后分配和释放都在这个链表上进行。
算法也是线性的。

I/O存取,其实这里就是对系统中断的处理了。
主要是鼠标和键盘中断,以及屏幕的绘制。

屏幕绘制,专门写了一个窗口和图层管理器。
这个算是代码量最大的一部分了,所以看起来这个操作系统代码很多,大部分都不是核心。
写屏幕很有意思,因为能够看到花花绿绿的结果,很开心,有成就感。
但实际上这些东西并不是操作系统的核心。

当然完成了这些东西,提供了一些API供应用程序使用,精力也就比较有限了。
所以文件系统没有实现,只是简单地在汇编阶段把软盘(而且只支持FAT12格式的软盘)内容全部读到内存中,再去操作内存。
同样,网络也没有实现,下学期学《计算机网络》的时候,倒不妨写一个TCP/IP玩玩。

在此期间我又读了读MINIX\cite{minix3}的源代码,这是一个分模块的操作系统实现,也从中学到了很多东西。
不过还没有去读Linux的内核源代码,这是计划了。

其实这个学期对我影响最深的是CS:APP——《深入理解计算机系统》\cite{csappv2}。
花了两个月的时间读完了。
这本书从头到尾把计算机硬件到软件讲了一遍,是很好的计算机入门书。
而且也正式从这本书中学到了汇编,对C的理解更深了一些。
不过学习的教学环境还没有这种课程,我对于这本书的理解也只是浅尝辄止,后面的作业其实没有做多少。
参考文献列在最后,\LaTeX 用来排版以及{\sc Bib}\TeX 用来列参考文献,倒是挺方便的。

操作系统原理性的东西,其实在上个世纪80年代已经全部搞明白了。
作为一个科班出身的程序员,能写OS也不是什么稀奇事了。
但是,玩具好做,真正能够把操作系统上的API设计得好,能够构建出一系列应用程序,
是一项非常浩大的工程。从内核构建到虚拟机,再到用户界面和应用程序,
只能说只有微软完成了这个完整地流程。
就算是Mac OS X和iOS,也是基于UNIX实现改过来的,说白了不过加个壳。
而Linux至今在桌面领域一团糟,衍生而来的Android虽然在移动领域二分天下,
不过也不是Google独立开发出来的,还面临着Java的版权问题。
所以这么一个庞大的生态系统,可以说难于上青天。当然现在上天倒是很随意的事情了。

话说回来,这个操作系统,还只能是玩具。
作为课程设计,这个可能有点太大了,但我觉得这一轮下来也是值得的。
这回课程设计,能够借机完成一个操作系统,也是我一直以来的愿望。
不过其实对这个玩具并不满意,除了可能GUI设计的扁平化了一些,看起来更像现代操作系统以外,
真是没有一点可爱的地方。本来还想实现汉字显示,结果折腾了半个月没弄好,
还好一直用git做版本控制,就回滚到能用的版本提交了这个版本。
如果有时间,我倒是想写一个UNIX兼容(POSIX)的内核,而不是现在这个自有API的东西。
当然这是后话了。

更多的探索还在后面,这个课程设计的完成也只是更大目标的开始。
要学的东西还有很多,所以就去更多地学习和实践了!

\vfill

\centering
\includegraphics[width=12cm]{image/laala.png}

\vspace*{1cm}


\cleardoublepage
\appendix
\addcontentsline{toc}{section}{参考文献}

\nocite{*}

\bibliographystyle{chinesebst}

\bibliography{reference}

\cleardoublepage
\section{代码清单}

\label{codes}

\lstset{language=C,
  numbers=left,
  numberstyle=\tiny,
  basicstyle=\tiny\tt,
  commentstyle=\color{gray},
  keywordstyle=\bfseries\color{violet},
  stringstyle=\color{teal},
  showstringspaces=false,
  frame=trBL,
  %     % identifierstyle=\sf,
%     breaklines=true,
  % escapeinside=` `,
%     % texcl=true,  % 为true时,注释内容被当做LaTeX代码来处理
  morekeywords=[1]{node,code,symfreq,symcode},
}

\linespread{1}

\lstinputlisting[language=bash]{exam/Makefile}

\lstinputlisting{exam/exam.c}

\lstinputlisting{exam/ui.h}
\lstinputlisting{exam/ui.c}

\lstinputlisting{exam/list.h}
\lstinputlisting{exam/list.c}

\lstinputlisting{exam/slist.h}
\lstinputlisting{exam/slist.c}

\lstinputlisting{exam/problem.h}
\lstinputlisting{exam/problem.c}

\lstinputlisting{exam/paper.h}
\lstinputlisting{exam/paper.c}

\lstinputlisting{exam/score.h}
\lstinputlisting{exam/score.c}

\lstinputlisting{exam/user.h}
\lstinputlisting{exam/user.c}

\lstinputlisting{exam/addon.h}
\lstinputlisting{exam/addon.c}

\linespread{1.5}


%%%% 结束 %%%%

\end{document}
