
\section{详细设计}

\subsection{位操作}

\begin{figure}[htp]
\centering
\umlDiagram[sizeX=4.4cm, sizeY=3.4cm]{
  \umlClass{code}
    {\umlAttribute[type=\texttt{unsigned long}]{\emph{len}}}
    {\umlAttribute[type=\texttt{unsigned char *}]{\emph{bits}}}
}
\caption{\label{code}码字数据类型}
\end{figure}

如图 \ref{code},定义数据类型 \verb|code| 用来存储位数据,其 \verb|len| 表示编码位长, \verb|bits| 是指向存储编码值得指针,一个 \verb|bits[]| 表示八位编码。

\function{bit\_len\_byte}
{unsigned long len}
{unsigned long}{字节长度}
{把位的长度转换成字节的长度,不足 8 位则进位。}

\function{get\_bit}
{unsigned char *bits, unsigned long i}
{unsigned char}{0 或 1}
{获取 $bits$ 的第 $i$ 位二进制位。}

\function{reversc\_bits}
{unsigned char *bits, unsigned long len}
{void}{无}
{把长度为 $len$ 的 $bits$ 依二进制位反转。}

\function{new\_code}
{const node *leaf}
{code *}{指向生成的编码结构体的指针}
{从叶结点 $leaf$ 沿 Huffman 树回溯到根结点,逆序生成该叶结点对应符号的 Huffman 编码序列。}

\function{free\_code}
{code *p}
{void}{无}
{销毁编码 $p$ 。}

\subsection{符号表}

定义符号表类型:
\begin{itemize}[topsep=0pt,partopsep=0pt,itemsep=0pt,parsep=0pt]
\item 符号频率表 \verb|symfreq[]| 是一个长为 256 的结点数组类型,主要用来存储符号出现频率以构建 Huffman 树;
\item 符号编码表 \verb|symcode[]| 是一个长为 256 的编码数组类型,用来存储符号对应的编码。
\end{itemize}

\function{init\_freq}
{symfreq *sf}
{void}{无}
{初始化符号频率表 $sf$。}

\function{get\_symfreq}
{symfreq *sf, FILE *in}
{unsigned int}{文件 {\tt in} 中的符号总数}
{读入文件 {\tt in} 中所有符号,统计符号总数,并把每个符号出现次数保存到符号频率表 $sf$ 中。}

\function{write\_code\_table}
{FILE *out, symcode *sc, uint32\_t syms}
{int}{成功返回 0}
{向文件 {\tt out} 中写出符号编码表 $sc$。写出的具体格式如下:
\begin{enumerate}[topsep=0pt,partopsep=0pt,itemsep=0pt,parsep=0pt]
  \item (4 字节) 编码大小 $n$,即 $sc$ 的大小,以网络字节顺序写出
  \item (4 字节) 待编码字节数 $syms$,即文件大小,以网络字节顺序写出
  \item 符号的具体编码 $code[1..n]$ ,每个 $code[i]$ 的编码格式为:
  \begin{enumerate}[topsep=0pt,partopsep=0pt,itemsep=0pt,parsep=0pt]
    \item (1 字节) 符号,即原字符本身
    \item (1 字节) 位长,即下面的编码位长度,单位是位
    \item 具体编码位数据。如果编码不是 8 的倍数的话,最后一字节会有用以补 0 的多余位
  \end{enumerate}
\end{enumerate}
}

\function{read\_code\_table}
{FILE *in, unsigned int *pdb}
{node *}{从文件中构建的 Huffman 树}
{从文件 {\tt in} 中读入符号编码表。以网络字节顺序,首先读入符号总数 $count$,然后读入待解码的符号总数 $pdb$。接着依据规则读入符号编码表,依此构建 Huffman 树,并把文件偏移设置到符号编码表末尾。}

\function{free\_encoder}
{symcode *sc}
{void}{无}
{销毁符号编码表 $sc$。}

\subsection{Huffman 树}

\begin{figure}[htp]
\centering
\umlDiagram[sizeX=4.8cm, sizeY=6.4cm]{
  \umlClass{node}{\umlAttribute[type=\texttt{bool}]{\emph{leaf}}
    \umlAttribute[type=\texttt{unsigned long}]{\emph{count}}
    \umlAttribute[type=\texttt{unsigned char *}]{\emph{bits}}
  }{\umlAttribute[visibility=-,type=\texttt{node *}]{\emph{zero}}
    \umlAttribute[visibility=-,type=\texttt{node *}]{\emph{one}}
    \umlAttribute[visibility=+,type=\texttt{unsigned char}]{\emph{symbol}}
  }
}
\caption{\label{node}结点数据类型}
\end{figure}

如图 \ref{node},定义数据类型 \verb|node| 用来存储 Huffman 树的结点。其 \verb|leaf| 标识是否为叶结点,如果是叶结点,联合体用 \verb|symbol| 存储该结点表示的符号;如果不是,则使用含有 \verb|zero| 和 \verb|one| 指针域的结构体来存储左右子树的地址。
 \verb|count| 存储该结点的频率(频数),指针 \verb|parent| 指向该结点的父亲。

\function{new\_leaf\_node}
{unsigned char symbol}
{node *}{生成的结点地址}
{生成一个叶结点,其保存了符号 $symbol$。}

\function{new\_node}
{unsigned long count, node *zero, node *one}
{node *}{生成的结点地址}
{生成一个普通结点,其保存了频数 $count$,左右孩子分别是 $zero$ 和 $one$。}

\function{sfcmp}
{const void *p1, const void *p2}
{int}{正数表示 $p1 > p2$,零表示相等,负数表示 $p2<p1$}
{对符号频率表进行排序的比较函数,用于 {\tt qsort()}。规则是低频率在前,高频率在后,频率为 0 即频率为空的在末尾。}

\function{calc\_code}
{symfreq *sf}
{symcode *}{已编码的数组指针}
{从符号频率表 $sf$ 建立 Huffman 树的主算法。首先获取符号数 $n$,然后对频率进行增序排序。接着构建 Huffman 树,每次取出并删除 $sf$ 最前面两个元素,合并成新的结点并加入 $sf$,调用 {\tt qsort()} 进行增序排序以维护有序性,如此进行 $n-1$ 次即可构建成功。最后调用 {\tt build\_symcode()} 生成符号编码表并返回。}

\function{build\_symcode}
{node *root, symcode *sc}
{void}{无}
{以 $root$ 为根遍历以构建好的 Huffman 树,建立符号编码表 $sc$。}

\function{free\_tree}
{node *root}
{void}{无}
{从根结点 $root$ 开始递归销毁这棵 Huffman 树。}
