\section{调试分析}

起初程序解压结果一直和原文不一致,经检查发现是在内存中和硬盘中字节存储顺序不一致,有必要采取统一编码方式。使用网络库是为了使用函数 \verb|ntohl()| 和 \verb|htonl()|,在网络字节顺序和本地字节顺序之间进行转换。之所以这样,是因为本地操作系统的二进制存储结构可能不尽相同(从大到小或从小到大),为了保证读写二进制文件时不发生错误,需要执行相应的转换保证字节编码方式一致。这两个函数在 Linux 和 Windows 下分别定义在 \verb|<netinet/in.h>| 和 \verb|<winsock2.h>| 中,需要编译时指明调用的库。

还有一个问题在于,如果对一个非本程序压缩的文件进行解压的话,将会产生错误。可以采用标识文件头来解决,这里省略。

Huffman 编码 \cite{huffman} 是最优的字符压缩编码,但不是最优的压缩编码。如果在一个文件中,各种符号出现的频率接近一致,那么 Huffman 编码并不能有效地减少文件体积,反而由于引入了编码表可能会导致文件变大,这点在非文本文件压缩中体现得尤为明显。甚至把一个文件进行多重压缩之后,文件体积会越来越大。当然还原文件还是可以正常工作的,不过这显然违背了压缩的原则——减小文件体积。所以在日常应用中,我们不使用 Huffman 编码。相应地,会采用 LZ77\cite{lz77}, LZ78\cite{lz78}, LZW\cite{lzw}, ZIP, RAR, LZMA 等算法,这些都是我们常见的压缩格式,实践证明这些也是有效的压缩算法。
