从选题目,确定题目难度1.2,又换题目,最终做这道1.1难度的题目,一直到昨天的运行成功,今天的调试完成。这次课程设计终于是告了一段落了。这4天,因为下周一的考试和这周五的考核,自己每天6点50就起床,7点10分到自习室一直学到临近中午,12点半准时把电脑搬到田劲锋宿舍,修修改改,查找翻阅,饭也顾不上吃的一直敲到晚上11点,虽然在大神或者老师看来仅仅几百行的代码而已,但是这些却可能是我大二上学期最有意义的4天了。

首先,万分感谢田劲锋同学愿意和我一起合作做这个题目,我知道自己编写代码的水平不算是大神,课题出来以后我就找到田劲锋,让他写接口,而我当“码农”,负责实现,算是对自己的一种磨练吧。和他合作了以后才知道水平的差距有多大,他所定义的函数,返回类型,文件的操作,vector数组的使用,都是我们没有学过的,简直有种“代码代沟”一样,大神也只是简单告诉我他的意思,具体实现要我自己网上或者搜索,这样子过了30个小时,从一开始的茫然,不知道这些是函数是什么,然后搜索用法,问大神诸如此类操作怎么实现等等,大神和度娘不厌其烦的给了我很多帮助。我难以想象如果没有大神,那么编译环境的兼容,vs2013的编译器,sublime这样精美的代码编辑器,添加环境变量等等这些我什么时候才会考虑到,什么时候才会去使用它,写到这里真是不得不再次感谢劲锋兄,简直让我有了一种豁然开朗的感觉。

最后一天的代码调试真是让我绞尽脑汁的苦差事,遇到一个bug可能几个小时都找不到哪里出了错误,或者就是有的代码你明明知道是这里错了,但是就是不知道解决方案,就像大神说的那句话,代码一天可能就写完了,但是你调试可能都要调两天。果然还是这样,出来混,总是要还的。终于明白了数据结构和算法这种科目的重要性但也只好临阵磨枪来弥补。

这次实验让我收获颇丰,我不仅感谢自己对自己真诚和认真,更感谢田劲锋同学耐心的解答。努力就会有收获,期待下学期的课程设计我能做的更好,

最后,感谢这次帮我的所有人,包括测试的王增辉同学和愿意换位子的飞飞同学。
