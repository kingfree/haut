想了想,我觉得我的总结还是放到最后比较合适。

分组在有了题目的时候,就可以决定了,和邢志鹏的合作也是比较合适的选择。我就考虑我来写设计文档和头文件,然后把具体实现交给邢志鹏。一方面也是体现团队合作的精神,一方面也是自己懒不想写太多代码。这个课程设计的题目,实际上在我看来难度都不算很大,所以一开始我就选了难度系数为 1.2 的股票交易系统。周日我开始着手写头文件,但是写完以后发现光头文件就达到了 400 行之多,想到去年的 C 语言课程设计我一个人写了 3000 行,这次岂不是要写 4000 行?一个人写确实可以写完,不过如果真的是团队合作的话,必然要照顾到组员的程度,减少工作量。所以讨论了一下,换成了难度系数稍低为 1.1 的猜数字游戏,这次的头文件写了不到 100 行,考虑到具体实现,应该不会超过 500 行,是可以完成的任务。至于这个股票交易系统的头文件,我放在了附录~\ref{stocksec}~中,可以参考。

实际上由于种种原因呢,这个学期的 C++ 课基本没去几节。虽然在第一节实验课上跟老师说明了情况,不过还是被忘了这个事……而且我觉得这么长的总结也会“太长不看”了,所以就当自己的一个阶段性总结写点吧。

本来我可以说是“精通” C,但是并不能说熟悉 C++。所以要做课程设计之前,我把 C++ 相关的书籍都找了一遍,因为图书馆没有 \cite{cppp} 和 \cite{cpppp},也没有原著的 \cite{cpplang},所以找到了两本大部头 \cite{proc_c} 和 \cite{vc_c},花了两天把 \cite{proc_c} 看完了一半,算是掌握了 C++ 的基本内容。所谓 C++ 的基本内容是什么呢,除了 C 语言的那些东西,还有引用、const、static、类、模板、字符串、容器和迭代器、异常这些东西。顺便学习了一些 C++ 11 的特性,虽然并没有在程序中用到。关于 C++ 的一些高级特性,如 STL 的扩展、new/delete 的重载、模板和泛型、多线程这些内容我还没有涉及到。至于设计模式,这个程序里只用到了一个简单的“单例模式”,也不能说是掌握了设计模式。至于面向对象的思想和编程方法,早在去年的 C 语言程序实践中就已经用 C 语言写出了面向对象的程序,所以转到 C++ 上也只是关键词的不同而已。

实际上这次的课程设计过程比较类似于“极限编程”的方法,当然强度并没有那么高。头文件设计好之后,就是撰写设计文档。文档还是一如既往地用~\LaTeXe~排版完成,\LaTeX~的好处主要在于可以比较方便地控制交叉引用、参考文献,绘图和排版也比较精确。经久不衰的软件总是值得信任的,比如我主要参考 \cite{latexdeng} 这本 2001 年的书来查阅命令,虽然十几年过去了,书中的叙述还是依然可用。文档和 UML 类图的绘制主要参考的是 \cite{apppc_c},虽然还没看完全书。基本上通过各种参考和查阅,完成了前期的准备工作。本来我还想编写一些单元测试,但是苦于找不到合适的单元测试库,CppUnit 还不是很熟悉,就放弃了这个想法。

同时邢志鹏的实现工作也在周一开始了,写完所以功能应该是用了两天来着,然后调试又调了两天。这其实也是个锻炼的过程,通过这个过程想来邢志鹏也是学到了不少东西了吧。编程还是主要靠自学,课上讲的东西,真的是不够。

围观了几天的编程,周四我忍不住自己上手花了两个小时写了一个完整的实现——速度快的原因其实是因为写了大量描述文档,以至于对这个程序太熟悉了。其实这个实现后来也被拿去做参考了,其实是有点担心不能按时完成课程设计的。

面向对象的编程方法是接在面向过程之后的,实际上面向对象之后还有个所谓面向数据的编程方法。但是呢,方法并不是唯一的,只要能解决问题,什么办法都是好的。
%这里我必须黑一下 C++,难用到死,编译慢,出错信息完全看不懂,模板类一团糟,为什么这垃圾语言还没死?好吧,活着总是有活着的原因的,主要是 C 语言太厉害了,C++ 沾着光也是屌得不行。把面向对象发挥得最好的语言其实是 Ruby,万物皆对象。当然非常火的 Java 也是非常不错的 OO 语言,同性的 C\# 也算不错。不过,OO 牺牲的是性能,这一点没有异议,虽然在现在计算机资源非常廉价的情况下这都不是问题了。
好吧,还是扯淡得有点远了。总之课程设计就这么完成了。不管是什么过程,学到东西就是好的。

