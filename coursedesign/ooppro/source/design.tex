\section{总体方案设计}

\subsection{总体功能框图}

考虑猜数功能的实现,为了实现最大程度的解耦,将与用户交互的操作全部封装在 UI 类中。

考虑对于全局唯一的用户,需要一个全局唯一的计分器,这里设计为单例模式的 Score 类。

考虑到每次运行程序用户会进行多次猜数,对于每次猜数都实例化一个 Number 对象,并提供对猜数的检查功能。

图~\ref{classes}~展示了各个类之间的关系。

\begin{figure}[htp]
  \pictext\small
\begin{tikzpicture}
\umlemptyclass{UI}
\umlemptyclass[x=4, y=3]{Score}
\umlemptyclass[x=-4, y=3]{Number}
\umlunicompo[geometry=--,mult=0..*]{Score}{Number}
\umluniassoc[geometry=-|,mult=1]{UI}{Score}
% \umlemptyclass[x=-2, y=-2]{TestCase}
% \umldep[geometry=-|]{TestCase}{Score}
% \umldep[geometry=-|]{TestCase}{Number}
\end{tikzpicture}
  \caption{\label{classes}猜数字游戏各个类之间的关系}
\end{figure}
