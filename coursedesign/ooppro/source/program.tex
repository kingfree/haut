\section{程序清单及注释}

\label{codes}

\lstset{language=C++,
  numbers=left,
  numberstyle=\tiny,
  basicstyle=\tiny\tt,
  commentstyle=\color{gray},
  keywordstyle=\bfseries\color{violet},
  stringstyle=\color{red!80!black},
  showstringspaces=false,
  frame=trBL,
  morekeywords=[1]{cout,cin,cerr,std,stdin,printf,scanf,perror},
  keywordstyle=[1]{\color{blue}},
  morekeywords=[2]{Score,Number,UI,string,vector,ifstream,ofstream,pair},
  keywordstyle=[2]{\color{teal}},
}

%\begin{quote}
{
\small\linespread{1}
\verbatiminput{number/LICENSE}
}
%\end{quote}

\subsection{主程序}
{\linespread{1}\lstinputlisting[caption={\tt cli.cpp}]{number/cli.cpp}}

\subsection{工程文件}
{\linespread{1}\lstinputlisting[language=bash,caption={\tt Makefile}]{number/Makefile}}
{\linespread{1}\lstinputlisting[language=bash,caption={\tt Makefile.vs}]{number/Makefile.vs}}

\subsection{头文件}
{\linespread{1}\lstinputlisting[caption={\tt Number.h}]{number/Number.h}}
{\linespread{1}\lstinputlisting[caption={\tt Score.h}]{number/Score.h}}
{\linespread{1}\lstinputlisting[caption={\tt UI.h}]{number/UI.h}}

\subsection{\xzp 的实现}
{\linespread{1}\lstinputlisting[caption={\tt Number.cpp}]{number/Number.cpp}}
{\linespread{1}\lstinputlisting[caption={\tt Score.cpp}]{number/Score.cpp}}
{\linespread{1}\lstinputlisting[caption={\tt UI.cpp}]{number/UI.cpp}}

\subsection{\tjf 的实现}
{\linespread{1}\lstinputlisting[caption={\tt Number\_tjf.cpp}]{number/Number_tjf.cpp}}
{\linespread{1}\lstinputlisting[caption={\tt Score\_tjf.cpp}]{number/Score_tjf.cpp}}
{\linespread{1}\lstinputlisting[caption={\tt UI\_tjf.cpp}]{number/UI_tjf.cpp}}

\subsection{辅助文件}
这个是用 C 语言一些跨平台控制台控制的代码,提供在这里供调用。
{\linespread{1}\lstinputlisting[caption={\tt mylib.h}]{number/mylib.h}}
{\linespread{1}\lstinputlisting[caption={\tt mylib.cpp}]{number/mylib.cpp}}
