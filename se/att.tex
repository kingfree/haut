\documentclass[cs4size,a4paper,nofonts]{ctexart}
\usepackage[utf8]{inputenc}
\def\tjf{{\tt{田劲锋}}}
\def\titlec{考勤管理系统需求规格说明书}
\def\version{(一)}
\usepackage[b5paper,margin=2cm]{geometry} % 页面设置
\usepackage[unicode,breaklinks=true,
colorlinks=true,linkcolor=black,anchorcolor=black,citecolor=black,urlcolor=black,
pdftitle={\titlec \version},pdfauthor={\tjf}]{hyperref}
%\CTEXsetup[number=\chinese{section}, format={\large\sf\bfseries}]{section}
\usepackage{latexsym,amsmath,amssymb,bm}

\setmainfont{Times New Roman}
\setCJKmainfont[BoldFont={SimHei}]{SimSun}  % 主要字体:宋体、黑体
\setCJKsansfont[BoldFont={STZhongsong}]{STFangsong} % 次要字体:仿宋、中宋
\setCJKmonofont{KFKai} % 等宽字体:楷体

\CJKsetecglue{\hspace{0.1em}}
\renewcommand\CJKglue{\hskip -0.3pt plus 0.08\baselineskip}
\frenchspacing
\widowpenalty=10000
\linespread{1.2} % 行距

\usepackage[inline]{enumitem} % 调整列表样式
\setlist{noitemsep}
\setlist[enumerate]{topsep=0pt,partopsep=0pt,itemsep=0pt,parsep=0pt}
\setlist[enumerate,2]{label={(\arabic*)}}

%\makeindex
\pagestyle{plain}

\begin{document}

%%%% 开始 %%%%

\title{\titlec~\version}
\author{软件工程1305班~\tjf~201316920311}
\maketitle

\section{业务需求}

\subsection{背景}

大学课堂上,教师需要有一套有效的考勤办法来记录学生的出勤情况。传统的人工点名办法存在一些缺点。

\subsection{业务机遇}

在软件工程课堂上,王珂老师提出利用一套软硬件设施进行考勤的需求,希望能够利用这一系统做到有效的考勤。

\subsection{业务目标}

系统初期要做到基本的考勤功能,随着扩展性的增强发展出一套可以商用的有效考勤系统。

\subsection{提供给客户的价值}

产品可以实现有效的考勤功能,并能够自动生成考勤结果,为任课教师减轻考勤负担和统计负担,减少课堂时间的浪费,使教育教学能够更加有效地进行。

\subsection{业务风险}

实验教学项目,风险极小,在理论上不会因为项目拖延或失败导致非常严重的问题。

\section{用户需求}

\subsection{产品用户分析}

产品的用户为大学的任课教师,也可以用于公司和其他情况的考勤。

\subsection{用户一览表}

教师:控制和使用该产品,管理和设置产品的运行。

学生:使用该产品系统,经受系统的检测,其行为记录在系统中。

\subsection{用户环境}

拥有前后门的大学教室。

\subsection{用户需求}

教师需要通过系统对学生来到教室的时间进行记录,并判断其是否旷课、迟到或早退。

学生要求尽可能简单地通过该系统,不必做过多的验证过程。

\section{系统需求}

\subsection{功能需求}

\begin{enumerate}
\item 检测学生是否按时到班,是否迟到。
\item 检测学生在上课时间是否在教室听课,是否早退。
\item 记录学生数据和到课情况,提供接口生成考勤报表。
\item 提供接口用以添加、删除和修改学生数据。
\item 要求有一定的容错性,能够排除非当课学生的情况。
\item 其他暂未考虑到的功能需求。
\end{enumerate}

\subsection{非功能需求}

\begin{enumerate}
\item 硬件系统要求便携,方便教师携带。
\item 要求系统安装和设置便捷,方便随时开启和关闭。
\item 检测过程不对学生和教师产生干扰,或干扰很小且无害。
\item 持续检测的电池续航能力和充电能力。
\item 系统界面的友好性和易操作性。
\item 可以通过移动设备访问。
\item 其他暂未考虑到的非功能需求。
\end{enumerate}

\begin{thebibliography}{10}
\bibitem{sre} 需求工程:实践者之路. Christof Ebert. 机械工业出版社, 2013.
\bibitem{se} 软件工程:实践者的研究方法. Roger S. Pressman. 机械工业出版社, 2011.
\end{thebibliography}

\end{document}
