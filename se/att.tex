\documentclass[cs4size,a4paper,nofonts]{ctexart}
\usepackage[utf8]{inputenc}
\def\tjf{{\tt{田劲锋}}}
\def\titlec{课堂考勤管理系统需求规格说明书}
\def\version{(二)}
\usepackage[b5paper,margin=2cm]{geometry} % 页面设置
\usepackage[unicode,breaklinks=true,
colorlinks=true,linkcolor=black,anchorcolor=black,citecolor=black,urlcolor=black,
pdftitle={\titlec \version},pdfauthor={\tjf}]{hyperref}
%\CTEXsetup[number=\chinese{section}, format={\large\sf\bfseries}]{section}
\usepackage{latexsym,amsmath,amssymb,bm}

\setmainfont{Times New Roman}
\setCJKmainfont[BoldFont={SimHei}]{SimSun}  % 主要字体:宋体、黑体
\setCJKsansfont[BoldFont={STZhongsong}]{STFangsong} % 次要字体:仿宋、中宋
\setCJKmonofont{KFKai} % 等宽字体:楷体

\CJKsetecglue{\hspace{0.1em}}
\renewcommand\CJKglue{\hskip -0.3pt plus 0.08\baselineskip}
\frenchspacing
\widowpenalty=10000
\linespread{1.2} % 行距

\usepackage[inline]{enumitem} % 调整列表样式
\setlist{noitemsep}
\setlist[enumerate]{topsep=0pt,partopsep=0pt,itemsep=0pt,parsep=0pt,label={\arabic{section}.\arabic{subsection}.\arabic*}}
\setlist[enumerate,2]{label*={.\arabic*}}

%\makeindex
\pagestyle{plain}

\begin{document}

%%%% 开始 %%%%

\title{\bf\titlec~\version}
\author{软件工程1305班~\quad\tjf\quad~201316920311}
\maketitle

\section{业务需求}

\subsection{背景}

大学课堂上,教师需要有一套有效的考勤办法来记录学生的出勤情况。传统的人工点名办法存在一些缺点,比如浪费课堂时间、不能准确识别学生。固定座位的办法也会损害学生自由选择座位的权利。

\subsection{业务机遇}

在软件工程课堂上,王珂老师提出利用一套软硬件设施进行考勤的需求,希望能够利用这一系统做到有效的考勤,以节省教师体力和课堂时间,克服传统方式的缺点。

\subsection{业务目标}

利用硬件结合软件管理日常出勤、迟到、旷课、早退,及时生成各种按班级或学生的汇总统计数据。

\subsection{提供给客户的价值}

产品可以实现有效的考勤功能,并能够自动生成考勤结果,为任课教师减轻考勤负担和统计负担,减少课堂时间的浪费,使教育教学能够更加有效地进行。

\subsection{业务风险}

实验教学项目,风险极小,在理论上不会因为项目拖延或失败导致非常严重的问题。另一方面,业务的时间限制比较宽泛,可以做到多次迭代,并可以进行大量测试保证产品的稳定性。

\section{用户需求}

\subsection{产品用户分析}

产品的用户为大学的任课教师,教师可以利用这套系统来进行有效考勤来减轻工作量。

学生使用这套系统可以规范行为,养成按时到课的习惯。

产品也可以用于公司和其他情况的考勤。

\subsection{用户一览表}

教师:控制和使用该产品,管理和设置产品的运行。系统提供用户界面给教师控制,教师拥有该科目的最高权限。

学生:使用该产品系统,经受系统的检测,其行为记录在系统中。学生只有在教师授权的情况下才能够使用该系统,也必须强制接受该系统的流程和限制。

\subsection{用户环境}

学生是在拥有前后门的大学教室上课的,拥有两个进出口需要管理,因此会需要两套同步的设备。

\subsection{用户需求}

教师导入或输入学生信息,通过系统对学生来到教室的时间进行记录,并判断其是否旷课、迟到或早退,能够生成考勤成绩报表。

学生将自己的识别信息输入到系统,在课堂上希望尽可能简单地通过该系统,不必做过多的验证过程,在持续验证过程中不受到过多干扰。

\section{系统需求}

\subsection{功能需求}

\begin{enumerate}
\item 学生和班级数据导入。
\begin{enumerate}
\item 从教务系统或电子表格中导入学生的学号、姓名等信息。
\item 从教务系统或电子表格中导入班级的信息。
\item 提供用户界面以控制导入过程和导入结果的正确性。
\end{enumerate}
\item 学生识别数据采集和整理。
\begin{enumerate}
\item 采集用以识别学生的数据。
\item 以摄像头采集虹膜、以指纹读取设备采集指纹等。
\item 采集学生的手机网卡MAC地址和SIM卡识别标签。
\item 采集成功后提示。
\item 保证识别数据的唯一性和易识别性。
\item 提供用户界面接口用以添加、删除和修改学生数据。
\end{enumerate}
\item 课程班数据导入和整理。
\begin{enumerate}
\item 从教务系统或电子表格中导入课程班的上课时间、教室等信息。
\item 提供用户界面以调整课程。
\end{enumerate}
\item 时钟功能。
\begin{enumerate}
\item 系统硬件时钟能够实现精确到秒的计时。
\item 通过时钟计时和课程安排来确定要考勤和正在考勤的课程。
\end{enumerate}
\item 检测学生是否按时到班,是否迟到、早退。
\begin{enumerate}
\item 通过进入教室时的识别数据验证来检测迟到。
\item 通过离开教室时的识别数据验证来检测早退。
\end{enumerate}
\item 检测学生在上课时间是否在教室听课。
\begin{enumerate}
\item 通过手机信号的随时检测来判断学生是否中途离开教室。
\item 识别学生的离开是正常下课短时间去卫生间,还是逃课等情况。
\end{enumerate}
\item 记录学生数据和到课情况,提供接口生成考勤报表。
\begin{enumerate}
\item 系统记录每个变化的状态,包括学生签到、离开等行为。
\item 提供用户界面查看这些记录。
\item 提供接口生成考勤报表和平时成绩单。
\end{enumerate}
\item 要求有一定的容错性,能够排除非当课学生的情况。
\begin{enumerate}
\item 对于不是该课程内人员的进出,只做记录不做识别。
\end{enumerate}
\item 设备同步和通信功能。
\item 其他暂未考虑到的功能需求。
\end{enumerate}

\subsection{非功能需求}

\begin{enumerate}
\item 硬件系统要求便携,方便教师携带。
\item 要求系统安装和设置便捷,方便随时开启和关闭。
\item 检测过程不对学生和教师产生干扰,或干扰很小且无害。
\item 持续检测的电池续航能力和充电能力。
\item 系统界面的友好性和易操作性。
\item 可以通过移动设备访问。
\item 其他暂未考虑到的非功能需求。
\end{enumerate}

\begin{thebibliography}{10}
\bibitem{sre} 需求工程:实践者之路. Christof Ebert. 机械工业出版社, 2013.
\bibitem{se} 软件工程:实践者的研究方法. Roger S. Pressman. 机械工业出版社, 2011.
\end{thebibliography}

\end{document}

% 一个整合微博、QQ、贴吧、论坛等信息流的手机客户端 田劲锋