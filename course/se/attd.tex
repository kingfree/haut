\documentclass[cs4size,a4paper,nofonts]{ctexart}
\usepackage[utf8]{inputenc}
\def\tjf{{\tt{田劲锋}}}
\def\titlec{课堂考勤管理系统}
\def\version{接口设计文档}
\usepackage[a4paper,margin=2.5cm]{geometry} % 页面设置
\usepackage[x11names]{xcolor}
\usepackage[unicode,breaklinks=true,
colorlinks=true,linkcolor=Blue4,anchorcolor=purple,citecolor=cyan,urlcolor=magenta,
pdftitle={\titlec \version},pdfauthor={\tjf}]{hyperref}

%\CTEXsetup[number=\chinese{section}, format={\large\sf\bfseries}]{section}
\usepackage{latexsym,amsmath,amssymb,bm}
\usepackage{wasysym}
\usepackage{marvosym}
\usepackage{multicol}
\usepackage{listings} % 排版代码
\usepackage{tikz}
\usetikzlibrary{arrows,shadows} % for pgf-umlsd
\usepackage[underline=true,rounded corners=false]{pgf-umlsd}

\usepackage{fancyvrb}
\DefineShortVerb{\|}
\fvset{frame=single}

\setmainfont{Times New Roman}
\setCJKmainfont[BoldFont={SimHei}]{SimSun}
\setCJKsansfont[BoldFont={STZhongsong}]{华文细黑}
\setCJKmonofont{KFKai} % 等宽字体:楷体

\CJKsetecglue{\hspace{0.1em}}
\renewcommand\CJKglue{\hskip -0.3pt plus 0.08\baselineskip}
\frenchspacing
\widowpenalty=10000
\linespread{1.2} % 行距

\usepackage[inline]{enumitem} % 调整列表样式
\setlist{noitemsep}
\setlist[itemize]{topsep=0pt,partopsep=0pt,itemsep=0pt,parsep=0pt,labelindent=\parindent,leftmargin=*,align=right}
\setlist[enumerate]{topsep=0pt,partopsep=0pt,itemsep=0pt,parsep=0pt,labelindent=\parindent,leftmargin=*,align=right}
\setlist[enumerate,1]{label={\arabic{section}.\arabic{subsection}.\arabic*}}
\setlist[enumerate,2]{label*={.\arabic*}}

\pagestyle{plain}

\lstset{language=Java,
  numbers=left,
  keywordstyle=\bfseries,
  frame=tb,
  %morekeywords=[1]{perror,assert,printf,fprintf,scanf,sscanf},
  %morekeywords=[2]{},
}

\newcommand{\des}[2]{\makebox[3em][l]{\bf #1}#2}
\newcommand{\function}[6][public]{\CTEXnoindent\vspace{1em}\par
% \begin{minipage}{\textwidth}
    \underline{\makebox[\textwidth][l]{{\sf {\bfseries #1} {{#4}} {\bfseries #2}(#3)}}}

    \des{返回}{#5}

    \des{描述}{#6}
% \end{minipage}
\par
\CTEXindent}
\newcommand{\interface}[3][public]{\CTEXnoindent\vspace{1em}\par
% \begin{minipage}{\textwidth}
    \underline{\makebox[\textwidth][l]{{\sf {\bfseries #1 interface} {{#2}} \{\} }}}

    \des{描述}{#3}
% \end{minipage}
\par
\CTEXindent}
\newcommand{\request}[6][GET]{\CTEXnoindent\vspace{1em}\par
% \begin{minipage}{\textwidth}
    \underline{\makebox[\textwidth]{{\tt\bfseries{#2}}\hfill{\it{#1}}}}

    \des{参数}{\hspace*{-3.23em}#3}

    \des{描述}{#5}

    \des{跳转}{{\tt{#4}}}
% \end{minipage}
\par
\CTEXindent}
\newcommand\kv[2]{\hspace*{3.125em}\makebox[6em][l]{\it #1} {#2}

}

\newcommand\pictext{\linespread{1}\centering}

\newcommand\method[2]{{#1}:~{\it #2}}
\newcommand\vart[2]{{#1}:~{\it #2}}
\newcommand\vars[2]{{#1}~{\itshape #2}}
\newcommand\argu[2]{{\sf #2}:~{\it #1}}
\newcommand\comt[1]{\hfill\quad{\tt //#1}}
\newcommand\comtn[1]{\\\comt{#1}}

\begin{document}

%%%% 开始 %%%%

\title{\bf\titlec~\version\footnote{本文档托管在 GitHub 上,PDF 文件:\url{https://github.com/kingfree/haut/raw/master/course/se/attd.pdf},\TeX 源文件:\url{https://github.com/kingfree/haut/blob/master/course/se/attd.tex}。}}
\author{软件工程1305班~\quad\tjf\quad~201316920311}
\maketitle

\begin{multicols}{2}
\tableofcontents
\end{multicols}

{\color{white}{ふぁっくゆ〜}}

\clearpage

本文档只描述对外的公开接口,{\bf 不}描述其具体实现细节(包括私有方法、私有类、私有属性,以及内部处理流程)。

考虑到服务端和客户端的统一性,以Java语言为基准,因为Java不仅可以有效开发Web应用(服务端前台)、系统应用(服务端后台),还可以开发基于Android的移动客户端应用。

由于时间不足和能力不够,不能做到十分完善,还请见谅。

\section{服务端}

\subsection{服务端前台}

服务端提供给教师可操作的前台界面,通过响应HTTP的GET/POST请求来实现交互。这实际上是一个Web服务器。

对于Web服务器而言,我们要提供的接口是URI,以及相应的请求类型和参数,为了简明起见,分成几个模块。

\subsubsection{课程管理模块}

\request{/subject/index/?}{
\kv{page}{页码}
\kv{search}{搜索关键字}
\kv{order}{排序方式}
\kv{asc}{排序顺序}
}{本页}
{显示课程列表。按照指定的页码和排序方式显示,如果有搜索关键字传入,则显示匹配的搜索结果。}

\request[(multipart)POST]{/subject/import/?file}{
\kv{type}{文件类型}
\kv{file}{待导入的文件}
}{/subject/import/?data}
{解析文件中的课程信息,跳转到自定义导入信息页面。}

\request[POST]{/subject/import/?data}{
\kv{data}{解析后的表格或映射表}
}{/subject/import/?status}
{导入课程。提供用户界面,允许选择表格行列与待导入字段对应。调用服务端后台接口写入数据库,跳转到导入状态页面。}

\request[POST]{/subject/import/?status}{
\kv{status}{导入状态}
\kv{data}{已导入的课程表}
}{本页}
{提示导入是否成功,显示成功导入的课程表。}

\request[POST]{/subject/bind/?data}{
\kv{data}{待绑定的课程和班级、学生}
}{/subject/bind/?status}
{绑定课头。用户自行编辑课程与班级或学生之间的对应关系。系统解析后调用服务端后台接口写入数据库,跳转到绑定状态页面。该界面也提供复选框来解绑课头。}

\request[POST]{/subject/bind/?status}{
\kv{status}{导入状态}
\kv{data}{已导入的课程表}
}{本页}
{提示导入是否成功,显示成功导入的课程表。}

\subsubsection{班级管理模块}

\request{/class/index/?}{
\kv{page}{页码}
\kv{search}{搜索关键字}
\kv{order}{排序方式}
\kv{asc}{排序顺序}
}{本页}
{显示班级列表。}

\request[(multipart)POST]{/class/import/?file}{
\kv{type}{文件类型}
\kv{file}{待导入的文件}
}{/class/import/?data}
{解析文件中的班级信息,跳转到自定义导入信息页面。}

\request[POST]{/class/import/?data}{
\kv{data}{解析后的表格或映射表}
}{/class/import/?status}
{导入班级。提供用户界面,允许选择表格行列与待导入字段对应。调用服务端后台接口写入数据库,跳转到导入状态页面。}

\request[POST]{/class/import/?status}{
\kv{status}{导入状态}
\kv{data}{已导入的班级}
}{本页}
{提示导入是否成功,显示成功导入的班级。}

\subsubsection{学生管理模块}

\request{/student/index/?}{
\kv{page}{页码}
\kv{search}{搜索关键字}
\kv{class}{班级}
\kv{order}{排序方式}
\kv{asc}{排序顺序}
}{本页}
{显示学生列表,可以指定班级。}

\request[(multipart)POST]{/student/import/?file}{
\kv{type}{文件类型}
\kv{file}{待导入的文件}
}{/student/import/?data}
{解析文件中的学生信息,跳转到自定义导入信息页面。}

\request[POST]{/student/import/?data}{
\kv{data}{解析后的表格或映射表}
}{/student/import/?status}
{导入学生。提供用户界面,允许选择表格行列与待导入字段对应。调用服务端后台接口写入数据库,跳转到导入状态页面。}

\request[POST]{/student/import/?status}{
\kv{status}{导入状态}
\kv{data}{已导入的学生}
}{本页}
{提示导入是否成功,显示成功导入的学生。}

\request{/student/collect/?}{
\kv{none}{}
}{/student/collect/?status}
{采集学生特征信息。教师在该页面向服务端后台发送要求开始采集信息的请求,后台接受请求开始处理,并实时返回采集进度。}

\request{/student/<id>/edit}{
\kv{id}{学生内部编号}
}{/student/<id>/?status}
{编辑学生信息。可以在此界面启动对该学生特征的重新采集。}

\subsubsection{考勤管理模块}

\request{/signin/do/?start}{
\kv{none}{}
}{本页}
{点名。开始一个新的签到进度,教师通过该页面向服务端后台发送请求,后台开始处理签到事务,并实时返回签到进度。}

\request{/signin/record/?id}{
\kv{id}{考勤记录编号}
}{本页}
{考勤记录。显示指定考勤具体数据。可以生成考勤报表。}

\subsection{服务端后台}

服务端后台是不可见的,服务端前台通过公共类的公共方法调用使用后台提供的公开接口,客户端应用程序则通过约定的通信协议(见第\pageref{sec:通信协议}页第\ref{sec:通信协议}节)与服务端后台进行交互。这并不是一个Web服务器。

\subsubsection{课程类 Subjects}

\function[public static]{import}
{\vars{JSONArray}{data}}
{JSONArray}{导入成功的数据。}
{导入课程表。解析传入的JSON数据为课程表,导入成功后写入数据库并返回导入成功的数据。导入失败则返回错误信息以供参考。}

\function[public static]{get}
{}
{JSONArray}{课程表。}
{获取全部课程表。没有参数,查询数据库中所有课程信息并返回。}

\function[public static]{get}
{\vars{int}{page}, \vars{int}{pageSize}, ... /* \vars{String}{order}, \vars{bool}{asc} */}
{JSONArray}{课程表。}
{获取指定页课程表。传入页码和每页数量,省略的参数为排序方式(默认为按编号顺序),返回按参数指定的课程表。}

\function[public static]{get}
{\vars{int}{id}}
{JSONObject}{课程信息。}
{获取课程信息。传入该课程的唯一ID,在数据库中查询后返回该课程信息。失败返回空。}

\function[public static]{update}
{\vars{int}{id}, \vars{JSONObject}{data}}
{JSONObject}{更新后的课程信息。}
{修改指定课程。传入该课程的唯一ID和要修改的内容,解析后在数据库中进行更新,返回更新成功后的该课程信息。失败则返回错误信息。}

\subsubsection{班级类 Classes}

\function[public static]{import}
{\vars{JSONArray}{data}}
{JSONArray}{导入成功的数据。}
{导入班级表。解析传入的JSON数据为班级表,导入成功后写入数据库并返回导入成功的数据。导入失败则返回错误信息以供参考。}

\function[public static]{get}
{}
{JSONArray}{班级表。}
{获取全部班级表。没有参数,查询数据库中所有班级信息并返回。}

\function[public static]{get}
{\vars{int}{page}, \vars{int}{pageSize}, ... /* \vars{String}{order}, \vars{bool}{asc} */}
{JSONArray}{班级表。}
{获取指定页班级表。传入页码和每页数量,省略的参数为排序方式(默认为按编号顺序),返回按参数指定的班级表。}

\function[public static]{get}
{\vars{int}{id}}
{JSONObject}{班级信息。}
{获取班级信息。传入该班级的唯一ID,在数据库中查询后返回该班级信息。失败返回空。}

\function[public static]{update}
{\vars{int}{id}, \vars{JSONObject}{data}}
{JSONObject}{更新后的班级信息。}
{修改指定班级。传入该班级的唯一ID和要修改的内容,解析后在数据库中进行更新,返回更新成功后的该班级信息。失败则返回错误信息。}

\subsubsection{学生类 Students}

\function[public static]{import}
{\vars{JSONArray}{data}}
{JSONArray}{导入成功的数据。}
{导入学生表。解析传入的JSON数据为学生表,导入成功后写入数据库并返回导入成功的数据。导入失败则返回错误信息以供参考。}

\function[public static]{get}
{}
{JSONArray}{学生表。}
{获取全部学生表。没有参数,查询数据库中所有学生信息并返回。}

\function[public static]{get}
{\vars{int}{page}, \vars{int}{pageSize}, ... /* \vars{String}{order}, \vars{bool}{asc} */}
{JSONArray}{学生表。}
{获取指定页学生表。传入页码和每页数量,省略的参数为排序方式(默认为按编号顺序),返回按参数指定的学生表。}

\function[public static]{get}
{\vars{int}{id}}
{JSONObject}{学生信息。}
{获取学生信息。传入该学生的唯一ID,在数据库中查询后返回该学生信息。失败返回空。}

\function[public static]{update}
{\vars{int}{id}, \vars{JSONObject}{data}}
{JSONObject}{更新后的学生信息。}
{修改指定学生。传入该学生的唯一ID和要修改的内容,解析后在数据库中进行更新,返回更新成功后的该学生信息。失败则返回错误信息。}

\subsubsection{签到类 Signin}

\function[public static]{do}
{\vars{int}{subjectId}}
{void}{无}
{开始一个签到进程。传入带签到的课程编号,系统检测当前时间来判断当前待签到学生信息。系统启动新线程开始签到,新线程向客户端发送签到请求进行签到。没有返回值。}

\function[public static]{status}
{}
{JSONObject}{签到状态。}
{获取当前签到状态。由服务器前端轮询,返回当前已经点到的学生信息。}

\subsubsection{识别类 Identification}

\function[public static]{init}
{\vars{int}{studentId}, \vars{JSONObject}{data}}
{JSONObject}{初始化情况。}
{初始化学生识别信息。传入参数为学生编号和采集到的数据,系统分析数据特征并存入数据库。返回初始化情况。}

\function[public static]{update}
{\vars{int}{studentId}, \vars{JSONObject}{data}}
{JSONObject}{更新情况。}
{更新学生识别信息。传入参数为学生编号和采集到的数据,系统分析数据特征并更新数据库。返回更新情况。}

\subsubsection{通信类 Communication}

通信类提供基于第\pageref{sec:通信协议}页第\ref{sec:通信协议}所述通信协议的收发,启动监听进程和发送进程属于具体实现而并非接口设计,这里不再阐述。

\function[public static]{send}
{\vars{JSONObject}{datagram}}
{void}{无。}
{发包。向客户端传送的数据包对象。}

\subsubsection{日志类 Log}

\function[public static]{log}
{\vars{int}{level}, \vars{String}{message}}
{void}{无。}
{写入日志。传入日志等级和日志消息,写出带时间的日志到日志文件和数据库。}

\section{客户端}

客户端与服务端后台进行通信,是C/S架构中的C。客户端本身采用MVC(模型-视图-控制器)的架构来设计,简述如下。由于没有实际的Android编程经验,该设计存在不可避免的错误。

\subsection{模型 model}

\subsubsection{学生类 StudentModel}

提供对学生访问和识别的方法。

\subsubsection{设备类 DeviceModel}

提供对设备唯一识别ID访问和确认的方法。

\subsubsection{图像类 ImageModel}

一个图像数据的包装类。

\subsubsection{声音类 AudioModel}

一个声音数据的包装类。

\subsection{视图 ui}

\subsubsection{主界面类 MainActivity}

{\noindent \sf MainActivity {\bfseries extends} Activity}

\function[{\itshape @Override} public]{onCreate}
{\vars{Bundle}{savedInstanceState}}
{void}{无。}
{创建主界面。载入主控制器。显示欢迎信息并等待接收指令。}

\subsubsection{登录类 LoginActivity}

{\noindent \sf LoginActivity {\bfseries extends} Activity}

\function[{\itshape @Override} public]{onCreate}
{\vars{Bundle}{savedInstanceState}}
{void}{无。}
{创建登录界面。允许学生用户输入学号和密码以登录系统。}

\subsubsection{签到类 SignInActivity}

{\noindent \sf SignInActivity {\bfseries extends} Activity}

\function[{\itshape @Override} public]{onCreate}
{\vars{Bundle}{savedInstanceState}}
{void}{无。}
{创建签到界面。提示用户要通过验证,应该做的动作和发出的声音。}

\subsubsection{验证类 VerifyActivity}

{\noindent \sf VerifyActivity {\bfseries extends} Activity}

\function[{\itshape @Override} public]{onCreate}
{\vars{Bundle}{savedInstanceState}}
{void}{无。}
{创建验证界面。即时显示从前置摄像头获取的图像信息,动态录取话筒接收的声音波形图。}

\subsection{控制器 logic}

\subsubsection{主控类 MainServices}

\function{login}
{\vars{String}{id}, \vars{String}{passwd}}
{void}{无。}
{执行登录。将用户名和密码(加密后)打包,调用通信控制器向服务端发送数据包,等待服务端返回登录状态。如果成功则跳转到登陆成功页面,失败则提示继续登录。}

\interface{signInListener}
{签到监听器。创建与服务端链接的套接字,并监听服务器的签到请求,跳转到签到界面。}

\function{signIn}
{}
{void}{无。}
{执行签到。获取当前环境的学生信息,向服务端发送待签到学生的识别信息,等待服务端返回签到内容,进入签到验证逻辑。}

\function{verify}
{}
{void}{无。}
{执行验证。获取用户的声音和图像信息,检测到可识别的图像和声音之后,与服务端建立连接并上传数据,等待服务端返回验证信息。如果验证成功,提示签到成功;如果验证失败,返回验证逻辑继续采集并签到。}

\subsubsection{通信类 Communication}

虽然作为客户端,但是我们需要监听服务端发送来的指令,所以也应提供相应接口。当然,说是自定义的通信类,实际上是对套接字类做了一个封装而已。

\function{Communication}
{}
{}{无。}
{创建与服务端连接的套接字。从配置文件中读入服务器地址,建立相应连接。失败则抛出异常,由调用者捕获。}

\function{recv}
{}
{JSONObject}{数据包。}
{接收数据。从服务器接收数据,监听服务器发送来的指令,当没有监听到时返回空,监听到时返回获得的数据。该方法需要轮询调用。}

\function{send}
{\vars{JSONObject}{数据包}}
{void}{无。}
{传送数据。向服务器发送该数据包,该方法将一直阻塞直到服务器确认数据已经收到。}

\function{close}
{}
{void}{无。}
{关闭连接。}

\subsubsection{图像处理类 Image{}Processor}

\function{isPortrait}
{\vars{Images}{图像}}
{bool}{是否为人像。}
{检查图像是否为人像。程序将图像处理为矩阵进行计算,判断是否为人的形象。当吻合率达到设定比率时返回真。注意,判断是否为本人的逻辑应在服务端完成而不是客户端!}

\subsubsection{音频处理类 Audio{}Processor}

\function{isSentence}
{\vars{Audios}{声音}, \vars{String}{句子}}
{bool}{是否为句子。}
{检查声音是否为指定句子。程序将声音分解为语音元素,判断语音是否与句子类似。当吻合率达到设定比率时返回真。同样注意,判断是否为本人的逻辑应在服务端完成而不是客户端!}

\section{数据处理约定}

为了简化格式解析处理流程,所有以字符串形式交换数据的格式,均采用以UTF-8编码的JSON表述。数据在传输过程中可以取消缩进和压缩以节省传输成本,但这里用缩进来明晰其嵌套层次。

\subsection{课程表格式}\label{sec:课程表格式}

课程表是一个数组,其中每个元素为一个课程对象。课程对象包含数个键值对以表示该课程。数组可以为空,表示没有相应数据,这样就可以在没有班级和学生的情况下导入单纯的课程。

\begin{Verbatim}
[
  {
    id:      1,
    name:    "软件工程概论",
    teacher: "王珂",
    course:  [
      {
        classes:     [2, 3], // 班级ID
        students:    [], // 学生ID,如重修跟班的学生
        weeks:       [1, 2, 3, 4, 5, 6, 7, 8],
        times:       [
          {
            weekday: 2,
            start:   "8:30",
            end:     "10:05"
          },
        ],
        classroom:   "4538"
      }
    ]
  }
]
\end{Verbatim}


\subsection{班级表格式}\label{sec:班级表格式}

班级表是一个数组,其中每个元素为一个班级对象。班级对象包含数个键值对以表示该班级。

\begin{Verbatim}
[
  {
    id:         3,
    name:       "软件工程1305班",
    shortName:  "软件1305",
    college:    "信息科学与工程学院",
    grade:      2013,
    students:   [1, 2, 3, 1098, 1099, 1100], // 学生编号
    memo:       "备注"
  }
]
\end{Verbatim}

\subsection{学生表格式}\label{sec:学生表格式}

学生表是一个数组,其中每个元素为一个学生对象。学生对象包含数个键值对以表示该学生,这里只给出基本键值对,如有需求可以随时添加。

\begin{Verbatim}
[
  {
    id:         1100,
    sid:        "201316920311"
    name:       "田劲锋",
    memo:       "备注"
  }
]
\end{Verbatim}

\subsection{消息格式}\label{sec:消息格式}

消息格式约定了调用公共方法后,出现异常情况时返回值的格式。

\begin{Verbatim}
{
  level:      "error",
  message:    "出错具体信息",
  timestamp:  1431416795,
  from:       "出错模块"
}
\end{Verbatim}

数组形式为:

\begin{Verbatim}
[
  "error",
  "出错具体信息",
  1431416795,
  "出错模块"
]
\end{Verbatim}

% \subsection{识别数据格式}\label{sec:识别数据格式}

% 略。

% \subsection{签到数据格式}\label{sec:签到数据格式}

% 略。

\subsection{音频数据格式}\label{sec:音频数据格式}

OGG。

\subsection{图像数据格式}\label{sec:图像数据格式}

PNG。

\subsection{通信协议}\label{sec:通信协议}

系统通信协议基于经典的TCP/IP协议,其中IP是IPv4。数据的拆解包可以由系统套接字接口实现,数据包中传递压缩的JSON格式字符串。关于TCP/IP的具体实现不再赘述。

\end{document}
