\documentclass[cs4size,a4paper,nofonts]{ctexart}
\usepackage[utf8]{inputenc}
\def\tjf{{\tt{田劲锋}}}
\def\titlec{课堂考勤管理系统}
\def\version{接口设计文档}
\usepackage[a4paper,margin=3cm]{geometry} % 页面设置
\usepackage[x11names]{xcolor}
\usepackage[unicode,breaklinks=true,
colorlinks=true,linkcolor=Blue4,anchorcolor=purple,citecolor=cyan,urlcolor=magenta,
pdftitle={\titlec \version},pdfauthor={\tjf}]{hyperref}

%\CTEXsetup[number=\chinese{section}, format={\large\sf\bfseries}]{section}
\usepackage{latexsym,amsmath,amssymb,bm}
\usepackage{wasysym}
\usepackage{marvosym}
\usepackage{multicol}
\usepackage{listings} % 排版代码
\usepackage{tikz}
\usetikzlibrary{arrows,shadows} % for pgf-umlsd
\usepackage[underline=true,rounded corners=false]{pgf-umlsd}

\setmainfont{Times New Roman}
\setCJKmainfont[BoldFont={SimHei}]{SimSun}  % 主要字体:宋体、黑体
\setCJKsansfont[BoldFont={STZhongsong}]{STFangsong} % 次要字体:仿宋、中宋
\setCJKmonofont{KFKai} % 等宽字体:楷体

\CJKsetecglue{\hspace{0.1em}}
\renewcommand\CJKglue{\hskip -0.3pt plus 0.08\baselineskip}
\frenchspacing
\widowpenalty=10000
\linespread{1.2} % 行距

\usepackage[inline]{enumitem} % 调整列表样式
\setlist{noitemsep}
\setlist[itemize]{topsep=0pt,partopsep=0pt,itemsep=0pt,parsep=0pt,labelindent=\parindent,leftmargin=*,align=right}
\setlist[enumerate]{topsep=0pt,partopsep=0pt,itemsep=0pt,parsep=0pt,labelindent=\parindent,leftmargin=*,align=right}
\setlist[enumerate,1]{label={\arabic{section}.\arabic{subsection}.\arabic*}}
\setlist[enumerate,2]{label*={.\arabic*}}

\pagestyle{plain}

\lstset{language=Java,
  numbers=left,
  keywordstyle=\bfseries,
  frame=tb,
  %morekeywords=[1]{perror,assert,printf,fprintf,scanf,sscanf},
  %morekeywords=[2]{},
}

\newcommand{\des}[2]{\makebox[3em][l]{\bf #1}#2}
\newcommand{\function}[5]{\CTEXnoindent\vspace{1em}\par
\begin{minipage}{\textwidth}
    \underline{\makebox[\textwidth][l]{\it{#1}}}\\
    {\tt {{#3}} {\bfseries #1}(#2);}\\
    \des{返回}{#4}\\
    \des{描述}{#5}
\end{minipage}\par
\CTEXindent}
\newcommand{\request}[6][GET]{\CTEXnoindent\vspace{1em}\par
\begin{minipage}{\textwidth}
    \underline{\makebox[\textwidth]{{\tt\bfseries{#2}}\hfill{\it{#1}}}}

    \des{参数}{\hspace*{-3.25em}#3}

    \des{描述}{#5}

    \des{跳转}{{\tt{#4}}}
\end{minipage}\par
\CTEXindent}
\newcommand\kv[2]{\hspace*{3em}\makebox[6em][l]{\it #1} {#2}

}

\newcommand\pictext{\linespread{1}\centering}

\newcommand\method[2]{{#1}:~{\it #2}}
\newcommand\vart[2]{{#1}:~{\it #2}}
\newcommand\argu[2]{{\sf #2}:~{\it #1}}
\newcommand\comt[1]{\hfill\quad{\tt //#1}}
\newcommand\comtn[1]{\\\comt{#1}}

\begin{document}

%%%% 开始 %%%%

\title{\bf\titlec~\version\footnote{本文档托管在 GitHub 上,PDF 文件:\url{https://github.com/kingfree/haut/raw/master/course/se/attd.pdf},\TeX 源文件:\url{https://github.com/kingfree/haut/blob/master/course/se/attd.tex}。}}
\author{软件工程1305班~\quad\tjf\quad~201316920311}
\maketitle

\begin{multicols}{2}
\tableofcontents
\end{multicols}

\clearpage

本文档只描述对外的公开接口,以Java语言为基准。

\section{服务端}

\subsection{服务端前台}

服务端提供给教师可操作的前台界面,通过响应HTTP请求来实现交互。这实际上是一个Web服务器。

对于Web服务器而言,我们要提供的接口是URI,以及相应的请求类型和参数,为了简明起见,分成几个模块。

\subsubsection{课程管理模块}

\request{/subject/index/?}{
\kv{page}{页码}
\kv{search}{搜索关键字}
\kv{order}{排序方式}
\kv{asc}{排序顺序}
}{本页}
{显示课程列表。按照指定的页码和排序方式显示,如果有搜索关键字传入,则显示匹配的搜索结果。}

\request[(multipart)POST]{/subject/import/?file}{
\kv{type}{文件类型}
\kv{file}{待导入的文件}
}{/subject/import/?data}
{解析文件中的课程信息,跳转到自定义导入信息页面。}

\request[POST]{/subject/import/?data}{
\kv{data}{解析后的表格或映射表}
}{/subject/import/?status}
{导入课程。提供用户界面,允许选择表格行列与待导入字段对应。调用服务端后台接口写入数据库,跳转到导入状态页面。}

\request[POST]{/subject/import/?status}{
\kv{status}{导入状态}
\kv{data}{已导入的课程表}
}{本页}
{提示导入是否成功,显示成功导入的课程表。}

\request[POST]{/subject/bind/?data}{
\kv{data}{待绑定的课程和班级、学生}
}{/subject/bind/?status}
{绑定课头。用户自行编辑课程与班级或学生之间的对应关系。系统解析后调用服务端后台接口写入数据库,跳转到绑定状态页面。该界面也提供复选框来解绑课头。}

\request[POST]{/subject/bind/?status}{
\kv{status}{导入状态}
\kv{data}{已导入的课程表}
}{本页}
{提示导入是否成功,显示成功导入的课程表。}

\subsubsection{班级管理模块}

\request{/class/index/?}{
\kv{page}{页码}
\kv{search}{搜索关键字}
\kv{order}{排序方式}
\kv{asc}{排序顺序}
}{本页}
{显示班级列表。}

\request[(multipart)POST]{/class/import/?file}{
\kv{type}{文件类型}
\kv{file}{待导入的文件}
}{/class/import/?data}
{解析文件中的班级信息,跳转到自定义导入信息页面。}

\request[POST]{/class/import/?data}{
\kv{data}{解析后的表格或映射表}
}{/class/import/?status}
{导入班级。提供用户界面,允许选择表格行列与待导入字段对应。调用服务端后台接口写入数据库,跳转到导入状态页面。}

\request[POST]{/class/import/?status}{
\kv{status}{导入状态}
\kv{data}{已导入的班级}
}{本页}
{提示导入是否成功,显示成功导入的班级。}

\subsubsection{学生管理模块}

\request{/student/index/?}{
\kv{page}{页码}
\kv{search}{搜索关键字}
\kv{class}{班级}
\kv{order}{排序方式}
\kv{asc}{排序顺序}
}{本页}
{显示学生列表,可以指定班级。}

\request[(multipart)POST]{/student/import/?file}{
\kv{type}{文件类型}
\kv{file}{待导入的文件}
}{/student/import/?data}
{解析文件中的学生信息,跳转到自定义导入信息页面。}

\request[POST]{/student/import/?data}{
\kv{data}{解析后的表格或映射表}
}{/student/import/?status}
{导入学生。提供用户界面,允许选择表格行列与待导入字段对应。调用服务端后台接口写入数据库,跳转到导入状态页面。}

\request[POST]{/student/import/?status}{
\kv{status}{导入状态}
\kv{data}{已导入的学生}
}{本页}
{提示导入是否成功,显示成功导入的学生。}

\subsubsection{}

\request{/student//?}{
\kv{}{}
}{}
{}

\request{/student//?}{
\kv{}{}
}{}
{}

\request{/student//?}{
\kv{}{}
}{}
{}

\request{/student//?}{
\kv{}{}
}{}
{}

\request{/student//?}{
\kv{}{}
}{}
{}

\request{/subject//?}{
\kv{}{}
}{}
{}

\request{/subject//?}{
\kv{}{}
}{}
{}

\request{/subject//?}{
\kv{}{}
}{}
{}

\request{/subject//?}{
\kv{}{}
}{}
{}

\subsection{服务端后台}

服务端后台是不可见的,服务端前台通过函数调用使用后台提供的公开接口,客户端应用程序则通过约定的通信协议(见第\pageref{sec:通信协议}页第\ref{sec:通信协议}节)与服务端后台进行交互。这并不是一个Web服务器。

\subsubsection{课程表类 Subjects}

\function{import}
{JSONObject data}
{}{}
{导入课程表,}

\section{客户端}

\section{通信协议}\label{sec:通信协议}

\end{document}
































