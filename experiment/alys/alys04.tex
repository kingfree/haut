\documentclass[cs4size,a4paper,nofonts]{ctexart}
\def\titlee{快速排序}
\usepackage[utf8]{inputenc}
\def\tjf{{\tt{田劲锋}}}
\def\titlec{《Java编程基础》实验报告}
\def\MyEnvironment{\begin{itemize}
\item javac 1.8.0\_40
\item java version "1.8.0\_40"
\item Java(TM) SE Runtime Environment (build 1.8.0\_40-b25)
\item Java HotSpot(TM) 64-Bit Server VM (build 25.40-b25, mixed mode)
\end{itemize}}
\usepackage[top=1.45cm,bottom=1.2cm,left=2.22cm,right=2.59cm,includeheadfoot,head=.05cm,foot=.55cm]{geometry} % 页面设置
\usepackage[unicode,breaklinks=true,
colorlinks=true,linkcolor=black,anchorcolor=black,citecolor=black,urlcolor=black,
pdftitle={\titlec},pdfauthor={\tjf}]{hyperref}
\usepackage{tikz} % 画图
\usetikzlibrary{shapes,arrows}
\usepackage{multicol} % 分栏
\usepackage{multirow} % 跨行
\usepackage{longtable} % 长表格
\usepackage{tabularx} % 变宽表格
\usepackage{booktabs} % 表格画线
\usepackage{graphicx} % 图形
\usepackage{color} % 颜色
\usepackage{xcolor} % 颜色
\usepackage{wallpaper} % 背景图片
\usepackage{listings} % 排版代码

\lstset{language=Java,
  numbers=left,
  numberstyle=\tiny,
  basicstyle=\small\tt,
  commentstyle=\color{gray},
  keywordstyle=\bfseries\color{violet},
  stringstyle=\color{red!80!black},
  showstringspaces=false,
  frame=trBL,
  % morekeywords=[1]{cout,cin,cerr,std,stdin,printf,scanf,perror},
  % keywordstyle=[1]{\color{blue}},
  % morekeywords=[2]{SHAPE,POINT,RECTANGLE,ELLIPSE,CIRCLE,CYLINDER,string,vector,ifstream,ofstream,pair},
  % keywordstyle=[2]{\color{teal}},
}

\usepackage{latexsym,amsmath,amssymb,bm}
\usepackage{verbatim} % 排版代码
\usepackage{url} % 排版链接
\usepackage{shortvrb}
\usepackage{pstricks} % 绘图
\usepackage{pst-tree} % 画树
% \usepackage{pst-uml} % 画 UML
\usepackage{uml} % 画 UML
%\usepackage{clrscode3e} % CLRS 伪代码
\usepackage{smartdiagram} % 智能画图
\usepackage{nameref}
\usepackage{rotating} % 横排大图
\usepackage{caption}
\captionsetup{font={small}} % 标题字体大小
\usepackage[inline]{enumitem} % 调整列表样式

\usepackage{tikz}
\usetikzlibrary{arrows,shadows} % for pgf-umlsd
\usepackage[underline=true,rounded corners=false]{pgf-umlsd}
\usepackage{tikz-uml}

\setmainfont{Times New Roman}
\setCJKmainfont[BoldFont={SimHei}]{SimSun}  % 主要字体:宋体、黑体
\setCJKsansfont[BoldFont={STZhongsong}]{STFangsong} % 次要字体:仿宋、中宋
\setCJKmonofont{KFKai} % 等宽字体:楷体
\setCJKfamilyfont{msyh}[BoldFont={* Bold}]{微软雅黑} \newcommand{\msyh}{\CJKfamily{msyh}} % 微软雅黑
\setCJKfamilyfont{micro}{文泉驿微米黑} \newcommand{\micro}{\CJKfamily{micro}} % 文泉驿微米黑

\CJKsetecglue{\hspace{0.1em}}
\renewcommand\CJKglue{\hskip -0.3pt plus 0.08\baselineskip}
\frenchspacing
\widowpenalty=10000
\linespread{1.2}
\setlength{\parskip}{2pt plus 2pt}
\renewcommand{\baselinestretch}{1.2}

\setlength{\abovecaptionskip}{1pt}
\setlength{\belowcaptionskip}{0pt}
\setlength{\intextsep}{8pt}

\makeindex
\pagestyle{plain}

\CTEXsetup[number=\chinese{section}, name={实验,}, format={\zihao{4}\bfseries\centering}, beforeskip={.1ex plus 1ex minus .2ex}, afterskip={.1ex plus .2ex minus .1ex}]{section}
\CTEXsetup[number=\chinese{subsection}, name={,.}, format={\normalsize\bfseries}, beforeskip={.1ex plus 1ex minus .2ex}, afterskip={.1ex plus .2ex minus .1ex}]{subsection}

\newcommand{\ctli}[2]{{#1}:\underline{{~#2~}}\hfill}
\newcommand{\ctliw}[3]{{#1}:\underline{\makebox[#3][c]{~#2~}}}

\setlength{\intextsep}{8pt}
\setlist{topsep=0pt,partopsep=0pt,itemsep=0pt,parsep=0pt}

\begin{document}
\begin{titlepage}

\begin{center}

\includegraphics[height=1cm]{image/haut.png}

\vspace*{1cm}
{\liti\fontsize{48pt}{50pt}{课\quad 程\quad 设\quad 计}}

\vspace*{4cm}
{\fontsize{36}{50}\sf\bfseries \titlec}

\vspace*{1cm}
{\huge\sf\bfseries \titlee}

\vfill
{\large
\newcommand{\ctline}[2]{\makebox[6em][s]{\bf #1}:\underline{\makebox[14em][c]{\qquad #2\qquad}}\\}
\ctline{课程设计名称}{数据结构课程设计}
\ctline{专业班级}{计算机 1303 班}
\ctline{学生姓名}{\tjf}
\ctline{学号}{201316920311}
\ctline{指导教师}{白\quad 浩}
\ctline{课程设计时间}{\today}
}

\end{center}

\end{titlepage}
\newpage

\section{实验题目}
快速排序
\section{实验目的}
\begin{enumerate}[topsep=0pt,partopsep=0pt,itemsep=0pt,parsep=0pt]
\item 掌握分治算法
\item 实现快速排序
\end{enumerate}
\section{实验要求}
使用快速排序算法对一系列数进行排序。

\begin{quote}
\verbatiminput{alys04/input.txt}
\end{quote}

\section{程序流程图}

与归并排序类似,快速排序也是基于分治算法思想的。
我们还是分三步来对子数组 $A[p..r]$ 进行处理:

\begin{enumerate}[topsep=0pt,partopsep=0pt,itemsep=0pt,parsep=0pt]
\item[\bf 分解] 把 $A[p..r]$ 分成 $A[p..q - 1]$ 和 $A[q + 1..r]$ 两个子数组,使得
$A[p.. q - 1]$ 中的每个元素都小于等于 $A[q]$,而 $A[q + 1..r]$ 中的每个元素则
都大于等于 $A[q]$。此步就是计算这个 $q$ 。
\item[\bf 解决] 递归对 $A[p..q-1]$ 和 $A[q+1..r]$ 进行排序。
\item[\bf 合并]  因为子数组已经有序,那么 $A[p..r]$ 排序完成。
\end{enumerate}

下面的就是快速排序过程:
\begin{quote}
\begin{codebox}
\Procname{\proc{Quicksort}($A, p, r$)}
\li \If $p < r$ \Then
\li   $q =$ \proc{Partition}($A, p, r$)
\li   \proc{Quicksort}($A, p, q - 1$)
\li   \proc{Quicksort}($A, q + 1, r$)
    \End
\end{codebox}
\end{quote}

对数组 $A$ 排序则调用 \proc{Quicksort}($A, 1, A.length$) 即可。

算法的关键是将 $A[p..r]$ 重新排列的划分过程 \proc{Partition}:
\begin{quote}
\begin{codebox}
\Procname{\proc{Partition}($A, p, r$)}
\li $x = A[r]$
\li $i = p - 1$
\li \For $j = p$ \To $r - 1$ \Do
\li   \If $A[j] \le x$ \Then
\li     $i++$
\li     交换 $A[i] \leftrightarrow A[j]$
       \End
     \End
\li 交换 $A[i + 1] \leftrightarrow A[r]$
\li \Return i + 1
\end{codebox}
\end{quote}


\proc{Partition} 每次选择一个元素 $x = A[r]$ 作为比较基准,
$i$ 和 $j$ 两个指针分别用来分割比 $x$ 小和比 $x$ 大的子数组,
3--6 行的循环用来把不合适的元素进行交换。


快速排序的均摊时间复杂度是 $O(n\lg n)$,在某些特殊情况下会退化成 $O(n^2)$。

\section{程序代码}
{\linespread{1}\lstinputlisting{alys04/alys04.cpp}}

\section{实验结果}
\begin{quote}
\verbatiminput{alys04/output.txt}
\end{quote}

\section{实验体会}

这次的伪代码依然来自《算法导论》\cite{clrs} 并按其改写成C++程序。
顾名思义,快速排序的算法在均摊意义上是基于比较的最快的排序算法,因此大多数的排序都采用快速排序。
快速排序是不稳定的,通过为每个元素分配唯一的辅助键值可以达成稳定排序的目的。
C \verb|<stdlib.h>| 中提供了 \verb|qsort| 函数,即快速排序;
C++ STL \verb|<algorithm>| 库中提供了 \verb|sort| 函数,对于大数据采用了快速排序。

\begin{thebibliography}{10}
\bibitem{clrs}{\it Introduction to Algorithms}, Third Edition, Thomas H. Cormen and Charles E. Leiserson and Ronald L. Rivest and Clifford Stein, 2011
\end{thebibliography}

\end{document}
