\documentclass[cs4size,a4paper,nofonts]{ctexart}
\def\titlee{二叉树插入}
\usepackage[utf8]{inputenc}
\def\tjf{{\tt{田劲锋}}}
\def\titlec{《Java编程基础》实验报告}
\def\MyEnvironment{\begin{itemize}
\item javac 1.8.0\_40
\item java version "1.8.0\_40"
\item Java(TM) SE Runtime Environment (build 1.8.0\_40-b25)
\item Java HotSpot(TM) 64-Bit Server VM (build 25.40-b25, mixed mode)
\end{itemize}}
\usepackage[top=1.45cm,bottom=1.2cm,left=2.22cm,right=2.59cm,includeheadfoot,head=.05cm,foot=.55cm]{geometry} % 页面设置
\usepackage[unicode,breaklinks=true,
colorlinks=true,linkcolor=black,anchorcolor=black,citecolor=black,urlcolor=black,
pdftitle={\titlec},pdfauthor={\tjf}]{hyperref}
\usepackage{tikz} % 画图
\usetikzlibrary{shapes,arrows}
\usepackage{multicol} % 分栏
\usepackage{multirow} % 跨行
\usepackage{longtable} % 长表格
\usepackage{tabularx} % 变宽表格
\usepackage{booktabs} % 表格画线
\usepackage{graphicx} % 图形
\usepackage{color} % 颜色
\usepackage{xcolor} % 颜色
\usepackage{wallpaper} % 背景图片
\usepackage{listings} % 排版代码

\lstset{language=Java,
  numbers=left,
  numberstyle=\tiny,
  basicstyle=\small\tt,
  commentstyle=\color{gray},
  keywordstyle=\bfseries\color{violet},
  stringstyle=\color{red!80!black},
  showstringspaces=false,
  frame=trBL,
  % morekeywords=[1]{cout,cin,cerr,std,stdin,printf,scanf,perror},
  % keywordstyle=[1]{\color{blue}},
  % morekeywords=[2]{SHAPE,POINT,RECTANGLE,ELLIPSE,CIRCLE,CYLINDER,string,vector,ifstream,ofstream,pair},
  % keywordstyle=[2]{\color{teal}},
}

\usepackage{latexsym,amsmath,amssymb,bm}
\usepackage{verbatim} % 排版代码
\usepackage{url} % 排版链接
\usepackage{shortvrb}
\usepackage{pstricks} % 绘图
\usepackage{pst-tree} % 画树
% \usepackage{pst-uml} % 画 UML
\usepackage{uml} % 画 UML
%\usepackage{clrscode3e} % CLRS 伪代码
\usepackage{smartdiagram} % 智能画图
\usepackage{nameref}
\usepackage{rotating} % 横排大图
\usepackage{caption}
\captionsetup{font={small}} % 标题字体大小
\usepackage[inline]{enumitem} % 调整列表样式

\usepackage{tikz}
\usetikzlibrary{arrows,shadows} % for pgf-umlsd
\usepackage[underline=true,rounded corners=false]{pgf-umlsd}
\usepackage{tikz-uml}

\setmainfont{Times New Roman}
\setCJKmainfont[BoldFont={SimHei}]{SimSun}  % 主要字体:宋体、黑体
\setCJKsansfont[BoldFont={STZhongsong}]{STFangsong} % 次要字体:仿宋、中宋
\setCJKmonofont{KFKai} % 等宽字体:楷体
\setCJKfamilyfont{msyh}[BoldFont={* Bold}]{微软雅黑} \newcommand{\msyh}{\CJKfamily{msyh}} % 微软雅黑
\setCJKfamilyfont{micro}{文泉驿微米黑} \newcommand{\micro}{\CJKfamily{micro}} % 文泉驿微米黑

\CJKsetecglue{\hspace{0.1em}}
\renewcommand\CJKglue{\hskip -0.3pt plus 0.08\baselineskip}
\frenchspacing
\widowpenalty=10000
\linespread{1.2}
\setlength{\parskip}{2pt plus 2pt}
\renewcommand{\baselinestretch}{1.2}

\setlength{\abovecaptionskip}{1pt}
\setlength{\belowcaptionskip}{0pt}
\setlength{\intextsep}{8pt}

\makeindex
\pagestyle{plain}

\CTEXsetup[number=\chinese{section}, name={实验,}, format={\zihao{4}\bfseries\centering}, beforeskip={.1ex plus 1ex minus .2ex}, afterskip={.1ex plus .2ex minus .1ex}]{section}
\CTEXsetup[number=\chinese{subsection}, name={,.}, format={\normalsize\bfseries}, beforeskip={.1ex plus 1ex minus .2ex}, afterskip={.1ex plus .2ex minus .1ex}]{subsection}

\newcommand{\ctli}[2]{{#1}:\underline{{~#2~}}\hfill}
\newcommand{\ctliw}[3]{{#1}:\underline{\makebox[#3][c]{~#2~}}}

\setlength{\intextsep}{8pt}
\setlist{topsep=0pt,partopsep=0pt,itemsep=0pt,parsep=0pt}

\begin{document}
\begin{titlepage}

\begin{center}

\includegraphics[height=1cm]{image/haut.png}

\vspace*{1cm}
{\liti\fontsize{48pt}{50pt}{课\quad 程\quad 设\quad 计}}

\vspace*{4cm}
{\fontsize{36}{50}\sf\bfseries \titlec}

\vspace*{1cm}
{\huge\sf\bfseries \titlee}

\vfill
{\large
\newcommand{\ctline}[2]{\makebox[6em][s]{\bf #1}:\underline{\makebox[14em][c]{\qquad #2\qquad}}\\}
\ctline{课程设计名称}{数据结构课程设计}
\ctline{专业班级}{计算机 1303 班}
\ctline{学生姓名}{\tjf}
\ctline{学号}{201316920311}
\ctline{指导教师}{白\quad 浩}
\ctline{课程设计时间}{\today}
}

\end{center}

\end{titlepage}
\newpage

\section{实验题目}
二叉树的插入
\section{实验目的}
\begin{enumerate}[topsep=0pt,partopsep=0pt,itemsep=0pt,parsep=0pt]
\item 实现二叉查找树的插入
\end{enumerate}
\section{实验要求}
给定一系列数,依次插入到二叉树中。

\begin{quote}
\verbatiminput{alys02/input.txt}
\end{quote}

\section{程序流程图}

二叉查找树\footnote{本节内容来源于我之前的论文\cite{sbt}。},是一棵空树,或是具有下列性质的二叉树:
\begin{enumerate}[topsep=0pt,partopsep=0pt,itemsep=0pt,parsep=0pt]
\item 若它的左子树不空,则左子树上所有结点的值均小于它的根结点的值;
\item 若它的右子树不空,则右子树上所有结点的值均大于它的根结点的值;
\item 它的左、右子树也分别为二叉查找树。
\end{enumerate}

二叉查找树是递归定义的,如图\ref{aBST}是一棵二叉查找树。
\begin{figure}[htbp]
\begin{center}
\pstree[levelsep=0.8cm]{\Tcircle{8}}{
    \pstree {\Tcircle{3}} {
        \Tcircle{1}
        \pstree {\Tcircle{6}} {
            \Tcircle{4}
            \Tcircle{7}
        }
    }
    \pstree {\Tcircle{10}} {
        \TC
        \pstree {\Tcircle{14}} {
            \Tcircle{13}
            \TC
        }
    }
}
\end{center}
\caption{\label{aBST}一棵三层的二叉查找树}
\end{figure}

我们用链表结构来存储二叉查找树,其中每一个节点都是一个对象。有$data$存储其值,$left$和$right$指向其左右儿子。

对于一个已知的二叉查找树,从小到大输出其节点的值,只需对其进行二叉树的中序遍历,即递归地先输出其左子树,再输出其本身,然后输出其右子树。遍历的时间复杂度为$O(n)$。这里,我们给出这一算法的伪代码,参见图\ref{BST-Inorder}。

\begin{figure}[htp]
\begin{quote}
\begin{codebox}
\Procname{\proc{Inorder}($x$)}
\li \If $x\ne\const{nil}$ \Then
\li   \proc{Inorder}($x.left$)
\li   输出 $x.data$
\li   \proc{Inorder}($x.right$)
    \End
\end{codebox}
\end{quote}
\caption{\label{BST-Inorder}中序遍历输出}
\end{figure}

将一个新值$v$插入到树$T$中,基本方法是类似于线性表中的二分查找,不断地在树中缩小范围定位,最终找到一个合适的位置插入。具体方法如下所述:
\begin{enumerate}[topsep=0pt,partopsep=0pt,itemsep=0pt,parsep=0pt,label=\arabic*.]
\item 从根节点开始插入;
\item 如果要插入的值小于等于当前节点的值,在当前节点的左子树中插入;
\item 如果要插入的值大于当前节点的值,在当前节点的右子树中插入;
\item 如果当前节点为空节点,在此建立新的节点,该节点的值为要插入的值,左右子树为空,插入成功。
\end{enumerate}

对于相同的元素,一种方法我们规定把它插入左边,另一种方法是我们在节点上再加一个域,记录重复节点的个数。上述方法为前者。插入的时间复杂度仍为$O(h)$。算法的伪代码参见图\ref{BST-Insert}。

\begin{figure}[htbp]
\begin{center}
\def\dedge{\ncline[linestyle=dotted]}
\pstree[levelsep=0.8cm]{\Tdia{12}}{
    \pstree {\Tcircle{5}} {
        \Tcircle{2}
        \Tcircle{9}
    }
    \pstree {\Tdia{18}} {
        \pstree {\Tdia{15}} {
            \Tdia[edge=\dedge]{13}
            \Tcircle{17}
        }
        \Tcircle{19}
    }
}
\end{center}
\caption{\label{iBST}将13插入一棵二叉查找树。菱形表示从根节点开始向下到插入位置的路径,虚线为在树中插入该节点而产生的链接。}
\end{figure}

\begin{figure}[htp]
\begin{quote}
\begin{codebox}
\Procname{\proc{Insert}($T, v$)}
\li \If $T=\const{nil}$ \Then
\li   new $T.key = v$ \Comment 新建节点并插入
\li \ElseIf $v \le T.key$ \Then
\li   $T.left.p = T$
\li   \proc{Insert}($T.left, v$)
\li \Else
\li   $T.right.p = T$
\li   \proc{Insert}($T.right, v$)
    \End
\end{codebox}
\end{quote}
\caption{\label{BST-Insert}二叉查找树的插入}
\end{figure}

\section{程序代码}
{\linespread{1}\lstinputlisting{alys02/alys02.cpp}}

\section{实验结果}
输出文件output.txt:
\begin{quote}
\verbatiminput{alys02/output.txt}
\end{quote}

\section{实验体会}
该实验题目说是二叉树的插入,其实准确来讲应该是二叉查找树的插入。关于二叉查找树的问题,无数先人已经探索出来了非常平整的道路了。包括随之拓展出来的二叉平衡树,也是充满了数学之美。三年前我也写过这样的论文,当然并没有什么创新价值,所以就没有发表,不过网络上应该是可以搜索到的。关于二叉树的问题我已经研究过很多了,随之产生的思考也是无穷尽的。

至于程序流程图,我还是一如既往地不画了。描述算法的最好办法依然是伪代码,没有人是用复杂的流程图来描述算法的。

程序是C++编写的,也是为了熟悉面向对象的程序设计方法。我对C++也不是很熟悉,调试也欠缺不少,比如如下的编译警告我暂时还没能解决,希望今后能找到解决办法。
\begin{quote}
\small
\begin{verbatim}
alys02.cpp: In constructor `BSTree<ElemType>::BSTNode::BSTNode(ElemType) [with E
lemType = int]':
alys02.cpp:44:   instantiated from `void BSTree<ElemType>::insert_node(BSTree<El
emType>::BSTNode*&, const ElemType&) [with ElemType = int]'
alys02.cpp:24:   instantiated from `void BSTree<ElemType>::insert(const ElemType
&) [with ElemType = int]'
alys02.cpp:72:   instantiated from here
alys02.cpp:13: warning: `BSTree<int>::BSTNode::data' will be initialized after
alys02.cpp:11: warning:   `BSTree<int>::BSTNode*BSTree<int>::BSTNode::left'
alys02.cpp:15: warning:   when initialized here
\end{verbatim}
\end{quote}


\begin{thebibliography}{10}
\bibitem{sbt}由BST到SBT, 田劲锋, 2011
\end{thebibliography}

\end{document}
