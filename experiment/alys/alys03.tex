\documentclass[cs4size,a4paper,nofonts]{ctexart}
\def\titlee{归并排序}
\usepackage[utf8]{inputenc}
\def\tjf{{\tt{田劲锋}}}
\def\titlec{《Java编程基础》实验报告}
\def\MyEnvironment{\begin{itemize}
\item javac 1.8.0\_40
\item java version "1.8.0\_40"
\item Java(TM) SE Runtime Environment (build 1.8.0\_40-b25)
\item Java HotSpot(TM) 64-Bit Server VM (build 25.40-b25, mixed mode)
\end{itemize}}
\usepackage[top=1.45cm,bottom=1.2cm,left=2.22cm,right=2.59cm,includeheadfoot,head=.05cm,foot=.55cm]{geometry} % 页面设置
\usepackage[unicode,breaklinks=true,
colorlinks=true,linkcolor=black,anchorcolor=black,citecolor=black,urlcolor=black,
pdftitle={\titlec},pdfauthor={\tjf}]{hyperref}
\usepackage{tikz} % 画图
\usetikzlibrary{shapes,arrows}
\usepackage{multicol} % 分栏
\usepackage{multirow} % 跨行
\usepackage{longtable} % 长表格
\usepackage{tabularx} % 变宽表格
\usepackage{booktabs} % 表格画线
\usepackage{graphicx} % 图形
\usepackage{color} % 颜色
\usepackage{xcolor} % 颜色
\usepackage{wallpaper} % 背景图片
\usepackage{listings} % 排版代码

\lstset{language=Java,
  numbers=left,
  numberstyle=\tiny,
  basicstyle=\small\tt,
  commentstyle=\color{gray},
  keywordstyle=\bfseries\color{violet},
  stringstyle=\color{red!80!black},
  showstringspaces=false,
  frame=trBL,
  % morekeywords=[1]{cout,cin,cerr,std,stdin,printf,scanf,perror},
  % keywordstyle=[1]{\color{blue}},
  % morekeywords=[2]{SHAPE,POINT,RECTANGLE,ELLIPSE,CIRCLE,CYLINDER,string,vector,ifstream,ofstream,pair},
  % keywordstyle=[2]{\color{teal}},
}

\usepackage{latexsym,amsmath,amssymb,bm}
\usepackage{verbatim} % 排版代码
\usepackage{url} % 排版链接
\usepackage{shortvrb}
\usepackage{pstricks} % 绘图
\usepackage{pst-tree} % 画树
% \usepackage{pst-uml} % 画 UML
\usepackage{uml} % 画 UML
%\usepackage{clrscode3e} % CLRS 伪代码
\usepackage{smartdiagram} % 智能画图
\usepackage{nameref}
\usepackage{rotating} % 横排大图
\usepackage{caption}
\captionsetup{font={small}} % 标题字体大小
\usepackage[inline]{enumitem} % 调整列表样式

\usepackage{tikz}
\usetikzlibrary{arrows,shadows} % for pgf-umlsd
\usepackage[underline=true,rounded corners=false]{pgf-umlsd}
\usepackage{tikz-uml}

\setmainfont{Times New Roman}
\setCJKmainfont[BoldFont={SimHei}]{SimSun}  % 主要字体:宋体、黑体
\setCJKsansfont[BoldFont={STZhongsong}]{STFangsong} % 次要字体:仿宋、中宋
\setCJKmonofont{KFKai} % 等宽字体:楷体
\setCJKfamilyfont{msyh}[BoldFont={* Bold}]{微软雅黑} \newcommand{\msyh}{\CJKfamily{msyh}} % 微软雅黑
\setCJKfamilyfont{micro}{文泉驿微米黑} \newcommand{\micro}{\CJKfamily{micro}} % 文泉驿微米黑

\CJKsetecglue{\hspace{0.1em}}
\renewcommand\CJKglue{\hskip -0.3pt plus 0.08\baselineskip}
\frenchspacing
\widowpenalty=10000
\linespread{1.2}
\setlength{\parskip}{2pt plus 2pt}
\renewcommand{\baselinestretch}{1.2}

\setlength{\abovecaptionskip}{1pt}
\setlength{\belowcaptionskip}{0pt}
\setlength{\intextsep}{8pt}

\makeindex
\pagestyle{plain}

\CTEXsetup[number=\chinese{section}, name={实验,}, format={\zihao{4}\bfseries\centering}, beforeskip={.1ex plus 1ex minus .2ex}, afterskip={.1ex plus .2ex minus .1ex}]{section}
\CTEXsetup[number=\chinese{subsection}, name={,.}, format={\normalsize\bfseries}, beforeskip={.1ex plus 1ex minus .2ex}, afterskip={.1ex plus .2ex minus .1ex}]{subsection}

\newcommand{\ctli}[2]{{#1}:\underline{{~#2~}}\hfill}
\newcommand{\ctliw}[3]{{#1}:\underline{\makebox[#3][c]{~#2~}}}

\setlength{\intextsep}{8pt}
\setlist{topsep=0pt,partopsep=0pt,itemsep=0pt,parsep=0pt}

\begin{document}
\begin{titlepage}

\begin{center}

\includegraphics[height=1cm]{image/haut.png}

\vspace*{1cm}
{\liti\fontsize{48pt}{50pt}{课\quad 程\quad 设\quad 计}}

\vspace*{4cm}
{\fontsize{36}{50}\sf\bfseries \titlec}

\vspace*{1cm}
{\huge\sf\bfseries \titlee}

\vfill
{\large
\newcommand{\ctline}[2]{\makebox[6em][s]{\bf #1}:\underline{\makebox[14em][c]{\qquad #2\qquad}}\\}
\ctline{课程设计名称}{数据结构课程设计}
\ctline{专业班级}{计算机 1303 班}
\ctline{学生姓名}{\tjf}
\ctline{学号}{201316920311}
\ctline{指导教师}{白\quad 浩}
\ctline{课程设计时间}{\today}
}

\end{center}

\end{titlepage}
\newpage

\section{实验题目}
归并排序
\section{实验目的}
\begin{enumerate}[topsep=0pt,partopsep=0pt,itemsep=0pt,parsep=0pt]
\item 掌握分治算法
\item 实现归并排序
\end{enumerate}
\section{实验要求}
使用归并排序算法对一系列数进行排序。

\begin{quote}
\verbatiminput{alys03/input.txt}
\end{quote}

\section{程序流程图}

归并排序基于如下分治算法:
\begin{enumerate}[topsep=0pt,partopsep=0pt,itemsep=0pt,parsep=0pt]
\item[\bf 分解] 把$n$个数的序列分成两个子序列;
\item[\bf 解决] 使用归并排序分别排序两个子序列;
\item[\bf 合并] 合并两个有序序列。
\end{enumerate}

如图\ref{Merge}是合并两个有序序列的算法。

\begin{figure}[htp]
\begin{quote}
\begin{codebox}
\Procname{\proc{Merge}($A, p, q, r$)}
\li $n_1 = q - p + 1$
\li $n_2 = r - q$
\li 新建数组 $L[1..n_1+1]$ 和 $R[1..n_2+1]$
\li \For $i = 1$ \To $n_1$ \Do
\li   $L[i] = A[p + i -1]$
     \End
\li \For $j = 1$ \To $n_2$ \Do
\li   $R[j] = A[q + j]$
     \End
\li $L[n_1 + 1] = R[n_2 + 1] = \infty$
\li $i = j = 1$
\li \For $k = p$ \To $r$ \Do
\li   \If $L[i] \le R[j]$ \Then
\li     $A[k] = L[i]$
\li     $i++$
\li   \Else
\li     $A[k] = R[j]$
\li     $j++$
       \End
     \End
\end{codebox}
\end{quote}
\caption{\label{Merge}合并算法}
\end{figure}

如图\ref{Merge-Sort}是归并排序的主算法,分别对两个子序列进行划分。

\begin{figure}[htp]
\begin{quote}
\begin{codebox}
\Procname{\proc{Merge-Sort}($A, p, r$)}
\li \If $p < r$ \Then
\li   $q = \lfloor (p + r) / 2 \rfloor$
\li   \proc{Merge-Sort}($A, p, q$)
\li   \proc{Merge-Sort}($A, q + 1, r$)
\li   \proc{Merge}($A, p, q, r$)
    \End
\end{codebox}
\end{quote}
\caption{\label{Merge-Sort}归并排序主算法}
\end{figure}

归并排序的时间复杂度是 $O(n\lg n)$,由主定理可证。

\section{程序代码}
{\linespread{1}\lstinputlisting{alys03/alys03.cpp}}

\section{实验结果}
\begin{quote}
\verbatiminput{alys03/output.txt}
\end{quote}

\section{实验体会}

归并排序是《算法导论》\cite{clrs} 中针对分治算法举出来的一个非常典型的例子,并通过详尽的分析应用主定理再次描述了时间复杂度的计算。程序伪代码来自与算法导论,并按其改写成C++程序。虽然使用了C++的vector标准模板类来简化数组操作,但仍需要注意C++中的数组是以0开始的,而非算法中所描述的下标1开始。其他一些细节,比如$\infty$表示为\verb|INT_MAX|,以及数组的引用传参,也需要注意。

分治算法作为一个重要的算法思想,依然发挥着其不可缺失的作用。

\begin{thebibliography}{10}
\bibitem{clrs}{\it Introduction to Algorithms}, Third Edition, Thomas H. Cormen and Charles E. Leiserson and Ronald L. Rivest and Clifford Stein, 2011
\end{thebibliography}

\end{document}
