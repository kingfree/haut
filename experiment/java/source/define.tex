\usepackage[utf8]{inputenc}
\def\tjf{{\tt{田劲锋}}}
\def\titlec{《Java编程基础》实验报告}
\def\MyEnvironment{\begin{itemize}
\item javac 1.8.0\_40
\item java version "1.8.0\_40"
\item Java(TM) SE Runtime Environment (build 1.8.0\_40-b25)
\item Java HotSpot(TM) 64-Bit Server VM (build 25.40-b25, mixed mode)
\end{itemize}}
\usepackage[top=1.45cm,bottom=1.2cm,left=2.22cm,right=2.59cm,includeheadfoot,head=.05cm,foot=.55cm]{geometry} % 页面设置
\usepackage[unicode,breaklinks=true,
colorlinks=true,linkcolor=black,anchorcolor=black,citecolor=black,urlcolor=black,
pdftitle={\titlec},pdfauthor={\tjf}]{hyperref}
\usepackage{tikz} % 画图
\usetikzlibrary{shapes,arrows}
\usepackage{multicol} % 分栏
\usepackage{multirow} % 跨行
\usepackage{longtable} % 长表格
\usepackage{tabularx} % 变宽表格
\usepackage{booktabs} % 表格画线
\usepackage{graphicx} % 图形
\usepackage{color} % 颜色
\usepackage{xcolor} % 颜色
\usepackage{wallpaper} % 背景图片
\usepackage{listings} % 排版代码

\lstset{language=Java,
  numbers=left,
  numberstyle=\tiny,
  basicstyle=\small\tt,
  commentstyle=\color{gray},
  keywordstyle=\bfseries\color{violet},
  stringstyle=\color{red!80!black},
  showstringspaces=false,
  frame=trBL,
  % morekeywords=[1]{cout,cin,cerr,std,stdin,printf,scanf,perror},
  % keywordstyle=[1]{\color{blue}},
  % morekeywords=[2]{SHAPE,POINT,RECTANGLE,ELLIPSE,CIRCLE,CYLINDER,string,vector,ifstream,ofstream,pair},
  % keywordstyle=[2]{\color{teal}},
}

\usepackage{latexsym,amsmath,amssymb,bm}
\usepackage{verbatim} % 排版代码
\usepackage{url} % 排版链接
\usepackage{shortvrb}
\usepackage{pstricks} % 绘图
\usepackage{pst-tree} % 画树
% \usepackage{pst-uml} % 画 UML
\usepackage{uml} % 画 UML
%\usepackage{clrscode3e} % CLRS 伪代码
\usepackage{smartdiagram} % 智能画图
\usepackage{nameref}
\usepackage{rotating} % 横排大图
\usepackage{caption}
\captionsetup{font={small}} % 标题字体大小
\usepackage[inline]{enumitem} % 调整列表样式

\usepackage{tikz}
\usetikzlibrary{arrows,shadows} % for pgf-umlsd
\usepackage[underline=true,rounded corners=false]{pgf-umlsd}
\usepackage{tikz-uml}

\setmainfont{Times New Roman}
\setCJKmainfont[BoldFont={SimHei}]{SimSun}  % 主要字体:宋体、黑体
\setCJKsansfont[BoldFont={STZhongsong}]{STFangsong} % 次要字体:仿宋、中宋
\setCJKmonofont{KFKai} % 等宽字体:楷体
\setCJKfamilyfont{msyh}[BoldFont={* Bold}]{微软雅黑} \newcommand{\msyh}{\CJKfamily{msyh}} % 微软雅黑
\setCJKfamilyfont{micro}{文泉驿微米黑} \newcommand{\micro}{\CJKfamily{micro}} % 文泉驿微米黑

\CJKsetecglue{\hspace{0.1em}}
\renewcommand\CJKglue{\hskip -0.3pt plus 0.08\baselineskip}
\frenchspacing
\widowpenalty=10000
\linespread{1.2}
\setlength{\parskip}{2pt plus 2pt}
\renewcommand{\baselinestretch}{1.2}

\setlength{\abovecaptionskip}{1pt}
\setlength{\belowcaptionskip}{0pt}
\setlength{\intextsep}{8pt}

\makeindex
\pagestyle{plain}

\CTEXsetup[number=\chinese{section}, name={实验,}, format={\zihao{4}\bfseries\centering}, beforeskip={.1ex plus 1ex minus .2ex}, afterskip={.1ex plus .2ex minus .1ex}]{section}
\CTEXsetup[number=\chinese{subsection}, name={,.}, format={\normalsize\bfseries}, beforeskip={.1ex plus 1ex minus .2ex}, afterskip={.1ex plus .2ex minus .1ex}]{subsection}

\newcommand{\ctli}[2]{{#1}:\underline{{~#2~}}\hfill}
\newcommand{\ctliw}[3]{{#1}:\underline{\makebox[#3][c]{~#2~}}}

\setlength{\intextsep}{8pt}
\setlist{topsep=0pt,partopsep=0pt,itemsep=0pt,parsep=0pt}

\newcommand{\titl}[3]{
\CTEXnoindent

\ctliw{实验日期}{#1~年~#2~月~#3~日}{9em}\hfill
\ctliw{班级}{软件 1305 班}{7em}

\ctliw{学号(后四位)}{0311}{6em}\hfill
\ctliw{姓名}{\tjf}{6em}\hfill
\ctliw{成绩}{}{7em}

\CTEXindent
}
