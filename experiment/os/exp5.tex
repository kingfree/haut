\documentclass[c5size,a4paper,nofonts]{ctexart}
\usepackage[utf8]{inputenc}
\def\tjf{{\tt{田劲锋}}}
\def\titlec{页式存储管理的页面置换算法模拟}

\usepackage[a4paper,margin=2.2cm]{geometry} % 页面设置
\usepackage[unicode,breaklinks=true,
colorlinks=true,linkcolor=black,anchorcolor=black,citecolor=black,urlcolor=black,
pdftitle={\titlec},pdfauthor={\tjf}]{hyperref}
\usepackage{latexsym,amsmath,amssymb,bm}
\usepackage{graphicx}
\usepackage{fancyvrb}
\DefineShortVerb{\|}
\fvset{frame=single}

\setmainfont{Times New Roman}
\setCJKmainfont[BoldFont={SimHei}]{SimSun}  % 主要字体:宋体、黑体
\setCJKsansfont[BoldFont={STZhongsong}]{STFangsong} % 次要字体:仿宋、中宋
\setCJKmonofont{KFKai} % 等宽字体:楷体

\CJKsetecglue{\hspace{0.1em}}
\renewcommand\CJKglue{\hskip -0.3pt plus 0.08\baselineskip}
\frenchspacing
\widowpenalty=10000
\linespread{1.2} % 行距

\CTEXsetup[name={实验,},number={\arabic{part}}]{part}
\CTEXsetup[name={,.},format={\large}]{section}

\usepackage[inline]{enumitem} % 调整列表样式
\setlist{noitemsep}
\setlist[itemize]{topsep=0pt,partopsep=0pt,itemsep=0pt,parsep=0pt}
\setlist[enumerate]{topsep=0pt,partopsep=0pt,itemsep=0pt,parsep=0pt}
\setlist[enumerate,1]{label={\arabic*.}}
\setlist[enumerate,2]{label={(\arabic*)}}

%\makeindex
\pagestyle{plain}

\newcommand{\image}[3][height=10cm]{%\begin{minipage}[t]{0.5\textwidth}
    \centering
    \includegraphics[#1]{#2}
    \caption{#3}
    \label{fig:#3}
%\end{minipage}
}

\linespread{1.1} % 行距
\fvset{frame=none}
\UndefineShortVerb{\|}

\begin{document}

%%%% 开始 %%%%

\setcounter{part}{4}
\begin{titlepage}

\begin{center}

\includegraphics[height=1cm]{image/haut.png}

\vspace*{1cm}
{\liti\fontsize{48pt}{50pt}{课\quad 程\quad 设\quad 计}}

\vspace*{4cm}
{\fontsize{36}{50}\sf\bfseries \titlec}

\vspace*{1cm}
{\huge\sf\bfseries \titlee}

\vfill
{\large
\newcommand{\ctline}[2]{\makebox[6em][s]{\bf #1}:\underline{\makebox[14em][c]{\qquad #2\qquad}}\\}
\ctline{课程设计名称}{数据结构课程设计}
\ctline{专业班级}{计算机 1303 班}
\ctline{学生姓名}{\tjf}
\ctline{学号}{201316920311}
\ctline{指导教师}{白\quad 浩}
\ctline{课程设计时间}{\today}
}

\end{center}

\end{titlepage}


% \iffalse
\section{实验目的}
通过请求页式存储管理中页面置换算法模拟设计,了解虚拟存储技术的特点,掌握请求页式存储管理的页面置换算法。

\section{实验环境}
装有操作系统Windows XP和开发工具VC++6.0,内存在256M以上的微机;

或者:装有Linux(Fedora 7)操作系统和gcc编译器,内存在256M以上的微机。

\section{实验内容}
\begin{enumerate}[label={(\arabic*)}]
\item 通过随机数产生一个指令序列,共320条指令。指令的地址按下述原则生成:
\begin{enumerate}[label={\textcircled{\small\arabic*}}]
\item 50\%的指令是顺序执行的;
\item 25\%的指令是均匀分布在前地址部分;
\item 25\%的指令是均匀分布在后地址部分;
\end{enumerate}

具体的实施方法是:
\begin{enumerate}[label={\textcircled{\small\arabic*}}]
\item 在$[0,319]$的指令地址之间随机选取一起点$m$;
\item 顺序执行一条指令,即执行地址为$m+1$的指令;
\item 在前地址$[0,m+1]$中随机选取一条指令并执行,该指令的地址为$m'$;
\item 顺序执行一条指令,其地址为$m'+1$的指令;
\item 在后地址$[m'+2,319]$中随机选取一条指令并执行;
\item 重复上述步骤\textcircled{\small1}~\textcircled{\small5},直到执行320次指令。
\end{enumerate}

\item 将指令序列变换为页地址流
\begin{enumerate}[label={\textcircled{\small\arabic*}}]
\item 设页面大小为1K;
\item 分配内存容量为4K到32K;
\item 用户虚存容量为32K。
\end{enumerate}

在用户虚存中,按每K存放10条指令排列虚存地址,即320条指令在虚存中的存放方式为:\\
第0条~第9条指令为第0页(对应虚存地址为$[0,9]$);\\
第10条~第19条指令为第1页(对应虚存地址为$[10,19]$);\\
……\\
……\\
第310条~第319条指令为第31页(对应虚存地址为$[310,319]$)。\\
按以上方式,用户指令可组成32页。
\item 计算先进先出(FIFO)算法或最近最少使用(LRU)算法在不同内存容量下的命中率。

其中,$\text{命中率}=1-\text{页面失效次数}/\text{页地址流长度}$
\end{enumerate}

\section{实验要求}
\begin{enumerate}[label={(\arabic*)}]
\item 将FIFO或者LRU算法的源程序及程序执行结果写入实验报告;
\item 将FIFO和LRU算法的工作机理写入实验报告。
\end{enumerate}

% \fi

\section{实验步骤}

\begin{enumerate}

\item %以下是{\tt priority.c}的源代码,注释已详细给出:

% {\small\linespread{1}\lstinputlisting[caption={\tt parent\_child.c}]{exp3/priority.c}}

% 我们为该程序准备了一个输入文件:

% \VerbatimInput[frame=lines]{exp3/pros.in}

% 编译并执行该程序:

% \begin{Verbatim}[frame=single]
% $ cc -Wall priority.c -o priority
% $ ./priority pros.in > 1
% \end{Verbatim}

% 得到输出结果如下,可以看到这个模拟程序按照既定的规则,共执行了20个时间片。

% \VerbatimInput[fontsize=\small,frame=lines]{exp3/1}

% \item 该算法即{\bf 高优先权优先调度算法},每次执行一次排序,并执行优先级最高的可执行的任务,直到执行完毕或进入阻塞。这种方法要求给出进程的优先级,调度程序动态调整其优先级,按照其“重要程度”顺序执行任务。适用于实时系统。

% 而{\bf 高响应比优先调度算法}的基本思想是把CPU分配给就绪队列中响应比(作业响应时间与作业执行时间的比值)最高的进程。这种方法兼顾了短作业与先后次序,且不会使长作业长期得不到服务。但是响应比计算用到了除法,增加了系统开销,所以更适合于批处理系统。

\end{enumerate}

\end{document}
