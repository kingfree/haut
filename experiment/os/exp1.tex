\documentclass[cs4size,a4paper,nofonts]{ctexart}
\usepackage[utf8]{inputenc}
\def\tjf{{\tt{田劲锋}}}
\def\titlec{Linux系统基本操作命令}

\usepackage[a4paper,margin=2.2cm]{geometry} % 页面设置
\usepackage[unicode,breaklinks=true,
colorlinks=true,linkcolor=black,anchorcolor=black,citecolor=black,urlcolor=black,
pdftitle={\titlec},pdfauthor={\tjf}]{hyperref}
\usepackage{latexsym,amsmath,amssymb,bm}
\usepackage{graphicx}
\usepackage{fancyvrb}
\DefineShortVerb{\|}
\fvset{frame=single}

\setmainfont{Times New Roman}
\setCJKmainfont[BoldFont={SimHei}]{SimSun}  % 主要字体:宋体、黑体
\setCJKsansfont[BoldFont={STZhongsong}]{STFangsong} % 次要字体:仿宋、中宋
\setCJKmonofont{KFKai} % 等宽字体:楷体

\CJKsetecglue{\hspace{0.1em}}
\renewcommand\CJKglue{\hskip -0.3pt plus 0.08\baselineskip}
\frenchspacing
\widowpenalty=10000
\linespread{1.2} % 行距

\CTEXsetup[name={实验,},number={\arabic{part}}]{part}
\CTEXsetup[name={,.},format={\large}]{section}

\usepackage[inline]{enumitem} % 调整列表样式
\setlist{noitemsep}
\setlist[itemize]{topsep=0pt,partopsep=0pt,itemsep=0pt,parsep=0pt}
\setlist[enumerate]{topsep=0pt,partopsep=0pt,itemsep=0pt,parsep=0pt}
\setlist[enumerate,1]{label={\arabic*.}}
\setlist[enumerate,2]{label={(\arabic*)}}

%\makeindex
\pagestyle{plain}

\newcommand{\image}[3][height=10cm]{%\begin{minipage}[t]{0.5\textwidth}
    \centering
    \includegraphics[#1]{#2}
    \caption{#3}
    \label{fig:#3}
%\end{minipage}
}


\begin{document}

%%%% 开始 %%%%

\setcounter{part}{0}
\begin{titlepage}

\begin{center}

\includegraphics[height=1cm]{image/haut.png}

\vspace*{1cm}
{\liti\fontsize{48pt}{50pt}{课\quad 程\quad 设\quad 计}}

\vspace*{4cm}
{\fontsize{36}{50}\sf\bfseries \titlec}

\vspace*{1cm}
{\huge\sf\bfseries \titlee}

\vfill
{\large
\newcommand{\ctline}[2]{\makebox[6em][s]{\bf #1}:\underline{\makebox[14em][c]{\qquad #2\qquad}}\\}
\ctline{课程设计名称}{数据结构课程设计}
\ctline{专业班级}{计算机 1303 班}
\ctline{学生姓名}{\tjf}
\ctline{学号}{201316920311}
\ctline{指导教师}{白\quad 浩}
\ctline{课程设计时间}{\today}
}

\end{center}

\end{titlepage}


\iffalse
\section{实验目的}
\begin{enumerate}
\item 熟悉Linux操作系统的基本操作命令;
\item 通过在Linux环境下对进程的基本操作,认识进程并区分与程序的区别。
\end{enumerate}

\section{实验环境}
一台装有Linux操作系统(Fedora 7),至少具有256M内存的微机。

\section{预备知识}
\begin{enumerate}
\item vi编辑器的使用
\item 系统管理命令\\
login, logout, man
\item 文件操作命令\\
ls, cat, cp, mv , rm,
\item 目录操作命令\\
cd, mkdir, rmdir, pwd
\item 进程操作命令\\
ps, kill, pstree
\end{enumerate}
\fi

\section{实验内容}
\begin{enumerate}[label={(\arabic*)}]
\item 用vi编辑器建立一个文件,用文件操作命令对该文件进行操作;
\item 使用目录操作命令进行创建目录、改变当前目录、删除目录、查看当前所在目录等操作;
\item 使用进程操作命令查看系统中的进程状态,杀死进程等;
\item 使用man命令了解各个命令的使用方法及参数。
\end{enumerate}

\section{实验要求}
\begin{enumerate}
\item 将各个命令行使用规格写入实验报告;
\item 将各命令操作过程及结果写入实验报告。
\end{enumerate}

\section{实验步骤}

\begin{enumerate}[label={(\arabic*)}]

\item 这里我使用Vim而不是|vi|,在终端中键入|vim|以启动:

\begin{Verbatim}
$ vim ~/.vimrc
\end{Verbatim}

\begin{figure}[htp]
\image{exp1/1-vi.png}{Vim编辑器}
\end{figure}

如图\ref{fig:Vim编辑器},这里编辑的是Vim的配置文件。

关于Vim的用法,|vimtutor|教程是个不错的入门手册,进阶也可参考Vim自带的文档(有中文版),剩下的就是熟能生巧了。

\item 如图\ref{fig:目录操作},使用如下命令建立、进入、删除了一个名为test的目录:

\begin{Verbatim}
$ ls
$ mkdir test
$ cd test
$ cd ..
$ rmdir test
$ ls
\end{Verbatim}

\begin{figure}[htp]
\image{exp1/2-dir.png}{目录操作}
\end{figure}

\item 如图\ref{fig:进程操作},

\begin{figure}[htp]
\image{exp1/3-ps.png}{进程操作}
\end{figure}

\item 如图\ref{fig:man命令},

\begin{figure}[htp]
\image{exp1/4-man.png}{man命令}
\end{figure}

\end{enumerate}

\end{document}
