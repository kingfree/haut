\documentclass[c5size,a4paper,nofonts]{ctexart}
\usepackage[utf8]{inputenc}
\def\tjf{{\tt{田劲锋}}}
\def\titlec{高(动态)优先权优先的进程调度算法模拟}

\usepackage[a4paper,margin=2.2cm]{geometry} % 页面设置
\usepackage[unicode,breaklinks=true,
colorlinks=true,linkcolor=black,anchorcolor=black,citecolor=black,urlcolor=black,
pdftitle={\titlec},pdfauthor={\tjf}]{hyperref}
\usepackage{latexsym,amsmath,amssymb,bm}
\usepackage{graphicx}
\usepackage{fancyvrb}
\DefineShortVerb{\|}
\fvset{frame=single}

\setmainfont{Times New Roman}
\setCJKmainfont[BoldFont={SimHei}]{SimSun}  % 主要字体:宋体、黑体
\setCJKsansfont[BoldFont={STZhongsong}]{STFangsong} % 次要字体:仿宋、中宋
\setCJKmonofont{KFKai} % 等宽字体:楷体

\CJKsetecglue{\hspace{0.1em}}
\renewcommand\CJKglue{\hskip -0.3pt plus 0.08\baselineskip}
\frenchspacing
\widowpenalty=10000
\linespread{1.2} % 行距

\CTEXsetup[name={实验,},number={\arabic{part}}]{part}
\CTEXsetup[name={,.},format={\large}]{section}

\usepackage[inline]{enumitem} % 调整列表样式
\setlist{noitemsep}
\setlist[itemize]{topsep=0pt,partopsep=0pt,itemsep=0pt,parsep=0pt}
\setlist[enumerate]{topsep=0pt,partopsep=0pt,itemsep=0pt,parsep=0pt}
\setlist[enumerate,1]{label={\arabic*.}}
\setlist[enumerate,2]{label={(\arabic*)}}

%\makeindex
\pagestyle{plain}

\newcommand{\image}[3][height=10cm]{%\begin{minipage}[t]{0.5\textwidth}
    \centering
    \includegraphics[#1]{#2}
    \caption{#3}
    \label{fig:#3}
%\end{minipage}
}

\linespread{1.1} % 行距
\fvset{frame=none}

\begin{document}

%%%% 开始 %%%%

\setcounter{part}{2}
\begin{titlepage}

\begin{center}

\includegraphics[height=1cm]{image/haut.png}

\vspace*{1cm}
{\liti\fontsize{48pt}{50pt}{课\quad 程\quad 设\quad 计}}

\vspace*{4cm}
{\fontsize{36}{50}\sf\bfseries \titlec}

\vspace*{1cm}
{\huge\sf\bfseries \titlee}

\vfill
{\large
\newcommand{\ctline}[2]{\makebox[6em][s]{\bf #1}:\underline{\makebox[14em][c]{\qquad #2\qquad}}\\}
\ctline{课程设计名称}{数据结构课程设计}
\ctline{专业班级}{计算机 1303 班}
\ctline{学生姓名}{\tjf}
\ctline{学号}{201316920311}
\ctline{指导教师}{白\quad 浩}
\ctline{课程设计时间}{\today}
}

\end{center}

\end{titlepage}


\iffalse
\section{实验目的}
通过动态优先权算法的模拟加深对进程概念和进程调度过程的理解。

\section{实验环境}
装有操作系统Windows XP和开发工具VC++6.0,内存在256M以上的微机;

或者:装有Linux(Fedora 7)操作系统和gcc编译器,内存在256M以上的微机。

\section{实验内容}
\begin{enumerate}[label={(\arabic*)}]
\item 用C语言来实现对N个进程采用动态优先权优先算法的进程调度。
\item 每个用来标识进程的进程控制块PCB用结构来描述,包括以下字段:
\begin{itemize}
\item 进程标识数ID;
\item 进程优先数PRIORITY,并规定优先数越大的进程,其优先权越高;
\item 进程已占用的CPU时间CPUTIME;
\item 进程还需占用的CPU时间NEEDTIME。当进程运行完毕时,NEEDTIME变为0;
\item 进程的阻塞时间STARTBLOCK,表示当进程再运行STARTBLOCK个时间片后,进程将进入阻塞状态;
\item 进程被阻塞的时间BLOCKTIME,表示已阻塞的进程再等待BLOCKTIME个时间片后,进程将转换成就绪状态;
\item 进程状态STATE;(READY, RUNNING, BLOCK, FINISH)
\item 队列指针NEXT,用来将PCB排成队列。
\end{itemize}
\item 优先数改变的原则:
\begin{itemize}
\item 进程在就绪队列中呆一个时间片,优先数增加1;
\item 进程每运行一个时间片,优先数减3。
\end{itemize}
\item 假设在调度前,系统中有5个进程,它们的初始状态如下:
\begin{Verbatim}
ID              0       1       2       3       4
PRIORITY        9      38      30      29       0
CPUTIME         0       0       0       0       0
NEEDTIME        3       3       6       3       4
STARTBLOCK      2      -1      -1      -1      -1
BLOCKTIME       3       0       0       0       0
STATE       READY   READY   READY   READY   READY
\end{Verbatim}
\item 为了清楚地观察进程的调度过程,程序应将每个时间片内的进程的情况显示出来,参照的具体格式如下:
\begin{Verbatim}
            RUNNING PROCESS: $id0
            READY QUEUE:  $id1->$id2
            BLOCK QUEUE:  $id3->$id4
    FINISH QUEUE:  $id0->$id1->$id2->$id3->$id4
====================================================================
ID    PRIORITY  CPUTIME   NEEDTIME   STATE   STARTBLOCK   BLOCKTIME
0       XX        XX       XX         XX       XX          XX
1       XX        XX       XX         XX       XX          XX
2       XX        XX       XX         XX       XX          XX
3       XX        XX       XX         XX       XX          XX
4       XX        XX       XX         XX       XX          XX   
====================================================================
\end{Verbatim}

\end{enumerate}

\section{实验要求}
\begin{enumerate}[label={(\arabic*)}]
\item 将源程序(|priority.c|)和程序运行结果写入实验报告。
\item 将该算法执行过程与高响应比优先调度算法的执行过程进行比较。
\end{enumerate}

\fi

\section{实验步骤}

\begin{enumerate}

\item 以下是{\tt priority.c}的源代码,注释已详细给出:

{\small\linespread{1}\lstinputlisting[caption={\tt parent\_child.c}]{exp3/priority.c}}

我们为该程序准备了一个输入文件:

\VerbatimInput[frame=lines]{exp3/pros.in}

编译并执行该程序:

\begin{Verbatim}[frame=single]
$ cc -Wall priority.c -o priority
$ ./priority pros.in > 1
\end{Verbatim}

得到输出结果如下,可以看到这个模拟程序按照既定的规则,共执行了19个时间片。

\VerbatimInput[fontsize=\small,frame=lines]{exp3/1}

\item 该算法即{\bf 高优先权优先调度算法},每次执行一次排序,并执行优先级最高的可执行的任务,直到执行完毕或进入阻塞。这种方法要求给出进程的优先级,调度程序动态调整其优先级,按照其“重要程度”顺序执行任务。适用于实时系统。

而{\bf 高响应比优先调度算法}的基本思想是把CPU分配给就绪队列中响应比(作业响应时间与作业执行时间的比值)最高的进程。这种方法兼顾了短作业与先后次序,且不会使长作业长期得不到服务。但是响应比计算用到了除法,增加了系统开销,所以更适合于批处理系统。

\end{enumerate}

\end{document}
