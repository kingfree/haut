\documentclass[cs4size,a4paper,nofonts]{ctexart}
\usepackage[utf8]{inputenc}
\def\tjf{{\tt{田劲锋}}}
\def\titlec{Linux进程控制与通信}
\usepackage[a4paper,margin=2.2cm]{geometry} % 页面设置
\usepackage[unicode,breaklinks=true,
colorlinks=true,linkcolor=black,anchorcolor=black,citecolor=black,urlcolor=black,
pdftitle={\titlec},pdfauthor={\tjf}]{hyperref}
\usepackage{latexsym,amsmath,amssymb,bm}
\usepackage{graphicx}
\usepackage{fancyvrb}
\DefineShortVerb{\|}
\fvset{frame=single}

\setmainfont{Times New Roman}
\setCJKmainfont[BoldFont={SimHei}]{SimSun}  % 主要字体:宋体、黑体
\setCJKsansfont[BoldFont={STZhongsong}]{STFangsong} % 次要字体:仿宋、中宋
\setCJKmonofont{KFKai} % 等宽字体:楷体

\CJKsetecglue{\hspace{0.1em}}
\renewcommand\CJKglue{\hskip -0.3pt plus 0.08\baselineskip}
\frenchspacing
\widowpenalty=10000
\linespread{1.2} % 行距

\CTEXsetup[name={实验,},number={\arabic{part}}]{part}
\CTEXsetup[name={,.},format={\large}]{section}

\usepackage[inline]{enumitem} % 调整列表样式
\setlist{noitemsep}
\setlist[itemize]{topsep=0pt,partopsep=0pt,itemsep=0pt,parsep=0pt}
\setlist[enumerate]{topsep=0pt,partopsep=0pt,itemsep=0pt,parsep=0pt}
\setlist[enumerate,1]{label={\arabic*.}}
\setlist[enumerate,2]{label={(\arabic*)}}

%\makeindex
\pagestyle{plain}

\begin{document}

%%%% 开始 %%%%

\setcounter{part}{1}
\part{\titlec}

\section{实验目的}
\begin{enumerate}
\item 进一步认识并发执行的概念,认识父子进程及进程创建原理;
\item 了解Linux系统中进程通信的基本原理。
\end{enumerate}

\section{实验环境}
一台装有Linux操作系统(Fedora 7),至少具有256M内存的微机。

\section{预备知识}
\begin{enumerate}
\item gcc编译器的使用
\item fork系统调用:创建一个新进程
\item getpid系统调用:获得一个进程的pid
\item wait系统调用:发出调用的进程等待子进程结束
\item pipe系统调用:建立管道
\item write系统调用:向文件中写数据
\item read系统调用:从文件中读数据
\end{enumerate}

\section{实验内容}
\begin{enumerate}[label={(\arabic*)}]
\item 编写一段程序(程序名为|parent_child.c|),使用系统调用fork()创建两个子进程,如果是父进程显示“Parent Process: A”,子进程分别显示“This is child1 (pid1 =xxxx )process: B”和“This is child1 (pid1 =xxxx )process: C”,其中“xxxx”分别指明子进程的pid号。
\item 编写一段程序(程序名为|comm.c|),父子进程之间建立一条管道,子进程向管道中写入“Child process 1 is sending a message!”,父进程从管道中读出数据,显示在屏幕上。
\end{enumerate}

\section{实验要求}
\begin{enumerate}
\item 将parentchild.c源程序,及程序执行结果写入实验报告;
\item 将fork()系统调用后内核的工作原理写入实验报告;
\item 将comm.c源程序,及程序执行结果写入实验报告;
\item 将Linux系统中管道通信的工作原理写入实验报告。
\end{enumerate}

\end{document}
