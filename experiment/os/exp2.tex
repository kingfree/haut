\documentclass[cs4size,a4paper,nofonts]{ctexart}
\usepackage[utf8]{inputenc}
\def\tjf{{\tt{田劲锋}}}
\def\titlec{Linux进程控制与通信}

\usepackage[a4paper,margin=2.2cm]{geometry} % 页面设置
\usepackage[unicode,breaklinks=true,
colorlinks=true,linkcolor=black,anchorcolor=black,citecolor=black,urlcolor=black,
pdftitle={\titlec},pdfauthor={\tjf}]{hyperref}
\usepackage{latexsym,amsmath,amssymb,bm}
\usepackage{graphicx}
\usepackage{fancyvrb}
\DefineShortVerb{\|}
\fvset{frame=single}

\setmainfont{Times New Roman}
\setCJKmainfont[BoldFont={SimHei}]{SimSun}  % 主要字体:宋体、黑体
\setCJKsansfont[BoldFont={STZhongsong}]{STFangsong} % 次要字体:仿宋、中宋
\setCJKmonofont{KFKai} % 等宽字体:楷体

\CJKsetecglue{\hspace{0.1em}}
\renewcommand\CJKglue{\hskip -0.3pt plus 0.08\baselineskip}
\frenchspacing
\widowpenalty=10000
\linespread{1.2} % 行距

\CTEXsetup[name={实验,},number={\arabic{part}}]{part}
\CTEXsetup[name={,.},format={\large}]{section}

\usepackage[inline]{enumitem} % 调整列表样式
\setlist{noitemsep}
\setlist[itemize]{topsep=0pt,partopsep=0pt,itemsep=0pt,parsep=0pt}
\setlist[enumerate]{topsep=0pt,partopsep=0pt,itemsep=0pt,parsep=0pt}
\setlist[enumerate,1]{label={\arabic*.}}
\setlist[enumerate,2]{label={(\arabic*)}}

%\makeindex
\pagestyle{plain}

\newcommand{\image}[3][height=10cm]{%\begin{minipage}[t]{0.5\textwidth}
    \centering
    \includegraphics[#1]{#2}
    \caption{#3}
    \label{fig:#3}
%\end{minipage}
}


\begin{document}

%%%% 开始 %%%%

\setcounter{part}{1}
\part{\titlec}

\iffalse
\section{实验目的}
\begin{enumerate}
\item 进一步认识并发执行的概念,认识父子进程及进程创建原理;
\item 了解Linux系统中进程通信的基本原理。
\end{enumerate}

\section{实验环境}
一台装有Linux操作系统(Fedora 7),至少具有256M内存的微机。

\section{预备知识}
\begin{enumerate}
\item gcc编译器的使用
\item fork系统调用:创建一个新进程
\item getpid系统调用:获得一个进程的pid
\item wait系统调用:发出调用的进程等待子进程结束
\item pipe系统调用:建立管道
\item write系统调用:向文件中写数据
\item read系统调用:从文件中读数据
\end{enumerate}
\fi

\section{实验内容}
\begin{enumerate}[label={(\arabic*)}]
\item 编写一段程序(程序名为|parent_child.c|),使用系统调用fork()创建两个子进程,如果是父进程显示“Parent Process: A”,子进程分别显示“This is child1 (pid1 =xxxx )process: B”和“This is child1 (pid1 =xxxx )process: C”,其中“xxxx”分别指明子进程的pid号。
\item 编写一段程序(程序名为|comm.c|),父子进程之间建立一条管道,子进程向管道中写入“Child process 1 is sending a message!”,父进程从管道中读出数据,显示在屏幕上。
\end{enumerate}

\section{实验要求}
\begin{enumerate}
\item 将parentchild.c源程序,及程序执行结果写入实验报告;
\item 将fork()系统调用后内核的工作原理写入实验报告;
\item 将comm.c源程序,及程序执行结果写入实验报告;
\item 将Linux系统中管道通信的工作原理写入实验报告。
\end{enumerate}

\section{实验步骤}

\begin{enumerate}[label={(\arabic*)}]

\item 以下是{\tt parent\_child.c}的源代码。

\lstinputlisting[caption={\tt parent\_child.c}]{exp2/parent_child.c}

首先要知道|fork()|函数定义在|<unistd.h>|中。在主函数中,9--11行尝试创建一个子进程1并将其进程ID赋值给|pid1|。12--14行判断程序在执行子进程1,打印出相应信息;注意|getpid()|会返回一个|pid_t|类型的值,应将其转换为|long|输出。15--24是父进程,其中18--20行创建子进程2,21--24行判断是子进程2打印出相应信息。
编译运行:

\begin{Verbatim}
$ cc parent_child.c -o parent_child
$ ./parent_child
Parent Process: A
This is child1 (pid1=13221) process: B
This is child2 (pid2=13222) process: C
\end{Verbatim}

这里使用了一个小技巧,在创建了第一个子进程后,应该在父进程中再建立另外一个子进程,所以使用了嵌套的if语句。
% 如果并行的话,实际上会创建出来3个子进程,而不是2个,这不是我们所期望的。
另外一种可行的方式是,使用|wait()|来等待子进程1结束后再|fork()|出来子进程2。

图\ref{fig:实际执行过程}是编译和执行该程序以及下面程序的过程。

\begin{figure}[htp]
\image{exp2/progs.png}{实际执行过程}
\end{figure}

\item 由|fork()|创建的是子进程。它被调用一次,返回两次。区别在于,子进程返回值是0,而父进程的返回值则是新建子进程的进程ID。子进程和父进程继续执行|fork()|调用之后的指令。子进程是父进程的副本,获得父进程的数据空间、堆和栈的副本。然而父进程和子进程并不共享这些存储部分,它们之间共享的是正文段。在实际实现中,操作系统通常并不完全复制其数据段和堆栈,而是采用了写时复制的技术来提高效率。

\item 以下是{\tt comm.c}的源代码。

\lstinputlisting[caption={\tt comm.c}]{exp2/comm.c}

|pipe()|也是在|<unistd.h>|中定义的。程序中,第6行定义了一个常量,表示一个文本行的最大长度。16-19行创建了一个管道,这个管道是从|fd[0]|读入,向|fd[1]|写出的。21-23行创建了一个子进程。24-26行,子进程关闭{\bf 读}描述符,向{\bf 写}描述符写出指定字符串,注意|write()|函数要求传入字符串的长度。27-31行是父进程,父进程关闭{\bf 写}描述符,从{\bf 读}描述符读取|n|个字节,写出到标准输出。这里使用了常量|STDOUT_FILENO|表示标准输出的文件号,默认是1。
编译运行:

\begin{Verbatim}
$ cc comm.c -o comm
$ ./comm
Child process 1 is sending a message!
\end{Verbatim}

这个程序同时用到了|read()|和|write()|用来读写文件描述符。

\item 	

\end{enumerate}

\end{document}
