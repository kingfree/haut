\documentclass[cs4size,a4paper,nofonts]{ctexart}
\usepackage[utf8]{inputenc}
\def\tjf{{\tt{田劲锋}}}
\def\titlec{shell编程}
\usepackage[a4paper,margin=2.2cm]{geometry} % 页面设置
\usepackage[unicode,breaklinks=true,
colorlinks=true,linkcolor=blue,anchorcolor=black,citecolor=black,urlcolor=blue,
pdftitle={\titlec},pdfauthor={\tjf}]{hyperref}
%\CTEXsetup[number=\chinese{section}, format={\large\sf\bfseries}]{section}
\usepackage{latexsym,amsmath,amssymb,bm}
\usepackage{graphicx}
\usepackage{xcolor}
\usepackage{subfigure}
\usepackage{wrapfig}
\usepackage{fancyvrb}
% \DefineShortVerb{\|}
\fvset{frame=single}

\setmainfont{Times New Roman}
\setCJKmainfont[BoldFont={SimHei}]{SimSun}  % 主要字体:宋体、黑体
\setCJKsansfont[BoldFont={STZhongsong}]{STFangsong} % 次要字体:仿宋、中宋
\setCJKmonofont{KFKai} % 等宽字体:楷体

\CJKsetecglue{\hspace{0.1em}}
\renewcommand\CJKglue{\hskip -0.3pt plus 0.08\baselineskip}
\frenchspacing
\widowpenalty=10000
\linespread{1.2} % 行距

\usepackage{listings} % 排版代码
\lstset{language=bash,
  numbers=left,
  numberstyle=\small,
  basicstyle=\tt,
  commentstyle=\color{gray},
  keywordstyle=\bfseries\color{blue},
  stringstyle=\color{teal},
  showstringspaces=false,
  frame=tb,
  extendedchars=false
}

\usepackage[inline]{enumitem} % 调整列表样式
\setlist{noitemsep,align=left}
\setlist[itemize]{topsep=0pt,partopsep=0pt,itemsep=0pt,parsep=0pt}
\setlist[enumerate]{topsep=0pt,partopsep=0pt,itemsep=0pt,parsep=0pt}
% \setlist[enumerate,1]{label={(\arabic*)}}
% \setlist[enumerate,2]{label={\arabic*)}}

\CTEXsetup[beforeskip={0pt},afterskip={0pt}]{paragraph}

%\makeindex
\pagestyle{plain}

\begin{document}

%%%% 开始 %%%%

\begin{titlepage}

\begin{center}

\includegraphics[height=1cm]{image/haut.png}

\vspace*{1cm}
{\liti\fontsize{48pt}{50pt}{课\quad 程\quad 设\quad 计}}

\vspace*{4cm}
{\fontsize{36}{50}\sf\bfseries \titlec}

\vspace*{1cm}
{\huge\sf\bfseries \titlee}

\vfill
{\large
\newcommand{\ctline}[2]{\makebox[6em][s]{\bf #1}:\underline{\makebox[14em][c]{\qquad #2\qquad}}\\}
\ctline{课程设计名称}{数据结构课程设计}
\ctline{专业班级}{计算机 1303 班}
\ctline{学生姓名}{\tjf}
\ctline{学号}{201316920311}
\ctline{指导教师}{白\quad 浩}
\ctline{课程设计时间}{\today}
}

\end{center}

\end{titlepage}


% \CTEXnoindent

\paragraph{实验题目:}\titlec

\paragraph{实验目的:}
(1)理解Shell脚本的作用;(2)掌握Shell脚本的基本语法,包括变量的运算、逻辑判断表达式等;(3)掌握Shell编程的控制结构,包括分支结构、循环结构等。

\paragraph{实验内容:}
\begin{enumerate}
\item 简单的编写一个脚本,打印“hello Shell”,并运行。
\item 要求用户输入一个目录,然后进行判断此目录下的文件哪些是具有读权限的?如果是,显示该文件的详细信息。
\item 输入一个成绩值,判定该成绩的等级。
\item 通过循环依次输入5个数,求其总值,要求输入数的个数大于5个时,使用break退出循环. 
\end{enumerate}

\paragraph{实验步骤:}\quad

\newcommand{\image}[3][width=0.9\textwidth]{
    \centering
        \includegraphics[#1]{images/exp8/#2.png}
    \caption{#3}
    \label{fig:#3}
}

本次实验在Mac OS X Yosemite上运行。

脚本的运行结果如图\ref{fig:脚本运行情况}。

\begin{enumerate}

\item 

{\linespread{1}\lstinputlisting[caption={\tt hello.sh}]{exp8/hello.sh}}

\item 

{\linespread{1}\lstinputlisting[caption={\tt dir.sh}]{exp8/dir.sh}}

\item 

{\linespread{1}\lstinputlisting[caption={\tt score.sh}]{exp8/score.sh}}

\item 

{\linespread{1}\lstinputlisting[caption={\tt loop.sh}]{exp8/loop.sh}}

\begin{figure}[htp]
\image{shell}{脚本运行情况}
\end{figure}

\end{enumerate}

\paragraph{实验体会:}\quad

Shell提供了常用的逻辑处理结构,实现了顺序、循环、选择三大结构,通过与应用程序和终端用户的交互,实现对常用事务的处理。在动态编程语言还不流行的时代,Shell脚本为处理系统事务提供了相当的便利。Shell也有很多衍生版本,在语法上存在细微的差异。

\end{document}
