\documentclass[cs4size,a4paper,nofonts]{ctexart}
\usepackage[utf8]{inputenc}
\def\tjf{{\tt{田劲锋}}}
\def\titlec{Linux系统管理与维护(2):网络配置管理}
\usepackage[a4paper,margin=2.2cm]{geometry} % 页面设置
\usepackage[unicode,breaklinks=true,
colorlinks=true,linkcolor=blue,anchorcolor=black,citecolor=black,urlcolor=blue,
pdftitle={\titlec},pdfauthor={\tjf}]{hyperref}
%\CTEXsetup[number=\chinese{section}, format={\large\sf\bfseries}]{section}
\usepackage{latexsym,amsmath,amssymb,bm}
\usepackage{graphicx}
\usepackage{subfigure}
\usepackage{wrapfig}
\usepackage{fancyvrb}
\DefineShortVerb{\|}
\fvset{frame=single}

\setmainfont{Times New Roman}
\setCJKmainfont[BoldFont={SimHei}]{SimSun}  % 主要字体:宋体、黑体
\setCJKsansfont[BoldFont={STZhongsong}]{STFangsong} % 次要字体:仿宋、中宋
\setCJKmonofont{KFKai} % 等宽字体:楷体

\CJKsetecglue{\hspace{0.1em}}
\renewcommand\CJKglue{\hskip -0.3pt plus 0.08\baselineskip}
\frenchspacing
\widowpenalty=10000
\linespread{1.2} % 行距

\usepackage[inline]{enumitem} % 调整列表样式
\setlist{noitemsep,align=left}
\setlist[itemize]{topsep=0pt,partopsep=0pt,itemsep=0pt,parsep=0pt}
\setlist[enumerate]{topsep=0pt,partopsep=0pt,itemsep=0pt,parsep=0pt}
\setlist[enumerate,1]{label={(\arabic*)}}
\setlist[enumerate,2]{label={\arabic*)}}

\CTEXsetup[beforeskip={0pt},afterskip={0pt}]{paragraph}

%\makeindex
\pagestyle{plain}

\begin{document}

%%%% 开始 %%%%

\begin{titlepage}

\begin{center}

\includegraphics[height=1cm]{image/haut.png}

\vspace*{1cm}
{\liti\fontsize{48pt}{50pt}{课\quad 程\quad 设\quad 计}}

\vspace*{4cm}
{\fontsize{36}{50}\sf\bfseries \titlec}

\vspace*{1cm}
{\huge\sf\bfseries \titlee}

\vfill
{\large
\newcommand{\ctline}[2]{\makebox[6em][s]{\bf #1}:\underline{\makebox[14em][c]{\qquad #2\qquad}}\\}
\ctline{课程设计名称}{数据结构课程设计}
\ctline{专业班级}{计算机 1303 班}
\ctline{学生姓名}{\tjf}
\ctline{学号}{201316920311}
\ctline{指导教师}{白\quad 浩}
\ctline{课程设计时间}{\today}
}

\end{center}

\end{titlepage}


% \CTEXnoindent

\paragraph{实验题目:}\titlec

\paragraph{实验目的:}
(1)理解网络的基本配置和相关文件;(2)掌握一些网络服务器的安装与配置。

\paragraph{实验内容:}
\begin{enumerate}
\item 查看IP地址配置文件,并解释文件的基本内容。
\item 查看主机名配置文件。
\item 使用命令查看主机名。
\item 使用命令更改主机名。
\item 域名服务器DNS的作用是什么?查看设置域名服务器DNS的配置文件。
\item 使用命令测试域名解析,以网站\url{www.haut.edu.cn}为例。 
\item 使用命令查看一个网卡的信息。
\item 使用命令卸载网卡,通过浏览器测试能否连接网络。再使用命令挂载网卡,再通过浏览器测试能否连接网络。
\item 使用命令跟踪一个发往某个地址的数据包路径。 
\item 简述DHCP的工作原理,并阐述如何安装、启动、停止DHCP服务器。
\end{enumerate}

\paragraph{实验步骤:}

\newcommand{\image}[3][width=0.8\textwidth]{
  %\begin{minipage}[t]{0.8\textwidth}
    \centering
        \includegraphics[#1]{images/exp7/#2.png}
    \caption{#3}
    \label{fig:#3}
  %\end{minipage}
}

本次实验是通过SSH连接上一台远端Ubuntu Linux服务器来进行的。

\begin{enumerate}

\item 最常用|ifconfig|来查看和配置网络,键入:
\begin{Verbatim}
$ ifconfig lo
\end{Verbatim}
来查看本地环回的网络配置,如图\ref{fig:IP配置}。也可以查看文件|/etc/network/interfaces|,然而这里并没有什么内容。根据Linux发行版的不同,配置IP的方法也各不相同;UNIX的不同实现更是五花八门。

\begin{figure}[htp]
\image{ip}{IP配置}
\end{figure}

\item 如图\ref{fig:主机名},键入:
\begin{Verbatim}
$ cat /etc/hostname
\end{Verbatim}
查看主机名配置文件。

\begin{figure}[htp]
\image{hostname}{主机名}
\end{figure}

\item 如图\ref{fig:主机名},键入:
\begin{Verbatim}
$ hostname
\end{Verbatim}
查看主机名。

\item 如图\ref{fig:主机名},键入:
\begin{Verbatim}
$ sudo hostname <新主机名>
\end{Verbatim}
更改主机名。这里我将主机名更改之后又改回去原来的了。

\item 域名系统(Domain Name System,DNS)简单来说就是将人类可读的域名解析为相应IP地址的系统。如图\ref{fig:DNS配置},键入:
\begin{Verbatim}
$ cat /etc/resolv.conf
\end{Verbatim}
来查看本机DNS服务器的配置。

\begin{figure}[htp]
\image{dns}{DNS配置}
\end{figure}

\item 如图\ref{fig:DNS配置},键入:
\begin{Verbatim}
$ nslookup www.haut.edu.cn
\end{Verbatim}
测试\url{www.haut.edu.cn}的域名解析,可以看到解析到的地址为123.15.36.150。 

\item 如图\ref{fig:IP配置},键入:
\begin{Verbatim}
$ ifconfig lo
\end{Verbatim}
来查看本地环回的网络配置。在|ifconfig|后面跟上诸如|eth0|可以查看相应网卡或虚拟网卡的配置信息。

\item 使用命令
\begin{Verbatim}
# ifdown <网卡>
\end{Verbatim}
卸载网卡。因为执行该操作会使远程服务器断开网络,造成服务中断,所以不再实验。使用命令
\begin{Verbatim}
# ifup <网卡>
\end{Verbatim}
可以挂载网卡。两个操作都需要超级用户权限。

\begin{figure}[htp]
\image{trace}{追踪}
\end{figure}

\item 如图\ref{fig:追踪},使用
\begin{Verbatim}
$ traceroute <IP地址>
\end{Verbatim}
可以追踪相关数据包。这里我追踪了本地环回的数据包和本机。追踪连接到该服务器的本机IP地址,可以看到,只有当我键入回车时,服务器才会收到数据包并处理,按|CTRL-C|终止追踪。

\item 动态主机设定协定(Dynamic Host Configuration Protocol,DHCP)用来动态分配IP网络地址。
DHCP工作过程分为四个步骤:
\begin{enumerate}
\item[第一步:]客户端发送广播查找可以给自己提供IP地址的DHCP服务器,
\item[第二步:]DHCP服务器发送广播提供一个可用的IP地址,并在地址池中将该地址打上标记,以防再次分配。
\item[第三步:]客户端收到广播后再次发送一个广播请求该地址,如果有多台DHCP响应,则第一个收到的优先。
\item[第四步:]DHCP服务器收到广播后再发送一个广播,确认该地址分配给这台主机使用,并在地址池中将该地址打上标记,以防再次分配。
\end{enumerate}

通过编辑|/etc/dhcpd.conf|可以配置DHCP服务器,|/etc/init.d/dhcpd|是服务项,可以用来启动和关闭服务。

\end{enumerate}

\paragraph{实验体会:}\quad

网络配置管理是一个非常复杂的系统工程,这是由于网络的先天复杂性所造成的。然而这又是一项很有趣的工作,和Linux服务器打交道的人不免要接触到各种各样的网络工具,而这些各种各样的工具,确实也为我们提供了不少方便。

\end{document}
