\documentclass[cs4size,a4paper,nofonts]{ctexart}
\usepackage[utf8]{inputenc}
\def\tjf{{\tt{田劲锋}}}
\def\titlec{Linux常用工具的使用(1):Linux文件系统的挂载与卸载工具}
\usepackage[a4paper,margin=2.2cm]{geometry} % 页面设置
\usepackage[unicode,breaklinks=true,
colorlinks=true,linkcolor=black,anchorcolor=black,citecolor=black,urlcolor=black,
pdftitle={\titlec},pdfauthor={\tjf}]{hyperref}
%\CTEXsetup[number=\chinese{section}, format={\large\sf\bfseries}]{section}
\usepackage{latexsym,amsmath,amssymb,bm}
\usepackage{graphicx}
\usepackage{subfigure}
\usepackage{wrapfig}
\usepackage{fancyvrb}
\DefineShortVerb{\|}
\fvset{frame=single}

\setmainfont{Times New Roman}
\setCJKmainfont[BoldFont={SimHei}]{SimSun}  % 主要字体:宋体、黑体
\setCJKsansfont[BoldFont={STZhongsong}]{STFangsong} % 次要字体:仿宋、中宋
\setCJKmonofont{KFKai} % 等宽字体:楷体

\CJKsetecglue{\hspace{0.1em}}
\renewcommand\CJKglue{\hskip -0.3pt plus 0.08\baselineskip}
\frenchspacing
\widowpenalty=10000
\linespread{1.2} % 行距

\usepackage[inline]{enumitem} % 调整列表样式
\setlist{noitemsep,align=left}
%\setlist[itemize]{topsep=0pt,partopsep=0pt,itemsep=0pt,parsep=0pt}
\setlist[enumerate]{topsep=0pt,partopsep=0pt,itemsep=0pt,parsep=0pt}
\setlist[enumerate,1]{label={(\arabic*)}}

\CTEXsetup[beforeskip={0pt},afterskip={0pt}]{paragraph}

%\makeindex
\pagestyle{plain}

\begin{document}

%%%% 开始 %%%%

\begin{titlepage}

\begin{center}

\includegraphics[height=1cm]{image/haut.png}

\vspace*{1cm}
{\liti\fontsize{48pt}{50pt}{课\quad 程\quad 设\quad 计}}

\vspace*{4cm}
{\fontsize{36}{50}\sf\bfseries \titlec}

\vspace*{1cm}
{\huge\sf\bfseries \titlee}

\vfill
{\large
\newcommand{\ctline}[2]{\makebox[6em][s]{\bf #1}:\underline{\makebox[14em][c]{\qquad #2\qquad}}\\}
\ctline{课程设计名称}{数据结构课程设计}
\ctline{专业班级}{计算机 1303 班}
\ctline{学生姓名}{\tjf}
\ctline{学号}{201316920311}
\ctline{指导教师}{白\quad 浩}
\ctline{课程设计时间}{\today}
}

\end{center}

\end{titlepage}


% \CTEXnoindent

\paragraph{实验题目:}\titlec

\paragraph{实验目的:}(1)理解Ubuntu系统的文件系统组成;(2)掌握挂载和卸载文件系统的方法

\paragraph{实验内容:}
\begin{enumerate}
\item 查看Ubuntu系统的目录结构,并写出每个目录的作用。
\item 在虚拟机上添加一块未分区的硬盘,创建文件系统,并挂载该文件系统,访问该分区,创建一个目录和文件,然后卸载该文件系统,再次访问刚刚创建的目录和文件,看是否能够被访问。
\end{enumerate}

\paragraph{实验步骤:}\quad

\newcommand{\image}[3][height=10cm]{%\begin{minipage}[t]{0.5\textwidth}
    \centering
    \includegraphics[#1]{images/exp4/#2.png}
    \caption{#3}
    \label{fig:#3}
%\end{minipage}
}

这次的实验我在Mac OS X上的Parallels Desktop中建立了一个虚拟机,并挂载上了Arch Linux的安装光盘镜像文件,以进行操作。

\begin{enumerate}

\item 不同的UNIX/Linux操作系统有着不同的根目录结构,但基本上还是一致的。键入:
\begin{Verbatim}
# ls -lh /
\end{Verbatim}
来查看根目录结构,如图\ref{fig:目录结构}。

\begin{figure}[htp]
\image{rootdir}{目录结构}{}
\end{figure}

\begin{itemize}

\item[\tt /bin/]   二进制可执行文件。
\item[\tt /boot/]  引导程序文件。
\item[\tt /dev/]   必要设备。
\item[\tt /etc/]   特定主机,系统范围内的配置文件。
% \begin{itemize}
% \item[\tt /etc/opt/]
% /opt/的配置文件
% \item[\tt /etc/X11/]
% X Window系统(版本11)的配置文件
% \item[\tt /etc/sgml/]
% SGML的配置文件
% \item[\tt /etc/xml/]
% XML的配置文件
% \end{itemize}

\item[\tt /home/]  用户的家目录。
\item[\tt /lib/]   必要的库文件。
\item[\tt /lib64/]   必要的库文件(64位平台)。
\item[\tt /lost+found/] 恢复文件。
\item[\tt /mnt/]   临时挂载的文件系统。
\item[\tt /opt/]   可选应用软件包。
\item[\tt /proc/]  虚拟文件系统,将内核与进程状态归档为文本文件。
\item[\tt /root/]  超级用户的家目录
\item[\tt /run]
代替|/var/run|目录。自最后一次启动以来运行中的系统的信息。
\item[\tt /sbin/]  必要的系统二进制文件。
\item[\tt /srv/]   服务器站点的具体数据,由系统提供。
\item[\tt /sys/]   系统链接。
\item[\tt /tmp/]   临时文件,在系统重启时目录中文件不会被保留。
\item[\tt /usr/]   用于存储只读用户数据的第二层次; 包含绝大多数的用户工具和应用程序。
\begin{itemize}
\item[\tt /usr/bin/]
非必要可执行文件 (在单用户模式中不需要);面向所有用户。
\item[\tt /usr/include/]
标准包含文件,即C语言头文件。
\item[\tt /usr/lib/]
|/usr/bin/|和|/usr/sbin/|中二进制文件的库。
\item[\tt /usr/sbin/]
非必要的系统二进制文件。
% \item[\tt /usr/share/]
% 体系结构无关(共享)数据。
% \item[\tt /usr/src/]
% 源代码,例如:内核源代码及其头文件。
% \item[\tt /usr/X11R6/]
% X Window系统 版本 11, Release 6
% \item[\tt /usr/local/]
% 本地数据的第三层次, 具体到本台主机。通常而言有进一步的子目录.
\end{itemize}

\item[\tt /var/]   变量文件——在正常运行的系统中其内容不断变化的文件。
% \begin{itemize}
% \item[\tt /var/cache/]
% 应用程序缓存数据。这些数据是在本地生成的一个耗时的I/O或计算结果。应用程序必须能够再生或恢复数据。缓存的文件可以被删除而不导致数据丢失。
% \item[\tt /var/lib/]
% 状态信息。 由程序在运行时维护的持久性数据。 例如:数据库、包装的系统元数据等。
% \item[\tt /var/lock/]
% 锁文件,一类跟踪当前使用中资源的文件。
% \item[\tt /var/log/]
% 日志文件,包含大量日志文件。
% \item[\tt /var/mail/]
% 用户的电子邮箱。
% \item[\tt /var/run/]
% 自最后一次启动以来运行中的系统的信息,例如:当前登录的用户和运行中的守护进程。现已经被/run代替[13]。
% \item[\tt /var/spool/]
% 等待处理的任务的脱机文件,例如:打印队列和未读的邮件。
% \item[\tt /var/tmp/]
% 在系统重启过程中可以保留的临时文件。
% \end{itemize}

\end{itemize}


\item 显示磁盘信息:
\begin{Verbatim}
# lsblk
\end{Verbatim}
如图\ref{fig:创建文件系统},这里我们看到有一未挂载的本地磁盘sda。

\begin{figure}[htp]
\image{mkfs}{创建文件系统}
\end{figure}

在|/dev/sda|上建立一个EXT4格式的文件系统:
\begin{Verbatim}
# mkfs.ext4 /dev/sda
\end{Verbatim}


然后如图\ref{fig:挂载文件系统},将其挂载到|/mnt|中:
\begin{Verbatim}
# mount /dev/sda /mnt
\end{Verbatim}

创建一个文件和目录:
\begin{Verbatim}
# cd /mnt
# touch file
# mkdir dir
# ls
\end{Verbatim}

\begin{figure}[htp]
\image{mount}{挂载文件系统}
\end{figure}

最后卸载之:
\begin{Verbatim}
# umount /mnt
\end{Verbatim}

如图\ref{fig:卸载文件系统},测试文件是否还能访问:
\begin{Verbatim}
# ls /mnt
\end{Verbatim}
当然结论是不能访问了(笑)。

\begin{figure}[htp]
\image{umount}{卸载文件系统}
\end{figure}

\end{enumerate}

\paragraph{实验体会:}\quad

Linux的目录结构继承UNIX,是经历实践检验有效的分类方式。

关于Linux文件系统,主要在安装系统中会遇到,尤其是Arch这种以命令行形式安装的发行版,需要手动分区和挂载安装。这里的实验也是直接利用了Arch的系统安装盘自带的一些功能。

\end{document}
