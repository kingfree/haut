\newpage
\section*{心得体会}
\addcontentsline{toc}{section}{心得体会}

数据结构课程设计,在我看来是锻炼语言掌握熟练度和熟悉数据结构与算法的重要途径。设计要求下发时正当讲到了 Huffman 算法,考虑到在高中竞赛中并没有怎么用到过这种算法,于是便想以此作为研究课题。

首先我简单用 Ruby 实现了一个简单的程序,它可以输入一个字符串编码成 01 字串,并在当前树上对 01 字串进行译码。Ruby 作为一种面向对象的动态编程语言,语言灵活性很大,实现起来很方便,但相对地运行速度就不是很理想。于是我开始考虑用 C 来实现。在前一段时间自学了 Linux 系统编程 \cite{linuxmnl} 的基础之上,是可以实现文件的二进制操作的。事实也证明,在文件和字节操作上的代码占了很大一部分。C 作为系统级别的编程语言,在这方面的库简单粗暴,以至于不得不考虑各种细节。典型的字节顺序问题就被我碰到了,还好可以通过查阅文档解决。有些位操作的函数似乎设计的有点复杂,不过我还没想到什么更好的办法。

开发跨平台应用程序一直以来都是我的习惯。所以这次也不例外,在 Windows 和 Linux 两个平台下进行调试开发,最终的成果也可以很好地在两个平台上运行。当然平台之间的差异显而易见,各种细节还是需要考虑到的。在用过各种各样的 Linux 的 X 桌面环境之后,我发现其实没必要纠结于这个问题,Linux 强大的是命令行,我喜欢 Linux 也是因为它的命令行。所以我现在把 Linux 安装在虚拟机中,并用支持 SSH 的终端程序连接进去来进行调试和编程,也是一种编程本质的回归。Windows 的 GUI 对于日常应用是足够了的,安装了 MinGW 也能够获得与 Linux 相近的 CLI 体验。

其实对于有些题目,比如五子棋,就可以用 JavaScript 来编写一个 Web 应用,可以直接在电脑甚至手机上的浏览器中运行,也是比较有趣的。至于占课题多数的什么“管理系统”,可以采用 LAMP 来开发一个网站应用来实现,不过这更多的是数据库操作而不是数据结构了,就是服务器的搭建也是一项工程,这种程序/软件往大了写就不是几个星期能完成的课程设计了。

另外,这个程序并没有进行大量的测试,可能还会存在某些问题。我还不擅长自动化测试技术,只能简单地做手工测试,不够严谨。当然,要编写好一套能测出大部分 Bug 的测试用例是一门学问,有必要今后再进行研究。

本来我还想做 LZ 系列算法的实现,但发现单做一个 Huffman 算法就已经写了五百行代码,就缩减了一开始的目标。实际上我也查阅了相关文献,对这些常用的压缩算法都有了个大概的了解。在此基础上,我巩固了对 7z(LZMA) 的喜爱。

整篇课程设计是用 \LaTeXe\ 排版的,期间又复习了一下 \LaTeX\ 各种宏包的用法。对于这种比较正式的文章,\TeX\ 给了我专心写作文字的权利和对版面精细调整的权力,也是一种美的享受。至于与 Word 等所见即所得的文字处理器的论战,我也早在四年前做过了,也就不费口舌。

在老师的要求下,我画了一个图。当然,为了画这个图我花了整整一下午。期间我参考了很多宏包,都不太理想。又想使用可视化的画图软件,但发现生成的图片非常难看,而且还要用鼠标手动调整大小,非常烦人。历经挫折,终于发现了 Ti{\it k}Z \& PGF,我参考了几个样例终于画好了,代码写了有 50 行,不过最后得到的效果很不错。当然,这个图形是不规范的,因为我不清楚相关规范是什么。这个程序本身不是面向对象的,所以画 UML 图的话不知从何下手(画 UML 图的宏包倒是有很多不错的)。当然这个图比较不错地表现了程序的结构,基本上一目了然。

在进行写作的过程中,参考了不少书籍资料和论文文献,我都已经在参考文献中列出。有利于河南工业大学校园网的帮助,查阅文献方便了很多,大部分都可以免费下载阅读。学校图书馆的藏书也为我的写作提供了不少参考。

白浩老师丰富的经验也让我受益匪浅,之前零零碎碎自学的算法知识在这里得到了系统的解释。不少之前不明确的问题也得到了解惑,由此得到的进步也得益于老师辛勤的耕耘。

这个课程设计只算一个玩具性质的实验,并未用到软件工程的系统实践。所以我还需要进一步学习相关知识和技术。学问浩如烟海,计算机学科虽是九牛一毛,却也广袤无垠。在有限的职业生命里,我不可能把所有领域都涉及到,但通用的技术和方法要掌握扎实。隐藏在编程语言和编程工具之后的,包括数据结构和算法,使用轮子和制造轮子,都应该重视起来并好好学习。新世纪的计算机技术还在不断地发展中,我也希望能够投入创造的潮流之中,推动产业的发展和进步。

{
\vfill
\centering
\includegraphics[height=5cm]{image/166.png}
\vfill
}