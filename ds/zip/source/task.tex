\newpage
\section*{\underline{计算机}专业课程设计任务书}
\addcontentsline{toc}{section}{课程设计任务书}
% \ThisCenterWallPaper{1}{source/task.pdf}
\CTEXnoindent
\begin{tabularx}{\textwidth}{|c|X|}\hline
{\bf 学生姓名} & \quad \tjf \quad \vil {\bf 专业班级} \vil 计算机 1303 班 \vil {\bf 学\quad 号} \vil 201316920311 \\\hline
{\bf 题\qquad 目} & \quad \titlec \\\hline
{\bf 课题性质} & \makebox[11em][c]{B.工程技术研究} \vil {\bf 课题来源} \vil \makebox[11em][c]{D.自拟课题} \\\hline
{\bf 指导教师} & \makebox[11em][c]{白浩} \vil {\bf 同组姓名} \vil \makebox[11em][c]{无} \\\hline
{\bf 主要内容} & \tabincell{l}{\begin{minipage}[c][5cm][c]{12cm}
探究 Huffman 编码算法;\\
利用 Huffman 算法来压缩和解压文件。
\end{minipage}} \\\hline
{\bf 任务要求} & \tabincell{l}{\begin{minipage}[c][5cm][c]{12cm}
明确算法的实现步骤,表现为伪代码和图形;\\
使用高级编程语言 C 实现算法;\\
测试算法的正确性,测试程序的健壮性。
\end{minipage}} \\\hline
{\bf 参考文献} & \tabincell{l}{\begin{minipage}[c][5cm][c]{12cm}
算法导论(第三版). Thomas H.Cormen 等. 机械工业出版社: 2012\\
算法精解(C语言描述). Kyle Loudon. 机械工业出版社: 2012\\
Introduction to Data Compression. Guy E. Blelloch. 2013\\
Data Compression. Debra A. Lelewer, Daniel S. Hirschberg. 1987\\
\LaTeXe 科技排版指南. 邓建松, 彭冉冉, 陈长松. 科学出版社: 2001
\end{minipage}} \\\hline
{\bf 审查意见} & \tabincell{l}{
    指导教师签字:\\
    \vspace*{2cm}\\
    教研室主任签字:\hspace*{10em} \qquad 年\qquad 月\qquad 日\quad\\[0.5em]
} \\\hline
\end{tabularx}

{\small 说明:本表由指导教师填写,由教研室主任审核后下达给选题学生,装订在设计(论文)首页}

\CTEXindent
\newpage
