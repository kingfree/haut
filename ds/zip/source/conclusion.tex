\section{程序总结}

其实本报告已经写得很详细了,没必要写程序的总结文字。但是,我觉得还是写一写比较好,好事多磨,多叙述几遍可能更容易理解。

程序代码符合 ANSI C89 标准,代码风格采用 K\&R 书中的格式。

程序本身,将主要逻辑划分出两个接口,放在头文件中,可以编译成库来让其他程序使用。主程序通过调用两个接口来对文件进行压缩解压操作,并提供较为友好的用户命令行界面。

程序主体封装在单独的文件里,与外界隔绝开来。第~\pageref{funcall}~页的横排大图 \ref{funcall} 清晰地显示出来这两个接口对内函数调用关系,也说明了其余四个重要的系统调用之间的关系。

程序使用了位操作,略显繁杂,也封装成若干个函数。

程序构建频率表和编码表,都使用了指针,导致程序难以理解,这也是 C 语言的通病。理想状态下,所有指针都得到了释放,不会有内存泄露问题。

本程序是开源的,遵循 BSD 许可证协议,全文如下:
\begin{quote}
\fontsize{8pt}{8pt}
\verbatiminput{program/LICENSE}
\end{quote}

代码托管在 Github 上,可以检出到本地编译运行,网址是 \url{https://github.com/kingfree/haut/tree/master/ds/zip/program}。