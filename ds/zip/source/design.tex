\section{概要设计}


\subsection{数据结构}

\subsection{位操作}

\begin{verbatim}
/* 把位转成字节 */
unsigned long bit2byte(unsigned long len)

/* 返回第 i 位 */
unsigned char get_bit(unsigned char *bits, unsigned long i)

/* 反转各位 */
void reversc_bits(unsigned char *bits, unsigned long len)
\end{verbatim}

\subsection{Huffman 树}

\begin{verbatim}
node *new_leaf_node(unsigned char symbol)

node *new_node(unsigned long count, node *zero, node *one)

void free_huffman_tree(node *root)

void free_code(code *p)

void free_encoder(symcode *sc)

/* 对叶节点生成编码 */
code *new_code(const node *leaf)
\end{verbatim}

\subsection{符号表}

\begin{verbatim}
#define MAX_SYMBOLS 256
/* 符号频率表 */
typedef node *symfreq[MAX_SYMBOLS];
/* 符号编码表 */
typedef code *symcode[MAX_SYMBOLS];

void init_freq(symfreq *sf)

unsigned int get_symfreq(symfreq *sf, FILE *in)

/* 排序比较函数,低频率在前,空在末尾 */
int sfcmp(const void *p1, const void *p2)

/* 遍历子树建立符号编码表 */
void build_symcode(node *root, symcode *sf)

/* 返回编码数组 */
symcode *calc_code(symfreq *sf)

/* 写出编码表。格式如下:
 * * 4 字节编码大小 n
 * * 4 字节已编码字节数
 * * 编码 [1..n] ,每个编码 [i] 的格式为:
 *   * 1 字节符号,1 字节位长,编码字节(以 bit2byte 编码)
 *   * 如果编码不是 8 的倍数的话,最后一字节可能会有多余位
 */
int write_code_table(FILE *out, symcode *sc, uint32_t syms)
\end{verbatim}

\subsection{文件操作}

\begin{verbatim}
/* 读入编码表,返回构建的 Huffman 树 */
node *read_code_table(FILE *in, unsigned int *pdb)

/* 对文件编码 */
int do_file_encode(FILE *in, FILE *out, symcode *sc)

/* 把 in 压缩成 out */
int huffman_encode_file(FILE *in, FILE *out)

/* 把 out 解压成 in */
int huffman_decode_file(FILE *in, FILE *out)
\end{verbatim}
