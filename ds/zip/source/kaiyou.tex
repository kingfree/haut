\section{概要设计}

% \subsection{函数调用关系}

\begin{sidewaysfigure}
\centering
\newcommand{\defun}[2]{{\it #2}\\ {#1}}
% \newcommand{\defun}[2]{{#1}}

% \smartdiagram[constellation diagram]{
% \defun{编码文件}{huffman encode file},
% \smartdiagram[sequence diagram]{
%   \defun{获取频率}{get symfreq},
%   \defun{新叶结点}{new leaf node},
%   \defun{建立符号编码表}{build symcode},
%   },
% \defun{计算编码}{calc code},
% \defun{写出编码表}{write code table},
% \defun{编码}{do file encode},
% \defun{销毁编码表}{free encoder},
% \defun{销毁树}{free tree}
% }
\begin{tikzpicture}[align=center]
% \draw[help lines] (0,0) grid (20,11);

\node (A) at (27em,10) [ellipse,draw] {\defun{压缩文件}{huffman\_encode\_file}};
  \node (B1) at (7em,8) [draw] {\defun{获取频率}{get\_symfreq}};
  \draw[<-] (node cs:name=B1,anchor=north) |- (7em,9) -| (node cs:name=A,anchor=south);
    \node (C11) at (0em,6) [draw] {\defun{排序比较函数}{sfcmp}};
    \draw[<-] (node cs:name=C11,anchor=north) |- (7em,7);
    \node (C12) at (7em,6) [draw] {\defun{新叶结点}{new\_leaf\_node}};
    \draw[<-] (node cs:name=C12,anchor=north) |- (node cs:name=B1,anchor=south);
    \node (C13) at (15em,6) [draw] {\defun{建立符号编码表}{build\_symcode}};
    \draw[<-] (node cs:name=C13,anchor=north) |- (7em,7);
      \node (D) at (15em,4) [draw] {\defun{创建编码}{new\_code}};
      \draw[<-] (node cs:name=D,anchor=north) |- (node cs:name=C13,anchor=south);
  \node (B2) at (25em,8) [draw] {\defun{计算编码}{calc\_code}};
  \draw[<-] (node cs:name=B2,anchor=north) |- (25em,9) -| (node cs:name=A,anchor=south);
    \node (C21) at (22em,6) [draw] {\defun{新结点}{new\_node}};
    \draw[<-] (node cs:name=C21,anchor=north) |- (22em,7) -| (node cs:name=B2,anchor=south);
    \node (C22) at (28em,6) [draw] {\defun{初始化频率表}{init\_freq}};
    \draw[<-] (node cs:name=C22,anchor=north) |- (28em,7) -| (node cs:name=B2,anchor=south);
  \node (B3) at (35em,8) [draw] {\defun{写出编码表}{write\_code\_table}};
  \draw[<-] (node cs:name=B3,anchor=north) |- (35em,9) -| (node cs:name=A,anchor=south);
    \node (C31) at (33.4em,6) [circle,draw] {\emph{fwrite}};
    \draw[<-] (node cs:name=C31,anchor=north) |- (35em,7) -| (node cs:name=B3,anchor=south);
    \node (C32) at (36.6em,6) [circle,draw] {\emph{htonl}};
    \draw[<-] (node cs:name=C32,anchor=north) |- (35em,7) -| (node cs:name=B3,anchor=south);
  \node (B4) at (44em,8) [draw] {\defun{进行文件编码}{do\_file\_encode}};
  \draw[<-] (node cs:name=B4,anchor=north) |- (44em,9) -| (node cs:name=A,anchor=south);
  \node (B5) at (44em,6) [draw] {\defun{销毁编码表}{free\_encoder}};
  \draw[<-] (node cs:name=B5,anchor=east) |- (50em,6) |- (35em,9) -| (node cs:name=A,anchor=south);
    \node (C5) at (44em,4) [draw] {\defun{销毁编码}{free\_code}};
    \draw[<-] (node cs:name=C5,anchor=north) |- (node cs:name=B5,anchor=south);
  \node (B6) at (40em,2) [draw] {\defun{销毁树}{free\_tree}};
  \draw[<-] (node cs:name=B6,anchor=east) |- (50em,2) |- (35em,9) -| (node cs:name=A,anchor=south);

\node (G) at (32em,0) [ellipse,draw] {\defun{解压缩文件}{huffman\_decode\_file}};
  \node (F) at (22em,2) [draw] {\defun{读入编码表}{read\_code\_table}};
  \draw[->] (node cs:name=G,anchor=north) |- (35em,1) -| (node cs:name=F,anchor=south);
    \node (E1) at (33.4em,4) [circle,draw] {\emph{fread}};
    \draw[->] (node cs:name=F,anchor=north) |- (35em,3) -| (node cs:name=E1,anchor=south);
    \node (E2) at (36.6em,4) [circle,draw] {\emph{ntohl}};
    \draw[->] (node cs:name=F,anchor=north) |- (35em,3) -| (node cs:name=E2,anchor=south);

  \draw[<-] (node cs:name=C21,anchor=south) |- (22em,3) -| (node cs:name=F,anchor=north);
  \draw[<-] (node cs:name=C12,anchor=south) |- (22em,3) -| (node cs:name=F,anchor=north);
\draw[<-] (node cs:name=B6,anchor=south) |- (35em,1) -| (node cs:name=G,anchor=north);

\end{tikzpicture}

% \begin{tikzpicture}
% \node {\defun{编码文件}{huffman encode file}}
%   child {node {\defun{计算编码}{calc code}}}
%   child {node {\defun{获取频率}{get symfreq}}
%     child {node {\defun{新叶结点}{new leaf node}}}
%     child {node {\defun{建立符号编码表}{build symcode}}}
%   }
%   child {node {\defun{写出编码表}{write code table}}}
%   child {node {\defun{编码}{do file encode}}}
%   child {node {\defun{销毁编码表}{free encoder}}}
%   child {node {\defun{销毁树}{free tree}}};
% \end{tikzpicture}

\caption{\label{funcall}两个接口对主要函数之间的调用关系}
\end{sidewaysfigure}


\subsection{程序运行逻辑}

我们首先考虑程序运行的主要逻辑。当键入程序名运行程序时,应该给出帮助信息,告诉用户应该怎样使用本程序。假定运行程序名为 {\sf huff}。

考虑到要压缩文件并输出,至少我们需要两个参数即\verb|<输入文件>|和\verb|<输出文件>|。所以,要压缩一个 {\sf 文件1} 到 {\sf 文件2} ,键入命令 \verb|huff 文件1 文件2| 即可。

对于解压文件,可以直接加一个参数 \verb|-u|。比如把 {\sf 文件2} 解压成 {\sf 文件3},应当键入命令 \verb|huff 文件2 文件3 -u|。

我们提供如下选项:
\textbf{-i} 输入文件;
\textbf{-o} 输出文件;
\textbf{-u} 解压缩;
\textbf{-z} 压缩;
\textbf{-h} 显示帮助;
\textbf{-v} 显示版本信息。
使用 \verb|getopt()| 函数即可处理这些参数。

默认情况下,对于正常完成的操作,不应该输出提示信息,直接结束即可。对于异常情况,则输出错误信息并结束程序,返回错误码。

\subsection{接口定义}

第~\pageref{funcall}~页的横排大图 \ref{funcall} 展示了这两个对外接口对主要函数之间的调用关系。

\function{huffman\_encode\_file}
{FILE *in, FILE *out}
{int}{成功返回 0}
{读入文件 {\tt in},编码之后输出到 {\tt out} 中。首先获取输入文件中符号出现频率,依此调用 Huffman 算法生成编码表。再次扫描文件,使用之前生成的编码表来编码,每读入一个位就进行对照编码表进行编码,满一个字节就写出,最后不满补 0。}

% \subsection{文件操作}
% \begin{verbatim}
% /* 对文件编码 */
% int do_file_encode(FILE *in, FILE *out, symcode *sc)
% \end{verbatim}

\function{huffman\_decode\_file}
{FILE *in, FILE *out}
{int}{成功返回 0}
{读入文件 {\tt in},解码之后输出到 {\tt out} 中。首先读入编码表,构建对应的 Huffman 树。接着按位读入后面的数据,不断查询 Huffman 树知道叶结点,输出解码结果,不断重复到文件直到末尾。}
