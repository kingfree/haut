\section{需求分析}

一个压缩解压实用程序,要读取输入文件,压缩之后写出输出文件,并可以把压缩文件解压还原成原来的文件。由此我们要实现这两个功能:
\begin{itemize}
\item 对文件进行编码
\item 对文件进行解码
\end{itemize}

\subsection{Huffman 编码}

哈夫曼编码 (Huffman coding),又译霍夫曼编码、哈弗曼编码、赫夫曼编码,是一种古老的压缩算法,它是一种基于最小冗余编码的压缩算法\cite{maic}。最小冗余编码是指,如果知道一组数据中符号出现的频率,就可以用相应的的位数来表示符号从而减少数据需要的空间。

由于所有文件都是二进制存放的,所以我这里不讨论文件编码问题,将中文字符和西文字符都看做字节,减少问题的复杂度。由于我们不安排编码,所以压缩解压后文件编码如原文一致。(注意:此编码非彼编码)

Huffman 算法实际上是一个贪心算法。