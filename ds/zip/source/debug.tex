\section{调试分析}


Huffman 编码是最优的字符压缩编码,但不是最优的压缩编码。如果在一个文件中,各种符号出现的频率接近一致,那么 Huffman 编码并不能有效地减少文件体积,反而由于引入了编码表可能会导致文件变大,这点在非文本文件压缩中体现得尤为明显。甚至把一个文件进行多重压缩之后,文件体积会越来越大。当然还原文件还是可以正常工作的,不过这显然违背了压缩的原则——减小文件体积。所以在日常应用中,我们不使用 Huffman 编码。相应地,会采用 LZ77\cite{lz77}, LZ78\cite{lz78}, LZW\cite{lzw}, ZIP\cite{zip}, RAR\cite{rar}, LZMA\cite{lzma} 等算法,这些都是我们常见的压缩格式,实践证明这些也是有效的压缩算法。
