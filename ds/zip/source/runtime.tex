\section{运行环境}
\label{runtime}

理论上,该程序没有硬件和软件限制。只要是现代计算机,有为该平台移植的 GNU C 编译器和网络库即可。为了显示运行结果,可能还需要一个终端实用程序。

对于 Linux 系统,安装了 GNU C 编译器即可,并包含网络库 \verb|<netinet/in.h>|。编译时,进入程序目录,输入 \verb|make| 即可生成 {\sf huff} 二进制可执行文件。

对于其他类 UNIX 系统,我没有进行测试,但只要支持 POSIX 标准,理论上是可以编译运行的。

对于 Windows 系统,安装了 MinGW 编译器即可编译该程序,要求编译器支持 ANSI C ,并包含网络库 \verb|<winsock2.h>| 。编译时,进入程序目录,输入 \verb|make WIN32=TRUE| 即可生成 {\sf huff.exe} 二进制可执行文件。注意由于源文件本身是 UTF-8 编码的,所以编译之后运行会产生乱码。这时可以使用文本编辑器打开 {\sf huff.c} 文件,保存成 GBK 编码格式,然后再进行编译。由于本文重点不在文字编码,所以不再进行详细讨论。
