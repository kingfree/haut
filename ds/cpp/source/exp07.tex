\section{创建 AHAPE 抽象类}
\hfill\ctli{实验时间}{~2015~年~1~月~7~日}
\subsection*{【实验目的】}
\begin{enumerate}[topsep=0pt,partopsep=0pt,itemsep=0pt,parsep=0pt,label={\arabic*、}]
\item 掌握多态性的概念。
\item 掌握虚函数概念及其与多态性的关系。
\end{enumerate}
\subsection*{【实验环境】}
\MyEnvironment
\subsection*{【实验内容】}
功能要求:定义一个抽象类SHAPE,抽象方法SHAPE包含X和Y两个属性的访问方法,VOLUME 方法,AREA抽象方法和GETNAME方法。不同的形状类,如POINT 类实现SHAPE 类,RECTANGLE类继承PIONT ,ELLIPSE 类继承RECTANGLE 类。CIRCLE 类继承ELLIPSE 类,CYLINDER类继承CIRCLE类。创建每个类的实例,并将每个类的实例存放于类型为SHAPE的数组中。以该SHAPE的数组作为参数,调用参数的类型为SHAPE 的数组的SHOWSHAPINFO方法,通过调用重写的方法为相应得图形对象计算表面积,体积并输出图形的名称
\subsection*{【详细分析】}
(此项由学生自己完成)
\subsection*{【实验源码】}
(此项由学生自己完成)
\subsection*{【实验结果】}
(截图给出实验结果)
\subsection*{【实验体会】}
(至少150字)
