\section{类 Date 的属性}
\hfill\ctli{实验时间}{~2015~年~1~月~12~日}
\subsection*{【实验目的】}
\begin{enumerate}[topsep=0pt,partopsep=0pt,itemsep=0pt,parsep=0pt,label={\arabic*、}]
\item 掌握重载的概念
\item 能够进行运算符重载。
\end{enumerate}
\subsection*{【实验环境】}
\MyEnvironment
\subsection*{【实验内容】}
日期类设计

定义Date类,参照实现:
\begin{enumerate}[topsep=0pt,partopsep=0pt,itemsep=0pt,parsep=0pt,label={(\arabic*)}]
\item 日期的加、减运算
\item 根据日期计算一年中的第几周星期几、一年中第几天为几月几日、该年是否为闰年
\item 输出日期对象
\end{enumerate}

完成相应应用程序设计
\subsection*{【详细分析】}
(此项由学生自己完成)
\subsubsection*{【实验源码】}
{\linespread{1}\lstinputlisting[caption={\tt Date.h}]{exp05/Date.h}}
{\linespread{1}\lstinputlisting[caption={\tt Date.cpp}]{exp05/Date.cpp}}
{\linespread{1}\lstinputlisting[caption={\tt exp01.cpp}]{exp05/exp01.cpp}}
\subsubsection*{【实验结果】}
\begin{figure}[htp]
\centering
\includegraphics[width=\textwidth]{exp05/exp01.png}
\caption{\label{out05_01}Date 类}
\end{figure}
\subsection*{【实验体会】}
对于最近几百年这样做是可以的,但是要注意,年月日是“历法单位”,不是“时间单位”,历法是人根据天文观测制定的,并不断修正,而“修正”这件事是没有规律的。

比如在欧洲大陆,1582年10月5日至10月14日,这10天就是不存在的,调整后的历法就是格里高利历;但是在英国,这个调整一直拖到了一百多年后,直到1752年,这一年的9月3日至13日这11天是不存在的;而在此期间的一百多年里两地的日期一直不相同。
