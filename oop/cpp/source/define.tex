\usepackage[utf8]{inputenc}
\def\tjf{{\tt{田劲锋}}}
\def\titlec{《面向对象程序设计》实验报告}
\def\MyEnvironment{gcc version 4.9.2 (tdm64-1)}
\usepackage[top=2.54cm,bottom=2.54cm,left=3.17cm,right=3.17cm]{geometry} % 页面设置
\usepackage[unicode,breaklinks=true,
colorlinks=true,linkcolor=black,anchorcolor=black,citecolor=black,urlcolor=black,
pdftitle={\titlec},pdfauthor={\tjf}]{hyperref}
\usepackage{tikz} % 画图
\usetikzlibrary{shapes,arrows}
\usepackage{multicol} % 分栏
\usepackage{multirow} % 跨行
\usepackage{longtable} % 长表格
\usepackage{tabularx} % 变宽表格
\usepackage{booktabs} % 表格画线
\usepackage{graphicx} % 图形
\usepackage{color} % 颜色
\usepackage{xcolor} % 颜色
\usepackage{wallpaper} % 背景图片
\usepackage{listings} % 排版代码
\lstset{language=C++,
  numbers=left,
  numberstyle=\tiny,
  basicstyle=\small\tt,
  commentstyle=\itshape\color{gray},
  keywordstyle=\bfseries\color{blue},
  stringstyle=\color{red},
  showstringspaces=false,
}
\usepackage{amsmath,bm}
\usepackage{verbatim} % 排版代码
\usepackage{url} % 排版链接
\usepackage{shortvrb}
\usepackage{pstricks} % 绘图
\usepackage{pst-tree} % 画树
% \usepackage{pst-uml} % 画 UML
\usepackage{uml} % 画 UML
\usepackage{clrscode3e} % CLRS 伪代码
\usepackage{smartdiagram} % 智能画图
\usepackage{nameref}
\usepackage{rotating} % 横排大图
\usepackage{caption}
\captionsetup{font={small}} % 标题字体大小
\usepackage[inline]{enumitem} % 调整列表样式

%\setmainfont{Times New Roman}
\setCJKmainfont[BoldFont={SimHei}]{SimSun}  % 主要字体:宋体、黑体
\setCJKsansfont[BoldFont={STZhongsong}]{STFangsong} % 次要字体:仿宋、中宋
\setCJKmonofont{KFKai} % 等宽字体:楷体
\setCJKfamilyfont{msyh}[BoldFont={* Bold}]{微软雅黑} \newcommand{\msyh}{\CJKfamily{msyh}} % 微软雅黑
\setCJKfamilyfont{micro}{文泉驿微米黑} \newcommand{\micro}{\CJKfamily{micro}} % 文泉驿微米黑
\setCJKfamilyfont{yaoti}{方正姚体} \newcommand{\yaoti}{\CJKfamily{yaoti}}

\CJKsetecglue{\hspace{0.1em}}
\renewcommand\CJKglue{\hskip -0.3pt plus 0.08\baselineskip}
\frenchspacing
\widowpenalty=10000
\linespread{1.2}
\setlength{\parskip}{2pt plus 2pt}
\renewcommand{\baselinestretch}{1.5}

\setlength{\abovecaptionskip}{1pt}
\setlength{\belowcaptionskip}{0pt}
\setlength{\intextsep}{8pt}

\makeindex
\pagestyle{plain}

\CTEXsetup[number=\chinese{part}, name={实验单元,}, format={\large\bfseries\centering}]{part}
\CTEXsetup[number=\chinese{section}, name={实验,}, format={\large\bfseries\centering}]{section}
\CTEXsetup[format={\normalsize\bfseries}]{subsection}

\newcommand{\ctli}[2]{{\bf #1}:\underline{{~#2~}}\hfill}
