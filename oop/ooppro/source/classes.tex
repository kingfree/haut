\subsection{类的设计说明}

\newcommand\method[2]{{#1}:~{\it #2}}
\newcommand\vart[2]{{#1}:~{\it #2}}
\newcommand\argu[2]{{\sf #2}:~{\it #1}}
\newcommand\comt[1]{\hfill\quad{\tt //#1}}
\newcommand\comtn[1]{\\\comt{#1}}

\subsubsection{Number 类}

\begin{figure}[htp]
  \pictext\small
\begin{tikzpicture}
\umlclass{Number}{
\# \argu{int}{number} \comt{待猜的数}\\
-- \argu{short[]}{numbers} \comt{数的拆分数组}\\
\# \argu{int}{count} \comt{猜数次数}\\
}{
+ {Number()} \comt{初始化产生随机数}\\
% + {Number(\argu{int}{number})} \comt{以指定数字初始化}\\
-- \method{genRand()}{int} \comt{产生符合要求的随机数}\\
+ \method{guess(\argu{int}{number})}{\argu{int}{x}, \argu{int}{y}} \comt{猜数并返回提示信息}\\
+ \method{detail(\argu{int}{number})}{int[]} \comt{猜数并返回正确的位数组}\\
+ \method{answer()}{int} \comt{返回正确的答案}\\
+ $\sim$Number() \comt{析构函数}\\
}
\end{tikzpicture}
  \caption{\label{Number}Number 类}
\end{figure}

\function{Number::Number}
{}
{}{构造函数}
{调用 Number::genRand() 来产生符合要求的随机数,调用 this$\to$setNumber() 来设置 this$\to$number 和 this$\to$numbers。初始化 this$\to$count 为 0。}

% \function{Number::Number}
% {int number}
% {}{构造函数}
% {指定待猜的数为 number,并判断该数字是否符合要求;如果不符合,则调用 Number::genRand(number) 置 this$\to$number 为距离 number 最近的符合要求的数字。初始化 this$\to$count 为 0。}

\function{Number::genRand}
{}
{int}{生成的随机数}
{生成符合要求的随机数。算法见第~\pageref{Number::genRand}~页的图~\ref{Number::genRand}。}

% \function{Number::genInt}
% {int number}
% {int}{符合要求的数}
% {如若 number 每位都不一样且为四位数,则直接返回;如果不符合要求,则找出距离 number 最近的符合要求的数,距离一样取较小的。算法见第~\pageref{Number::genInt}~页的图~\ref{Number::genInt}。}

% \function{Number::isGood}
% {int number}
% {bool}{是否符合要求}
% {判断 number 是否为每位都不一样且为四位数。算法见第~\pageref{Number::isGood}~页的图~\ref{Number::isGood}。}

\function{Number::setNumber}
{int number}
{void}{无}
{将 number 放置到 this$\to$number 里,并将 this$\to$numbers 置为对应的拆分数组,这两个是一对一的关系,不可分拆,只是单纯的方便后面的计算。}

\function{Number::guess}
{int number}
{std::pair<int, int>}{提示信息$(x,y)$}
{猜数。用户猜的是 number,判断是否正确并返回提示信息$(x,y)$ ,$x$表示数字、位置都匹配的个数,$y$表示数字匹配但位置不匹配的个数。全部正确时应该返回$(4,0)$。每调用该方法时 this$\to$count 应该自增 1。}

\function{Number::detail}
{int number}
{std::vector<int>}{详细信息的数组}
{对应 8888 键的功能。用户上一次猜的是 number,返回一个数组,数组的每个元素表示哪一位正确了。全部正确时应该返回$\{1,2,3,4\}$。每调用该方法时 this$\to$count 应该自增 1。}

\function{Number::answer}
{}
{int}{正确答案}
{对应 7777 键的功能。返回正确答案的 this$\to$numbers。每调用该方法时 this$\to$count 应该自增 1。}

\subsubsection{Score 类}

\begin{figure}[htp]
  \pictext\small
\begin{tikzpicture}
\umlclass{Score}{
\underline{\# \argu{int}{PLUS} = 20} \comt{加分分值}\\
\underline{\# \argu{int}{MINUS} = 40} \comt{减分分值}\\
\# \argu{int}{score} \comt{得分}\\
+ \method{getScore()}{int} \comt{获取得分}\\
\# \argu{int}{lastNumber} \comt{用户上一次猜的数}\\
\# \argu{Number[]}{numbers} \comt{存储每次猜数对象}\\
-- \argu{string}{password} \comt{密码}\\
}{
-- {Score()} \comt{构造函数}\\
\underline{+ \method{getInstance()}{Score\&}} \comt{获取实例}\\
+ {$\sim$Score()} \comt{析构函数}\\
+ \method{newGame()}{void} \comt{新建游戏}\\
+ \method{newGame(\argu{int}{number})}{void} \comt{指定数字新建游戏}\\
+ \method{guess(\argu{int}{number})}{bool, string} \comt{猜数}\\
\# \method{plus()}{void} \comt{加分}\\
\# \method{minus()}{void} \comt{减分}\\
+ \method{read()}{int} \comt{读入分数}\\
+ \method{write()}{void} \comt{写出分数}\\
+ \method{checkPassword(\argu{string}{password})}{bool} \comt{检查密码}\\
}
\end{tikzpicture}
  \caption{\label{Score}Score 类}
\end{figure}

\function{Score::Score}
{}
{}{构造函数}
{单例模式的构造函数是 private 的。构造函数尝试调用 this$\to$read() 从文件中读取分数和密码,初始化 this$\to$score 和 this$\to$password。}

\function{Score::getInstance}
{}
{static Score\&}{唯一的 Score 实例引用}
{单例模式中,获取唯一的 Score 实例。这是个静态方法。}

\function{Score::newGame}
{}
{void}{无}
{创建一个新的 Number 实例,添加到 this$\to$numbers 数组的尾部。}

\function{Score::newGame}
{int number}
{void}{无}
{以指定的 number 为参数,创建一个新的 Number 实例,并添加到 this$\to$numbers 数组的尾部。}

\function{Score::guess}
{int number}
{std::pair<bool, std::string>}{是否猜对以及提示信息}
{用户输入了 number。

对于特殊情况,如果 number 是 8888,则调用当前 Number 对象(即 this$\to$numbers 数组的最后一个元素)的 detail() 方法,参数为 this$\to$lastNumber,并返回猜数失败和详细的帮助信息,即“第 $a,b,c$ 位数字正确”;如果是 7777,则调用 answer() 方法来查看答案,并返回猜数失败和正确答案,即“正确答案是 $number$”。特殊情况不加减分数。

对于一般的猜数,则调用 guess() 进行正常的猜数流程,如果猜对则返回猜数成功和祝贺信息;猜错则返回猜数失败和提示信息 $(x,y)$,即“数位匹配 $x$ 个,数匹配位不符 $y$ 个”。猜对的加分,猜错的则减分。}

\function{Score::plus}
{}
{void}{无}
{给 this$\to$score 加上 Score::PLUS 的分值。}

\function{Score::minus}
{}
{void}{无}
{给 this$\to$score 减去 Score::MINUS 的分值。}

\function{Score::read}
{}
{int}{读入的得分}
{从文件中读取得分和密码,分别放到 this$\to$score 和 this$\to$password 里。}

\function{Score::write}
{}
{void}{无}
{向文件中写出得分 this$\to$score 和密码 this$\to$password。}

\function{Score::checkPassword}
{std::string password}
{bool}{密码是否正确}
{对比 password 和 this$\to$password 是否一致。}

\subsubsection{UI 类}

\begin{figure}[htp]
  \pictext\small
\begin{tikzpicture}
\umlclass[type=utility]{UI}{
}{
\underline{+ \method{Main()}{void}} \comt{主循环}\\
\underline{+ \method{MainMenu()}{int}} \comt{主菜单}\\
\underline{+ \method{NewGame()}{void}} \comt{新游戏}\\
\underline{+ \method{GuessNumber()}{bool}} \comt{猜数字}\\
\underline{+ \method{ViewDetail()}{void}} \comt{8888}\\
\underline{+ \method{ViewAnswer()}{void}} \comt{7777}\\
\underline{+ \method{ShowScore()}{void}} \comt{显示得分}\\
\underline{+ \method{InputPassword()}{bool}} \comt{输入密码}\\
}
\end{tikzpicture}
  \caption{\label{UI}UI 类}
\end{figure}

\function{UI::Main}
{}
{void}{无}
{循环调用 UI::MainMenu() ,根据其返回值调用 UI::NewGame() 或者 UI::ShowScore()。直到其返回 0 表示退出,此时中止循环。}

\function{UI::MainMenu}
{}
{void}{无}
{显示主菜单,等待用户输入选项,输入后返回选项值。

定义:(1) 新游戏;(2) 显示得分;(0) 退出。

对于用户的其他不合理输入,一律解析为 0,表示退出。}

\function{UI::NewGame}
{}
{void}{无}
{开始新游戏。首先调用单例 Score 的 newGame 方法,然后显示提示信息,调用 UI::GuessNumber() 进入猜数流程。循环调用直到该方法返回真表示猜对,显示祝贺信息。}

\function{UI::GuessNumber}
{}
{bool}{是否猜对}
{等待用户输入。判断用户的输入,如果是 8888 则调用 UI::ViewDetail(),如果是 7777 则调用 UI::ViewAnswer,如果是 0000 则返回上一级,如果是负数则直接退出程序。对于正常的输入,则直接调用 Score 实例的 guess() 方法,显示其返回的提示字符串,显示当前得分。}

\function{UI::ViewDetail}
{}
{void}{无}
{8888 功能。调用 Score 实例的 guess() 方法,并显示提示字符串。}

\function{UI::ViewAnswer}
{}
{void}{无}
{7777 功能。首先调用 UI::InputPassword() 来进行密码输入和验证,允许三次密码输入,如果验证失败则直接返回;如果验证成功,则调用 Score 实例的 guess() 方法,并显示提示字符串。}

\function{UI::ShowScore}
{}
{void}{无}
{显示得分。}

\function{UI::InputPassword}
{}
{bool}{密码是否验证通过}
{提示用户输入密码,设置控制台属性,等待用户输入。在 Windows 下,密码输入后回显成星号 “*”;在 Linux 下,密码输入不回显,但退格键仍应可用。这些操作可以调用 getpass() 函数,在 {\tt Password.h} 中提供。密码输入后,调用 Score 实例的 checkPassword() 方法来验证密码的正确性。}

















