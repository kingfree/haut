\subsection{主要算法流程图}

产生一个指定的随机数的 Number::genRand() 方法,即每次从 0..9 中选取一个数放到数组里。由于产生的是四位数,要求千位不能为 0,所以产生千位时采用了特殊处理。具体算法的伪代码描述见图~\ref{Number::genRand}。

\begin{figure}[htp]
\begin{quote}
\begin{codebox}
\Procname{\proc{Number::genRand}()}
\li $S = \{\}$
\li $T = \{1, 2, \cdots, 9\}$
\li $S[0] = $从 $T$ 中随机选取一个数
\li $T = T - \{S_0\} + \{0\}$
\li \For $i = 1$ \To $3$ \Do
\li   $S[i] = $从 $T$ 中随机选取一个数
\li   $T = T - \{S_i\}$
	\End
\li $n = \overline{S_0S_1S_2S_3}$
\li \Return $n$
\end{codebox}
\end{quote}
\caption{\label{Number::genRand}生成符合要求的随机数}
\end{figure}

验证猜数并返回指定的 $(x, y)$ 的 Number::guess() 方法,首先需要从 this$\to$numbers 获取答案的分解,然后把传进来的数字也进行分解。分解后的两个数组按位比较,累加后得 $x$。两个数组求交集之后得到 $x + y$。具体算法见图~\ref{Number::guess}。

\begin{figure}[htp]
\begin{quote}
\begin{codebox}
\Procname{\proc{Number::guess}($n$)}
\li $x = 0$
\li $y = 0$
\li $S = $ this$\to$numbers
\li $\overline{A_0A_1A_2A_3} = n$
\li \For $i = 0$ \To $3$ \Do
\li   \If $A[i] == S[i]$ \Then
\li     $x++$
      \End
    \End
\li $y = |S \cap A| - x$
\li this$\to$count $++$
\li \Return $(x, y)$
\end{codebox}
\end{quote}
\caption{\label{Number::guess}验证猜数并返回指定的 $(x, y)$}
\end{figure}

验证猜数并返回详细信息的 Number::detail() 方法,和 Number::guess() 非常类似,只是这回返回的是一个数组。由于要求首位表示为 1,所以需要加一处理。具体算法伪代码见图~\ref{Number::detail}。

\begin{figure}[htp]
\begin{quote}
\begin{codebox}
\Procname{\proc{Number::detail}($n$)}
\li $V = \{\}$
\li $S = $ this$\to$numbers
\li $\overline{A_0A_1A_2A_3} = n$
\li \For $i = 0$ \To $3$ \Do
\li   \If $A[i] == S[i]$ \Then
\li     $V = V + \{i + 1\}$
      \End
    \End
\li this$\to$count $++$
\li \Return $V$
\end{codebox}
\end{quote}
\caption{\label{Number::detail}验证猜数并返回详细信息}
\end{figure}

% \begin{figure}[htp]
% \begin{quote}
% \begin{codebox}
% \Procname{\proc{Number::genInt}()}

% \end{codebox}
% \end{quote}
% \caption{\label{Number::genInt}根据给的的数生成符合要求的数}
% \end{figure}

语言细节相关的实现,不是前提设计的重点,这里不再赘述算法具体应该怎样用语言实现,请参考语言相关文档和标准库文档。
