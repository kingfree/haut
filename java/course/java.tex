\documentclass[cs5size,b5paper,nofonts,twoside]{ctexart}
\CTEXsetup[number=\chinese{section}, name={,、}, format={\Large\bfseries}]{section}

\usepackage[utf8]{inputenc}
\def\tjf{{\tt{田劲锋}}}
\def\titlec{《算法设计与分析》综合性实验实验报告}
\usepackage[top=2.54cm,bottom=2.54cm,left=3.17cm,right=3.17cm]{geometry} % 页面设置
\usepackage[unicode,breaklinks=true,
colorlinks=true,linkcolor=black,anchorcolor=black,citecolor=black,urlcolor=black,
pdftitle={\titlec},pdfauthor={\tjf}]{hyperref}
\usepackage{tikz} % 画图
\usetikzlibrary{shapes,arrows}
\usepackage{multicol} % 分栏
\usepackage{multirow} % 跨行
\usepackage{longtable} % 长表格
\usepackage{tabularx} % 变宽表格
\usepackage{booktabs} % 表格画线
\usepackage{graphicx} % 图形
\usepackage{color} % 颜色
\usepackage{xcolor} % 颜色
\usepackage{wallpaper} % 背景图片
\usepackage{listings} % 排版代码
\lstset{language=C++,
  numbers=left,
  numberstyle=\tiny,
  basicstyle=\small\tt,
  commentstyle=\color{gray},
  keywordstyle=\bfseries\color{violet},
  stringstyle=\color{teal},
  showstringspaces=false,
}
\usepackage{amsmath,bm}
\usepackage{verbatim} % 排版代码
\usepackage{url} % 排版链接
\usepackage{shortvrb}
\usepackage{pstricks} % 绘图
\usepackage{pst-tree} % 画树
% \usepackage{pst-uml} % 画 UML
\usepackage{uml} % 画 UML
\usepackage{clrscode3e} % CLRS 伪代码
\usepackage{smartdiagram} % 智能画图
\usepackage{nameref}
\usepackage{rotating} % 横排大图
\usepackage{caption}
\captionsetup{font={small}} % 标题字体大小
\usepackage[inline]{enumitem} % 调整列表样式

%\setmainfont{Times New Roman}
\setCJKmainfont[BoldFont={SimHei}]{SimSun}  % 主要字体:宋体、黑体
\setCJKsansfont[BoldFont={STZhongsong}]{STFangsong} % 次要字体:仿宋、中宋
\setCJKmonofont{KFKai} % 等宽字体:楷体
\setCJKfamilyfont{msyh}[BoldFont={* Bold}]{微软雅黑} \newcommand{\msyh}{\CJKfamily{msyh}} % 微软雅黑
\setCJKfamilyfont{micro}{文泉驿微米黑} \newcommand{\micro}{\CJKfamily{micro}} % 文泉驿微米黑
\setCJKfamilyfont{yaoti}{方正姚体} \newcommand{\yaoti}{\CJKfamily{yaoti}}

\CJKsetecglue{\hspace{0.1em}}
\renewcommand\CJKglue{\hskip -0.3pt plus 0.08\baselineskip}
\frenchspacing
\widowpenalty=10000
\linespread{1.5} % 1.5 倍行距
\setlength{\parskip}{2pt plus 2pt}
\renewcommand{\baselinestretch}{1.5}

\setlength{\abovecaptionskip}{1pt}
\setlength{\belowcaptionskip}{0pt}
\setlength{\intextsep}{8pt}

\makeindex
\pagestyle{plain}


\begin{document}

\begin{titlepage}
\begin{center}
{\zihao{-1}\bf 《Java编程基础》上机指导书}
\end{center}
\begin{center}
{\zihao{4}\bf 实验报告注意事项}
\end{center}
\begin{enumerate}
\item 试验报告按要求装订好,用{\textcolor{red}{\bf B5纸}}打印,{\bf 本页也必须打印}。已打印的实验页为封皮,后面的实验内容可以写到正式实验报告纸或者作业纸上,但必须规范。
\item 报告的内容主要为完成的程序(关键代码)。
\item 实验报告上交的最后日期为下一次实验前,过时不收。
\item 上机、实验报告的完成情况会作为平时成绩,在总成绩100分中占20分。
\item 如有抄袭,被炒和抄袭者本次实验都为0分。
\end{enumerate}
\end{titlepage}

\clearpage
\section{JDK 安装、配置及 Java 程序的编译和运行}
\titl{2015}{3}{16}

\subsection{实验目的}
\begin{enumerate}
\item 熟悉JDK的安装、配置。
\item 学会如何编辑、编译、运行Java程序。
\end{enumerate}
\subsection{实验内容}
完成如下任务或编写能够满足如下条件的程序(3、4、5中任选两题):
\begin{enumerate}
\item 安装JDK,并设置编译、运行Java程序需要的环境。 
\item 编写一个简单的程序,输出“Welcome to java world”。
\item 计算一个整数各位数字之和。
\item 编程求解234是否是一个水仙花数。所谓“水仙花数”是指一个3位数,其各位数字立方和等于该数。
\item 求数组的和、平均值。
\end{enumerate}

% {\linespread{1}\lstinputlisting{exp1/Hello/src/Hello.java}}

% {\linespread{1}\lstinputlisting{exp1/SumCount/src/SumCount.java}}

% {\linespread{1}\lstinputlisting{exp1/NarcissisticNumber/src/NarcissisticNumber.java}}

% {\linespread{1}\lstinputlisting{exp1/SumAverage/src/SumAverage.java}}

\clearpage
\section{Java基本语法}
\titl{2015}{3}{16}
\subsection{实验目的}
\begin{enumerate}
\item 熟悉Java语言中的数据类型、变量声明、数组、运算符号、流程控制语句。
\item 学会定义类和方法,利用方法传递参数,得到方法的返回值。
\end{enumerate}
\subsection{实验内容}
编写能够满足如下条件的程序:
\begin{enumerate}
\item 计算两个3$\times$3矩阵的和,int[][]或double[][]都可以。方法接受参数,并返回计算的结果。
\item 声明一个类,定义一个方法以计算一维数组中的最大值并返回该值,参数为int[]或double[]。在main方法中调用该方法,传递不同长度的数组,得到返回值并输出。
\item 用$\frac{\pi}{4}\approx 1-\frac{1}{3}+\frac{1}{5}-\frac{1}{7}$公式求$\pi$的近似值,直到最后一项绝对值小于$10^{-6}$。
\item (选做)求100至200间的全部素数。
\item (选做)输出100~999之间所有的“水仙花数”。
\item (选做)求Fibonacci数列的前40个数。即$F_1=1$,$F_2=1$,$F_n=F_{n-1}+F_{n-2} (n\ge 3)$。
\item (选做)在一个方法中实现从一个数组中找到该数组的最大值和次大值并返回。
\item (选做)一个数如果恰好等于它的因子之和,这个数就是完数。例如6的因子为1、2、3,而6=1+2+3,因此6是一个完数。编程求出1000之内的所有完数。
\end{enumerate}

% {\linespread{1}\lstinputlisting{exp2/src/fibonacci/Fibonacci.java}}

% {\linespread{1}\lstinputlisting{exp2/src/matrix/MatrixMultiply.java}}

% {\linespread{1}\lstinputlisting{exp2/src/max2ndElement/MaxSecondElement.java}}

% {\linespread{1}\lstinputlisting{exp2/src/maxElement/MaxElement.java}}

% {\linespread{1}\lstinputlisting{exp2/src/narcissistic/NarcissisticNumbers.java}}

% {\linespread{1}\lstinputlisting{exp2/src/perfect/PerfectNumber.java}}

% {\linespread{1}\lstinputlisting{exp2/src/pi/CalculatePI.java}}

% {\linespread{1}\lstinputlisting{exp2/src/prime/CalculatePrime.java}}

\clearpage
\section{Java中的类继承机制、接口}
\titl{2015}{3}{16}
\subsection{实验目的}
\begin{enumerate}
\item 实现Java中的类继承机制。
\item 体会继承的好处:重用和封装。
\end{enumerate}
\subsection{实验内容}
编写能够满足如下条件的程序:
\begin{enumerate}
\item
\begin{enumerate}
\item 声明一个Person类,有name(String类型)、age(int类型)、sex(char类型)属性。

      通过构造方法进行赋值。
	  
      一个show方法,返回String类型,内容如下:
         \begin{verbatim}某某 男(女) 年龄\end{verbatim}
\item 声明一个Student类,继承Person类,增加id(int,学号)属性,通过构造方法,利用super调用父类构造方法来进行变量赋值。Override父类的show方法,返回String类型,内容如下:
         \begin{verbatim}某某 男(女) 年龄 学号\end{verbatim}
提示:利用super调用父类的show方法得到除学号部分的String,然后加上学号的信息。
\item 声明一个Teacher类,继承Person,增加course(String,所教课程)属性,通过构造方法,利用super调用父类构造方法来进行变量赋值。Override父类的show方法,返回String类型,内容如下:
      \begin{verbatim}某某 男(女) 年龄 所教课程\end{verbatim}
提示:利用super调用父类的show方法得到除所教课程部分的String,然后加上所教课程的信息。
\item 声明PersonApp类,在其中的main方法中分别声明Person、Student、Teacher类型的变量,并通过构造方法初始化,然后显示各自的信息。
\end{enumerate}
\item 声明一个Shape接口,其中有计算面积(area)、周长(perimeter)的方法,有以下几个实现:Circle(圆),Rectangle(矩形),Triangle(三角形),都有计算面积、周长的方法。
\end{enumerate}

\clearpage
\section{Java中的输入机制}
\titl{2015}{3}{16}
\subsection{实验目的}
\begin{enumerate}
\item Java如何操作文件。
\item 了解Java中的输入机制:如何从控制台输入,如何用InputStream和Reader显示文件中的内容。
\end{enumerate}
\subsection{实验内容}
编写能够满足如下条件的程序(任选两个):
\begin{enumerate}
\item 递归遍历目录,显示其中的文件名。
\item 用InputStream的子类读入一个英文文本文件,并用System.out显示其中的内容。
\item 用Reader的子类读入一个一个字符文件,并用System.out显示其中的内容。
\item 从控制台输入Student类的信息,包括学号、姓名、年龄,如输入错误,提示用户重新输入。创建该类,并在toString方法中显示个人信息。
\item 增加1中的功能,显示文件的大小,目录的话显示其中包含的所有文件的大小。注意显示文件大小的单位(KB或MB)。
\end{enumerate}

%\clearpage
\section{Java的输出机制}
\hfill\ctliw{成绩}{}{7em}
\subsection{实验目的}
\begin{enumerate}
\item 掌握Java中的输出机制,会使用OutputStream、Writer输出。
\item 能够结合输入、输出拷贝文件内容。
\end{enumerate}
\subsection{实验内容}
编写能够满足如下条件的程序(任选两个):
\begin{enumerate}
\item 把从控制台输入的内容写入文件中。
\item 用InputStream和OutputStream拷贝一个图片。
\item 用Reader和Writer拷贝一个txt文件。
\end{enumerate}

\clearpage
\section{简单学生信息管理系统}
\titl{2015}{3}{16}
\subsection{实验目的}
\begin{enumerate}
\item 了解JDBC的作用,掌握通过JDBC访问数据库的方法。
\item 能够实现对数据库中数据的添加、删除、修改和查询。
\end{enumerate}
\subsection{实验内容}
注意:在写报告时,不需要写数据库连接的四个参数,也不需要写Class.forName(......),只要写上“Connection con=....”就行。其它的不能省略。

编写能够满足如下条件的程序:
\begin{enumerate}
\item (选做)声明Student类,该类实现Serializable接口以表明该类可以进行序列化。该类有姓名、学号(long),math、os、java用来存放对应的成绩,在构造方法中进行姓名、学号、课程成绩的赋值。Override由Object继承来的toString方法以便以友好格式显示自己的属性,格式为:
     \begin{verbatim}张三 12 os:90 java:90 math: 90 \end{verbatim}
\item 建立一个类,利用数据库来存储多个Student,写完一个方法在main中写一段测试代码,运行以保证目前所做工作的正确性。有以下方法:
\begin{enumerate}
     \item add(Student stu):可以向其中增加新的学生,并保存在数据库中。

	 测试add方法是否正确:用add方法向数据库增加一个新的学生,然后在数据库的图形管理界面中查询,确认是否增加。
     \item dispAll():可以显示所有的学生信息。
     \item findById(long id):可以按照学号来查找,并显示符合条件的学生信息,查无该人的话显示错误信息。
     \item findByName(String name):可以按照姓名查找学生,找到后显示其信息,查无此人显示错误信息。
     \item delById(long id):可以按照id删除学生的信息,然后显示找到该人。若查无此人,显示相应的错误信息。
\end{enumerate}
\item (选做)在控制台显示菜单,并实现相应的功能。菜单如下:
   \begin{verbatim}1 显示所有学生信息    2 按学号查找    3 按姓名查找
4 按学号删除   5按成绩排序 6 退出
请输入数字(1-6)\end{verbatim}
   用switch-case判断输入的内容。当输入2或4时,显示:
      \begin{verbatim}请输入学号:\end{verbatim}
   当输入3时,显示:
      \begin{verbatim}请输入姓名:\end{verbatim}
   当输入5时,显示:
      \begin{verbatim}1 按math成绩 2 按os成绩 3 按java成绩,请输入(1-3)\end{verbatim}
\item (选做)在3的菜单中增加“添加学生”的功能,当选中该项时,分别显示信息,提示输入姓名、学号、各科的成绩,保存该学生信息,并显示所有学生信息以证明信息已输入。
\item (选做)在JComboBox中显示所有的班级,在JList中显示该班的学生学号、姓名,双击时弹出对话框,显示该学生的信息。
\end{enumerate}


\end{document}
