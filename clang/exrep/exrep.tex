\documentclass[cs4size,a4paper,nofonts]{ctexart}
\usepackage[utf8]{inputenc}
\def\tjf{{\tt{田劲锋}}}
\usepackage[top=2.5cm,bottom=2.5cm,left=3.1cm,right=3.1cm]{geometry} % 页面设置
\usepackage[unicode,breaklinks=true,
colorlinks=true,linkcolor=black,anchorcolor=black,citecolor=black,urlcolor=black]{hyperref}
\usepackage{amsmath,amsopn}
\usepackage{graphicx} % 图形
\usepackage{color} % 颜色
\usepackage{xcolor} % 颜色
\usepackage{wallpaper} % 背景图片
\usepackage{listings} % 排版代码
\usepackage{verbatim} % 排版代码
\usepackage{url} % 排版链接
\usepackage{shortvrb}
\usepackage{fancyvrb}
\usepackage{nameref}
\usepackage{caption}
\captionsetup{font={small}} % 标题字体大小
\usepackage[inline]{enumitem} % 调整列表样式

% \setmainfont{Times New Roman}
\setCJKmainfont[BoldFont={SimHei}]{SimSun}  % 主要字体:宋体、黑体
\setCJKsansfont[BoldFont={STZhongsong}]{STFangsong} % 次要字体:仿宋、中宋
\setCJKmonofont{KFKai} % 等宽字体:楷体
\setCJKfamilyfont{liti}{隶书} \newcommand{\liti}{\CJKfamily{liti}}

\CJKsetecglue{\hspace{0.1em}}
\renewcommand\CJKglue{\hskip -0.3pt plus 0.08\baselineskip}
\frenchspacing
\widowpenalty=10000
\linespread{1}
\setlength{\parskip}{2pt plus 2pt}
\renewcommand{\baselinestretch}{1}

\setlength{\abovecaptionskip}{1pt}
\setlength{\belowcaptionskip}{0pt}
\setlength{\intextsep}{8pt}

\lstset{language=C,
  numbers=left,
  numberstyle=\tiny,
  basicstyle=\small\tt,
  commentstyle=\color{gray},
  keywordstyle=\bfseries\color{violet},
  stringstyle=\color{teal},
  showstringspaces=false,
  frame=tb,
  %     % identifierstyle=\sf,
%     breaklines=true,
  escapeinside=` `,
%     % texcl=true,  % 为true时,注释内容被当做LaTeX代码来处理
}

\DeclareMathOperator{\Ack}{Ack}

\makeindex
\pagestyle{plain}

\begin{document}

%%%% 开始 %%%%
\title{程序设计实践 实验报告}
\author{\tjf}
\date{201316920311}
\maketitle

\tableofcontents

\section{函数与程序结构}

\subsection{P123 调试示例}

主要是预编译指令的写法,和工程的文件包含。

\lstinputlisting[language=C,caption={\tt error10\_1\_main.c}]{../ex1/error10_1_main.c}
\lstinputlisting[language=C,caption={\tt error10\_1\_cal.c}]{../ex1/error10_1_cal.c}
\lstinputlisting[language=C,caption={\tt error10\_1\_vol.c}]{../ex1/error10_1_vol.c}

运行结果:
\begin{verbatim}
    1 - 计算球体体积
    2 - 计算圆柱体体积
    3 - 计算圆锥体体积
 其他 - 退出程序运行
请输入对应指令:1
请输入球体的半径:2
球体体积为:33.51
    1 - 计算球体体积
    2 - 计算圆柱体体积
    3 - 计算圆锥体体积
 其他 - 退出程序运行
请输入对应指令:3
请输入圆锥体的底面半径和高:2.4 3
圆锥体体积为:18.10
    1 - 计算球体体积
    2 - 计算圆柱体体积
    3 - 计算圆锥体体积
 其他 - 退出程序运行
请输入对应指令:2
请输入圆柱体的底面半径和高:2.4 3
圆柱体体积为:54.29
    1 - 计算球体体积
    2 - 计算圆柱体体积
    3 - 计算圆锥体体积
 其他 - 退出程序运行
请输入对应指令:4
\end{verbatim}

另外,一些复杂的预编译指令在 Microsoft Visual C++ 6.0 上是无法编译通过的,需要使用符合 C89 标准的编译器 如 GCC。

\subsection{P132 编程题 (5)}

实现阿克曼函数:
\[\Ack(m, n) = \left\{\begin{array}{ll}
n + 1, & m = 0 ,\\
\Ack(m - 1, 1), & n = 0 ,\\
\Ack(m-1, \Ack(m, n-1)), & m > 0 \text{\,且\,} n > 0 .\\
\end{array}\right.\]

使用递归可以实现:
\lstinputlisting[language=C,caption={\tt t5.c}]{../ex1/t5.c}

运行结果:
\begin{verbatim}
请输入 m: 2
请输入 n: 3
Ackerman(2, 3) = 9
\end{verbatim}

\subsection{P132 编程题 (7)}

求斐波那契数列项:
\[f(n) = \left\{\begin{array}{ll}
0, & n = 0, \\
1, & n = 1, \\
f(n - 2) + f(n - 1), & n > 1. \\
\end{array}\right.\]

使用递归完成:
\lstinputlisting[language=C,caption={\tt t7.c}]{../ex1/t7.c}

运行结果:
\begin{verbatim}
请输入 n: 6
fib(6) = 8
\end{verbatim}

\subsection{P132 编程题 (8)}

实现二进制转换函数 {\tt dectobin(n)},其中 $n$ 是正整数。

二除取余,倒序输出。仍然可以使用递归:
\lstinputlisting[language=C,caption={\tt t8.c}]{../ex1/t8.c}

运行结果:
\begin{verbatim}
请输入 n: 100
1100100
\end{verbatim}

\section{指针进阶}

指针虽好,减少滥用。

\subsection{P138 编程题 (2)}

输入星期的英文单词,输出对应序号。

顺序查找比较即可:
\lstinputlisting[language=C,caption={\tt t2.c}]{../ex2/t2.c}

运行结果:
\begin{verbatim}
Tuesday
3

Kinyoubi                           # 星期五(金曜日「きにょうび」)
-1
\end{verbatim}

\subsection{P138 编程题 (3)}

统计输入的最长字符串。

小心指针类型的问题,分清楚指针数组和数组指针。这里我每读入一个字符串,就分配相应大小的内存:
\lstinputlisting[language=C,caption={\tt t3.c}]{../ex2/t3.c}

运行结果:
\begin{verbatim}
n=4
ao                                 # 蓝色(青「あお」)
kiiro                              # 黄色(黄色「きいろ」)
aka                                # 红色(赤「あか」)
midori                             # 绿色(緑「みどり」)
6
\end{verbatim}

\subsection{P140 调试示例}
\subsection{P145 编程题 (1)}
\subsection{P145 编程题 (2)}
\subsection{P145 编程题 (3)}

\section{文件}

\subsection{P148 改错题}
\subsection{P150 编程题 (3)}
\subsection{P150 编程题 (7)}

%%%% 结束 %%%%

\end{document}
