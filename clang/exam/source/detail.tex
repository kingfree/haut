\section{详细设计}

\subsection{已实现功能}

整个系统有七个大类,上百个方法。由于时间关系,不能够一一说明,我这里拣几个比较主要的方法来说明其程序流程,其他代码实现还请查看附录~\ref{codes}。

\subsubsection{插入题目}

图~\ref{insert}~展示了插入题目的流程,首先实例化一个链表,将题目数据库读入,提示用户输入题目信息,将题目编号设置为最大编号加一,插入链表尾部,最后将链表写入文件,释放内存。

\begin{figure}[htp]
  \pictext\small
\begin{tikzpicture}
\begin{umlseqdiag}
\umlactor[class=UI]{ui}
\umlmulti[class=List, fill=blue!20]{list}
\umlobject[class=Problem, fill=green!20]{p}
\begin{umlcall}[op={list\_new()},return={list\_free(list)}]{ui}{list}
  \begin{umlcall}[op={problem\_read\_file(list)}]{list}{list}
  \end{umlcall}
  \begin{umlcall}[op={ui\_output\_count(list)}]{ui}{list}
  \end{umlcall}
  \begin{umlcall}[op={ui\_input\_problem()}, return={ui\_output\_problem(p)}]{ui}{p}
    \begin{umlcall}[op={p.id}, return={list.max\_id + 1}]{p}{list}
    \end{umlcall}
  \end{umlcall}
  \begin{umlcall}[op={list\_insert(list, p)}]{p}{list}
  \end{umlcall}
  \begin{umlcall}[op={problem\_write\_file(list)}]{list}{list}
  \end{umlcall}
\end{umlcall}
\end{umlseqdiag}
\end{tikzpicture}
  \caption{\label{insert}ui\_teacher\_insert()}
\end{figure}

删除和修改题目与此同理,都是将文件数据全部读入实例化的链表中,对链表进行操作,再将链表全部写入文件。这对于小数据量是有效且便利的。老师提出,删除时不应读入内存,存在大量数据的情况会使内存溢出。为了换取这样的空间代价,我们不得不用时间上非常慢的对硬盘文件的读取来实现,这也得不偿失。实际上,对文件的读写在操作系统里会映射为对内存的读写,操作系统还是会把文件读入内存(虽然不是全部,但一般文件没有大到足以分页),这种优化实际上是徒劳的。基于这种思想的删除和修改在实验最后一题里有实现(见实验报告),但不适用于本程序。

\subsubsection{模糊查询}

模糊查询实际上是统合了所有属性的查询,这种查询方式很类似于搜索引擎。当然相对于搜索引擎来说,还是非常简单的,不需要语法分析什么的,只是简单地关键字比对。

在输入关键字 key 之后,只需要调用 ui\_select\_output(list, by\_mul, key) 即可将所有查找结果输出出来。其中回调函数 by\_mul 实现如 Listing~\ref{by_mul},将关键字与各个域进行比对并返回元素本身。


\lstset{language=C,
  numbers=left,
  numberstyle=\tiny,
  basicstyle=\small,
  stringstyle=\tt,
  keywordstyle=\bfseries,
  keywordstyle=[2]{\bfseries\color{blue!50!black}},
  showstringspaces=false,
  breaklines=true,
  frame=tb,
  morekeywords=[1]{perror,assert,printf,fprintf,scanf,sscanf},
  morekeywords=[2]{Problem,List,SList,Paper,User,Score},
}

{\linespread{1}
\begin{lstlisting}[caption={\label{by_mul}by\_mul()}]
void *by_mul(SList *item, void *data)
{
    Problem *p = (Problem *)item->userdata;
    char *key = (char *)data;
    int a, b;
    if ('0' <= key[0] && key[0] <= '9') {
        sscanf(key, "%d", &a);
        if (p->id == a || p->tag == a || p->dif == a
         || p->chapter == a || p->section == a) {
            return item;
        }
        if (sscanf(key, "%d.%d", &a, &b) == 2
         && p->chapter == a && p->section == b) {
            return item;
        }
    }
    if (by_des(item, data) != NULL) {
        return item;
    }
    if (by_opt(item, data) != NULL) {
        return item;
    }
    return NULL;
}
\end{lstlisting}
}

\subsubsection{生成试卷}

生成试卷的算法有多个,其中按标签和按章节生成的算法比较类似。Listing~\ref{paper_generate_tags}~是按标签生成的算法,将题目列表和指定的题目数、标签列表传入,限制每个标签的题目数量不超过$n/m+1$,遍历题目列表凡是符合条件的加入试卷,直到满足数量或超出限制。

{\linespread{1}
\begin{lstlisting}[caption={\label{paper_generate_tags}paper\_generate\_tags()}]
int paper_generate_tags(Paper *pa, List *list, int n, int tags[], int m)
{
    if (list->count < n) {
        n = list->count;
    }
    SList *s = list->slist;
    Problem *p = NULL;
    int i = 0, j = 0;
    int k = n / m + 1;
    for (i = 0; i < m; i++) {
        j = 0;
        while ((s = (SList *)slist_find(s, by_tags, tags + i)) != NULL) {
            p = (Problem *)s->userdata;
            if (p->id == paper_insert_pid(pa, p->id)) {
                j++;
                if (j >= k) {
                    break;
                }
            }
            if (pa->length >= n) {
                goto end;
            }
            s = slist_tail(s);
        }
    }
end:
    return pa->length;
}
\end{lstlisting}
}

随机生成的算法比较简单,每次生成合法的随机题目编号插入试卷,直到达到指定数量为止。

按难度区间生成的算法,需要在区间中均匀分布题目难度,也有一个控制因子在起作用。

\subsubsection{用户登录}

用户登录逻辑比较简单,如果没有用户数据库,就注册一个教师用户 root;如果没有该用户名,就注册一个学生类型的账户。注册完后直接登录。

逻辑比较简单,方便用户操作,只是为了简单分模块逻辑。密码明文存储,也不提供修改密码什么的操作,没有必要。

\subsubsection{学生考试}

学生考试功能是 UI 类最复杂的,一共对三个类进行了操作,实例化了五个对象,逻辑比较复杂。Listing~\ref{ui_student_test}~展示了该方法的具体实现代码,首先读取试卷数据库,让用户选择试卷编号,实例化试卷;接着将试卷对应的题目读取出来,用用户和试卷来实例化成绩类,对每道题目调用 ui\_do\_problem 来做题并记录;最后读取成绩列表,插入成绩再写回硬盘。

{\linespread{1}
\begin{lstlisting}[caption={\label{ui_student_test}ui\_student\_test()}]
void ui_student_test(User *u)
{
    List *list = list_new();
    if (paper_read_list(list) < 0) {
        perror("读取试卷数据库失败");
        return;
    }
    list_each_call(list, ui_each_paper_show, list);
    int id = -1;
    printf("试卷编号:\n$ ");
    id = ui_input_number();
    Paper *p = (Paper *)list_find(list, by_id, &id);
    if (p == NULL) {
        perror("没有这套试卷");
        return;
    }
    List *problist = list_new();
    if (problem_read_file(problist) < 0) {
        perror("读取题目数据库失败");
        return;
    }
    cls();
    Score *s = score_new(u, p);
    paper_problem_call(p, problist, ui_do_problem, s);
    list_free(list);
    list_free(problist);

    List *scorelist = list_new();
    if (score_read_list(scorelist) < 0) {
        perror("读取成绩数据库失败");
        goto end;
    }
    s->id = scorelist->max_id + 1;
    if (list_insert(scorelist, s) < 0) {
        perror("插入成绩失败");
        goto end;
    }
    if (score_write_file(scorelist) < 0) {
        perror("写入成绩数据库失败");
        goto end;
    }
end:
    printf("考试完成!\n");
    ui_output_score(s);
    list_free(scorelist);
    pause();
}
\end{lstlisting}
}

\subsection{未实现功能}

\subsubsection{多项选择题}

系统完成的时候,老师提出了加入多选题功能的要求。由于时间关系,不再具体实现,大概叙述一下实现的思路。

在存储单项选择题的时候,我固定了四个选项,并且用一个字符表示正确选项的编号。实际上,选项可以增长到无限,正确答案也可能不止一个。那么如何存储呢?考虑使用$n$个二进制位来标识每个选项的正确与否,这样一个8位的字符实际上可以保存8个选项的状态。要实现选项答案的分离,只需要用到简单的位操作。Listing~\ref{get_bit}~实现了取二进制指定位的操作,并且支持用多个字符即字符串来存储更多选项。

{\linespread{1}
\begin{lstlisting}[caption={\label{get_bit}get\_bit()}]
unsigned char get_bit(unsigned char *bits, unsigned long i) {
  return (bits[i / 8] >> i % 8) & 1;
}
\end{lstlisting}
}

至于多项选择题的输入判断,可以直接读入字符串,进行大小写转换,取其 ASCII 编码值直接对应到每个选项中,取对应位进行对比判断即可。

至于评分标准,应该支持自定义。一般地,多选只要有错选就不给分,不选不给分。特殊的,少选的给部分分或者不给分。新课标高考物理的评分标准是“全部选对的得6分,选对但不全的得3分,有选错的得0分”。类似于这样的标准可以通过选项配置来解决。

\subsubsection{词法分析}

实际上我们的查询可以做得更好,利用状态机来将用户一定规则的输入与数据进行匹配,可以实现更为丰富的查询,甚至实现批量删除和修改。当然我们会发现这些功能都是 SQL 的标准配置。个人实现一个 SQL 还是比较困难的,尤其是没有编译原理的相关基础,所以这部分就放弃不做了。

\subsubsection{图形界面}

其实不管怎么说,这个系统的难度在于使用了大量的底层操作。终端界面有不少限制,并不适合这种互动性非常高的应用。想象了许多功能,在终端中运营都不会达到非常好的效果。终端的优势在于使用脚本语言处理大量文本数据,进行系统编程和系统管理维护。以普通用户为中心的应用程序,应当拥有良好的图形用户界面\footnote{关于用户界面的设计,参考《界面设计模式》\cite{dinterface}。}。

比较好的解决方案是采用 C/S 架构,分离底层与用户界面。本程序这点做的非常不好,也是水平所限。
