\section{详细设计}

整个系统有七个大类,上百个方法。由于时间关系,不能够一一说明,我这里拣几个比较主要的方法来说明其程序流程,其他代码实现还请查看附录~\ref{codes}。

\subsection{已实现功能}

\subsubsection{插入题目}

图~\ref{insert_problem}~展示了插入题目的流程。

\begin{figure}[htp]
  \pictext\small
\begin{tikzpicture}
\begin{umlseqdiag}
\umlactor[class=UI]{ui}
\umlmulti[class=List, fill=blue!20]{list}
\umlobject[class=Problem]{p}
\begin{umlcall}[op={list\_new()},return={list\_free(list)}]{ui}{list}
  \begin{umlcall}[op={problem\_read\_file(list)}]{list}{list}
  \end{umlcall}
  \begin{umlcall}[op={ui\_output\_count(list)}]{ui}{list}
  \end{umlcall}
  \begin{umlcall}[op={ui\_input\_problem()}, return={ui\_output\_problem(p)}]{ui}{p}
    \begin{umlcall}[op={p.id}, return={list.max\_id + 1}]{p}{list}
    \end{umlcall}
  \end{umlcall}
  \begin{umlcall}[op={list\_insert(list, p)}]{p}{list}
  \end{umlcall}
  \begin{umlcall}[op={problem\_write\_file(list)}]{list}{list}
  \end{umlcall}
\end{umlcall}
\end{umlseqdiag}
\end{tikzpicture}
  \caption{\label{insert_problem}插入题目的流程}
\end{figure}

\subsubsection{修改题目}
\subsubsection{模糊查询}
\subsubsection{生成试卷}
\subsubsection{用户登录}
\subsubsection{学生考试}


\subsection{未实现功能}

系统完成的时候,老师提出了加入多选题功能的要求。由于时间关系,不再具体实现,大概叙述一下实现的思路。






















