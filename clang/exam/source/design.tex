\section{概要设计}

实际上,C 作为系统编程语言,久经考验,使用好的方法便可以写出有良好表现和维护性的代码。按传统的过程式编程方法,对于一个较大型的项目来说,有点难以驾驭。

为了程序维护和修改方便,以及程序通用性,我采用了面向对象的编程方法,参考了面向对象和设计模式的相关书籍资料\footnote{主要是《设计模式解析》\cite{dpatt} \cite{dpatt_c}。}。
对于 C 语言来说,实现面向对象有点麻烦\footnote{学习《系统程序员成长计划》\cite{sys_grow} 中介绍的经验。},不过基本上还是模块化编程的思想\footnote{参考《模块化 C:怎样编写可重用可维护的 C 语言代码》\cite{modc}。}。

\subsection{类的划分}

\begin{figure}[htp]
  \pictext\small
\begin{tikzpicture}
\umlemptyclass{UI}
\umlemptyclass[x=5, y=2]{SList}
\umlemptyclass[x=3, y=2]{List}
\umlemptyclass[x=-6, y=1]{Paper}
\umlemptyclass[x=-3, y=0]{Problem}
\umlemptyclass[x=-4, y=-3]{Score}
\umlemptyclass[x=2, y=-4]{User}
\umlaggreg[geometry=-|-]{List}{SList}
\umlcompo[geometry=|-]{Paper}{Problem}
\umlcompo[geometry=|-|]{Score}{Problem}
\umlcompo[geometry=-|]{Score}{Paper}
\umlaggreg[geometry=|-]{Score}{User}
\umldep[geometry=-|]{UI}{User}
\umldep[geometry=|-]{UI}{Paper}
\umldep[geometry=|-]{UI}{Problem}
\umldep[geometry=|-]{UI}{Score}
\umldep[geometry=-|]{UI}{List}
% \umluniassoc[geometry=|-,mult=0..*]{Paper}{List}
% \umluniassoc[geometry=|-|,mult=0..*]{Problem}{List}
% \umluniassoc[geometry=--,mult=0..*]{Score}{List}
\end{tikzpicture}
  \caption{\label{classes}各个类之间的关系}
\end{figure}

通过研究程序中出现的对象,我们可以抽象出若干个类。图~\ref{classes}~展示了各个类之间的关系。

首先,我决定使用 UI 类来控制所有与用户交互有关的操作,所有屏幕可显示字符串都存放在 UI 类中,便于修改和移植。

最重要的类是 Problem 类,提供题目存取和相关操作。Problem 类聚合可以生成试卷,即 Paper 类,用来存取试卷信息。学生和教师共用 User 类来存取,其中学生的成绩使用 Score 类存取。Score 类由 Paper 类和 User 类聚合而成,并且依赖于 Problem 类。

由于没有数据库,我设计一个 List 类来实现对文件的读取、写入和增、删、改、查操作。通过 UI 类将 List 类与其他类连接起来操作(这里也是本程序设计的不佳之处)。List 类是对单链表 SList 类的一个封装。

\subsection{List 类}

\newcommand\method[2]{{#1}:~{\it #2}}
\newcommand\vart[2]{{#1}:~{\it #2}}
\newcommand\argu[2]{{\sf #2}:~{\it #1}}
\newcommand\comt[1]{\hfill\quad{\tt //#1}}
\newcommand\comtn[1]{\\\comt{#1}}

List 类(图~\ref{list})存储一个链表,记录链表元素个数和元素中最大编号。除了初始化和销毁操作之外,提供从文件中读取、向文件中写入功能。对于链表本身,提供插入、删除、查找和遍历、查找遍历功能,通过传入回调函数来实现具体功能。

\begin{figure}[htp]
  \pictext\small
\begin{tikzpicture}
\umlclass{List}{
-- \argu{SList *}{slist} \comt{单链表头指针}\\
+ \argu{int}{max\_id} \comt{链表中元素最大编号}\\
+ \argu{int}{count} \comt{链表中元素个数}
}{
+ \method{list\_new()}{List *} \comt{创建链表}\\
+ \method{list\_free(\argu{List *}{list})}{void} \comt{销毁链表}\\
+ \method{read\_file\_to\_list(\argu{char *}{filename}, \argu{List *}{list}, \argu{size\_t}{size})}{int} \comt{读取文件到链表}\\
-- \method{list\_restore(\argu{List *}{list})}{void} \comt{排序整理链表}\\
\# \method{file\_data\_count(\argu{FILE *}{file}, \argu{size\_t}{size})}{int} \comt{获取文件大小}\\
+ \method{write\_list\_to\_file(\argu{char *}{filename}, \argu{List *}{list}, \argu{size\_t}{size})}{int} \comt{写出链表到文件}\\
+ \method{list\_insert(\argu{List *}{list}, \argu{void *}{p})}{int} \comt{插入到链表}\\
+ \method{list\_remove(\argu{List *}{list}, \argu{回调函数}{find}, \argu{void *}{data})}{void *} \comt{从列表删除}\\
+ \method{list\_find(\argu{List *}{list}, \argu{回调函数}{find}, \argu{void *}{data})}{void *} \comt{在列表中查找}\\
+ \method{list\_each\_call(\argu{List *}{list}, \argu{回调函数}{call}, \argu{void *}{data})}{void *} \comt{遍历链表}\\
+ \method{list\_find\_each\_call(\argu{List *}{list}, \argu{回调}{find}, \argu{void *}{data}, \argu{回调}{call}, \argu{void *}{userdata})}{void *} \comtn{遍历链表中符合条件的元素}\\
}
\end{tikzpicture}
  \caption{\label{list}List 类}
\end{figure}

List 类实际上是对单链表 SList 类的一个封装,同时提供了文件操作,在程序中实际上是作为数据库类存在的。

\subsection{SList 类}

SList 类是 GNU Libltdl 库的组成部分之一,经过了长时间的考验,是非常健壮的。尊重 LGPL 协议,我未对其代码做任何改动。本来我是想改动 SList 类来实现现在 List 类的功能的,但是我还是决定尊重版权,并且提高代码重用性。

\begin{figure}[htp]
  \pictext\small
\begin{tikzpicture}
\umlclass{SList}{
-- \argu{SList *}{next} \comt{指向下一元素地址的指针}\\
+ \argu{const void *}{userdata} \comt{指向数据域指针}\\
}{
+ \method{slist\_concat(\argu{SList *}{head}, \argu{SList *}{tail})}{SList *} \comt{链接两个单链表}\\
+ \method{slist\_cons(\argu{SList *}{item}, \argu{SList *}{slist})}{SList *} \comt{插入链表元素}\\
+ \method{slist\_delete(\argu{SList *}{slist}, \argu{回调函数}{delete\_fct})}{SList *} \comt{销毁链表}\\
+ \method{slist\_remove(\argu{SList **}{phead}, \argu{回调函数}{find}, \argu{void *}{matchdata})}{SList *} \comt{删除指定元素}\\
+ \method{slist\_reverse(\argu{SList *}{slist})}{SList *} \comt{反转链表}\\
-- \method{slist\_sort\_merge(\argu{SList *}{left}, \argu{SList *}{right}, \argu{比较函数}{compare}, \argu{void *}{userdata})}{SList *}\\
+ \method{slist\_sort(\argu{SList *}{slist}, \argu{比较函数}{compare}, \argu{void *}{userdata})}{SList *} \comt{快速排序}\\
+ \method{slist\_tail(\argu{SList *}{slist})}{SList *} \comt{取下一个元素}\\
+ \method{slist\_nth(\argu{SList *}{slist}, \argu{size\_t}{n})}{SList *} \comt{取第$n$个元素}\\
+ \method{slist\_find(\argu{SList *}{slist}, \argu{回调函数}{find}, \argu{void *}{matchdata})}{void *} \comt{查找指定元素}\\
+ \method{slist\_length(\argu{SList *}{slist})}{size\_t} \comt{取链表长度}\\
+ \method{slist\_foreach(\argu{SList *}{slist}, \argu{回调函数}{find}, \argu{void *}{userdata})}{void *} \comt{遍历每个元素}\\
+ \method{slist\_box(\argu{const void *}{userdata})}{SList *} \comt{封装数据}\\
+ \method{slist\_unbox(\argu{SList *}{item})}{void *} \comt{拆包数据}\\
}
\end{tikzpicture}
  \caption{\label{slist}SList 类}
\end{figure}

图~\ref{slist}~描述了 SList 类。每个 SList 对象都是一个链表元素,包含指向下一元素地址的指针和本结点元素的数据指针。

SList 类提供了链表操作的通用方法,包括附加、连接、销毁、删除、查找、遍历操作,还可以对其进行排序(使用快速排序)、反转,一般的取长度和取指定元素方法也必不可少。

\subsection{Problem 类}

\begin{figure}[htp]
  \pictext\small
\begin{tikzpicture}
\umlclass{Problem}{
+ \argu{int}{id} \comt{题目编号}\\
+ \argu{char[]}{des} \comt{题目描述}\\
+ \argu{char[4][]}{opt} \comt{选项}\\
+ \argu{char}{ans} \comt{答案}\\
+ \argu{short}{dif} \comt{难度系数}\\
+ \argu{short}{tag} \comt{知识点标签}\\
+ \argu{short}{chapter} \comt{章}\\
+ \argu{short}{section} \comt{节}\\
}{
+ \method{problem\_new()}{Problem *} \comt{实例化题目}\\
+ \method{problem\_read\_file(\argu{List *}{list})}{int} \comt{读取题目数据库}\\
+ \method{problem\_write\_file(\argu{List *}{list})}{int} \comt{写出题目数据库}\\
}
\end{tikzpicture}
  \caption{\label{problem}Problem 类}
\end{figure}

图~\ref{problem}~所示的 Problem 类,有着题目的数据域。为了简便起见,采用四个选项的单选形式。

题目可以通过访问数据域直接操作,所以其本身的方法并不是很多,都托管给 UI 类了。

\subsection{Paper 类}

\begin{figure}[htp]
  \pictext\small
\begin{tikzpicture}
\umlclass{Paper}{
+ \argu{int}{id} \comt{试卷编号}\\
+ \argu{int}{length} \comt{试卷题目数}\\
+ \argu{int[]}{pid} \comt{题目编号数组}\\
+ \argu{char[]}{title} \comt{试卷标题}\\
}{
+ \method{papar\_new()}{Papar *} \comt{实例化试卷}\\
+ \method{papar\_free(\argu{Paper *}{pa})}{void *} \comt{析构试卷}\\
+ \method{papar\_read\_list(\argu{List *}{list})}{int} \comt{读取试卷数据库}\\
+ \method{papar\_write\_list(\argu{List *}{list})}{int} \comt{写出试卷数据库}\\
+ \method{paper\_insert\_pid(\argu{Paper *}{pa}, \argu{int}{pid})}{int} \comt{插入题目}\\
+ \method{paper\_problem\_call(\argu{Paper *}{pa}, \argu{List *}{list}, \argu{回调}{call}, \argu{void *}{userdata})}{void} \comtn{遍历试卷题目}\\
+ \method{paper\_generate\_random(\argu{Paper *}{pa}, \argu{List *}{list}, \argu{int}{n})}{int} \comtn{随机生成试卷}\\
+ \method{paper\_generate\_tags(\argu{Paper *}{pa}, \argu{List *}{list}, \argu{int}{n}, \argu{int[]}{tags}, \argu{int}{m})}{int} \comtn{按标签生成试卷}\\
+ \method{paper\_generate\_secs(\argu{Paper *}{pa}, \argu{List *}{list}, \argu{int}{n}, \argu{double[]}{secs}, \argu{int}{m})}{int} \comtn{按章节生成试卷}\\
+ \method{paper\_generate\_dif(\argu{Paper *}{pa}, \argu{List *}{list}, \argu{int}{n}, \argu{int}{a}, \argu{int}{b})}{int} \comtn{在难度区间$[a,b]$生成试卷}\\
}
\end{tikzpicture}
  \caption{\label{paper}Paper 类}
\end{figure}

图~\ref{paper}~所示的 Paper 类存储试卷,用数组来存储题目数据库中的题目编号(所以试卷依赖于题目)。

在试卷中插入题目,要求题目不重复,方法会遍历题目数组保证题号唯一。生成试卷时指定的题目数量如果过多,会自动缩减到合适大小。

\subsection{User 类}

\begin{figure}[htp]
  \pictext\small
\begin{tikzpicture}
\umlclass{User}{
+ \argu{int}{id} \comt{用户编号}\\
+ \argu{char[]}{username} \comt{用户名}\\
-- \argu{char[]}{passwd} \comt{密码}\\
+ \argu{bool}{teacher} \comt{是否为教师}\\
}{
+ \method{user\_new()}{User *} \comt{实例化用户}\\
+ \method{user\_free(\argu{User *}{u})}{void *} \comt{析构用户}\\
-- \method{user\_init()}{int} \comt{初始化根用户}\\
+ \method{user\_reg(\argu{User *}{u})}{int} \comt{用户注册}\\
+ \method{user\_login(\argu{User *}{u})}{int} \comt{用户登录}\\
}
\end{tikzpicture}
  \caption{\label{user}User 类}
\end{figure}

图~\ref{user}~是 User 类,这个类自己在内部实现了文件操作。

密码本来应该是密文存储,但是没有必要为这样一个玩具系统写 Hash 函数,不如明文存储。这里目的是展示用户登录逻辑,没有必有实现特别复杂的用户管理。

\subsection{Score 类}

\begin{figure}[htp]
  \pictext\small
\begin{tikzpicture}
\umlclass{Score}{
+ \argu{int}{id} \comt{成绩编号}\\
+ \argu{int}{user\_id} \comt{用户编号}\\
+ \argu{char[]}{username} \comt{用户名}\\
+ \argu{int}{paper\_id} \comt{试卷编号}\\
+ \argu{int}{paper\_count} \comt{试卷题目数}\\
+ \argu{char[]}{answer} \comt{用户提交的答案}\\
+ \argu{int}{right} \comt{用户题目正确个数}\\
+ \argu{time\_t}{date} \comt{开始做题时间}\\
-- \argu{int}{now} \comt{当前做到的题目指针}\\
}{
+ \method{score\_new(\argu{User *}{u}, \argu{Paper *}{p})}{Score *} \comt{实例化成绩}\\
+ \method{score\_read\_list(\argu{List *}{list})}{int} \comt{读取成绩数据库}\\
+ \method{score\_write\_list(\argu{List *}{list})}{int} \comt{写出成绩数据库}\\
+ \method{score\_did(\argu{Score *}{s}, \argu{char}{c}, \argu{char}{ans})}{int} \comt{记录做题}\\
}
\end{tikzpicture}
  \caption{\label{score}Score 类}
\end{figure}

图~\ref{score}~是 Score 类,实现了成绩的记录。

\subsection{公共回调函数}
\label{sec_callback}

Slist 类提供了两个接口,其中回调函数接口提供了对 SList 对象的带参数调用操作,排序比较函数接口用于 SList 对象的大小比较,类似于 \verb|qsort()| 的比较函数。

\begin{figure}[htp]
  \pictext\small
\begin{tikzpicture}
\umlclass[type=interface]{SListCallback}{
\# \argu{SList *}{item}\\
\# \argu{void *}{userdata}\\
}{
+ write\_userdata() \comt{写出数据到文件}\\
+ by\_id() \comt{各类编号(查找)}\\
+ by\_des() \comt{题目描述(查找)}\\
+ by\_opt() \comt{题目选项(查找)}\\
+ by\_difr() \comt{题目难度区间(查找)}\\
+ by\_difs() \comt{题目难度(查找)}\\
+ by\_tags() \comt{题目标签(查找)}\\
+ by\_secs() \comt{题目章节(查找)}\\
+ by\_mul() \comt{模糊查询(查找)}\\
+ by\_user\_id() \comt{试卷中的用户编号(查找)}\\
}
\end{tikzpicture}  \caption{\label{callback}回调函数}
\end{figure}

回调函数可以用于查找、输出,主要传递给 foreach 方法进行调用。由于传递进去的参数只有一个泛型指针,所以我设计了 sel\_num 和 ID 、 Block 数据类型来存储和传递参数,具体实现比较丑陋,还有待改进。遵循此接口的方法列在图~\ref{callback}~中。

\begin{figure}[htp]
  \pictext\small
\begin{tikzpicture}
\umlclass[type=interface]{SListCompare}{
\# \argu{SList *}{item1}\\
\# \argu{SList *}{item2}\\
\# \argu{void *}{userdata}\\
}{
+ cmp\_id() \comt{比较编号}\\
}
\end{tikzpicture}  \caption{\label{compare}比较函数}
\end{figure}

比较函数用于排序,目前只有按编号排序(图~\ref{compare}),并且隐藏在读入方法里调用。其他的关键字排序可以以此接口来编写调用,因为用途不大,不再编写。

\subsection{UI 类}

UI 类实际上相当于一个上帝类,掌控全局,并且可以访问其他类的几乎所有方法和属性。
