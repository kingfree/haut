\newpage
\newcommand\tasktext{
开发出一个标准化考试系统,所谓标准化考试系统即仅支持选择题,也是为方便自动批改的功能的实现。要求实现以下{\bf 基本功能}:
\begin{asparaenum}[(1)]
\item 提供给教师添加试题的功能(试题信息用文件保存)——输入;
\item 试题的整体浏览功能;
\item 能够抽取试题组合成一套试卷(组卷的策略:可以是随机的,当然若教师添加的试题时有知识点、章节等信息,亦可以实现按照一定的组卷策略实现出题:如每个知识点抽取若干题目,最终组合一套试卷);
\item 教师实现题目信息的管理,比如删除、修改等;
\item 查询功能(至少一种查询方式)、排序功能(至少一种排序方式)。
\end{asparaenum}
\CTEXnoindent

{\bf 扩展功能:}可以按照自己的程度进行扩展。比如
\begin{inparaenum}[(1)]
\item 简单的权限处理;
\item 成绩报表打印功能;
\item 甚至根据自己情况,可以加上学生信息和考试成绩信息的管理,并扩充为{\bf 广义}的考试系统。即学生输入账号密码登陆,进行考试,交卷后显示成绩;
\item 模糊查询;
\item 综合查询;
\item 统计、分析等功能。
\end{inparaenum}
总之,可以根据自己需求进行分析功能。

{\bf 特别说明:}尽可能地运用自己已经学习过的数据结构的知识去展现。
}
\begin{center}
\section*{课程报告任务书}
% \linespread{1}
\zihao{5}
\begin{tabular}{|c|c|c|c|c|c|c|c|c|}\hline
{\bf 题目} & \multicolumn{8}{c|}{\parbox[c][8.1mm][c]{10em}{\titlec}} \\\hline
\multirow{1}{2em}{\bf 主要\\内容} & \multicolumn{8}{l|}{\begin{minipage}{36em}
\CTEXindent\quad\par
\tasktext\par
\quad\par
\end{minipage}}\\\hline
\multirow{1}{2em}{\bf 任务\\要求} & \multicolumn{8}{l|}{\begin{minipage}{36em}
\CTEXindent\quad\par

一、提交材料应包括:
\begin{inparaenum}[(1)]
\item 系统源代码
\item 课程报告
\end{inparaenum}

二、整个设计过程具体要求
\begin{asparaenum}[(1)]
\item {\bf 需求分析}\quad 要求学生对案例系统进行分析,设计出需要完成的功能,完善各个模块的调用关系;
\item {\bf 设计过程}\quad 要求学生进一步明确各模块调用关系,进一步完善模块函数细节(函数名、参数、返回值等);
\item {\bf 实现过程}\quad 要求学生养成良好的编码习惯、完成各个模块并进行测试,最终完成系统整体测试;
\item {\bf 总结阶段}\quad 按照要求完成系统设计和实现报告,并进行总结、答辩。
\end{asparaenum}
\quad\par
\end{minipage}}\\\hline
\multirow{5}{2em}{\bf 成绩\\评定}
  & \multicolumn{3}{c|}{报告撰写情况(30分)} & \multicolumn{2}{c|}{系统完成情况(30分)} 
  & \multicolumn{2}{c|}{答辩情况(40分)} & \multirow{3}{2em}{总分}           \\\cline{2-8}
& 内容 & 规范程度 & 程序测试 & 基本功能 & 扩展功能 & 自述情况 & 答辩情况    & \\
& 20分        & 5分      & 5分      & 20分     & 10分     & 10分     & 30分 & \\\cline{2-9}
&             &          &          &          &          &          &      & \\[1em]\cline{2-8}
& \multicolumn{3}{c|}{} & \multicolumn{2}{c|}{} & \multicolumn{2}{c|}{} &     \\[1em]\hline
\end{tabular}
\end{center}

{\CTEXnoindent \zihao{-3} \bf 成绩评定教师:}

\newpage

% 三、评分细则
% 1. 报告撰写情况
% 1)内容
% 包含程序流程图、数据结构定义,程序调试分析透彻,总结充分得19~20;有流程图但无数据结构定义得16~18分;有简单的需求分析及初步设计、过于简略,但报告基本完整得12~14分
% 2)规范程度
% 格式规范,无明显格式错误,报告整洁得5分;出现明显错行、不当居中等格式混乱得3~4分
% 3)程序测试
% 调试分析充分,有测试数据,功能测试抓图完整得5分,有分析但不详实得4分,无分析但有完整功能测试抓图得3分;有简单的分析、测试,但功能测试抓图不完整得2分
% 2. 系统完成情况
% 1)基本功能
% 按任务书的要求,能实现规定的信息录入、文件保存、信息浏览输出、查询功能、排序及信息的删除与修改等基本功能得20分;基本功能少1项扣2~3分;
% 2)扩展功能
% 每实现扩展功能1项加2分,但最高不超过10分
% 3. 答辩情况
% 1)自述情况
% 能完整叙述系统开发的基本情况,系统的优点、缺点,开发过程的难点及设计收获的得10分;其余的酌情给分。
% 2)答辩情况
%     能准确无误地回答答辩老师提出的质询得30分;回答问题较为准确的得27~28分;不能完整回答问题,对答辩老师提出的个别问题不理解的得24~26分;回答问题不准确,能看出所开发系统的全部或部分非本人开发的酌情得18~23分。