\documentclass[cs5size,b5paper,nofonts]{ctexart}
\usepackage[utf8]{inputenc}
\def\tjf{{\tt{田劲锋}}}
\def\titlec{现代萌系动画的美术特征}
\def\titlee{我的《美术鉴赏》课}
\usepackage[b5paper,margin=2cm]{geometry} % 页面设置
\usepackage[unicode,breaklinks=true,
colorlinks=true,linkcolor=black,anchorcolor=black,citecolor=black,urlcolor=black,
pdftitle={\titlec},pdfauthor={\tjf}]{hyperref}
\usepackage{multicol} % 分栏
\CTEXsetup[number=\chinese{section}, format={\large\sf\bfseries}]{section}

\setmainfont{Times New Roman}
\setCJKmainfont[BoldFont={SimHei}]{SimSun}  % 主要字体:宋体、黑体
\setCJKsansfont[BoldFont={SimHei}]{STFangsong} % 次要字体:仿宋、中宋
\setCJKmonofont{KFKai} % 等宽字体:楷体
\setCJKfamilyfont{mincho}{MS PMincho} \newcommand{\ja}{\CJKfamily{mincho}}


\CJKsetecglue{\hspace{0.1em}}
\renewcommand\CJKglue{\hskip -0.3pt plus 0.08\baselineskip}
\frenchspacing
\widowpenalty=10000
\linespread{1.2} % 行距

\makeindex
\pagestyle{plain}

\begin{document}

%%%% 开始 %%%%

% 格式:题目(二号字) 正文(五号字)
% 字数:2000以上

\title{{\zihao{2}\titlec} \\ ——\titlee}
\newcommand{\ctline}[2]{\makebox[2em][s]{\bf #1}:\underline{\makebox[8em][c]{\, #2\,}} \quad}
\author{\ctline{姓名}{\tjf} 
\ctline{学校}{河南工业大学} \\
\ctline{班级}{软件 1305 班} 
\ctline{学号}{201316920311}
}
\maketitle

\begin{abstract}
以美少女为主体的日本萌系动画的美术特征,是基于素描技巧,对眼睛和发型的夸张,面部的简化,身材的协调以及服饰的得体体现出来的。而“美少女”一词的定义,具体地来说,年龄设定多在12--25岁之间,以高中生和初中生为主,有着这样的特征:
较大的眼睛、
较小的鼻子、
较扁平的脸、
较高的瞳孔、
约5--7头身的身材、
较细小的肢体、
相对较大的头、
多彩多色的头发、
幼化的脸。
以水手服为主的美少女制服也是萌系角色重要的美术特征。
文章就这些内容展开较为详细地阐述和介绍,以期能够分享对于萌系动画美的感受。

{\bf 关键词:}日本; 动画; 美少女; 萌; 作画; 上色; 水手服; 美术

\end{abstract}

\begin{multicols}{2}

拿到这个题目以后我是很不安的。本来论文是要有论点的,或是展示技术,或是解释现象,或是辨析是非。所谓“\titlee ”,是要写写上课的感想,发一篇课后感吗?这根本不是什么论文嘛。或者是把课上讲的东西复述一遍,这也是没有什么创造的重复劳动。好的办法是就其中一点展开来阐述,针对某个单独的问题来进行研究。我个人的着眼点,还是希望能够在擅长的领域下手,针对日本动画来进行一些研究\footnote{实际情况是,本身熟悉的书法领域,因为很久不接触了,逐渐变得生疏,不敢拿十年前的认识来班门弄斧。而对于一般意义上的绘画、工艺、建筑、雕塑等,本人自以为还是相当不了解,更不敢拈指一二。}。这个领域在课上并没有讲到,但是确实是本人非常想探究的一个东西。由于日本动画本身也是覆盖面非常广的,我单纯选取萌系动画为研究对象,相对浅显地探寻相关内容。

\bigskip

首先是对于“动漫”“动画”“漫画”这几个词的辨析,这是圈内的老生常谈,是面向圈外人第一道科普。在日本,是没有动漫一词的,动画(Anime/Animation)、漫画(Manga/Comics)、游戏(Game)都是不同的概念。国内称呼动漫则是对动画和漫画的统称,并非单指动画。中文圈统称动画、漫画、游戏则是ACG这一缩略词,若加上轻小说(Light Novel)则统称ACGN,是比较广泛接受的称呼。

科普不是本文的重点,所以关于日本动画的制作以及相关产业的问题不再赘述,如有兴趣可参考相关文献。这里单就动画中的萌系动画进行研讨,由于美术鉴赏的性质,内容着重放在对于作画、上色、摄影这些美术相关的内容上。

\pagebreak

要研究萌系动画,首先要理解“萌”这一词的含义。这一词本义在中国还是日本都是动词,意为植物发芽。由于一系列演变,与{\ja 「燃え」}(moe)同音的形容词{\ja 「萌え」}被创造了出来,专用以形容可爱的美少女。这一概念随着美少女恋爱游戏(Gal Game)的兴盛传播开来,并逐渐发生了含义的扩充。萌不再单纯地形容美少女,而是可以用于一切可以让人产生喜爱之情的事物上面;萌在形容词上的这层含义也扩充到了动词——这是女性群体创造出来并广泛使用的含义。虽然在一定程度上“萌”可以用“可爱”“喜欢”来替代,但是由于其本身意义一直在发生扩充,并不是所有场合都可以充分解释清楚萌的含义的。而“美少女”一词的定义,具体地来说,年龄设定多在12--25岁之间,以高中生和初中生为主,有着这样的特征:
较大的眼睛、
较小的鼻子、
较扁平的脸、
较高的瞳孔、
约5--7头身的身材、
较细小的肢体、
相对较大的头、
多彩多色的头发、
幼化的脸。
这里对萌系动画划下一个界限来避免分歧,单指以美少女为主角的动画作品。

我所要讨论的萌系动画,再加上现代一个限定词,则是指的九十年代以及之后的萌系动画作品。萌系动画再分,则根据有无男性主角,划分成百合动画和后宫动画。早期的萌系动画多是有男主的,而随着受众群体对代入感需求的逐渐减退,男主的地位逐渐弱化,以描写女生之间故事的百合动画越来越多。典型的后宫动画有(编码序):
《CLANNAD》
《Fate》系列、
《凉宫春日的忧郁》
《出包王女》
《只有神知道的世界》
《天降之物》
《旋风管家》
《某魔法的禁书目录》
《灼眼的夏娜》
《零之使魔》
等;而偏向于百合动画的有:
《Love Live!》
《偶像大师》系列、
《光之美少女》系列、
《天才麻将少女》
《幸运星》
《摇曳百合》
《某科学的超电磁炮》
《美少女战士》
《轻音少女》
《魔卡少女樱》
《魔法少女奈叶》
《魔法少女小圆》
等。列举的只是相对比较有名的作品,相对于日本动画产量可谓是九牛一毛。下面我结合具体的作品来阐述它们是如何展示出“萌”这一元素的。由于动画相关图片具有版权,在此不再刊登,有兴趣可以在网络上进行搜索。

\smallskip

能够令人唤起萌的感觉的角色特征,称为萌属性。通过萌属性的堆砌,可以创造出来一个有特色的美少女。萌系动画之所以为萌系动画,在于其人设(人物设定)是美少女。人设决定了角色在美术上萌属性的表现。人设,首先是对于线条的把握。传统的铅笔作画——即基于素描的漫画绘制法是最得人心的办法,通过草稿的擦写来产生逐渐明晰人物的形象。草稿既定,一般使用蘸水笔或者针笔对其描边,是所谓清稿、誊稿。线条这时已经描述了角色的绝大多数特征,而黑色的描边线条也会被保留到最终版本,这正是日本动画的特征——相对于CG动画刻意抹去线条而言。一张原稿出来,要进行上色才能算得上完整的人设。实际上由于黑白漫画的统治地位,少见直接在原稿上用水彩色进行上色作业的。对于漫画家,他们需要将涂黑的地方单纯地涂黑,需要颜色的部分用网点纸粘贴上去。对于动画原画师,则使用彩色铅笔标记颜色变化的轮廓,如高光、阴影等;接下来就是扫描入计算机,由仕上来进行上色。动画的上色多采用色块填充,这就是用线条标记的好处,对于节省成本和时间来说这是个最优解。Gal Game之类的单张非动画,则会上色得更为细致,包括渐变、光影、透明感都会表现出来,这也是建立在时间和资金上面的质量提升。总之,原画和仕上完成后,接下来就是交给摄影来将一张张画面连接起来,配合背景,加入补帧和特效等,来组合成可以动的动画。

回到人物特征的表现,一般来说,头部是最主要的部分,而决定人物特征的则是眼睛和发型。头部是卵圆形,一般会将下颌修尖。根据“三庭五眼”的规则,眼睛被安排在合适的位置上。眼睛很大,在一定程度上越大越可爱,一般来说会有脸长的1/5--1/3高度,宽度则标准地浮动在1/5脸宽。瞳色以黑色为基调,从四周向中间渐变成彩色,正中间是黑色的瞳孔,通常与眼珠区别得比较明显。角色的灵魂在于高光,在眼球上部会有一个相当大的白点作为高光,下部也会有弧形的高亮线,以及在相应位置的特别高光处理——可以说一个角色就这么完成了,只需要眼睛的传神。眼白比较大且不会过多刻画,上睫毛的刻画相对比较丰富,基本上都是黑色并且较粗,下睫毛和眉毛一般都被一笔带过。

次之重要的是发型,有长发、短发、中长发这些常见的长短分别。与瞳色一样,发色也是表现人物特征的有效手段。粉色常常作为女主人公的发色,金黄色则会显得角色的气质非凡,蓝色和绿色也是常见的战队发色, 红色会让人感受到热情,紫色会塑造出角色奇异的处境。相对的贴近自然发色的棕色、橙色、偏黑的深色也是通常的形象塑造色,而黑色则是非常传统的发色,少量运用会产生不错的效果。“黑长直”便指的是的黑色长发,是一种经典的萌属性。发型也分很多种,马尾、双马尾是其中最有特色的了,尤其是双马尾已经成为了一种文化影响深刻。“双马尾”一词在日语中读作“{\ja ツインテール}(Twin Tail)”,与“傲娇”的发音“{\ja ツンデレ}(tsun dere)”相似,所以用双马尾来表现人物的傲娇性格已经成为了一个惯例。日本动画最特别地是创造出来了“呆毛”这一萌属性,指的是头发丛中翘起的一缕或几缕头发。呆毛随着人物心情发生的抖动是人物传神的重要途径之一。

眼睛和发型已经展示出来了人物的大部分特征,而口鼻相对会弱化。鼻子会弱化到一点甚至没有,而嘴巴则根据表情产生几个不同的变化。由于日语的元音只有“a i u e o”五个,那么人物的嘴型只需要五个便可以完全表示语言。嘴巴的弱化在于对嘴唇的消失化,以及牙齿和舌头的简化。如果对这些进行加强,则会产生审美的反效果,会用在展现特殊表情上面,用于丑化表情和性格。五官除此之外还有耳朵,这个也是经过简化的,同样的越写实约丑陋。部分角色会拥有所谓“兽耳”属性,这不再属于人类特征,不予讨论。

头部占身体高度一般在1/7--1/5之间,称为“七头身”“五头身”等。人物的特征主要在于头部,而身体肢体的动作表现则是叙事的手段。关于人体构造以及动作捕捉相关,这里略去,直接步入对于人物服饰的鉴赏。

校服,即水手服,是传统且普遍的美少女着装。水手服整体一般是白色;有着大的领子,在前方折叠成三角形,背后是三角形或者方形,通常是蓝色或红色,也有其他颜色,一般会加上白色线条用印装饰;红领巾是水手服的一大特征,但也不全是红色的。裙子是百褶裙,是对应上衣的单一颜色,设有隐藏的口袋。一般会配套白色的袜子,从短袜、中筒袜、过膝袜各种都有,非白色的彩色袜子也有,条纹也是存在的。鞋子一般是皮鞋,日本要求室内地面保持干净,所以学校会统一派发室内鞋共在学校教学楼内活动。水手服是日本动画最大特征,也是日本学生的主要着装。由于日本动画主要描写的是初中生的高中生,所以作为校服的水手服是标准配置。同样作为校服的学校泳装、体育短裤、运动衫、体操服也是经常出现的校园着装。制服除了水手服还有西服等,不再赘述。

至于私服,即私人私下穿着的服饰,则种类更加繁多。典型地如海边回的泳装、夏日祭回的浴衣(和服的一种),以及虚拟设定中女仆所穿着的女仆装,来源于欧洲的萝莉塔长裙等,实质上在为观众提供视觉享受的同时,也不断丰富着美术的创作和审美的提升。从作画的角度来看,服装的设定和作画是醉繁杂的部分了。简单的水手服都会有百褶裙的褶皱产生的阴影所带来的细节描绘,更复杂的服饰则需要更多的线条去表现。我记得当时画过时崎狂三的精灵装,大量蕾丝花纹的绘制花去了整张作画的一半时间。当然我知道专业画师不会这么没效率的,不过就我个人的经验来说,服装的华丽是要花大心血的。基本上服装就是这些,实际上要是展开来说,包括上衣、裙子、裤子、袜子、帽子、饰品、内衣等,都可以展开一个相当大的话题,这里没有机会去一一详述。

\smallskip

总的来说,以美少女为主体的日本萌系动画的美术特征,是基于素描技巧,对眼睛和发型的夸张,面部的简化,身材的协调以及服饰的得体体现出来的。如果没有大量阅览的经验,可能在感官上认为这样的美少女表现得过于“幼稚”。实际上,萌的表现形式之一就是幼稚。针对4--14女生所制作的“幼女向动画”,诸如《光之美少女》系列、《偶像活动》系列、《美妙旋律》系列,由于其人设属于典型的萌系动画,也吸引着大量19--40年龄阶段男性观众\footnote{一般称呼这些人为“大友”,即“大朋友”,与“小朋友”对应。}。值得一提的是,国产动画的《巴拉拉小魔仙》系列也逐渐得到了不少人的肯定,虽然前期骂声一片,但不得不说近来越做越好了,这也是值得高兴的事实。关于动画本身传达了怎样的价值观,我们不做深入讨论。单就其美术价值来看,我们还需要进一步学习和进步。国内动画目前还是主要以简陋的Flash动画和粗陋3D为主,像《西游记》《大闹天宫》等优秀作画动画现在已经很少见了。而讽刺的是,日本动画制作公司会将中间帧的作画外包给中国的廉价劳动力。艺术和经济都得不偿失的中国动画产业确实应该做改变了啊。

\smallskip

这篇所谓论文呢,实际上算不上论文,虽然尝试去找了个明确的阐述目标,但是几千字下来发现并没有多少创新。这方面国内的参考文献还不是很多,于是参考了很多日文文献,在此列出,有必要可以进行查阅。文章论点可能有失偏颇,小生毕竟才疏学浅,能力所不能及,还请见谅;论据流程如有错讹之处,还望指出\footnote{我的联系方式: \url{anytjf@live.com}},不胜感激。最后向辛勤工作的老师致敬。

\begin{thebibliography}{10}
\item 高晨. 日本文化空间下日本动画的“狂欢化”色彩——兼论日本动画的无法模仿性. 东疆学刊, 2013, 30(4).
\item {\ja 山口康男}. 日本动画全史:日本动画制霸世界的奇迹. {\ja 日本のアニメ全史: 世界を制した日本アニメの奇跡}. Ten-Books, 2004.
\item {\ja ササキバラ・ゴウ}. “美少女”的现代史:“萌”与角色. {\ja 「美少女」 の現代史:「萌え」 とキャラクター』}. {\ja 講談社}. 9. 2004.
\item {\ja 増田のぞみ}. “幼女向电视动画”调查——《光之美少女》系列的挑战. {\ja 「女の子向けテレビアニメ」 を問う---『プリキュア』 シリーズの挑戦}. {\ja 年報 『少女』 文化研究}, 2009, 3: 106-118.
\item 萌娘百科. \url{http://zh.moegirl.org}.
\end{thebibliography}

\end{multicols}

\end{document}
