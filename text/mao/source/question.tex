\section{调查问卷}
\label{wenquan}

\subsubsection{网络语言测试}

请完成以下选择题,在每一题中选出您认为恰当的答案:
\begin{enumerate}
\item “愤怒哥”面对记者的镜头问:“我能说脏话吗?”请问,他对何事如此愤怒?
  \\\begin{enumerate*}
  \item 房价上涨 %
  \item 油价上涨
  \item 飞机燃油附加费上涨
  \item 他的自行车被偷了
  \end{enumerate*}
\item 日本人河源启一郎骑自行车环游世界,到中国哪个城市时,自行车被偷了?
  \\\begin{enumerate*}
  \item 西安
  \item 南京
  \item 武汉 %
  \item 长沙
  \end{enumerate*}
\item 被恶搞的“杜甫很忙”,是中学语文课本中哪首诗的插图?
  \\\begin{enumerate*}
  \item 《石壕吏》
  \item 《登高》 %
  \item 《客至》
  \item 《春望》
  \end{enumerate*}
\item “哲♂学”的主要人物是?
  \\\begin{enumerate*}
  \item 花泽香菜
  \item 姚明
  \item 郭敬明
  \item 比利$\cdot$海灵顿 %
  \end{enumerate*}
\item 下列哪个不是“Aji人”的特点?
  \begin{enumerate}
  \item 喜欢玩扩散性百万亚瑟王、偶像大师 灰姑娘女孩、LoveLive!学园偶像祭等课金游戏
  \item 单纯地喜欢 LoveLive! 这一企划 %
  \item 经常盲目跟风大型建造却毫无收获以至于垃圾人群内经常出现满屏哀嚎,夜以继日肝船
  \item 看到朋友出UR或者自己想要却抽不到的SR卡会开玩笑地骂人和打人
  \end{enumerate}
\end{enumerate}

\subsubsection{熟悉度}

对于下列网络语言,请根据自己对它们的了解程度填写表格:
\newcounter{thei}[subsubsection]
\newcommand\ii{\stepcounter{thei}\arabic{thei}}
\begin{longtable}{|c|c|c|c|c|c|c|}
\hline& 网络语言单字 & 熟悉 & 了解 & 一般 & 不清楚 & 没见过\\\hline
\ii & 囧 &  &  &  &  & \\\hline
\ii & 雷 &  &  &  &  & \\\hline
\ii & 槑 &  &  &  &  & \\\hline
\ii & 宅 &  &  &  &  & \\\hline
\ii & 腐 &  &  &  &  & \\\hline
\ii & 艹 &  &  &  &  & \\\hline
\ii & 萌 &  &  &  &  & \\\hline
\ii & 赞 &  &  &  &  & \\\hline
\ii & 顶 &  &  &  &  & \\\hline
\ii & 踩 &  &  &  &  & \\\hline
\end{longtable}
\begin{longtable}{|c|c|c|c|c|c|c|}
\hline & 网络语言词语 & 熟悉 & 了解 & 一般 & 不清楚 & 没见过\\\hline
\ii & 233 &  &  &  &  & \\\hline
\ii & NB &  &  &  &  & \\\hline
\ii & prpr &  &  &  &  & \\\hline
\ii & wwww &  &  &  &  & \\\hline
\ii & 兹磁 &  &  &  &  & \\\hline
\ii & 喜大普奔 &  &  &  &  & \\\hline
\ii & 拙计 &  &  &  &  & \\\hline
\ii & 楼主 &  &  &  &  & \\\hline
\ii & 吐槽 &  &  &  &  & \\\hline
\ii & 卧槽 &  &  &  &  & \\\hline
\end{longtable}
\begin{longtable}{|c|l|c|c|c|c|c|}
\hline & 网络语言成语 & 熟悉 & 了解 & 一般 & 不清楚 & 没见过\\\hline
\ii & 你妈喊你回家吃饭 &  &  &  &  & \\\hline
\ii & $\times\times$大法好 &  &  &  &  & \\\hline
\ii & 图样图森破 &  &  &  &  & \\\hline
\ii & 开门,顺丰快递 &  &  &  &  & \\\hline
\ii & 德国骨科世界第一 &  &  &  &  & \\\hline
\ii & 我兔一盘大棋 &  &  &  &  & \\\hline
\ii & 你国药丸 &  &  &  &  & \\\hline
\ii & 我和我的小伙伴们都惊呆了 &  &  &  &  & \\\hline
\ii & 我觉得自己是零 &  &  &  &  & \\\hline
\ii & 神秘代码呢 &  &  &  &  & \\\hline
\end{longtable}

\subsubsection{喜欢网络语言的原因}

\begin{enumerate}
\item 你喜欢使用网络语言吗?
  \\\begin{enumerate*}
  \item 非常喜欢
  \item 喜欢
  \item 一般
  \item 不喜欢
  \item 讨厌
  \end{enumerate*}
\item 你喜欢使用网络语言的原因是什么(多选):
  \\\begin{enumerate*}
  \item 简单易懂
  \item 幽默生动
  \item 个性时尚
  \item 方便快捷
  \item 气氛轻松
  \item 更好表达
  \item 跟风从众
  \end{enumerate*}
\item 你不喜欢网络语言的原因是什么(多选):
  \\\begin{enumerate*}
  \item 完全看不懂
  \item 觉得没必要
  \item 不够规范
  \item 破坏语言规则
  \item 难以掌握
  \item 其他:
  \end{enumerate*}
\end{enumerate}

\subsubsection{使用范围及态度}

\begin{enumerate}
\item 您在以下媒介中能听到或看到网络语言符号吗(多选):
  \\\begin{enumerate*}
  \item 广播
  \item 电视
  \item 报纸
  \item 杂志
  \item 手机
  \end{enumerate*}
\item 您会在网络聊天时使用网络语言吗?
  \\\begin{enumerate*}
  \item 频繁
  \item 经常
  \item 一般
  \item 偶尔
  \item 从不
  \end{enumerate*}
\end{enumerate}

\subsubsection{对网络语言的认知}



\subsubsection{被调查者上网情况}



\subsubsection{性别对比}


