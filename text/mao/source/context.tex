\section{绪论}

众多网民通过不断进化的网络技术形态的变化,不断改变着自己在网络上的生存状态,他们活跃于网络的各个角落,蛰伏于网络的每一个毛孔,通过电子邮件、电子公告系统、社区论坛、博客、微博客、社会化网络、即时通信、网络游戏以及网络新闻等建立了自己的或庞大或微小的网络空间,而构建这片疆土的有力工具就是网民们惯常熟悉的“网络语言”。

传播与语言相辅共生,现代传播从语言开始。网络语言的混血性质,不遵循日常交往的话轮原则,即量的准则、质的准则、关系准则和方式准则。

\section{研究设计与技术路线}

在语料库的基础之上,对网络语言的以下问题展开研究:
\begin{enumerate}
\item 网络语言缘何产生?网络语言的定义、分类的理据何在?网络语言的根本特点是什么?传播规律是什么?
\item 网络语言在不同的网络通讯方式中,各自具有怎样的表征?使用网络语言的原则是什么?
\item 网络语言符号能指、所指产生了什么变化?意指如何实现?网络传播符号如何改变?
\item 网络语言的传播途径是什么?
\item 网络语言属于什么类型的传播?它传播了什么内容?网民使用网络语言的具体行为表现是什么?
\end{enumerate}

通过以下方法来进行研究:
\begin{enumerate}
\item 自然观察法;
\item 参与观察法;
\item 问卷和访谈法;
\item 搜集、整理、分类、归纳法;
\item 语料库。
\end{enumerate}

由于时间和人力所限,我们研究的内容不能涉及到方方面面,不能够获得大量有效的数据,也不可能得出足够信服的结论。

\section{网络语言使用人群调查}

根据 CNNIC \cite{cnnic},截至 2013 年 12 月,中国网民规模达 6.18 亿,全年共计新增网民 5358 万人。互联网普
及率为 45.8\%,较 2012 年底提升 3.7 个百分点。学生网民占网民总数的 25.5\%,依然是网络语言的主要生力军。个体户/自由职业者构成网民第二大群体,占比 18.6\%。企业公司中管理人员占比为
2.5\%,一般职员占比为 11.4\%。

\subsection{调查方法}

由于条件限制,我们不能够做到大范围的系统的抽样调查。于是我们从网络上收集了一些数据,并结合《网络语言传播导论》\cite{cao}一书中的数据分析,进行了数据统合和分析。同时为了数据一定程度上反映真实情况,我们制作了面向周围同学的调查问卷(见第\pageref{wenquan}页附录\ref{wenquan}),得到了一些宝贵的数据。

\subsection{调查对象}

根据\cite{cao}给出的五个已抽样调查的群体:初中生、高中生、本科生、硕士生和博士生。

\subsection{调查结果及分析}

\subsubsection{网络语言测试}

网络语言测试题得分普遍偏低,说明五个群体对所测试的网络流行语不是非常熟悉。

\subsubsection{熟悉度}

调查显示,被调查对象对于网络语言都不是很熟悉。相对而言,高中生对网络语言最为熟悉;其次是初中生;再次是硕士生和博士生;本科生掌握的最少,也是令人震惊的结果。

究其原因,我们发现,在大城市成长的、接触网络较早的学生更为熟悉网络语言,而大学本科生中存在许多之前并没有接触很多网络的农村学生或者经济困难的学生,也是导致这一现象的原因。

\subsubsection{喜欢网络语言的原因}

除了博士生之外,其他四个群体中喜欢网络语言的人数要比不喜欢和无所谓的比例高,尤其是高中生和本科生更能接受网络语言,多数人都喜欢网络语言。

\subsubsection{使用范围及态度}



\subsubsection{对网络语言的认知}



\subsubsection{被调查者上网情况}



\subsubsection{性别对比}



\subsection{调查结论}



\section{总结}


