\section{绪论}

众多网民通过不断进化的网络技术形态的变化,不断改变着自己在网络上的生存状态,他们活跃于网络的各个角落,蛰伏于网络的每一个毛孔,通过电子邮件、电子公告系统、社区论坛、博客、微博客、社会化网络、即时通信、网络游戏以及网络新闻等建立了自己的或庞大或微小的网络空间,而构建这片疆土的有力工具就是网民们惯常熟悉的“网络语言”。

传播与语言相辅共生,现代传播从语言开始。网络语言的混血性质,不遵循日常交往的话轮原则,即量的准则、质的准则、关系准则和方式准则。

\section{研究设计与技术路线}

在语料库的基础之上,对网络语言的以下问题展开研究:
\begin{enumerate}
\item 网络语言缘何产生?网络语言的定义、分类的理据何在?网络语言的根本特点是什么?传播规律是什么?
\item 网络语言在不同的网络通讯方式中,各自具有怎样的表征?使用网络语言的原则是什么?
\item 网络语言符号能指、所指产生了什么变化?意指如何实现?网络传播符号如何改变?
\item 网络语言的传播途径是什么?
\item 网络语言属于什么类型的传播?它传播了什么内容?网民使用网络语言的具体行为表现是什么?
\end{enumerate}

\section{正文}



\section{总结}


