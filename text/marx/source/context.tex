\section{绪论}

众多网民通过不断进化的网络技术形态的变化,不断改变着自己在网络上的生存状态,他们活跃于网络的各个角落,蛰伏于网络的每一个毛孔,通过电子邮件、电子公告系统、社区论坛、博客、微博客、社会化网络、即时通信、网络游戏以及网络新闻等建立了自己的或庞大或微小的网络空间,而构建这片疆土的有力工具就是网民们惯常熟悉的“网络语言”。

传播与语言相辅共生,现代传播从语言开始。网络语言的混血性质,不遵循日常交往的话轮原则,即量的准则、质的准则、关系准则和方式准则。

\section{研究设计与技术路线}

在语料库的基础之上,对网络语言的以下问题展开研究:
\begin{enumerate}
\item 网络语言缘何产生?网络语言的定义、分类的理据何在?网络语言的根本特点是什么?传播规律是什么?
\item 网络语言在不同的网络通讯方式中,各自具有怎样的表征?使用网络语言的原则是什么?
\item 网络语言符号能指、所指产生了什么变化?意指如何实现?网络传播符号如何改变?
\item 网络语言的传播途径是什么?
\item 网络语言属于什么类型的传播?它传播了什么内容?网民使用网络语言的具体行为表现是什么?
\end{enumerate}

通过以下方法来进行研究:
\begin{enumerate}
\item 自然观察法;
\item 参与观察法;
\item 问卷和访谈法;
\item 搜集、整理、分类、归纳法;
\item 语料库。
\end{enumerate}

由于时间和人力所限,我们研究的内容不能涉及到方方面面,不能够获得大量有效的数据,也不可能得出足够信服的结论。

\section{网络语言使用人群调查}

根据 CNNIC \cite{cnnic},截至 2013 年 12 月,中国网民规模达 6。18 亿,全年共计新增网民 5358 万人。互联网普
及率为 45。8\%,较 2012 年底提升 3。7 个百分点。学生网民占网民总数的 25。5\%,依然是网络语言的主要生力军。个体户/自由职业者构成网民第二大群体,占比 18。6\%。企业公司中管理人员占比为
2。5\%,一般职员占比为 11。4\%。

\subsection{调查方法}

由于条件限制,我们不能够做到大范围的系统的抽样调查。于是我们从网络上收集了一些数据,并结合《网络语言传播导论》\cite{cao}一书中的数据分析,进行了数据统合和分析。同时为了数据一定程度上反映真实情况,我们制作了面向周围同学的调查问卷(见第\pageref{wenquan}页附录\ref{wenquan}),得到了一些宝贵的数据。

以下只给出概述,没有具体数据,请见谅。

\subsection{调查对象}

根据\cite{cao}给出的五个已抽样调查的群体:初中生、高中生、大学本科生、硕士生和博士生。

\subsection{调查结果及分析}

\subsubsection{网络语言测试}

网络语言测试题得分普遍偏低,说明五个群体对所测试的网络流行语不是非常熟悉。实际上,测试用的 5 个题目都是来自于不同的社交圈子的,一般人回答正确 1~2 个题目也是比较正常的现象。

\subsubsection{熟悉度}

调查显示,被调查对象对于网络语言都不是很熟悉。相对而言,高中生对网络语言最为熟悉;其次是初中生;再次是硕士生和博士生;本科生掌握的最少,也是令人震惊的结果。

究其原因,我们发现,在大城市成长的、接触网络较早的学生更为熟悉网络语言,而大学本科生中存在许多之前并没有接触很多网络的农村学生或者经济困难的学生,也是导致这一现象的原因。

\subsubsection{喜欢网络语言的原因}

除了博士生之外,其他四个群体中喜欢网络语言的人数要比不喜欢和无所谓的比例高,尤其是高中生和本科生更能接受网络语言,多数人都喜欢网络语言。

喜欢网络语言的主要原因是“个性时尚”和“幽默互动”,“跟风从众”也占了相当大的比例。网络语言这种新颖而富有创意的语言形式,偏离了日常的语言规范,契合了青少年张扬自我、突出个性的表意追求。

不喜欢网络语言的原因集中在“完全看不懂”和“难以掌握”上,究其原因是学习上的惰性和态度上的抵制。

\subsubsection{使用范围及态度}

由于移动互联网的兴起,手机端显示出非常大的优势。在网络语言传播上,计算机和手机是两大主力军平台。

还可以看到人们使用网络语言也是分场合进行的,不会滥用在不合适的地方。

对于网络语言的态度趋于中立,五个选项呈标准的正态分布。

\subsubsection{被调查者上网情况}

初中生使用网络语言的对象多数是同学;高中生使用网络语言的对象最多的是朋友,其次是同学和网友;大学本科生使用网络语言对象最多的为网友,其次为朋友;硕士生和博士生主要在于同学和朋友。

使用网络的环境,大部分被调查者都有可以上网的环境,主要以手机为主。在 QQ 上使用网络语言是大家的共识;微博也是网络语言的一大广场;贴吧和论坛也占了相当一部分;电子邮件中使用网络语言的比较少,这一点和国外有所区别。另外,在网络游戏中使用网络语言也是常见的现象。

使用网络语言的男性多于女性,数据与被调查群体人口密集度相似,但男性相对偏高。说明男性在网络中使用网络语言的人数多于女性,男性还是网络上的主力军。

\subsection{调查结论}

\begin{enumerate}
\item 网络语言已成为青年学生开展人际传播与群体传播的重要媒介语言。时尚、快捷、便利,实现交际最大化是青年网民使用网络语言的重要动力。
\item 文件与访谈调查数据显示,尽管西部青年在网络语言使用方面没有东部青年那么频繁,但是网民并不陌生网络语言,甚至乐于使用与传播网络语言。
\item 数据同时显示,网络语言成为青年网民群体内部交往的重要工具,使用网络语言的频率于人际熟悉度呈正相关关系。
\item 网络语言的使用剩余传播还呈现出青年网民的性别差异,青年男女、女网农民在网络语言使用的场合、频次、对象对具有统计学意义上点额差异。这种差异在一定意义上呈现出社会语言学的性别差异特征。
\end{enumerate}

为了进一步验证问卷调差的信度与效度,调查人员就受试是否拥有个人计算机,上网时间。是否了解网络语言。使用网络语言的的情况、对网络语言的认识、对网络语言的使用环境的了解、对网络语言所持的态度等开展了访谈。访谈结果说明,网络语言渐入人心,越来越多的年轻人乐意接受并使用网络语言。

\section{网络语言分类}

在研究网络语言传播的同时,我们也研究了网络语言的特点,并做了一定的分类归纳和整理。

《中国网络语言词典》\cite{webt}将网络语言大体上分为三类:
\begin{enumerate}
\item 与网络有关的专业术语;
\item 与网络有关的特别用语;
\item 网民在聊天室和论坛上的常用词语。
\end{enumerate}

我们来分类透视网络语言传播规律,综合网络语言语汇的结构、符号特征与意义、符号组合与搭配以及各类表现形态,分为如下十大类:

\begin{enumerate}

\item 缩写——语言浓缩

网上交流把语言传播搬到了网络虚拟空间,用文字的形式来“说话”,网络语言的绘画特征要求人们在网上交流时,要在最短的时间内最快的完成传播任务,这就对打字速度提出了极高的要求。而缩略语可以大大提高打字速度,进而提高传播的速度。网络缩略语中的绝大部分都是字母、数字和标点符号的排列组合。

\item 语码混用——语言交融

语码混用是语言传播的必然结果。随着传播面的扩大,传播速度的加快,各种语言的语码混用那个情况就越发频繁。语料库数据显示,语码混用,尤其是汉语与外语混用的频率并不是很高,占总语料数的8.8\%。这里面可能与语言使用者的受教育程度,文化水平有关。

\item 谐音——语义嫁接

谐音是常见的一种语言表现形式,是提高语言表达效果的一种重要手段。他可以用来表达含蓄深沉,微弯曲折,风趣诙谐或幽默生动各种情感。

\item 词义转换——语义延伸

词义转换中,汉语词义歪解、旧词新意、旧子新意占得比例分别为0.9\%、7.3\%、0.3\%。语言构词必须遵循一定的规范,即词性规定和词语的特定搭配。但网络用语全然颠覆了这种原则,经常对原有的词汇进行引申、仿造、甚至曲解,造出新的词汇,网友们常常使用这种修辞手段来标新自己的意蕴。

\item 新造字词——语言张力

这是对语言的继续发展,亦为语言张力的体现。新造字、词是网络传播的有一个规律。 

\item 借词——语言动力

网络借词不仅存在词汇层面,甚至使语句,文法,和表达方式都产生了变化。借词主要体现三种相互影响:语音,句法和词汇的相互影响。现代汉语网络语言的借词现象主要来自于英语和日语。

\item 非语言符号——语言伴侣

网络非语言符号主要由表情符号与肢体符号构成。这类符号占网络语言的34.7\%。小类中键盘符号构成表情语言占34.6\%,网络自带表情符号占0.1\%。

\item 网络语法——法无定法

所谓网络语法既包括词法,也包括句法。网络语言的语法并没有一定之规,也不存在规定的元语言。

\item 网络语汇——舆论之源

这类语言表明了网民关注社会环境,生存状态,创作戏谑式的语言游戏以及反映了网民对就业的恐慌。也表明了人际关系淡漠,网民渴望亲情。

\item 火星文——语言游戏

这一类网络语言占本文语料库的3.6\%。主要类似这样严重偏离语言规则的交流内容和形式,是语言文字游戏性质的新型的网络语言。

\end{enumerate}

\section{新生网络词语}

网络语言除了上面提到的语法方面的有别于日常语言的异变外,更大的变素表现在其词汇方
面。这些网络语言中的特殊词汇,既包括了各种特殊用法的汉字词,也包括了非汉字的英文字
母、象形符号及阿拉伯数字等。

网络语言词语另一明显特征是其经济性。网络词语除了字数上的节俭外,还突破了原有汉语书
写语言符号的局限,改变了汉字的形音意的约定俗成,创造了一种新的形音意表达结合符号,
如以下将提到的数字字母组合词、数字词,图形符号等。网语用最简洁的“型”来提高交流速
度,缩短网络交际中的宝贵上网时间。网语中多用短句,长句复杂句极少使用,将长句断开使
用或缩略及其常见,原因一可能许多年轻网民为了节省上网费用。

交际网络语言词汇又另可分为以下几类:
\begin{enumerate}

\item 取谐音的网络词汇

汉字取谐音的网络词汇,即给予原本汉字词没有的其他意义,也就是扩大了词义范围。如在网
上“斑竹”即“版主”的意思。

数字谐音的网络词汇。比如
“520”等于“我爱你”、“514”等于“无意思”等。这种数字谐音词当然在进入手机网络时
代后,不再使用率那么频繁,但是在聊天时为了提高打字速度或者在有些时候打出汉字觉得尴
尬的时候,网民们还是加以利用。比如在聊天时突然紧急情况下,需要退出聊天,迫于时间的
要求可能一个“53”就能代替“我闪”、一个“94”就代表“就是”(对谈话对手意见观点
同意时的附和)等。

谐音缩写,更进一步的对进行了拼音化并且简写。一种是
可采用于称呼上,比如“姐姐”、“妹妹”、“哥哥”、“弟弟”相对应变成“jj”、
“mm”、“gg”、“dd”,是属于社会称呼的一种,已经失去其原有的亲戚关系意义。另
一种可采用于人名上,比如“文革”可拼音化后简写为“WG”、或者“温家宝”,“WJB”或
者“WBB”,“温宝宝”;还有一种直接运用英文的缩写,或者中英文缩写混杂,
比如“plmm,ILU,CU”意思即“漂亮妹妹,我爱你,再见。”

\item 对传统汉字词的改造

外形的改造:夸奖别人为“强”的时候写成“弓虽”,起到强调及吸引人眼光的作用。说别人
“好棒呀”写成“女子木奉口牙”来达到吸引人目光的作用。
词义的改造:“写手”,按传统的理解应该是写的手或者写作的手,但是在网络中是“网络文
学作者”的意思;又比如“见光死”按字面理解是“一见光就死”的意思,但是在网络语言中
是“网络上恋爱的对象到彼此真正见面时,却都不满意对方,不愿意继续恋情”的意思。
最典型的是“囧”“雷”“槑”这些单字词,已经耳熟能详。

\item 英语借词与日语借词

英语借词大致可分为两种:一为音译借词,二为意译借词。比如“download”中国官方借用为
意译“下载”,而在网络上为“当”,是将“download”缩减音译,使得其词语的经济性提高
了。又比如“E-mail”的官方借用也是意译“电子邮件”,但是在网络语言中为“伊妹儿”。

日语词“腹黑({\mincho 腹黒「はらぐろ」})”是外表看不出来的“恶人”;“{\mincho 攻略}({\mincho こうりゃく})”是成功的要诀;“地下铁({\mincho 地下鉄「ちかてつ」})”就是“地
铁”;“卡哇伊({\mincho 可愛い「かわいい」})”是可爱;“{\mincho 宅}({\mincho オタク})”
就是喜欢呆在家看动画的人;“萌({\mincho 萌え「もえ」})”
用来形容对某件事物的喜爱之情。同英语借词很大的一点不同是,日语借可以直接用汉字表示出来,
不用像英语要先翻译成汉字,并且也不需要翻译读音,仍然用汉字发音。另外日本流行文化
在中国大陆的兴盛,如动漫和游戏,也是日语外来词盛行的原因之一。此外,一些日语学习者
为了展现自己的语言优势,也喜欢用日语音译或者意译词来突显自己的身份或者群体感。当然
也不能排除用外来词汇描述日常事物的确能够带来新鲜感。

\item 方言借词

方言借词也是网络语言词汇中的一大特点,尤其是近几年来网络的普及,以及台湾综艺节目在
大陆网民间的流行,导致了台湾方言或流行语的导入。比如台语的“那 A 按哪?”成为了许多
网民的口头禅,它的意思是“那怎么办呢?”或者“白目”在大陆日常语言中应该是“没眼
色”,但是在网络上“白目”显然使用率更多。还有现在越来越常用的“老大”,在网络上已
经没有“黑道老大”这一意思,而是跟多代表一种无奈的情绪称呼对方为“老兄”,这一用法
来自于香港民间的用法;“粉”是客家话“很”的意思,现在在网络语言中也广为使用,这是
在一综艺节目中其主持人使用后得到观众的回应,继而在网络上流传开来。以上提到的例子只
是冰上一角,真正实际运用的此类词语还有很多。

\end{enumerate}

从上面提到的网络语言词汇的全部变体来看,网络语言并非自己独特的一套语言。首先,它作
为一套独立的语言系统,缺乏其独有的基本词汇,仅仅在一些所谓的词汇变体上有自己的一些
特点,但这些特点也仅仅是建立在日常语言(中文也好,外文也好,方言也好)的基础上。单
单因为这些就语言系统里一小部分变体来说明“网络语言”是一种独立的语言的说法是完全不
成立的。也就是说,在这样的情况下,“网语”无法形成一套独有的语言系统,而仅仅是作为
社会方言,也就是一种语言变体的形式存在着。

\section{结论}

网络语言藉由网络的各种通信方式而产生,凭借网络传播符号表达的形式可能性而发展,经由模因而播撒,通过网民群体而传播。

网络语言表现出了高度的去中心化、平面化、部落化、碎片化和草根化特征,此乃网络语言生存的土壤和传播基础。

在网络传播中,传播符号的形式生产与意义创造成为网民才智的开放式自由竞赛。网络传播的大背景,网民社会文化心理的驱使,技术逻辑的演变,传播环境的制约导致网络传播符号的形式与意义的膨胀与流失。

网络模因传播成为网络语言传播的基本途径,凭藉各种网络通信方式,网络语言广为流传。

网络人际传播主要依靠即时通信、博客与微博、电子邮件和论坛实现。网络群体传播则有赖于这些虚拟社区。网络上的“圈”或“群”把网民划分成为一个个自由的社交组合群体,在这样的群体中,网络语言呈现“多”“杂”“散”“匿”的特征。

对于自然语言来说,网络语言既不是超越,更不是替代,而是网络人际交流的补充手段。网络语言的存在使得人类从新的视角看待语言与言语、口语与书面语、个人语言与社会语言的多重关系。

网络语言和汉语一样,从外语中输入了不少语汇(主要是英语和日语),同时也输出到国外,促成文化的交流与融合。

网络语言是一种语言的杂糅现象,体现了语言简化和语言交融的发展规律。这也正是马克思指出的人类社会的发展趋势。
