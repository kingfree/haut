\begin{titlepage}

\includegraphics[height=24.6mm]{image/haut.jpg}
\hfill
分数 \underline{\makebox[6em][c]{}}

\begin{center}

{\fangso\bfseries\zihao{2}{2013--2014--2}}

\vspace*{1cm}

{\fangso\bfseries\zihao{2}{《马克思主义基本原理概论》调研报告}}

\vfill
{\large
\newcommand{\ctline}[2]{\makebox[4em][s]{\bf #1}:\underline{\makebox[15em][c]{#2}}\par}
\newcommand{\ctlin}[2]{\makebox[4em][s]{\bf #1}\quad\underline{\makebox[15em][c]{#2}}\par}
\ctline{题目}{\titlec}
\ctline{学院}{信息科学与工程学院}
\ctline{专业班级}{计算机 1303 班}
\ctline{小组成员}{\tjf}
\ctlin{}{王飞飞}
\ctlin{}{邢志鹏}
\ctlin{}{郭嵩浩}
\ctlin{}{钟军毅}
\ctline{提交时间}{\today}
}

\vspace*{2cm}

{\large\tt 河南工业大学思政学院原理教研室}

\end{center}

\end{titlepage}


% 社会实践步骤要求
% 一、社会实践步骤:
%    1、分组:每小组人数不宜过多,以5-6人为宜。
%    2、选题:实践前同学根据参考选题和本组兴趣自拟题目(尽可能具体),写出调查计划、提纲交任课老师检查、指导。
%    3、撰写调查报告:调查结束后要认真撰写调查报告,实践活动进行总结
% 二、撰写调查报告的要求:
%    1.调查报告包括标题、署名、正文及辅助材料。
%    2.报告的正文由三个部分构成,即:前言、主体和结束语。前言,扼要说明调查的目的;时间、地点;对象或范围;做了哪些调查;本文所要报告的主要内容是什么。主体,主要是运用所学理论对调查内容进行归纳、总结和分析。一般把调查的主要情况、经验或问题归纳为几个方面,分为几个小部分来写。每个小部分有一个中心,加上序码来表明,或加上小标题来提示、概括这部分的内容,使之眉目清楚。结束语,可对全文进行提纲挈领的总结,深化主题;也可提出相应“对策”,给人启迪;还可指出发展方向,展望发展前景,给人以鼓舞,还可以对调研活动的得失进行反思。
%    3.文章必须紧扣马克思主义基本原理,用马克思主义相关原理、立场、观点方法分析问题、解决问题。   
%    4.调查报告正文不低于三千字,要求使用A4纸打印。
%    5.辅助材料包括调查报告相关的图片、问卷、访谈记录。
%    6.期中上交辅助材料,课程结束时以组为单位上交调研报告。
% 三、评分与交流
%    1.评分:任课老师根据调研情况和调查报告质量对各小组社会实践进行成绩评定,写出分数、评语,调研所有资料期末存档。
%    2.交流:任课老师根据成绩和调查情况可组织社会实践成绩突出的小组进行经验交流和总结。
%    3.5月初全体老师在教研室汇报实践课教学进行情况。
