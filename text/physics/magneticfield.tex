\section*{稳恒磁场}

\subsubsection*{一. 选择题}

1. 在磁感强度为$\vec B$的均匀磁场中作一半径为$r$的半球面$S$,$S$边线所在平面的发现方向单位矢量$\vec n$与$\vec B$的夹角为$\alpha$,则通过半球面$S$的磁通量为\fbox{$-\pi r^2B\cos\alpha$}.(取弯曲面向外为正)

2. 边长为$a$的正方形的四个角上固定有四个电荷均为$q$的点电荷.此正方形以角速度$\omega$绕$AC$轴旋转时,在中心$O$点产生的磁感强度大小为$B_1$;此正方形以同样的角速度$\omega$绕过$O$点垂直于正方形平面的轴旋转时,在$O$点产生的磁感强度的大小为$B_2$,则\fbox{$B_1=\frac{1}{2}B_2$}.

3. 通有电流$I$的无限长直导线有如图三种形状(直线、“L”型、下半圆),则$P,Q,O$各点磁感强度大小关系为\fbox{$B_O>B_Q>B_P$}.

4. 载流的圆形线圈(半径$a_2$)与正方形线圈(边长$a_2$)通有相同电流$I$.若两个线圈的中心$O_1,O_2$处的磁感强度大小相同,则半径$a_1$与边长$a_2$的比$a_1:a_2$为\fbox{$\sqrt2\pi:8$}.

5. 无限长直导线在$P$弯成半径为$R$的圆,当通以电流$I$时,则在圆心$O$的磁感强度大小等于\fbox{$\frac{\mu_0I}{2R}(1-\frac{1}{\pi})$}.

6. 一个电流元$I\d\vec l$位于直角坐标系原点,电流沿$z$轴方向,点$P(x,y,z)$的磁感强度沿$x$轴的分量是\fbox{$-(\mu_0/4\pi)Iy\d l/(x^2+y^2+z^2)^{3/2}$}.

9. 电流由长直导线1沿半径方向经$a$点流入一由电阻均匀的导线构成的圆环,再由$b$点沿半径方向从圆环流出,经长直导线2返回电源.已知直导线上电流强度为$I$,$\angle aOb=30\degree$.若长直导线1,2和圆环中的电流在圆心$O$点产生的磁感强度分别用$\vec B_1,\vec B_2,\vec B_3$表示,则圆心$O$点的磁感强度大小\fbox{$B=0$,因为$B_1=B_2=B_3=0$}.

14. 有一无限长通电流的扁平铜片,宽度为$a$,厚度不计,电流$I$在铜片上均匀分布,在铜片外与铜片共面,离铜片右边缘为$b$处的$P$点的磁感强度$\vec B$的大小为\fbox{$\frac{\mu_0I}{2\pi a}\ln\frac{a+b}{b}$}.

15. 若空间存在两根无限长直载流导线,空间的磁场分布就不具有简单的对称性,则该磁场分布\ul{可以用安培环路定理和磁感强度的叠加原理求出}.

16. 如图,流出纸面的电流为$2I$,流进纸面的电流为$I$,则正确的是\fbox{$\oint_{L_4}\vec B\cdot\d\vec l=-\mu_0I$}.

17. 在图(a)和(b)中各有一半径相同的圆形回路$L_1,L_2$,圆周内有电流$I_1,I_2$,其分布相同,且均在真空中,但在(b)图中$L_2$回路外有电流$L_3$,$P_1,P_2$位两圆形回路上的对应点,则\fbox{$\oint_{L_1}\vec B\cdot\d\vec l=\oint_{L_2}\vec B\cdot\d\vec l$, $B_{P_1}\neq B_{P_2}$}.

20. 如图所示,在磁感强度为$\vec B$的均匀磁场中,有一圆形载流导线,$a,b,c$是其上三个长度相等的电流元,则它们所受安培力大小关系为\fbox{$F_b>F_c>F_a$}.

21. 如图,无限长直载流导线与正三角形载流线圈在同一平面内,若长直导线固定不动,则载流三角形线圈将\fbox{向着长直导线平移}.

22. 长直电流$I_2$与圆形电流$I_1$共面,并与其一直径相重合,如图(但两者间绝缘),设长直电流不懂,则圆形电流将\fbox{向右运动}.

25. 如图,匀强磁场中有一矩形铜电线圈,它的平面与磁场平行,在磁场作用下,线圈发生转动,其方向是\fbox{$ab$边转入纸内,$cd$边转出纸外}.

26. 把通电的直导线放在蹄形磁铁磁极的上方,如图所示.导线可以自由活动,且不计重力.当导线内通以如图所示的电流时,导线将\fbox{顺时针方向转动,然后下降}(从上往下看).

\subsubsection*{二. 填空题}

38. 真空中稳恒电流$I$流过两个半径分别为$R_1,R_2$的同心半圆型导线,两半圆导线间由沿直径的指导线连接,电流沿直导线流入.
    (1) 如果两个半圆共面(图1),圆心$O$点的磁感强度$\vec B_0$的大小为\fbox{$\frac{\mu_0I}{4}(\frac{1}{R_2}-\frac{1}{R_1})$},方向\fbox{垂直纸面向外};
    (2) 如果两个半圆正交(图2),则圆心$O$点的磁感强度$\vec B_0$的大小为\fbox{$\frac{\mu_0I}{4}\sqrt{\frac{1}{R_1^2}+\frac{1}{R_2^2}}$},$\vec B_0$的方向与$y$轴的夹角为\fbox{$\frac{2}{\pi}+\arctan\frac{R_2}{R_1}$}.

39. 一弯曲的载流导线在同一平面内,形状如图($O$点是半径为$R_1$和$R_2$的两个半圆弧的共同圆心,电流自无穷远来到无穷远去),则$O$点磁感强度的大小是\fbox{$B_0=\frac{\mu_0I}{4R_1}+\frac{\mu_0I}{4R_2}-\frac{\mu_0I}{4\pi R_2}$}.

49. 如图,在无限长直载流导线的右侧有面积为$S_1$和$S_2$的两个矩形回路.两回路与长直载流导线共面,且矩形回路的一边与长直载流导线平行.则通过面积为$S_1$的矩形回路的磁通量与通过面积为$S_2$的矩形回路的磁通量之比为\fbox{$1:1$}.

50. 有一同轴电缆,其尺寸如图所示,它的内外两导体中的电流均为$I$,且在横截面上均匀分布,点两者电流的流向正相反,则
    (1) 在$r<R_1$处磁感强度大小为\fbox{$\mu_0rI/(2\pi R_1^2)$};
    (2) 在$r>R_2$处磁感强度大小为\fbox{$0$}.

51. 两根长直导线通有电流$I$,图示有三种环路;在每种情况下,$\oint_L\vec B\cdot\d\vec l$等于:
    \fbox{$\mu_0I$}(对环路$a$),
    \fbox{$0$}(对环路$b$),
    \fbox{$2\mu_0I$}(对环路$c$).

65. 如图所示,在真空中有一半径为$a$的$3/4$圆弧形导线,其中通以稳恒电流$I$,导线置于均匀外磁场$\vec B$中,且$\vec B$与导线所在平面垂直.则该载流导线${bc}$所受磁力大小为\fbox{$\sqrt2aIB$}.

66. 如图所示,在真空中有一半圆形闭合线圈,半径为$a$,流过稳恒电流$I$,则圆心$O$处的电流元$I\d\vec l$所受的安培力$\d\vec F$的大小为\fbox{$\mu_0I^2\d l/4a$},方向\fbox{垂直电流元背向半圆弧(即向左)}.

\subsubsection*{三. 计算题}

69. 有一条载有电流$I$的导线弯成如图示$abcda$形状.其中$ab,cd$是直线段,其余为圆弧.两段圆弧的长度和半径分别为$I_1,R_1$和$I_2,R_2$,且两段圆弧共面共心.求圆心$O$处的磁感强度$\vec B$的大小.
\SL{两段圆弧在$O$处产生的磁感强度为$B_1=\frac{\mu_0Il_1}{4\pi R_1^2}$,$B_2=\frac{\mu_0Il_2}{4\pi R_2^2}$.\\
两段直导线在$O$处产生的磁感强度为$B_3=B_4=\frac{\mu_0I}{4\pi R_1\cos\frac{l_1}{2R_1}}(-\sin\frac{l_1}{2R_1}+\sin\frac{l_2}{2R_2})$.\\
取向里为正$B=B_1+B_3+B_4-B_2=\frac{\mu_0I}{2\pi R_1\cos\frac{l_1}{2R_1}}(-\sin\frac{l_1}{2R_1}+\sin\frac{l_2}{2R_2})+\frac{\mu_0I}{4\pi}(\frac{l_1}{R_1^2}-\frac{l_2}{R_2^2})$.
方向垂直版面向里.
}
71. 已知半径为$R$的载流圆线圈与边长为$a$的载流正方形线圈的磁矩之比为$2:1$,且载流圆线圈在中心$O$处产生的磁感应强度为$B_0$,求在正方形线圈中心$O'$处磁感强度的大小.
\SL{设圆线圈磁矩为$p_{m_1}$,方线圈磁矩为$p_{m_2}$,则$p_{m_1}=I_1\pi R^2$, $p_{m_2}=I_1a^2$,\\
所以$I_2=\pi R^2I_1/(2a^2)$.
正方形一边在其中心处产生的磁感强度\pp
$B_1=\frac{\mu_0I_2}{4\pi(a/2)}(\cos\frac{\pi}{4}-\frac{3\pi}{4})=\frac{\sqrt2\mu_0I_2}{2\pi a}$,
【答案缺失】
}
73. 将通有电流$I$的导线在同一平面内弯成如图所示形状,求$D$点的磁感强度$\vec B$的大小.
\SL{【答案缺失】
}
78. 用两根彼此平行的半无限长直导线$L_1,L_2$把半径为$R$的均匀导体圆环联到电源上,如图所示.已知直导线中的电流为$I$,求圆环中心$O$点的磁感强度.
\SL{【答案缺失】
$\vec B_3+\vec B_4=0$.\\
故$O$点的磁感强度$|\vec B_0|=|\vec B_1+\vec B_2+\vec B_3+\vec B_4|=\frac{\mu_0I}{4\pi R}$,方向垂直图面向外.
}
79. 如图所示,一无限长载流平板宽度为$a$,线电流密度为$\delta$(即沿$x$方向单位长度上的电流),求与平板共面且距平板一边为$b$的任意点$P$的磁感强度.
\SL{利用无限长载流直导线的公式求解.\\
取离$P$点为$x$宽度为$\d x$的无限长载流细条,它的电流$\d i=\delta\d x$,\\
这载流长条在$P$点产生的磁感强度
$\d B=\frac{\mu_0\d i}{2\pi x}=\frac{\mu_0\delta\d x}{2\pi x}$,
方向垂直纸面向里.\\
所有载流长条在$P$点产生的磁感强度方向都相同,所以载流平板在$P$点产生的磁感强度\pp
$B=\int\d B=\frac{\mu_0\delta}{2\pi x}\int_b^{a+b}\frac{\d x}{x}=\frac{\mu_0\delta}{2\pi x}\ln\frac{a+b}{b}$,
方向垂直纸面向里.
}
80. 在一无限长的半圆筒形的金属箔片中,沿轴向流有电流,在垂直电流方向单位长度的电流为$i=k\sin\theta$,其中$k$为常量,$\theta$如图所示.求半圆筒轴线上的磁感强度.
\SL{设轴线上任意点的磁感强度为$B$,半圆筒半径为$R$.\\
先将半圆筒面分成许多平行轴线宽度为$\d l$的无限长直导线,其中流过的电流为\pp
$\d I=i\d l=k\sin\theta\cdot\d l=k\sin\theta R\d\theta$,\\
它在轴线上产生的元磁感强度为
$\d B=\frac{\mu_0\d I}{2\pi R}$,方向如图.\\
由对称性可知:
$\d\vec B$在$z$轴向的分量为0,在$y$轴的分量叠加中互相抵消,\\
只需要考虑$\d\vec B$在$x$轴的分量
$\d B_x=\d B\sin\theta=\frac{\mu_0\d I}{2\pi R}\sin\theta=\frac{\mu_0k\sin^2\theta}{2\pi}\d\theta$,\\
积分
$B=\int\d B_x=\int_0^\pi\frac{\mu_0k\sin^2\theta}{2\pi}\d\theta=\frac{\mu_0k}{2\pi}\int_0^\pi\frac{1-\cos2\theta}{2}\d\theta=\frac{\mu_0k}{4}$.
}
81. 如图所示,载有电流$I_1$和$I_2$的长直导线$ab$和$cd$相互平行,相距为$3r$,今有载有电流$I_3$的导线$MN=r$,水平放置,且其两端$MN$分别与$I_1,I_2$的距离都是$r$, $ab,cd$和$MN$共面,求导线$MN$所受的磁力大小和方向.
\SL{载流导线$MN$上任一点处的磁感强度大小为
$B=\frac{\mu_0I_1}{2\pi(r+x)}-\frac{\mu_0I_2}{2\pi(2r-x)}$.\\
$MN$上电流元$I_3\d x$所受磁力
$\d F=I_3B\d x=I_3[\frac{\mu_0I_1}{2\pi(r+x)}-\frac{\mu_0I_2}{2\pi(2r-x)}]\d x$,\pp
$F\,= I_3\int_0^r[\frac{\mu_0I_1}{2\pi(r+x)}-\frac{\mu_0I_2}{2\pi(2r-x)}]\d x
   = \frac{\mu_0I_1}{2\pi}[\int_0^r\frac{I_1}{r+x}\d x-\int_0^r\frac{I_2}{2r-x}\d x]
$\\  
$  = \frac{\mu_0I_1}{2\pi}[I_1\ln\frac{2r}{r}+I_2\ln\frac{r}{2r}]
   = \frac{\mu_0I_1}{2\pi}[I_1\ln2-I_2\ln2]
   = \frac{\mu_0I_1}{2\pi}(I_1-I_2)\ln2$.\\
若$I_2>I_1$,则$\vec F$的方向向下;
若$I_2<I_1$,则$\vec F$的方向向上.
}
