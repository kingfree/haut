\documentclass[10pt,a4paper,nofonts]{ctexart}
\usepackage[utf8]{inputenc}
\def\tjf{{\tt{田劲锋}}}
\def\titlec{概率论与数理统计}
\usepackage[a4paper,margin=1.5cm]{geometry} % 页面设置
\usepackage[unicode,breaklinks=true,
colorlinks=true,linkcolor=black,anchorcolor=black,citecolor=black,urlcolor=black,
pdftitle={\titlec},pdfauthor={\tjf}]{hyperref}
\usepackage{latexsym,amsmath,amssymb,bm}
\usepackage[inline]{enumitem} % 调整列表样式
\usepackage{multicol} % 分栏

\setCJKmainfont[BoldFont={SimHei}]{SimSun}  % 主要字体: 宋体、黑体
\setCJKsansfont[BoldFont={STZhongsong}]{STFangsong} % 次要字体: 仿宋、中宋
\setCJKmonofont{KFKai} % 等宽字体: 楷体

\CJKsetecglue{\hspace{0.1em}}
\renewcommand\CJKglue{\hskip -0.3pt plus 0.08\baselineskip}
\frenchspacing
\widowpenalty=10000
\linespread{1} % 行距

\pagestyle{empty}

\newcommand{\D}{\displaystyle}
\renewcommand{\d}{{\mathrm{d}\:\!}}
\newcommand{\e}{{\mathrm{e}}}
\DeclareMathOperator{\Var}{Var}
\DeclareMathOperator{\Cov}{Cov}
\setlist{noitemsep}
\setlist[enumerate]{topsep=0pt,partopsep=0pt,itemsep=0pt,parsep=0pt}
\setlist[enumerate,2]{label={(\arabic*)}}

\CTEXsetup[beforeskip={0ex},afterskip={0ex},number={\chinese{section}},format={\centering\Large\sf\bfseries}]{section}
\CTEXsetup[beforeskip={0ex},afterskip={1ex},name={第,章},number={\chinese{subsection}}]{subsection}
\CTEXsetup[beforeskip={0ex},afterskip={0ex},name={,.},number={\arabic{subsubsection}}]{subsubsection}

\begin{document}

%\CTEXnoindent

\begin{multicols}{2}

\section{公式和性质}

\subsection{随机事件}

\begin{enumerate}
\item {\bf 概率}的性质: 
\begin{enumerate}
\item $P(\varnothing)=0$
\item $A_i$两两互斥, $P(\bigcup_{i=1}^{n}{A_i})=\sum_{i=1}^{n}{P(A_i)}$
\item $P(\overline{A})=1-P(A)$
\item $P(B-A)=P(B)-P(A)$
\item (加法公式) $P(A\cup B)=P(A)+P(B)-P(AB)$
\end{enumerate}

\item {\bf 加法公式}: $P(A\cup B)=P(A)+P(B)-P(AB)$
\begin{enumerate}
\item 互不相容: $P(A\cup B)=P(A)+P(B)$
\item 相互独立: $P(A\cup B)=1-P(\overline{A})P(\overline{B})$
\end{enumerate}

\item {\bf 乘法公式}: $P(AB)=P(A)P(B|A)=P(B)P(A|B)$
\begin{enumerate}
\item 相互独立: $P(AB)=P(A)P(B)$
\end{enumerate}
\item {\bf 条件概率公式}: $P(B|A)=P_B(A)=\D\frac{P(AB)}{P(A)}$
\item {\bf 全概率公式}: $P(A)=\sum_{i=1}^{n}{P(B_i)P(A|B_i)}$
\item {\bf 贝叶斯公式}: $P(B_i|A)=\D\frac{P(B_i)P(A|B_i)}{\sum_{j=1}^{n}{P(B_j)P(A|B_j)}}$

\end{enumerate}

\subsection{随机变量}

\begin{enumerate}
\item 概率密度函数$f(x)$: $P\{a<X\le b\}=\int_a^b f(x)\d x$
\item 概率分布函数: $F(x)=P\{X\le x\}$
\begin{enumerate}
\item (单调不减性) $P\{a<X\le b\}=F(b)-F(a)$
\item (有界性) $F(-\infty)=0, F(+\infty)=1$
\item $F(x)=\sum_{x_k\le x}p_k$
\end{enumerate}
\item $F(x)=P\{X\le x\}=\int_{-\infty}^{x}f(t)\d t$\\$f(x)=F'(x)$

\item 已知连续性随机变量$X$的概率密度$f_X(x)$, 求随机变量$Y=g(X)$的概率密度$f_Y(y)$
\begin{enumerate}
\item {\bf 积分转化法} $F_Y(y)=P\{g(X)\le y\}$,\\$f_Y(y)=F'_Y(y)$
\item {\bf 单调函数公式法} 对函数关系$y=g(x)$, 给出反函数$x=h(y)$. \\
$f_Y(y)=\left\{\begin{array}{cl}
f(h(y))\cdot|h'(y)|, & a< y< b,\\
0, & \text{其他}
\end{array}\right.$
\end{enumerate}

\end{enumerate}

\subsubsection*{离散型随机变量}

\begin{enumerate}
\item {\bf 两点分布} $X\sim B(1,p)$\\
\fbox{$P\{X=1\}=p, P\{X=0\}=1-p$}
\begin{enumerate}[label={\sf 性质\arabic*}]
\item $E(X)=p, D(X)=p(1-p)$
\end{enumerate}
$\hat{p}=\overline{X}={\sum{X}}/{n}$是$p$的矩估计和极大似然估计

\item {\bf 二项分布} $X\sim B(n,p)$ 伯努利实验\\
\fbox{$b(k;n,p)=P\{X=k\}={{n}\choose{k}}p^k(1-p)^{(n-k)}$}
\begin{enumerate}[label={\sf 性质\arabic*}]
\item $E(X)=np, D(X)=np(1-p)$
\item $X$可以表示成$n$个同分布为$B(1,p)$的独立随机变量$X_i$
\item 若$X_i$相互独立, 且$X_i\sim B(n_i,p)$, \\则$X=\sum_{i=1}^{m}X_i\sim B(n,p)$
\end{enumerate}
$\hat{p}={\overline{X}}/{n}={\sum{X}}/{(nN)}$是$p$的矩估计和极大似然估计

\item {\bf 泊松分布} $X\sim P(\lambda)$\\
\fbox{$\D b(k;\lambda)=P\{X=k\}=\frac{\lambda^k}{k!}e^{-\lambda}$}
\begin{enumerate}[label={\sf 性质\arabic*}]
\item $E(X)=\lambda, D(X)=\lambda$
\item 若$X_i$相互独立, 且$X_i\sim P(\lambda_i)$, \\则$X=\sum_{i=1}^{m}X_i\sim P(\lambda)$
\item 若$X_1, X_2$相互独立, 且$X_i\sim P(\lambda_i)$, 则条件分布$X_i|(X_1+X_2=n)$是二项分布, 且$X_i|(X_1+X_2=n)\sim B(n, \frac{\lambda_i}{\lambda_1+\lambda_2})$
\end{enumerate}
某医院每天前来就诊的病人数, 某地区一段时间间隔内发生火灾的次数, 一段时间间隔内容器内部细菌数, 某地一年内发生暴雨的次数, 每条床单上的疵点数.

\item 二项分布和泊松分布的近似公式: 当$n\to\infty, p\to0$时, $b(k;n,p)\approx b(k;\lambda), \lambda=np$

\end{enumerate}

\subsubsection*{连续型随机变量}

\begin{enumerate}
\item {\bf 均匀分布} $X\sim U[a,b ]$\\
\fbox{$f(x)=\left\{\begin{array}{cl}
{1}/{(b-a)}, & a\le x\le b,\\
0, & \text{其他}
\end{array}\right.$}
\begin{enumerate}[label={\sf 性质\arabic*}]
\item $\D E(X)={(a+b)}/{2}, D(X)={(b-a)^2}/{12}$
\item 对任意满足$a\le c<d\le d$的$c, d$, \\有$P\{c\le X\le d\}=\D\frac{d-c}{b-a}$
\item $F(X)\sim U(0,1)$
\end{enumerate}

\item {\bf 指数分布} $X\sim EP(\lambda)$ 寿命分布\\
\fbox{$f(x)=\left\{\begin{array}{cl}
\lambda\e^{-\lambda x}, & x\ge 0,\\
0, & x < 0
\end{array}\right.$}
\begin{enumerate}[label={\sf 性质\arabic*}]
\item $E(X)=\lambda^{-1}, D(X)=\lambda^{-2}$
\item (无记忆性) $P\{X>x+y|X>y\}=P\{X>x\}$
%\item 若$X_i$独立同分布为$EP(\lambda)$, 则$X=\sum_{i=1}^{n}X_i\sim \Gamma(n,\lambda)$
%\item 若$X\sim U(0,1)$, 则$Y=-\ln x\sim EP(1)$
%\item 若$X\sim EP(1)$, 则$Y=(\beta X)^{\frac{1}{\alpha}}+\delta \sim W(\alpha,\beta,\delta)$
\end{enumerate}

\item {\bf 正态分布} $X\sim N(\mu,\sigma^2)$ 高斯分布\\
\fbox{$f(x)=\D\frac{1}{\sqrt{2\pi}\sigma}\e^{-\frac{(x-u)^2}{2\sigma^2}}$\\}
\begin{enumerate}
\item 标准正态分布: $X\sim N(0,1)$
\item 概率密度: $\D\varphi(x)\frac{1}{\sqrt{2\pi}}\e^{-\frac{x^2}{2}}$
\item 分布函数: $\D\varPhi(x)=\int_{-\infty}^{x}\frac{1}{\sqrt{2\pi}}\e^{-\frac{u^2}{2}}\d u$
\item $\varPhi(-x)=1-\varPhi(x)$
\item 标准正态分布上的$\alpha$分位点, $\varPhi(Z_\alpha)=1-\alpha$, \\$\D P\{X>Z_\alpha\}=\int_{Z_\alpha}^{+\infty}\varphi(x)\d x=\alpha$
\end{enumerate}
\begin{enumerate}[label={\sf 性质\arabic*}]
\item $E(X)=\mu, D(X)=\sigma^2$
\item $\D \sum_{i=1}^{n}c_iX_i+d\sim N\left(\sum_{i=1}^{n}c_i\mu_i+d,\sum_{i=1}^{n}c_i^2\sigma_i^2\right)$
\item 若$X\sim N(\mu,\sigma^2)$, 则$Y=\D\frac{X-\mu}{\sigma}\sim N(0,1)$;\\
若$X\sim N(0,1)$, 则$Y=\sigma X+\mu\sim N(\mu,\sigma^2)$
\item $\overline{X}=\D\frac{\sum_{i=1}^{n}X_i}{n}\sim N(\mu, \D\frac{\sigma^2}{n})$
\item $\D P\{a<X<b\}=\varPhi\Big(\frac{b-\mu}{\sigma}\Big)-\varPhi\Big(\frac{a-\mu}{\sigma}\Big)$\\
$\D P\{X<b\}=\varPhi\Big(\frac{b-\mu}{\sigma}\Big)$\\
$\D P\{X>a\}=1-\varPhi\Big(\frac{a-\mu}{\sigma}\Big)$
%\item 若$U_1, U_2 \sim U(0,1)$且相互独立, 令\\$X_1=\sqrt{-2\pi\ln U_1}\sin(2\pi U_2)$, \\$X_2=\sqrt{-2\pi\ln U_1}\cos(2\pi U_2)$, \\则$X_1, X_2 \sim N(0,1)$, 且相互独立
\end{enumerate}
某地区成年男性的身高, 某零件长度的测量误差, 半导体器件中的热噪声电流.

\end{enumerate}

\subsection{随机向量}

\begin{enumerate}
\item $F(x,y)=P\{X\le x,Y\le y\}$,\\$P\{X=x_i,Y=y_i\}=p_{ij}$
\begin{enumerate}
\item 离散型: $F(x,y)=\sum_{x_i\le x}\,\sum_{y_i\le y}p_{ij}$
\item 连续型: $F(x,y)=\int_{-\infty}^{y}\int_{-\infty}^{x}f(u,v)\d u\d v$
\begin{enumerate}
\item $f(x,y)\ge 0$
\item $F(+\infty,+\infty)=1$
\item $\D\frac{\partial^2 F(x,y)}{\partial x \partial y}=f(x,y)$
\item $P\{(X,Y)\in D\}=\iint_D f(x,y)\d x\d y$
\end{enumerate}
\end{enumerate}
\item 二维均匀分布
$f(x,y)=\left\{\begin{array}{cl}
{1}/{d}, & (x,y)\in D,\\
0, & \text{其他}
\end{array}\right.$
\item 二维正态分布 $(X,Y)\sim N(\mu_1,\mu_2,\sigma_1^2,\sigma_2^2,\rho)$\\
$\D f(x,y)=\frac{1}{2\pi\sigma_1\sigma_2\sqrt{1-\rho^2}}\e^{-\frac{1}{2(1-\rho^2)}\left[\frac{(x-\mu_1)^2}{\sigma_1^2}-2\rho\frac{(x-\mu_1)(y-\mu_2)}{\sigma_1\sigma_2}+\frac{(y-\mu_2)^2}{\sigma_2^2}\right]}$
\item {\bf 边缘概率分布}:\\$F_X(x)=F(x,+\infty), F_Y(x)=F(+\infty,y)$
\begin{enumerate}
\item 二维离散型随机向量: $p_{i\cdot}=\sum_j p_{ij}$, $p_{\cdot j}=\sum_i p_{ij}$
\item 二维连续型随机变量: $\begin{array}{l}
f_X(x)=\int_{-\infty}^{+\infty}f(x,y)\d y\\
f_Y(x)=\int_{-\infty}^{+\infty}f(x,y)\d x
\end{array}$
\end{enumerate}
\item {\bf 条件概率密度}: $f_{X|Y}(u|y)=\D\frac{f(x,y)}{f_Y(y)}$
\item 独立性的判断: $\forall x,y$, $F(x,y)=F_X(x)\cdot F_Y(y)$, \\对于连续型变量$f(x,y)=f_X(x)\cdot f_Y(y)$

\item $Z=X+Y$的密度: $\begin{array}{l}
f_Z(z)=\int_{-\infty}^{+\infty}f(x,z-x)\d x,\\
f_Z(z)=\int_{-\infty}^{+\infty}f(z-y, y)\d y.
\end{array}$
\begin{enumerate}
\item 相互独立: $\begin{array}{l}
f_Z(z)=\int_{-\infty}^{+\infty}f_X(x)f_Y(z-x)\d x,\\
f_Z(z)=\int_{-\infty}^{+\infty}f_X(z-y)f_Y(y)\d y.
\end{array}$
\end{enumerate}
\item $Z=\max\{X,Y\}$和$Z=\min\{X,Y\}$的分布: \\
$F_{\max}(z)=F_X(z)\cdot F_Y(z)$\\
$F_{\min}(z)=1-(1-F_X(z))\cdot(1-F_Y(z))$
\end{enumerate}

\subsection{数字特征}

\begin{enumerate}

\item {\bf 期望} $E(X)$
\begin{enumerate}
\item 离散型: $E(X)=\sum_i x_i p_i$
\item 连续型: $E(X)=\int_{-\infty}^{+\infty}xf(x)\d x$
\item $E(g(X))=\int_{-\infty}^{+\infty}g(x)f(x)\d x$
\end{enumerate}
\begin{enumerate}[label={\sf 性质\arabic*}]
\item $E(c) = c$
\item $E(kX) = kE(X)$
\item $E(X+Y) = E(X)+E(Y)$
\item 若$X,Y$相互独立, 则$E(XY) = E(X)E(Y)$
\end{enumerate}

\item {\bf 方差} $D(X)=\Var(X)=E\{[X-E(X)]^2\}$
\begin{enumerate}
\item 离散型: $D(X)=\sum_i [x_i-E(X)]^2 p_i$
\item 连续型: $D(X)=\int_{-\infty}^{+\infty}[x_i-E(X)]^2 f(x)\d x$
\item (计算公式) $D(X)=E(X^2)-E^2(X)$
\end{enumerate}
\begin{enumerate}[label={\sf 性质\arabic*}]
\item $D(c) = 0, D(X+c)=D(X)$
\item $D(kX) = k^2 D(X), D(-X)=D(X)$
\item $D(X+Y) = D(X)+D(Y)+2Cov(X,Y)$. 若$X,Y$相互独立, 则$D(X+Y) = D(X)+D(Y)$
\item 若$X,Y$相互独立, 则$D(aX\pm bY)=a^2 D(X)+b^2 D(Y)$
\end{enumerate}

\item {\bf 协方差} $\Cov(X,Y)=E\{[X-E(X)][Y-E(Y)]\}$
\begin{enumerate}[label={\sf 性质\arabic*}]
\item $\Cov(X,Y)=\Cov(Y,X)$
\item $\Cov(aX+b,cY+d)=ac\Cov(X,Y)$
\item $\Cov(X_1+X_2,Y)\Cov(X_1,Y)+\Cov(X_2,Y)$
\item (计算公式) $\Cov(X,Y)=E(XY)-E(X)E(Y)$
\item $\Cov^2(X,Y)\le D(X)D(Y)$, 等号成立当且仅当$\exists a,b, Y=aX+b$
\end{enumerate}

\item {\bf 相关系数} $\rho_{XY}=\D\frac{\Cov(X,Y)}{\sqrt{D(X)D(Y)}}$
\begin{enumerate}
\item $|\rho_{XY}|\le 1$
\item 互不相关: $\rho_{XY}=0$
\end{enumerate}

\item $E\{[X-E(X)]^k\}$称为$X$的$k$阶{\bf 中心矩}
\item 记$c_{ij}=\Cov(X_i,Y_j)$, 称$\bm C =(c_{ij})_{n\times n}$\\为$(X_1,X_2,\cdots,X_n)$的{\bf 协方差矩阵}

\end{enumerate}

\subsection{极限定理}

\begin{enumerate}

\item {\bf 切比雪夫不等式}: $\forall \varepsilon>0$,\\
$\begin{array}{l}
P\{|X-\mu|\ge\varepsilon\}\le\sigma^2/\varepsilon^2,\\
P\{|X-\mu|<\varepsilon\}\ge1-\sigma^2/\varepsilon^2
\end{array}$

\item {\bf 大数定律}: $X_i$相互独立, $Y_n=\frac{1}{n}\sum_{i=1}^{n}X_i$, $\forall \varepsilon>0$,\\
$\begin{array}{l}
\lim_{n\to\infty}P\{|Y_n-\mu|<\varepsilon\}=1.\\
\lim_{n\to\infty}P\{|n_A/n-p|<\varepsilon\}=1.
\end{array}$

\item 独立同分布的{\bf 中心极限定理}: \\$X_i$相互独立且服从同一分布, 则$Y_n=\D\frac{\sum_{i=1}^{n}X_i-n\mu}{\sqrt{n}{\sigma}}$\\的分布函数$F_n(x)=P\{Y_n\le x\}$收敛于$\varPhi(x)$,\\ 即$\lim_{x\to\infty}F_n(x)=\varPhi(x).$

\item (棣莫夫-拉普拉斯定理) $X_i$相互独立且服从$B(1,p)$, 则$\forall x$, $\D\lim_{x\to\infty}P\left\{\frac{\sum_{i=1}^{n}X_i-np}{\sqrt{np(1-p)}}\le x\right\}=\varPhi(x).$

\end{enumerate}

\subsection{样本与统计量}

\begin{enumerate}
\item {\bf 样本均值} $\overline{X}=\frac{1}{n}\sum_{i=1}^{n}X_i$ 是$\mu$的估计
\item {\bf 样本方差} $S^2=\frac{1}{n-1}\sum_{i=1}^{n}(X_i-\overline{X})^2$ 是$\sigma^2$的估计
\item 样本标准差 $S=\sqrt{\frac{1}{n-1}\sum_{i=1}^{n}(X_i-\overline{X})^2}$
\item {\bf 样本原点矩} $A_k=\frac{1}{n}\sum_{i=1}^{n}X_i^k$
\item {\bf 样本中心矩} $M_k=\frac{1}{n}\sum_{i=1}^{n}(X_i-\overline{X})^k$
\item 样本均值的{\bf 大样本分布}: $\D\overline{X}\sim N(\mu,\frac{\sigma^2}{n})$.\\
$\D F(x)\approx\varPhi\Big(\frac{x-\mu}{\sigma/\sqrt{n}}\Big)$,\\
$\D P\{|\overline{X}-\mu|\le c\}\approx 2\varPhi\Big(\frac{c}{\sigma/\sqrt{n}}\Big)-1$

\end{enumerate}

\subsubsection*{正态总体的抽样分布}

\begin{enumerate}
\item {\bf $\chi^2$分布} $X\sim \chi^2_n$\\
$X_i$独立同分布于$N(0,1)$, $X=\sum_{i=1}^{n}X_i^2$\\
\fbox{$f(x)=\left\{\begin{array}{cl}
\D\frac{1}{2^{\frac{n}{2}}\Gamma(\frac{n}{2})}x^{\frac{n}{2}-1}\e^{-\frac{x}{2}}, & x > 0,\\
0, & x \le 0.
\end{array}\right.$}\\
$\D\Gamma\Big(\frac{n}{2}\Big)=\left\{\begin{array}{cl}
(\frac{n}{2}-1)(\frac{n}{2}-2)\cdots3\cdot2\cdot1, & \text{若$n$为偶数},\\
(\frac{n}{2}-1)(\frac{n}{2}-2)\cdots\frac{3}{2}\cdot\frac{1}{2}\sqrt{\pi}, & \text{若$n$为奇数}.
\end{array}\right.$
\begin{enumerate}[label={\sf 性质\arabic*}]
\item $E(X^k)=2^k\frac{\Gamma(\frac{n}{2}+k)}{\Gamma(\frac{n}{2})}$,\\ $E(X)=n, D(X)=2n$
\item 若$X_i\sim\chi_{n_i}^2$独立同分布, 则$\sum_{i=1}^{m}X_i\sim\chi_n^2$, \\其中$n=\sum_{i=1}^{m}n_i$
\end{enumerate}
\item $\chi_n^2(\alpha)$为$\chi_n^2$分布上的$\alpha$分位点:\\$P\{\chi_n^2>\chi_n^2(\alpha)\}=\int_{\chi_n^2(\alpha)}^{+\infty}f(x)\d x=\alpha$

\item {\bf $t$分布} $T\sim t_n$ 学生分布\\
$X\sim N(0,1), Y\sim\chi_n^2, T=\D\frac{X}{\sqrt{Y/n}}$\\
\fbox{$f(x)=\D\frac{\Gamma(\frac{n+1}{2})}{\sqrt{n\pi}\Gamma(\frac{n}{2})}\Big(1+\frac{x^2}{n}\Big)^{-\frac{n+1}{2}}$}
\begin{enumerate}[label={\sf 性质\arabic*}]
\item $E(F)=0, D(F)=n/(n-2), n=3,4,\cdots$
\item $Y=\D\frac{F_2-F_1}{2\sqrt{F_1F_2/n}}\sim t_n$
\item $\D\frac{\overline{F}-\mu}{S/\sqrt{n}}\sim t_{n-1}$
\end{enumerate}
\item $t_n(\alpha)$为$t_n$分布上的$\alpha$分位点:\\$P\{T>t_n(\alpha)\}=\int_{t_n(\alpha)}^{+\infty}f(x)\d x=\alpha$

\item {\bf $F$分布} $F\sim F_{m,n}$\\
$X\sim\chi_m^2, Y\sim\chi_n^2, F=\D\frac{X/m}{Y/n}$\\
\fbox{$f(x)=\left\{\begin{array}{cl}
\D\frac{\Gamma(\frac{m+n}{2})}{\Gamma(\frac{m}{2})\Gamma(\frac{n}{2})}
\Big(\frac{m}{n}\Big)^{\frac{m}{2}}
x^{\frac{m}{2}-1}
\Big(1+\frac{m}{n}x\Big)^{-\frac{m+n}{2}}
, & x > 0,\\
0, & x \le 0.
\end{array}\right.$}
\begin{enumerate}[label={\sf 性质\arabic*}]
\item $E(F)={n}/{(n-2)}, n > 2;$\\$\D D(F)=\frac{2n^2(m+n-2)}{m(n-2)^2(n-4)}, n>4$
\item $1/F\sim F_{m,n}$
\item 若$X\sim t_n$, 则$X^2\sim F_{1,n}$
\end{enumerate}
\item $F_{m,n}(\alpha)$为$F_{m,n}$分布上的$\alpha$分位点:\\$P\{F>F_{m,n}(\alpha)\}=\int_{F_{m,n}(\alpha)}^{+\infty}f(x)\d x=\alpha$

\item (基本定理) 样本$X_i$来自正态总体$N(\mu,\sigma^2)$
\begin{enumerate}
\item $\D \overline{X}\sim N\Big(\mu,\frac{\sigma^2}{n}\Big)$
\item $\D \frac{(n-1)S^2}{\sigma^2}\sim\chi_{n-1}^2$
\item $\overline{X}$与$S^2$相互独立
\item $\D \frac{\overline{X}-\mu}{S/\sqrt{n}}\sim t_{n-1}$
\end{enumerate}

\end{enumerate}

\subsection{参数估计}

\begin{enumerate}
\item 
\end{enumerate}

\subsection{假设检验}

\begin{enumerate}
\item 
\end{enumerate}

\subsection{回归分析与方差分析}

\begin{enumerate}
\item 
\end{enumerate}

\section{重要分布表}

\begin{enumerate}
\item 泊松分布
$$P\{X\ge x\}=\sum_{r=x}^{\infty}\frac{\e^{-r}\lambda^r}{r!}$$
\item 标准正态分布
$$\varPhi(x)=\int_{-\infty}^{x}\frac{1}{\sqrt{2\pi}}\e^{-\frac{u^2}{2}}\d u$$
\item $t$ 分布
\item $\chi^2$ 分布
\item $F$ 分布

\end{enumerate}

\section{常用结论}

\begin{enumerate}
\item 人群中有相同生日: $\D P(A)=\frac{A_N^n}{N^n}$

\item 把$n$个物品分成$k$组, 每组恰有$n_i$个, 不同的分组方法有$\D\frac{n!}{\prod_{i=1}^{k}n_i!}$种

\item 将$n$个球放入$M$个盒子,有球的盒子数\\$E(X)=M\left[1-(1-1/M)^n\right]$

\end{enumerate}

\end{multicols}

\end{document}
