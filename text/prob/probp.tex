\documentclass[10pt,a4paper,nofonts]{ctexart}
\usepackage[utf8]{inputenc}
\def\tjf{{\tt{田劲锋}}}
\def\titlec{概率论与数理统计}
\usepackage[papersize={11cm,32cm},text={10.5cm,27cm}]{geometry} % 页面设置
% 22*18->7*18->10.5*27
\usepackage[unicode,breaklinks=true,
colorlinks=true,linkcolor=black,anchorcolor=black,citecolor=black,urlcolor=black,
pdftitle={\titlec},pdfauthor={\tjf}]{hyperref}
\usepackage{latexsym,amsmath,amssymb,bm,mathrsfs}
\usepackage[inline]{enumitem} % 调整列表样式
\usepackage{multicol} % 分栏
\usepackage{color,xcolor}

\setCJKmainfont[BoldFont={SimHei}]{SimSun}  % 主要字体: 宋体、黑体
\setCJKsansfont[BoldFont={STZhongsong}]{STFangsong} % 次要字体: 仿宋、中宋
\setCJKmonofont{KFKai} % 等宽字体: 楷体

\CJKsetecglue{\hspace{0.1em}}
\renewcommand\CJKglue{\hskip -0.3pt plus 0.08\baselineskip}
\frenchspacing
\widowpenalty=10000
\linespread{1} % 行距

\pagestyle{plain}

\newcommand{\D}{\displaystyle}
\renewcommand{\d}{{\mathrm{d}\:\!}}
\newcommand{\e}{{\mathrm{e}}}
\renewcommand{\c}{{\mathrm{c}}}
\renewcommand{\T}{{\mathrm{T}}}

\renewcommand{\le}{\leqslant}
\renewcommand{\ge}{\geqslant}

\newcommand{\defeq}{\stackrel{\rm def}{=}}
\newcommand{\<}{\langle}
\renewcommand{\>}{\rangle}

\newcommand{\net}{\backslash\!\!\!\!}
%\newcommand{\nexists}{\,\net\,\exists}

\DeclareMathOperator{\MSE}{MSE}
\DeclareMathOperator{\Var}{Var}
\DeclareMathOperator{\Cov}{Cov}
\DeclareMathOperator{\sh}{sh}
\DeclareMathOperator{\ch}{ch}

\DeclareMathOperator{\card}{card}
\DeclareMathOperator{\dom}{dom}
\DeclareMathOperator{\ran}{ran}
\DeclareMathOperator{\FLD}{FLD}
\DeclareMathOperator{\COV}{COV}
\DeclareMathOperator{\LUB}{LUB}
\DeclareMathOperator{\GLB}{GLB}

\setlist{noitemsep}
\setlist[enumerate]{topsep=0pt,partopsep=0pt,itemsep=0pt,parsep=0pt}
\setlist[enumerate,2]{label={(\arabic*)}}

\CTEXsetup[beforeskip={0ex},afterskip={0ex},name={第,章},number={\chinese{subsection}}]{subsection}
\CTEXsetup[beforeskip={0ex},afterskip={0ex},name={,.},number={\arabic{subsubsection}}]{subsubsection}

\begin{document}

%\CTEXnoindent
\section*{概率论与数理统计}
\subsection{随机事件}

\begin{enumerate}
\item {\bf 概率}的性质: 
\begin{enumerate}
\item $P(\varnothing)=0$
\item $A_i$两两互斥, $P(\bigcup{A_i})=\sum{P(A_i)}$
\item $P(\overline{A})=1-P(A)$
\item $P(B-A)=P(B)-P(A)$
\item (加法公式) $P(A\cup B)=P(A)+P(B)-P(AB)$
\end{enumerate}

\item {\bf 加法公式}: $P(A\cup B)=P(A)+P(B)-P(AB)$
\begin{enumerate}
\item 互不相容: $P(A\cup B)=P(A)+P(B)$
\item 相互独立: $P(A\cup B)=1-P(\overline{A})P(\overline{B})$
\end{enumerate}

\item {\bf 乘法公式}: $P(AB)=P(A)P(B|A)=P(B)P(A|B)$
\begin{enumerate}
\item 相互独立: $P(AB)=P(A)P(B)$
\end{enumerate}
\item {\bf 条件概率公式}: $P(B|A)=P_B(A)=\D\frac{P(AB)}{P(A)}$
\item {\bf 全概率公式}: $P(A)=\sum_{i=1}^{n}{P(B_i)P(A|B_i)}$
\item {\bf 贝叶斯公式}: $P(B_i|A)=\D\frac{P(B_i)P(A|B_i)}{\sum_{j=1}^{n}{P(B_j)P(A|B_j)}}$

\end{enumerate}

\subsection{随机变量}

\begin{enumerate}
\item 概率密度函数$f(x)$: $P\{a<X\le b\}=\int_a^b f(x)\d x$
\item 概率分布函数: $F(x)=P\{X\le x\}$
\begin{enumerate}
\item (单调不减性) $P\{a<X\le b\}=F(b)-F(a)$
\item (有界性) $F(-\infty)=0, F(+\infty)=1$
\item $F(x)=\sum_{x_k\le x}p_k$
\item $\sum F(x)=1$
\end{enumerate}
\item $F(x)=P\{X\le x\}=\int_{-\infty}^{x}f(t)\d t$\\$f(x)=F'(x)$

\item 已知连续性随机变量$X$的概率密度$f_X(x)$, 求随机变量$Y=g(X)$的概率密度$f_Y(y)$
\begin{enumerate}
\item {\bf 积分转化法} $F_Y(y)=P\{g(X)\le y\}$, $f_Y(y)=F'_Y(y)$
\item {\bf 单调函数公式法} 反函数$x=h(y)$ \\
$f_Y(y)=\left\{\begin{array}{cl}
f(h(y))\cdot|h'(y)|, & a< y< b,\\
0, & \text{其他}
\end{array}\right.$
\end{enumerate}

\end{enumerate}

\subsubsection*{离散型随机变量}

\begin{enumerate}
\item {\bf 两点分布} $X\sim B(1,p)$\\
\fbox{$P\{X=1\}=p, P\{X=0\}=1-p$}
\begin{enumerate}[label={\sf 性质\arabic*}]
\item $E(X)=p, D(X)=p(1-p)$
\end{enumerate}
$\hat{p}=\bar{X}={\sum{X}}/{n}$是$p$的矩估计和极大似然估计

\item {\bf 二项分布} $X\sim B(n,p)$ 伯努利实验\\
\fbox{$b(k;n,p)=P\{X=k\}={{n}\choose{k}}p^k(1-p)^{(n-k)}$}
\begin{enumerate}[label={\sf 性质\arabic*}]
\item $E(X)=np, D(X)=np(1-p)$
\item $X$可表示成$n$个$X_i\sim B(1,p)$的独立随机变量
\item 若$X_i$相互独立, 且$X_i\sim B(n_i,p)$, \\则$X=\sum X_i\sim B(n,p)$
\end{enumerate}
$\hat{p}={\bar{X}}/{n}={\sum{X}}/{(nN)}$是$p$的矩估计和极大似然估计

\item {\bf 泊松分布} $X\sim P(\lambda)$\\
\fbox{$\D b(k;\lambda)=P\{X=k\}=\frac{\lambda^k}{k!}e^{-\lambda}$}
\begin{enumerate}[label={\sf 性质\arabic*}]
\item $E(X)=\lambda, D(X)=\lambda$
\item 若$X_i$相互独立, 且$X_i\sim P(\lambda_i)$, 则$X=\sum X_i\sim P(\lambda)$
\item 若$X_1, X_2$相互独立, 且$X_i\sim P(\lambda_i)$, 则$X_i|(X_1+X_2=n)\sim B(n, \frac{\lambda_i}{\lambda_1+\lambda_2})$
\end{enumerate}
某医院每天前来就诊的病人数, 某地区一段时间间隔内发生火灾的次数, 一段时间间隔内容器内部细菌数, 某地一年内发生暴雨的次数, 每条床单上的疵点数.
\item \colorbox{lightgray}{$P\{X\ge x\}=\sum_{r=x}^{\infty}\frac{\e^{-r}\lambda^r}{r!}$}

\item 当$n\to\infty, p\to0$时, $b(k;n,p)\approx b(k;\lambda), \lambda=np$

\end{enumerate}

\subsubsection*{连续型随机变量}

\begin{enumerate}
\item {\bf 均匀分布} $X\sim U[a,b ]$\\
\fbox{$f(x)=\left\{\begin{array}{cl}
{1}/{(b-a)}, & a\le x\le b,\\
0, & \text{其他}
\end{array}\right.$}
\begin{enumerate}[label={\sf 性质\arabic*}]
\item $\D E(X)={(a+b)}/{2}, D(X)={(b-a)^2}/{12}$
\item $(\forall c,d) a\le c<d\le d$, 有$P\{c\le X\le d\}=\D\frac{d-c}{b-a}$
\item $F(X)\sim U(0,1)$
\end{enumerate}

\item {\bf 指数分布} $X\sim EP(\lambda)\sim e(\lambda^{-1})$ 寿命分布\\
\fbox{$f(x)=\left\{\begin{array}{cl}
\lambda\e^{-\lambda x}, & x\ge 0,\\
0, & x < 0
\end{array}\right.$}
\begin{enumerate}[label={\sf 性质\arabic*}]
\item $E(X)=\lambda^{-1}, D(X)=\lambda^{-2}$
\item (无记忆性) $P\{X>x+y|X>y\}=P\{X>x\}$
%\item 若$X_i$独立同分布为$EP(\lambda)$, 则$X=\sum_{i=1}^{n}X_i\sim \Gamma(n,\lambda)$
%\item 若$X\sim U(0,1)$, 则$Y=-\ln x\sim EP(1)$
%\item 若$X\sim EP(1)$, 则$Y=(\beta X)^{\frac{1}{\alpha}}+\delta \sim W(\alpha,\beta,\delta)$
\end{enumerate}

\item {\bf 正态分布} $X\sim N(\mu,\sigma^2)$ 高斯分布\\
\fbox{$f(x)=\D\frac{1}{\sqrt{2\pi}\sigma}\e^{-\frac{(x-u)^2}{2\sigma^2}}$\\}
\begin{enumerate}
\item 标准正态分布: $X\sim N(0,1)$
\item 概率密度: $\varphi(x)\frac{1}{\sqrt{2\pi}}\e^{-\frac{x^2}{2}}$
\item 分布函数: \colorbox{lightgray}{$\varPhi(x)=\int_{-\infty}^{x}\frac{1}{\sqrt{2\pi}}\e^{-\frac{u^2}{2}}\d u$}
\item $\varPhi(-x)=1-\varPhi(x)$
\item 标准正态分布上的$\alpha$分位点, $\varPhi(Z_\alpha)=1-\alpha$, \\$P\{X>Z_\alpha\}=\int_{Z_\alpha}^{+\infty}\varphi(x)\d x=\alpha$
\end{enumerate}
\begin{enumerate}[label={\sf 性质\arabic*}]
\item $E(X)=\mu, D(X)=\sigma^2$
\item $\sum_{i=1}^{n}c_iX_i+d\sim N\left(\sum_{i=1}^{n}c_i\mu_i+d,\sum_{i=1}^{n}c_i^2\sigma_i^2\right)$
\item 若$X\sim N(\mu,\sigma^2)$, 则$Y=\frac{X-\mu}{\sigma}\sim N(0,1)$;\\
若$X\sim N(0,1)$, 则$Y=\sigma X+\mu\sim N(\mu,\sigma^2)$
\item $\bar{X}=\frac{\sum_{i=1}^{n}X_i}{n}\sim N(\mu, \frac{\sigma^2}{n})$
\item $P\{a<X<b\}=\varPhi (\frac{b-\mu}{\sigma} )-\varPhi (\frac{a-\mu}{\sigma} )$\\
$P\{X<b\}=\varPhi (\frac{b-\mu}{\sigma} )$\\
$P\{X>a\}=1-\varPhi (\frac{a-\mu}{\sigma} )$
%\item 若$U_1, U_2 \sim U(0,1)$且相互独立, 令\\$X_1=\sqrt{-2\pi\ln U_1}\sin(2\pi U_2)$, \\$X_2=\sqrt{-2\pi\ln U_1}\cos(2\pi U_2)$, \\则$X_1, X_2 \sim N(0,1)$, 且相互独立
\end{enumerate}
某地区成年男性的身高, 某零件长度的测量误差, 半导体器件中的热噪声电流.

\end{enumerate}

\subsection{随机向量}

\begin{enumerate}
\item $F(x,y)=P\{X\le x,Y\le y\}$,\\$P\{X=x_i,Y=y_i\}=p_{ij}$
\begin{enumerate}
\item 离散型: $F(x,y)=\sum_{x_i\le x}\,\sum_{y_i\le y}p_{ij}$
\item 连续型: $F(x,y)=\int_{-\infty}^{y}\int_{-\infty}^{x}f(u,v)\d u\d v$
\begin{enumerate}
\item $f(x,y)\ge 0$
\item $F(+\infty,+\infty)=1$
\item $\D\frac{\partial^2 F(x,y)}{\partial x \partial y}=f(x,y)$
\item $P\{(X,Y)\in D\}=\iint_D f(x,y)\d x\d y$
\end{enumerate}
\end{enumerate}
\item 二维均匀分布
$f(x,y)=\left\{\begin{array}{cl}
{1}/{d}, & (x,y)\in D,\\
0, & \text{其他}
\end{array}\right.$
\item 二维正态分布 $(X,Y)\sim N(\mu_1,\mu_2,\sigma_1^2,\sigma_2^2,\rho)$\\
$f(x,y)=\frac{1}{2\pi\sigma_1\sigma_2\sqrt{1-\rho^2}}\e^{-\frac{1}{2(1-\rho^2)}\left[\frac{(x-\mu_1)^2}{\sigma_1^2}-2\rho\frac{(x-\mu_1)(y-\mu_2)}{\sigma_1\sigma_2}+\frac{(y-\mu_2)^2}{\sigma_2^2}\right]}$
\item {\bf 边缘概率分布}:\\$F_X(x)=F(x,+\infty), F_Y(x)=F(+\infty,y)$
\begin{enumerate}
\item 二维离散型随机向量: $p_{i\cdot}=\sum_j p_{ij}$, $p_{\cdot j}=\sum_i p_{ij}$
\item 二维连续型随机变量: $\begin{array}{l}
f_X(x)=\int_{-\infty}^{+\infty}f(x,y)\d y\\
f_Y(x)=\int_{-\infty}^{+\infty}f(x,y)\d x
\end{array}$
\end{enumerate}
\item {\bf 条件概率密度}: $f_{X|Y}(u|y)=\D\frac{f(x,y)}{f_Y(y)}$
\item 独立性的判断: $\forall x,y$, $F(x,y)=F_X(x)\cdot F_Y(y)$, \\对于连续型变量$f(x,y)=f_X(x)\cdot f_Y(y)$

\item $Z=X+Y$的密度: $\begin{array}{l}
f_Z(z)=\int_{-\infty}^{+\infty}f(x,z-x)\d x,\\
f_Z(z)=\int_{-\infty}^{+\infty}f(z-y, y)\d y.
\end{array}$
\begin{enumerate}
\item 相互独立: $\begin{array}{l}
f_Z(z)=\int_{-\infty}^{+\infty}f_X(x)f_Y(z-x)\d x,\\
f_Z(z)=\int_{-\infty}^{+\infty}f_X(z-y)f_Y(y)\d y.
\end{array}$
\end{enumerate}
\item $Z=\max\{X,Y\}$和$Z=\min\{X,Y\}$的分布: \\
$F_{\max}(z)=F_X(z)\cdot F_Y(z)$\\
$F_{\min}(z)=1-(1-F_X(z))\cdot(1-F_Y(z))$
\end{enumerate}

\subsection{数字特征}

\begin{enumerate}

\item {\bf 期望} $E(X)$
\begin{enumerate}
\item 离散型: $E(X)=\sum_{i=1}^n x_i p_i$
\item 连续型: $E(X)=\int_{-\infty}^{+\infty}xf(x)\d x$
\item $E(g(X))=\int_{-\infty}^{+\infty}g(x)f(x)\d x$
\end{enumerate}
\begin{enumerate}[label={\sf 性质\arabic*}]
\item $E(c) = c$
\item $E(kX) = kE(X)$
\item $E(X+Y) = E(X)+E(Y)$
\item 若$X,Y$相互独立, 则$E(XY) = E(X)E(Y)$
\end{enumerate}

\item {\bf 方差} $D(X)=\Var(X)=E\{[X-E(X)]^2\}$
\begin{enumerate}
\item 离散型: $D(X)=\sum_{i=1}^n [x_i-E(X)]^2 p_i$
\item 连续型: $D(X)=\int_{-\infty}^{+\infty}[x_i-E(X)]^2 f(x)\d x$
\item (计算公式) $D(X)=E(X^2)-E^2(X)$
\end{enumerate}
\begin{enumerate}[label={\sf 性质\arabic*}]
\item $D(c) = 0, D(X+c)=D(X)$
\item $D(kX) = k^2 D(X), D(-X)=D(X)$
\item $D(X+Y) = D(X)+D(Y)+2\Cov(X,Y)$. \\若$X,Y$相互独立, 则$D(X+Y) = D(X)+D(Y)$
\item 若$X,Y$相互独立, 则$D(aX\pm bY)=a^2 D(X)+b^2 D(Y)$
\end{enumerate}

\item {\bf 协方差} $\Cov(X,Y)=E\{[X-E(X)][Y-E(Y)]\}$
\begin{enumerate}[label={\sf 性质\arabic*}]
\item $\Cov(X,Y)=\Cov(Y,X)$
\item $\Cov(aX+b,cY+d)=ac\Cov(X,Y)$
\item $\Cov(X_1+X_2,Y)\Cov(X_1,Y)+\Cov(X_2,Y)$
\item (计算公式) $\Cov(X,Y)=E(XY)-E(X)E(Y)$
\item $\Cov^2(X,Y)\le D(X)D(Y)$, 等号成立 iff $(\exists a,b) Y=aX+b$
\end{enumerate}

\item {\bf 相关系数} $\rho_{XY}=\D\frac{\Cov(X,Y)}{\sqrt{D(X)D(Y)}}$
\begin{enumerate}
\item $|\rho_{XY}|\le 1$
\item 互不相关: $\rho_{XY}=0$
\end{enumerate}

\item $E\{[X-E(X)]^k\}$称为$X$的$k$阶{\bf 中心矩}
\item 记$c_{ij}=\Cov(X_i,Y_j)$, 称$\bm C =(c_{ij})_{n\times n}$\\为$(X_1,X_2,\cdots,X_n)$的{\bf 协方差矩阵}

\end{enumerate}

\subsection{极限定理}

\begin{enumerate}

\item {\bf 切比雪夫不等式}: $\forall \varepsilon>0$,\\
$\begin{array}{l}
P\{|X-\mu|\ge\varepsilon\}\le\sigma^2/\varepsilon^2,\\
P\{|X-\mu|<\varepsilon\}\ge1-\sigma^2/\varepsilon^2
\end{array}$

\item {\bf 大数定律}: $X_i$相互独立, $Y_n=\frac{1}{n}\sum_{i=1}^{n}X_i$, $\forall \varepsilon>0$,\\
$\begin{array}{l}
\lim_{n\to\infty}P\{|Y_n-\mu|<\varepsilon\}=1.\\
\lim_{n\to\infty}P\{|n_A/n-p|<\varepsilon\}=1.
\end{array}$

\item 独立同分布的{\bf 中心极限定理}: \\$X_i$相互独立且服从同一分布, 则$Y_n=\D\frac{\sum_{i=1}^{n}X_i-n\mu}{\sqrt{n}{\sigma}}$\\的分布函数$F_n(x)=P\{Y_n\le x\}$收敛于$\varPhi(x)$,\\ 即$\lim_{x\to\infty}F_n(x)=\varPhi(x).$

\item (棣莫夫-拉普拉斯定理) $X_i$相互独立且服从$B(1,p)$, 则$\forall x$, \\$\D\lim_{x\to\infty}P\left\{\frac{\sum_{i=1}^{n}X_i-np}{\sqrt{np(1-p)}}\le x\right\}=\varPhi(x).$

\end{enumerate}

\subsection{样本与统计量}

\begin{enumerate}
\item {\bf 样本均值} $\bar{X}=\frac{1}{n}\sum_{i=1}^{n}X_i$ 是$\mu$的估计
\item {\bf 样本方差} $S^2=\frac{1}{n-1}\sum_{i=1}^{n}(X_i-\bar{X})^2$ 是$\sigma^2$的估计
\item 样本标准差 $S=\sqrt{\frac{1}{n-1}\sum_{i=1}^{n}(X_i-\bar{X})^2}$
\item {\bf 样本原点矩} $A_k=\frac{1}{n}\sum_{i=1}^{n}X_i^k$
\item {\bf 样本中心矩} $M_k=\frac{1}{n}\sum_{i=1}^{n}(X_i-\bar{X})^k$
\item 样本均值的{\bf 大样本分布}: $\D\bar{X}\sim N(\mu,\frac{\sigma^2}{n})$.\\
$\D F(x)\approx\varPhi\Big(\frac{x-\mu}{\sigma/\sqrt{n}}\Big)$,\\
$\D P\{|\bar{X}-\mu|\le c\}\approx 2\varPhi\Big(\frac{c}{\sigma/\sqrt{n}}\Big)-1$

\end{enumerate}

\subsubsection*{正态总体的抽样分布}

\begin{enumerate}

\item $\D\Gamma\Big(\frac{n}{2}\Big)=\left\{\begin{array}{cl}
(\frac{n}{2}-1)(\frac{n}{2}-2)\cdots3\cdot2\cdot1, & \text{$n$为偶数},\\
(\frac{n}{2}-1)(\frac{n}{2}-2)\cdots\frac{3}{2}\cdot\frac{1}{2}\sqrt{\pi}, & \text{$n$为奇数}.
\end{array}\right.$

\quad

\item {\bf $\chi^2$分布} $X\sim \chi^2_n$\\
$X_i$独立同分布于$N(0,1)$, $X=\sum_{i=1}^{n}X_i^2$\\
\fbox{$f(x)=\left\{\begin{array}{cl}
\D\frac{1}{2^{\frac{n}{2}}\Gamma(\frac{n}{2})}x^{\frac{n}{2}-1}\e^{-\frac{x}{2}}, & x > 0,\\
0, & x \le 0.
\end{array}\right.$}
\begin{enumerate}[label={\sf 性质\arabic*}]
\item $E(X^k)=2^k\frac{\Gamma(\frac{n}{2}+k)}{\Gamma(\frac{n}{2})}$,\\ $E(X)=n, D(X)=2n$
\item 若$X_i\sim\chi_{n_i}^2$独立同分布, 则$\sum_{i=1}^{m}X_i\sim\chi_n^2$, \\其中$n=\sum_{i=1}^{m}n_i$
\end{enumerate}
\item $\chi_n^2(\alpha)$为$\chi_n^2$分布上的$\alpha$分位点:\\
\colorbox{lightgray}{$P\{\chi_n^2>\chi_n^2(\alpha)\}=\int_{\chi_n^2(\alpha)}^{+\infty}f(x)\d x=\alpha$}

\item {\bf $t$分布} $T\sim t_n$ 学生分布\\
$X\sim N(0,1), Y\sim\chi_n^2, T=\D\frac{X}{\sqrt{Y/n}}$\\
\fbox{$f(x)=\D\frac{\Gamma(\frac{n+1}{2})}{\sqrt{n\pi}\Gamma(\frac{n}{2})}\Big(1+\frac{x^2}{n}\Big)^{-\frac{n+1}{2}}$}
\begin{enumerate}[label={\sf 性质\arabic*}]
\item $E(F)=0, D(F)=n/(n-2), n=3,4,\cdots$
\item $Y=\D\frac{F_2-F_1}{2\sqrt{F_1F_2/n}}\sim t_n$
\item $\D\frac{\overline{F}-\mu}{S/\sqrt{n}}\sim t_{n-1}$
\end{enumerate}
\item $t_n(\alpha)$为$t_n$分布上的$\alpha$分位点:\\
\colorbox{lightgray}{$P\{T>t_n(\alpha)\}=\int_{t_n(\alpha)}^{+\infty}f(x)\d x=\alpha$}

\iffalse
\item {\bf $F$分布} $F\sim F_{m,n}$\\
$X\sim\chi_m^2, Y\sim\chi_n^2, F=\D\frac{X/m}{Y/n}$\\
\fbox{$f(x)=\left\{\begin{array}{cl}
\D\frac{\Gamma(\frac{m+n}{2})}{\Gamma(\frac{m}{2})\Gamma(\frac{n}{2})}
\Big(\frac{m}{n}\Big)^{\frac{m}{2}}
x^{\frac{m}{2}-1}
\Big(1+\frac{m}{n}x\Big)^{-\frac{m+n}{2}}
, & x > 0,\\
0, & x \le 0.
\end{array}\right.$}
\begin{enumerate}[label={\sf 性质\arabic*}]
\item $E(F)={n}/{(n-2)}, n > 2;$\\$\D D(F)=\frac{2n^2(m+n-2)}{m(n-2)^2(n-4)}, n>4$
\item $1/F\sim F_{m,n}$
\item 若$X\sim t_n$, 则$X^2\sim F_{1,n}$
\end{enumerate}
\item $F_{m,n}(\alpha)$为$F_{m,n}$分布上的$\alpha$分位点:\\
\colorbox{lightgray}{$P\{F>F_{m,n}(\alpha)\}=\int_{F_{m,n}(\alpha)}^{+\infty}f(x)\d x=\alpha$}
\fi

\item (基本定理) 样本$X_i$来自正态总体$N(\mu,\sigma^2)$
\begin{enumerate}
\item $\bar{X}\sim N (\mu,{\sigma^2}/{n} )$
\item ${(n-1)S^2}/{\sigma^2}\sim\chi_{n-1}^2$
\item $\bar{X}$与$S^2$相互独立
\item $\D\frac{\bar{X}-\mu}{S/\sqrt{n}}\sim t_{n-1}$
\end{enumerate}

\end{enumerate}

\subsection{参数估计}

\begin{enumerate}
\item 矩估计: $\alpha_m=E(X^m), A_m=\sum X_i^m, \alpha_m(\theta_i)=A_m$
\begin{enumerate}
\item 正态分布: $\hat\mu=\bar{X}, \hat\sigma^2=\frac{1}{n}\sum_{i=1}^n(X_i-\bar{X})^2$
\item 均匀分布: $\hat a=\bar{X}-\sqrt{3}\hat\sigma, \hat b=\bar{X}+\sqrt{3}\hat\sigma$
\end{enumerate}
\item 极大似然估计: $L(x_i;\theta_j)=\prod f(x_i,\theta_j)$,\\
$\ln L(\theta)=\sum \ln f(x_i; \theta_j)$,
$\frac{\partial\ln L(\theta)}{\partial\theta_i}=0$
\begin{enumerate}
\item 正态分布: $\mu^*=\bar{X}, \sigma^{2*}=\frac{1}{n}\sum_{i=1}^n(X_i-\bar{X})^2$
\item 泊松分布: $\lambda^*=\bar{X}$
\item 均匀分布: $a^*=\min\{X_i\}, b^*=\max\{X_i\}$
\end{enumerate}
\item 无偏估计: $E[\hat\theta(X_i)]=\theta$
\begin{enumerate}
\item $E(\bar{X})=\mu$
\item $E(S^2)=\sigma^2$
\item $S$不是$\sigma$的无偏估计
\end{enumerate}
\item 均方误差: $\MSE(\hat\theta)=E[(\hat\theta-\theta)^2]=D(\hat\theta)+[E(\hat\theta)-\theta]^2$
\begin{enumerate}
\item 无偏估计: $\MSE(\hat\theta)=D(\hat\theta)$
\item 两个估计中哪个方差小, 哪个优
\end{enumerate}
\item $P\{\theta\in[\hat\theta_1,\hat\theta_2]\}\ge1-\alpha$\\
置信区间 $[\hat\theta_1,\hat\theta_2]$, 置信系数 $1-\alpha$
\item $X_i\sim N(\mu,\sigma^2)$, 置信系数为$1-\alpha$
\begin{enumerate}
\item $\mu$的置信区间: $[\bar{X}-\frac{\sigma}{\sqrt{n}}Z_{\frac{\alpha}{2}},\bar{X}+\frac{\sigma}{\sqrt{n}}Z_{\frac{\alpha}{2}}]$\\
$\sigma^2$未知时: $[\bar{X}-\frac{S}{\sqrt{n}}t_{n-1}(\frac{\alpha}{2}),\bar{X}+\frac{S}{\sqrt{n}}t_{n-1}(\frac{\alpha}{2})]$
\item $\sigma^2$的置信区间: $[\frac{(n-1)S^2}{\chi_{n-1}^2(\frac{\alpha}{2})},\frac{(n-1)S^2}{\chi_{n-1}^2(1-\frac{\alpha}{2})}]$
\end{enumerate}
\item $X_i\sim N(\mu_1,\sigma_1^2), Y_i\sim N(\mu_2,\sigma_2^2)$互相独立
\begin{enumerate}
\item $\bar{X}-\bar{Y}\sim N(\mu_1-\mu_2,\sigma_1^2/m+\sigma_2^2/n)$,\\
或$\frac{(\bar{X}-\bar{Y})-(\mu_1-\mu_2)}{\sqrt{\sigma_1^2/m+\sigma_2^2/n}}\sim N(0,1)$
\item $\frac{(\bar{X}-\bar{Y})-(\mu_1-\mu_2)}{S\sqrt{1/m+1/n}}\sim t_{m+n-2}$,\\
其中$S^2=\frac{(m-1)S_1^2+(n-1)S_2^2}{m+n-2}$
\item $\mu_1-\mu_2$的置信区间: $[\bar{X}-\bar{Y}\pm Z_{\frac{\alpha}{2}}\sqrt{\frac{\sigma_1^2}{\sqrt{m}}+\frac{\sigma_2^2}{\sqrt{n}}}]$\\
$\sigma_1^2=\sigma_2^2=\sigma^2$: $[\bar{X}-\bar{Y}\pm t_{m+n-2}{(\frac{\alpha}{2})}S\sqrt{\frac{1}{\sqrt{m}}+\frac{1}{\sqrt{n}}}]$
\end{enumerate}
\item 大样本参数$\mu$的置信区间: $[\bar{X}\pm\frac{\sigma}{\sqrt{n}}Z_{\frac{\alpha}{2}}]$
\item 二项分布参数$p$的置信区间: $[\hat{p}\pm Z_{\frac{\alpha}{2}}\sqrt{\hat{p}(1-\hat{p})/n}]$,\\
其中$\hat{p}=Y_n/n=\sum_{i=1}^n X_i/n$
\item 泊松分布参数$\lambda$的置信区间: $[\bar{X}\pm Z_{\frac{\alpha}{2}}\sqrt{\bar{X}/n}]$
\end{enumerate}

\subsection{假设检验}

\begin{enumerate}
\item 检验$H_0$有可能犯的两类错误
\begin{enumerate}
\item 第一类错误 “弃真”错误: $H_0$正确但是被拒绝了
\item 第二类错误 “采伪”错误: $H_0$错误但是被接受了
\end{enumerate}
\item 检验的显著性水平: $P\{\text{犯第一类错误}\}\le\alpha\in(0,1)$
\end{enumerate}

\subsubsection{正态总体均值的检验}

\begin{enumerate}

\item 单个正态总体$N(\mu,\sigma^2)$均值$\mu$的检验
\begin{enumerate}
\item $H_0:\mu=\mu_0\leftrightarrow H_1:\mu\neq\mu_0$
\begin{enumerate}
\item 拒绝域: $\frac{|\bar{X}-\mu_0|}{\sigma/\sqrt{n}}\ge Z_{\frac{\alpha}{2}}$
\item 拒绝域: $\frac{|\bar{X}-\mu_0|}{S/\sqrt{n}}\ge t_{n-1}{(\frac{\alpha}{2})}$
\end{enumerate}
\item $H_0:\mu=\mu_0\leftrightarrow H_1:\mu>\mu_0$
\begin{enumerate}
\item 拒绝域: $\frac{\bar{X}-\mu_0}{S/\sqrt{n}} \ge t_{n-1}{(\alpha)}$
\end{enumerate}
\end{enumerate}

\item 两个正态总体$N(\mu_1,\sigma_1^2)$和$N(\mu_2,\sigma_2^2)$均值的比较
\begin{enumerate}
\item $H_0:\mu_1=\mu_2\leftrightarrow H_1:\mu_1\neq\mu_2$
\begin{enumerate}
\item 拒绝域: $|\bar{X}-\bar{Y}|\ge Z_{\frac{\alpha}{2}}\sqrt{\sigma_1^2/m+\sigma_2^2/n}$
\item 拒绝域: $|\bar{X}-\bar{Y}|\ge t_{m+n-2}{(\frac{\alpha}{2})}\sqrt{\frac{m+n}{mn}}S$
\end{enumerate}
\item $H'_0:\mu_1-\mu_2\ge0\leftrightarrow H'_1:\mu_1-\mu_2<0$
\begin{enumerate}
\item 拒绝域: $\bar{X}-\bar{Y}\le -t_{m+n-2}{(\alpha)}\sqrt{\frac{m+n}{mn}}S$
\end{enumerate}
\item $H''_0:\mu_1-\mu_2\le0\leftrightarrow H''_1:\mu_1-\mu_2>0$
\begin{enumerate}
\item 拒绝域: $\bar{X}-\bar{Y}\ge t_{m+n-2}{(\alpha)}\sqrt{\frac{m+n}{mn}}S$
\end{enumerate}
\end{enumerate}

\item 成对数据的$t$检验
\begin{enumerate}
\item $H_0:\mu=0\leftrightarrow H_1:\mu\neq0$, $d_i=X_i-Y_i$
\begin{enumerate}
\item 拒绝域: $|\bar{d}|\ge t_{n-1}{(\frac{\alpha}{2})} \frac{S_d}{\sqrt{n}}$
\end{enumerate}
\end{enumerate}

\end{enumerate}

\subsubsection{正态总体方差的检验}

\begin{enumerate}

\item 单个正态总体方差的$\chi^2$检验
\begin{enumerate}
\item $H_0:\sigma^2=\sigma^2_0\leftrightarrow H_1:\sigma^2\neq\sigma^2_0$
\begin{enumerate}
\item 拒绝域: $(n-1)S^2/\sigma^2_0\le\chi^2_{n-1}(1-\frac{\alpha}{2})$
\item 拒绝域: $(n-1)S^2/\sigma^2_0\ge\chi^2_{n-1}(\frac{\alpha}{2})$
\end{enumerate}
\item $H_0:\sigma^2\le\sigma^2_0\leftrightarrow H_1:\sigma^2>\sigma^2_0$
\begin{enumerate}
\item 拒绝域: $(n-1)S^2/\sigma^2_0\ge\chi^2_{n-1}(\alpha)$
\end{enumerate}
\end{enumerate}

\iffalse
\item 两个正态总体方差比的$F$检验
\begin{enumerate}
\item $H_0:\sigma^2_1=\sigma^2_2\leftrightarrow H_1:\sigma^2_1\neq\sigma^2_2$
\begin{enumerate}
\item 拒绝域: $S^2_1/S^2_2\le F_{m-1,n-1}(1-\frac{\alpha}{2})$
\item 拒绝域: $S^2_1/S^2_2\ge F_{m-1,n-1}(\frac{\alpha}{2})$
\end{enumerate}
\item $H'_0:\sigma^2_1\le\sigma^2_2\leftrightarrow H'_1:\sigma^2_1>\sigma^2_2$
\begin{enumerate}
\item 拒绝域: $S^2_1/S^2_2\ge F_{m-1,n-1}(\alpha)$
\end{enumerate}
\end{enumerate}
\fi

\end{enumerate}

\subsubsection{拟合优度检验}

\begin{enumerate}
\item $H_0:$样本$X_i$的总体分布为$F(x)$
\begin{enumerate}
\item 把$(-\infty,+\infty)$分割为$k$个区间:\\$I_1=(a_0,a_1], \cdots, I_k=(a_{k-1},a_k)$
\item 计算每个区间的实际频数$f_i$
\item 计算理论频数$p_i(\theta)=F(a_i,\theta)-F(a_{i-1},\theta)$
\item 计算偏差平方和$\chi^2=\sum_{i=1}^k\frac{[f_i-np_i(\theta^*)]^2}{np_i(\theta^*)}$
\item 假设$H_0$的水平$\alpha$的检验拒绝域为$\chi^2\ge\chi^2_{k-r-1}(\alpha)$
\end{enumerate}
\end{enumerate}

\subsubsection{独立性检验}

\begin{enumerate}
\item $H_0:p_{ij}=p_{i\cdot}p_{\cdot j}$
\begin{enumerate}
\item 拒绝域: $\chi^2\ge\chi^2_{(a-1)(b-1)}(\alpha)$,\\
其中$\chi^2=\sum_{i=1}^a\sum_{j=1}^b\frac{(nn_{ij}-n_{i\cdot}n_{\cdot j})^2}{n_{ij}n_{i\cdot}n_{\cdot j}}$
\end{enumerate}
\end{enumerate}


\subsection{回归分析}

\begin{enumerate}

\item 一元线性回归模型 $y_i=\beta_0+\beta_1x_i+e_i$
\begin{enumerate}[label={\sf 性质\arabic*}]
\item $E(\hat\beta_0)=\beta_0,E(\hat\beta_1)=\beta_1$为无偏估计
\item 假设$e_i\sim N(0,\sigma^2)$, $S_{xx}=\sum(x_i-\bar x)^2$,则\\
$\hat\beta_0\sim N(\beta_0,(1/n+\bar x^2/S_{xx})\sigma^2)$,\\
$\hat\beta_1\sim N(\beta_1,\sigma^2/S_{xx})$.
\end{enumerate}
\item 高斯-马尔可夫假设: $E(e_i)=0$,$D(e_i)=\sigma^2, \Cov(e_i,e_j)=0$
\item $L_{xx}=\sum x_i^2-n\bar x^2$,\\
$L_{xy}=\sum x_iy_i-n\overline{xy}$,\\
$L_{yy}=\sum y_i^2-n\bar y^2$.
\item 最小二乘估计$\Bigg\{\!\begin{array}{l}
\hat a=\bar{y}-\hat b\bar{x},\\
\hat b=\frac{\sum x_iy_i -n\overline{xy}}{\sum x_i^2-n\bar x^2}=\frac{L_{xy}}{L_{xx}}.\\
\end{array}$
\item 经验回归直线方程 $\hat y=\hat b x+\hat a$
\item 残差 $\hat e_i=y_i-(\hat\beta_0+\hat\beta_1x_i)$
\item $\hat\sigma^2=\sum\hat e^2_i/(n-2)$
\begin{enumerate}
\item $\hat\sigma^2$是$\sigma^2$的无偏估计
\item $(n-2)\hat\sigma^2/\sigma^2\sim\chi^2_{n-2}$,且$\hat\sigma^2$与$\hat\beta_0,\hat\beta_1$相互独立
\end{enumerate}

\item 回归方程的显著性检验
\begin{enumerate}
\item $H_0:\beta_1=0\leftrightarrow H_1:\beta_1\neq0$
\begin{enumerate}
\item 拒绝域: $r=\frac{L_{xy}}{\sqrt{L_{xx}L_{yy}}}\ge r_{\alpha=0.01}(n-2)$
\item 拒绝域: $|T|= \big|\frac{\hat\beta_1^2}{\hat\sigma/\sqrt{S_{xx}}}\big|\ge t_{n-2}(\frac{\alpha}{2})$
\item 拒绝域: $F=\frac{\hat\beta_1^2}{\hat\sigma^2/{S_{xx}}}\ge F_{1,n-2}$
\end{enumerate}
\end{enumerate}

\item 判定系数/确定系数 $R^2=\frac{\sum(\hat y_i-\bar y)^2}{\sum(y_i-\bar y)^2}$
\begin{enumerate}
\item 回归平方和 $SS_{\text{回}}=\sum(\hat y_i-\bar y)^2$
\item 总平方和 $SS_{\text{总}}=\sum(y_i-\bar y)^2$
\item 误差平方和 $SS_{\text{误}}=\sum\hat e^2_i=SS_{\text{回}}-SS_{\text{总}}$
\end{enumerate}

\item 回归参数的区间估计
$\hat\sigma_{\hat\beta_0}=\sqrt{\frac{1}{n}+\frac{\bar x^2}{S_{xx}}}\hat\sigma$,
$\hat\sigma_{\hat\beta_1}=\frac{\hat\sigma}{\sqrt{S_{xx}}}$.\\
$\beta_i$的置信区间: $[\hat\beta_i\pm t_{n-2}(\frac{\alpha}{2})\hat\sigma_{\hat\beta_i}]$

\end{enumerate}

\subsection*{常用结论}

\begin{enumerate}
\item 人群中有相同生日: $P(A)=\frac{A_N^n}{N^n}$

\item 把$n$个物品分成$k$组, 每组恰有$n_i$个, 不同分组方法$\frac{n!}{\prod_{i=1}^{k}n_i!}$种

\item 将$n$个球放入$M$个盒子, 有球的盒子数$E(X)=M\left[1-(1-\frac{1}{M})^n\right]$

\begin{multicols}{2}

\item {\bf 基本积分表}
\begin{enumerate}
\item $\int k\d x=kx+C$
\item $\int x^\mu\d x=\frac{x^{\mu+1}}{\mu+1}+C$
\item $\int \frac{\d x}{x}=\ln |x|+C$
\item $\int \frac{\d x}{1+x^2}=\arctan x+C$
\item $\int \frac{\d x}{\sqrt{1-x^2}}=\arcsin x+C$
\item $\int\cos x\d x=\sin x+C$
\item $\int\sin x\d x=-\cos x+C$
%\item $\int \frac{\d x}{\cos^2 x}=\int\sec^2 x\d x=\tan x+C$
%\item $\int \frac{\d x}{\sin^2 x}=\int\csc^2 x\d x=-\cot x+C$
%\item $\int\sec x\tan x\d x=\sec x+C$
%\item $\int\csc x\cot x\d x=-\csc x+C$
\item $\int\e^x\d x=\e^x+C$
\item $\int a^x\d x=\frac{a^x}{\ln a}+C$
%\item $\int\sh x\d x=\ch+C$
%\item $\int\ch x\d x=\sh+C$
%\item $\int\tan x\d x=-\ln|\cos x|+C$
%\item $\int\cot x\d x=\ln|\sin x|+C$
%\item $\int\sec x\d x=\ln|\sec x+\tan x|+C$
%\item $\int\csc x\d x=\ln|\csc x-\cot x|+C$
\iffalse
\item $\int\frac{\d x}{x^2+a^2}=\frac{1}{a}\arctan\frac{x}{a}+C$
\item $\int\frac{\d x}{x^2-a^2}=\frac{1}{2a}\ln|\frac{x-a}{x+a}|+C$
\item $\int\frac{\d x}{\sqrt{a^2-x^2}}=\arcsin\frac{x}{a}+C$
\item $\int\frac{\d x}{\sqrt{x^2+a^2}}=\ln(x+\sqrt{x^2+a^2})+C$
\item $\int\frac{\d x}{\sqrt{x^2-a^2}}=\ln|x+\sqrt{x^2-a^2}|+C$
\fi
\end{enumerate}

\item 微分法则\\
$\d(u\pm v)=\d u\pm\d v$\\
$\d(Cu)=C\d u$\\
$\d(uv)=v\d u+u\d v$\\
$\d(\frac{u}{v})=\frac{v\d u-u\d v}{v^2}$

\end{multicols}


\item 积分换元法\\
$\int f[\varphi(x)]\varphi'(x)\d x=[\int f(u)\d u]_{u=\varphi(x)}$\\
$\int f(x)\d x=[f[\psi(t)]\psi'(t)\d t]_{t=\psi^{-1}(x)}$

\item 二重积分的计算\\
$\iint_D f(x,y)\d\sigma=\int_a^b\d x\int_{\varphi_1(x)}^{\varphi_2(x)}f(x,y)\d y$\\
$\iint_D f(x,y)\d\sigma=\int_c^d\d y\int_{\psi_1(y)}^{\psi_2(y)}f(x,y)\d x$

\end{enumerate}

\newpage
\setcounter{subsection}{0}
\section*{离散数学}
\subsection{数理逻辑}

\begin{enumerate}

\item 逻辑表达式
\begin{enumerate}
\item[0] $0000_2$ $\bm F$永假公式({\bf 矛盾式})
\item[1] $0001_2$ $A\wedge B$合取; $A\cdot B, AB$与 \hfill $A\cap B$交集
\item[2] $0010_2$ $A\stackrel{c}{\to} B$逆条件; $A\wedge\neg B$ \hfill $A-B,A\setminus B$差集
\item[\bf 3] $0011_2$ $A$
\item[4] $0100_2$ $B\stackrel{c}{\to} A$逆条件; $\neg A\wedge B$
\item[\bf 5] $0101_2$ $B$
\item[6] $0110_2$ $A\bar\vee B$不可兼或取; $A\oplus B$异或 \hfill $A\oplus B$对称差
\item[7] $0111_2$ $A\vee B$析取; $A+B$或 \hfill $A\cup B$并集
\item[8] $1000_2$ $A\downarrow B$或非; $\neg(A\vee B)$
\item[9] $1001_2$ $A\leftrightarrow B, A\leftrightarrows B$双条件; $A\odot B$同或; iff %$A=B$; 
\item[10] $1010_2$ $\neg B$非
\item[11] $1011_2$ $B\to A$条件; $A\vee\neg B$
\item[12] $1100_2$ $\neg A$非 \hfill $\sim\! A, \overline{A}, A^\c, A'$补集
\item[13] $1101_2$ $A\to B$条件; $\neg A\vee B$
\item[14] $1110_2$ $A\uparrow B$与非; $\neg(A\wedge B)$
\item[15] $1111_2$ $\bm T$永真公式({\bf 重言式})
\end{enumerate}

\item 布尔合取(小项): $m_i\wedge m_j=\bm F$, $\sum_{i=0}^{2^n-1} m_i=\bigvee m_i\Leftrightarrow \bm T$\\
布尔析取(大项): $M_i\vee M_j=\bm T$, $\prod_{i=0}^{2^n-1} M_i=\bigwedge M_i\Leftrightarrow \bm F$

\item 证明方法
\begin{enumerate}
\item[P规则] 前提引入 \hfill P, P(附加前提)
\item[T规则] 结论引用 \hfill T$(i)E$, T$(i)I$
\item[CP规则] 由$(S\wedge R) \Rightarrow C$, 证得$S \Rightarrow (R\to C)$ \hfill CP
\item[US规则] 全称指定规则 ${(\forall x)P(x)}, \quad {\therefore P(c)}$  \hfill US$(i)$
\item[UG规则] 全称推广规则 ${P(c)}, \quad {\therefore (\forall x)P(c)}$  \hfill UG$(i)$
\item[ES规则] 存在指定规则 ${(\exists x)P(x)}, \quad {\therefore P(c)}$  \hfill ES$(i)$
\item[EG规则] 存在推广规则 ${P(x)}, \quad {\therefore (\exists x)P(x)}$  \hfill EG$(i)$
\end{enumerate}

\item 常用蕴含式
\begin{enumerate}[label={$I_{\arabic*}$}]
\item $P\wedge Q \Rightarrow P$
\item $P\wedge Q \Rightarrow Q$
\item $P \Rightarrow P\vee Q$
\item $Q \Rightarrow P\vee Q$
\item $\neg P \Rightarrow P\to Q$
\item $Q \Rightarrow P\to Q$
\item $\neg(P\to Q) \Rightarrow P$
\item $\neg(P\to Q) \Rightarrow \neg Q$
\item $P, Q \Rightarrow P\wedge Q$
\item $\neg P, P\vee Q \Rightarrow Q$
\item $P, P\to Q \Rightarrow Q$
\item $\neg Q, P\to Q \Rightarrow \neg P$
\item $P\to Q, Q\to R \Rightarrow P\to R$
\item $P\vee Q, P\to R, Q\to R \Rightarrow R$
\item $A\to B \Rightarrow (A\vee C)\to(B\vee C)$
\item $A\to B \Rightarrow (A\wedge C)\to(B\wedge C)$
\item $(\forall x)A(x)\vee(\forall x)B(x) \Rightarrow (\forall x)(A(x)\vee B(x))$
\item $(\exists x)(A(x)\wedge B(x)) \Rightarrow (\exists x)A(x)\wedge(\exists x)B(x)$
\item $(\exists x)A(x)\to (\forall x)B(x) \Rightarrow (\forall x)(A(x)\to B(x))$
\end{enumerate}

\item 常用等价式 (改换符号可得集合的运算律)
\begin{enumerate}[label={$E_{\arabic*}$}]
\item (对合律) $\neg\neg P \Leftrightarrow P$
\item (交换律) $P\wedge Q \Leftrightarrow Q\wedge P$
\item (交换律) $P\vee Q \Leftrightarrow Q\vee P$
\item (结合律) $(P\wedge Q)\wedge R \Leftrightarrow P\wedge(Q\wedge R)$
\item (结合律) $(P\vee Q)\vee R \Leftrightarrow P\vee(Q\vee R)$
\item (分配律) $P\wedge(Q\vee R) \Leftrightarrow (P\wedge Q)\vee(P\wedge R)$
\item (分配律) $P\vee(Q\wedge R) \Leftrightarrow (P\vee Q)\wedge(P\vee R)$
\item (德摩根律) $\neg(P\wedge Q) \Leftrightarrow \neg P\vee \neg Q$
\item (德摩根律) $\neg(P\vee Q) \Leftrightarrow \neg P\wedge \neg Q$
\item (幂等律) $P\vee P \Leftrightarrow P$
\item (幂等律) $P\wedge P \Leftrightarrow P$
\item (同一律) $R\vee(P\wedge \neg P) \Leftrightarrow R$
\item (同一律) $R\wedge(P\vee \neg P) \Leftrightarrow R$
\item (零律) $R\vee(P\vee \neg P) \Leftrightarrow \bm T$
\item (零律) $R\wedge(P\wedge \neg P) \Leftrightarrow \bm F$
\item $P\to Q \Leftrightarrow \neg P\wedge Q$
\item $\neg(P\to Q) \Leftrightarrow P\wedge\neg Q$
\item $P\to Q \Leftrightarrow \neg Q\to\neg P$
\item $P\to (Q\to R) \Leftrightarrow (P\wedge Q)\to R$
\item $P\leftrightarrow Q \Leftrightarrow (P\to Q)\wedge(Q\to P)$
\item $P\leftrightarrow Q \Leftrightarrow (P\wedge Q)\wedge(\neg P\wedge \neg Q)$
\item $\neg(P\leftrightarrow Q) \Leftrightarrow P\leftrightarrow\neg Q$
\item $(\exists x)(A(x)\vee B(x)) \Leftrightarrow (\exists x)A(x)\vee(\exists x)B(x)$
\item $(\forall x)(A(x)\wedge B(x)) \Leftrightarrow (\forall x)A(x)\wedge(\forall x)B(x)$
\item $\neg(\exists x)A(x) \Leftrightarrow (\forall x)\neg A(x)$
\item $\neg(\forall x)A(x) \Leftrightarrow (\exists x)\neg A(x)$
\item $(\forall x)(A\vee B(x)) \Leftrightarrow A\vee(\forall x)B(x)$
\item $(\exists x)(A\wedge B(x)) \Leftrightarrow A\wedge(\exists x)B(x)$
\item $(\exists x)(A(x)\to B(x)) \Leftrightarrow (\forall x)A(x)\to(\exists x)B(x)$
\item $(\forall x)A(x)\to B \Leftrightarrow (\exists x)(A(x)\to B)$
\item $(\exists x)A(x)\to B \Leftrightarrow (\forall x)(A(x)\to B)$
\item $A\to(\forall x)B(x) \Leftrightarrow (\forall x)(A\to B(x))$
\item $A\to(\exists x)B(x) \Leftrightarrow (\exists x)(A\to B(x))$
\end{enumerate}

\end{enumerate}


\subsection{集合论}

\begin{enumerate}

%\item 空集$\varnothing$; 全集$U,E$.
%\item 基数$|A|=\card(A)$.
%\item 相等$A=B$, 不相等$A\neq B$.
%\item 子集$A\subseteq B$,$B\supseteq A$, 真子集$A\subset B$.
%\item 罗素悖论 $S=\{A|A\notin A\}$.
%\item 交集$A\cap B$; 并集$A\cup B$; 差集$A-B$;\\ 补集$\sim\! A, \overline{A}, A^{\rm c}, A'$; 对称差$A+B, A\oplus B$.
\item 幂集$\mathscr{P}(A)=\{X|X\subseteq A\}$, $|\mathscr{P}(A)|=2^{|A|}$.\
\item 序偶$\<x,y\>\defeq \big\{\{x\}, \{x,y\}\big\}$, $\<x,y,z\>\defeq \big\<\<x,y\>,z\big\>$.
\item 笛卡尔积(直积): $A\times B=\{\<u,v\>|u\in A\wedge v\in B\}$.
\begin{enumerate}
\item $A^n=A\times A\times\cdots\times A$
\item $A\times \varnothing=\varnothing$, $\varnothing\times A=\varnothing$
\item $A\times B\neq B\times A, (A\times B)\times C\neq A\times(B\times C)$
\item $*\defeq \cup, \cap, -$,\\
左分配 $A\times(B*C)=(A\times B)*(A\times C)$\\
右分配 $(B*C)\times A=(B\times A)*(C\times A)$
\item $A\subseteq B\Leftrightarrow A\times C\subseteq B\times C\Leftrightarrow C\times A\subseteq C\times B$
\item $A\times B\subseteq C\times D\Leftrightarrow A\subseteq C, B\subseteq D$
\end{enumerate}
\item 覆盖$\bigcup S_i=A$; 划分$\bigcup S_i=A$, 且$S_i\cap S_j=\varnothing$.
\item 鸽笼原理: 把$n$个物体放入$m$个盒子, \\则至少有一个盒子里有$\lceil n/m \rceil$个物体.
\item 容斥原理(加法原理): $|A\cup B|=|A|+|B|-|A\cap B|$,\\
$|A\cup B\cup C|=|A|+|B|+|C|-|A\cap B|-|B\cap C|-|C\cap A|+|A\cap B\cap C|$.

\item 关系$xRy\defeq \<x,y\>\in R$. 恒等关系$I_A=\{\<x,x\>|x\in A\}$.
\item 定义域$\dom R$, 值域$\ran R$, 域$\FLD R=\dom R\cup\ran R$.
%\item 关系矩阵$M_R=(r_{ij})_{m\times n}$, 关系图$G_R$.
\item 逆关系$R^\c=\{\<b,a\>|\<a,b\>\in R\}$, $\<a,b\>\in R\Leftrightarrow\<b,a\>\in R_\c$.
\item 复合关系$R_1\circ R_2$, $R^{(m)}=R\circ\cdots\circ R$.\\
$c_{ij}=\bigvee_{k=1}^n(a_{ik}\wedge b_{kj})$. $(R_1\circ R_2)^\c=R_1^\c\circ R_2^\c$.

\item 关系的基本类型
\begin{enumerate}
\item[自反] iff $I_A\subseteq R \Leftrightarrow r(R)=R$ \hfill 如$\cong$ \\
    $\Leftrightarrow$ $(\forall x)(x\in A\to xRx)$\\
    $\Leftrightarrow$ $M_R$的主对角元全为1\\
    $\Leftrightarrow$ $G_R$每一结点有自回路
\item[对称] iff $R^\c=R \Leftrightarrow s(R)=R$ \hfill 如$=$,等势,同余 \\
    $\Leftrightarrow$ $(\forall x, y)(x, y\in A\wedge xRy\to yRx)$\\
    $\Leftrightarrow$ $M_R$是对称矩阵\\
    $\Leftrightarrow$ $G_R$有向边成对出现
\item[传递] iff $R\circ R\subseteq R \Leftrightarrow t(R)=R$ \hfill 如$=$,$<$,$\le$,$\subset$,$\subseteq$,整除,等势,同余 \\
    $\Leftrightarrow$ $(\forall x, y, z)(x,y,z\in A\wedge xRy\wedge yRz\to xRz)$\\
    $\Leftrightarrow$ $G_R$若从$a$到$b$有一条路径,则从$a$到$b$有一条弧
\item[反自反] iff $R\cap I_A=\varnothing$ \hfill 如$<$,$>$ \\
    $\Leftrightarrow$ $(\forall x)(x\in A\to x\net R x)$\\
    $\Leftrightarrow$ $M_R$的主对角元全为0\\
    $\Leftrightarrow$ $G_R$每一结点无自回路
\item[反对称] iff $R\cap R^\c\subseteq I_A$ \hfill 如$\le$,$\subseteq$ \\
    $\Leftrightarrow$ $(\forall x,y)(x,y\in A\wedge xRy\wedge yRx\to x=y)$\\
    $\Leftrightarrow$ $M_R$中$i\neq j\wedge(a_{ij}=1)\to(a_{ji}=0)$\\
    $\Leftrightarrow$ $G_R$若从$a$到$b$有一条弧,则必无$b$到$a$的弧
\end{enumerate}

\item 关系的闭包运算和构造闭包的方法
\begin{enumerate}
\item 自反闭包 $r(R)=R\cup I_A$,\\
$M_r=M+I$;
\item 对称闭包 $s(R)=R\cup R^\c$,\\
$M_s=M+M^\T$;
\item 传递闭包 $t(R)=\bigcup_{i=1}^{\infty}R^i\defeq R^+$,\\
$M_t=M+M^2+M^3+\cdots$,\\
Warshall-Floyd算法: $A[i,j]\gets A[i,k]+A[k,j]$.
\end{enumerate}

\item
\begin{enumerate}
\item $rs(R)=sr(R)$;
\item $rt(R)=tr(R)$;
\item $st(R)\subseteq ts(R)$.
\end{enumerate}

\item 等价关系$R$: 自反、对称、传递.
\begin{enumerate}
\item 等价类: $[a]_R=\{x|x\in A\wedge aRx\}$. $aRb\Leftrightarrow [a]_R=[b]_R$.
\item 商集$A/R=\{[a]_R|a\in A\}$. $R_1=R_2$ iff $A/R_1=A/R_2$.
\item 一个等价关系确定一个划分, 一个划分确定一个等价关系.
\end{enumerate}

\item 相容关系$r$: 自反、对称.
\begin{enumerate}
\item 相容类: $a_1 r a_2$.
\item 最大相容类: $(\forall C) C\subseteq C_r$.
\item 完全覆盖: $C_r(A)$.
\item 覆盖$\{A_i\}$确定的关系$r=\bigcup A_i\times A_i$是相容关系.
\end{enumerate}

\item 偏序关系$\preccurlyeq$: 自反、反对称且传递; 偏序集$\<A,\preccurlyeq\>$.
\item 哈斯图: 表达盖住关系. $\COV A=\{\<x,y\>|x,y\in A; $\\$x\preccurlyeq y, x\neq y; (  z\in A)(x\preccurlyeq z\preccurlyeq y)\}$
\item 
\begin{enumerate}
\item 极大元$b$: $(\forall x\in B)(b\preccurlyeq x\to x=b)$. 最末端的元素;
\item 极小元$b$: $(\forall x\in B)(x\preccurlyeq b\to x=b)$. 最底层的元素.
\end{enumerate}
\item 
\begin{enumerate}
\item 最大元$b$: $(\forall x)(x\in B\to x\preccurlyeq b)$;
\item 最小元$b$: $(\forall x)(x\in B\to b\preccurlyeq x)$.
\end{enumerate}
\item 设$\<A,\preccurlyeq\>, B\subseteq A, a\in A$:
\begin{enumerate}
\item 上界$a$: $(\forall x)(x\in B\to x\preccurlyeq a)$, 上确界$\LUB B=\{a\}$;
\item 下界$b$: $(\forall x)(x\in B\to b\preccurlyeq x)$, 下确界$\GLB B=\{b\}$.
\end{enumerate}

\item 良序: 每一非空子集总含有最小元. 良序集合:$\<A,\preccurlyeq\>$.\\
良序集合一定是全序集合, 有限全序集合一定是良序集合.

\end{enumerate}


\subsection{图论}

\begin{enumerate}

\item 

\end{enumerate}


\end{document}
