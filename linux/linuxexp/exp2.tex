\documentclass[cs4size,a4paper,nofonts]{ctexart}
\usepackage[utf8]{inputenc}
\def\tjf{{\tt{田劲锋}}}
\def\titlec{Linux常用命令(一)}
\usepackage[a4paper,margin=2.2cm]{geometry} % 页面设置
\usepackage[unicode,breaklinks=true,
colorlinks=true,linkcolor=black,anchorcolor=black,citecolor=black,urlcolor=black,
pdftitle={\titlec},pdfauthor={\tjf}]{hyperref}
%\CTEXsetup[number=\chinese{section}, format={\large\sf\bfseries}]{section}
\usepackage{latexsym,amsmath,amssymb,bm}
\usepackage{graphicx}
\usepackage{subfigure}
\usepackage{wrapfig}
\usepackage{fancyvrb}
\DefineShortVerb{\|}
\fvset{frame=single}

\setmainfont{Times New Roman}
\setCJKmainfont[BoldFont={SimHei}]{SimSun}  % 主要字体:宋体、黑体
\setCJKsansfont[BoldFont={STZhongsong}]{STFangsong} % 次要字体:仿宋、中宋
\setCJKmonofont{KFKai} % 等宽字体:楷体

\CJKsetecglue{\hspace{0.1em}}
\renewcommand\CJKglue{\hskip -0.3pt plus 0.08\baselineskip}
\frenchspacing
\widowpenalty=10000
\linespread{1.2} % 行距

\usepackage[inline]{enumitem} % 调整列表样式
\setlist{noitemsep}
\setlist[itemize]{topsep=0pt,partopsep=0pt,itemsep=0pt,parsep=0pt}
\setlist[enumerate]{topsep=0pt,partopsep=0pt,itemsep=0pt,parsep=0pt}
\setlist[enumerate,1]{label={(\arabic*)}}

\CTEXsetup[beforeskip={0pt},afterskip={0pt}]{paragraph}

%\makeindex
\pagestyle{plain}

\begin{document}

%%%% 开始 %%%%

\begin{titlepage}

\begin{center}

\includegraphics[height=1cm]{image/haut.png}

\vspace*{1cm}
{\liti\fontsize{48pt}{50pt}{课\quad 程\quad 设\quad 计}}

\vspace*{4cm}
{\fontsize{36}{50}\sf\bfseries \titlec}

\vspace*{1cm}
{\huge\sf\bfseries \titlee}

\vfill
{\large
\newcommand{\ctline}[2]{\makebox[6em][s]{\bf #1}:\underline{\makebox[14em][c]{\qquad #2\qquad}}\\}
\ctline{课程设计名称}{数据结构课程设计}
\ctline{专业班级}{计算机 1303 班}
\ctline{学生姓名}{\tjf}
\ctline{学号}{201316920311}
\ctline{指导教师}{白\quad 浩}
\ctline{课程设计时间}{\today}
}

\end{center}

\end{titlepage}


% \CTEXnoindent

\paragraph{实验题目:}Linux常用命令(一)

\paragraph{实验目的:}(1)掌握图形方式下启动Shell程序的方法;(2)理解目录操作命令,包括ls命令、cd命令、pwd命令、mkdir命令和rmdir命令;(3)理解文件操作的基本命令,包括touch命令、cat命令、cp命令、rm命令、mv命令和chmod命令。

\paragraph{实验内容:}
\begin{enumerate}
\item 列举出目录/etc下的子目录和文件(包括隐藏文件)的详细内容,并解释执行结果中某一行的各个部分的具体含义。
\item 使用命令进入当前登录用户的宿主目录。
\item 假设登录用户为ubuntu,在该用户的宿主目录下,使用一行命令建立多级目录,例如dir/dir1/dir2。
\item 在上例建立的目录dir1下创建一个空文件file.txt,然后使用目录删除命令删除目录dir2和dir1,看是否能够成功删除。
\item 在用户ubuntu用户的宿主目录下,使用cat建立两个文件a.txt和b.txt,分别写入“hello”和“world”,然后使用cat命令把a.txt和b.txt文件内容加上行号后将内容复制到c.txt文件中。
\item 在用户ubuntu用户的宿主目录下,使用cat建立一个文件d.txt,并将其内容追加到文件c.txt中。
\item 在用户ubuntu用户的宿主目录下建立两个新目录newdir1和newdir2,在目录newdir1下建立一个新文件file.txt和新目录newdir11,试完成下列任务:1)将目录newdir1本身连同其下的子目录和文件一并拷贝到目录newdir2下,查看目录newdir2,测试是否成功。2)将目录newdir1下的子目录和文件(不包括目录newdir1本身)一并拷贝到目录newdir2下,查看目录newdir2,测试是否成功。
\end{enumerate}

\paragraph{实验步骤:}

\newcommand{\image}[3][height=10cm]{%\begin{minipage}[t]{0.5\textwidth}
    \centering
    \includegraphics[#1]{images/#2.png}
    \caption{#3}
    \label{fig:#3}
%\end{minipage}
}

\newcommand{\subimage}[4][height=10cm]{
    \centering
    \subfigure[]{
        \includegraphics[#1]{images/#2.png}
        \label{fig:#3:#4}
    }
}

\newcommand{\images}[3][width=\textwidth]{\begin{minipage}[t]{0.33\textwidth}
\centering
\includegraphics[#1]{images/#2.png}
\caption{#3}
\label{fig:#3}
\end{minipage}}

首先,我将终端配色设置为了“白底黑字”,为了打印出的实验报告清晰可辨识。也给出了相应命令的文字版。

\begin{enumerate}

\item 如图\ref{fig:列举目录}所示,使用~|ls|~命令,键入:
\begin{Verbatim}
$ ls -lha /etc
总用量 1.2M
drwxr-xr-x 133 root root    12K  4月 27 12:52 .
drwxr-xr-x  23 root root   4.0K 11月  4 18:04 ..
drwxr-xr-x   3 root root   4.0K  4月 25  2013 acpi
-rw-r--r--   1 root root   3.0K  4月 25  2013 adduser.conf
... ...
drwxr-xr-x  10 root root   4.0K  4月 25  2013 X11
... ...
-rw-r--r--   1 root root    349  2月 27  2013 zsh_command_not_found
\end{Verbatim}
显示出~|/etc|~目录下所有文件的详细信息,并给出了目录的总大小(不递归计算)。

其中一行:
\begin{verbatim}
drwxr-xr-x   3 root root   4.0K  4月 25  2013 acpi
\end{verbatim}
表示~|acpi|~是个目录,所有者可读可写可执行,用户组可读可执行,其他用户可读可执行;目录下有3个文件或目录;所有者是root,用户组是root,目录大小是4.0K(并不是目录下文件大小的总和),最后修改时间是2013年4月25日。

\begin{figure}[htp]
\subimage{2015-04-27-(1)-1}{列举目录}{1}
\subimage{2015-04-27-(1)-2}{列举目录}{2}
\caption{列举目录}\label{fig:列举目录}
\end{figure}

\item 进入宿主目录,如图\ref{fig:进入主目录},使用~|cd|~命令:
\begin{Verbatim}
$ cd ~
\end{Verbatim}
或者直接使用:
\begin{Verbatim}
$ cd
\end{Verbatim}
都可以。

\begin{figure}[htp]
%\image{2015-04-27-(0)}{初始界面}
\image{2015-04-27-(2)}{进入主目录}
\end{figure}

\item 如图\ref{fig:建立目录}所示,使用|mkdir -p|命令:
\begin{Verbatim}
$ mkdir -p dir1/dir2
\end{Verbatim}
建立指定的多级目录。使用:
\begin{Verbatim}
$ ls
dir1  公共的  模板  视频  图片  文档  下载  音乐  桌面
$ ls dir1/
dir2
$ ls dir1/dir2/
\end{Verbatim}
来检查是否成功创建。

\begin{figure}[htp]
\image{2015-04-27-(3)}{建立目录}
\end{figure}

\item 如图\ref{fig:删除目录},首先使用~|touch|~创建一个文件:
\begin{Verbatim}
$ touch dir1/file.txt
$ ls dir1/
dir2  file.txt
\end{Verbatim}
直接用|rmdir|删除该目录,提示目录非空:
\begin{Verbatim}
~$ rmdir dir1/
rmdir: 删除 "dir1/" 失败: 目录非空
\end{Verbatim}

\begin{figure}[htp]
\image{2015-04-27-(4)}{删除目录}
\end{figure}

\item 如图\ref{fig:文件合并},首先使用~|cat|~配合标准输出重定向符号~|>|~创建两个文件:
\begin{Verbatim}
$ cat > a.txt
hello
$ cat > b.txt
world
\end{Verbatim}
去掉重定向符可以显示文件内容:
\begin{Verbatim}
$ cat a.txt
hello
$ cat b.txt
world
\end{Verbatim}
最后带着行号合并到一个新文件~|c.txt|~中:
\begin{Verbatim}
$ cat -n a.txt b.txt > c.txt
$ cat c.txt 
     1	hello
     2	world
\end{Verbatim}

\begin{figure}[htp]
\image{2015-04-27-(5)}{文件合并}
\end{figure}

\item 如图\ref{fig:文件追加},首先创建一个文件:
\begin{Verbatim}
$ cat > d.txt
pretty
rhythm   
\end{Verbatim}
然后把新的文件内容用~|>>|~追加到~|c.txt|~的末尾:
\begin{Verbatim}
$ cat -n d.txt >> c.txt
$ cat c.txt 
     1	hello
     2	world
     1	pretty
     2	rhythm
\end{Verbatim}

\begin{figure}[htp]
\image{2015-04-27-(6)}{文件追加}
\end{figure}

\item 如图\ref{fig:复制目录},首先创建两个目录:
\begin{Verbatim}
$ mkdir newdir1 newdir2
\end{Verbatim}
进入该目录,创建一个文件和一个目录:
\begin{Verbatim}
$ cd newdir1
$ touch file.txt
$ mkdir newdir11
$ cd ..
\end{Verbatim}
\begin{enumerate}
\item 将~|newdir1|~复制到~|newdir2|~目录下:
\begin{Verbatim}
$ ls newdir2
$ cp -r newdir1 newdir2/
$ find newdir2
newdir2
newdir2/newdir1
newdir2/newdir1/newdir11
newdir2/newdir1/file.txt
\end{Verbatim}
\item 将~|newdir1|~下的文件递归复制到~|newdir2|~目录中:
\begin{Verbatim}
$ ls newdir2
newdir1
$ cp -r newdir1/* newdir2
$ find newdir2
newdir2
newdir2/newdir11
newdir2/newdir1
newdir2/newdir1/newdir11
newdir2/newdir1/file.txt
newdir2/file.txt
\end{Verbatim}
\end{enumerate}
实质上只是一个通配符~|*|~的区别而已。注意上面的~|newdir2/newdir1|~目录是上一次复制的遗留产物。

\begin{figure}[htp]
%\subimage{2015-04-27-(7)-1}{复制目录}{1}
\subimage{2015-04-27-(7)-2}{复制目录}{2}
\subimage{2015-04-27-(7)-3}{复制目录}{3}
\caption{复制目录}\label{fig:复制目录}
\end{figure}

\end{enumerate}

\paragraph{实验体会:}\quad

这次实验是练习Shell中非常基本的几个文件和目录操作命令。这些命令在日常应用中都是很常用的。

最有趣的其实是~|cat|~,明明只是简单把输入的内容输出出来,却由于重定向而变得非常强大。我最喜欢用~|cat|~来复制粘贴文本了。

\end{document}
